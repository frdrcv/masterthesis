%!TEX root = memoire-maitrise.tex
\chapter*{Avant-propos}         % ne pas numéroter
\phantomsection\addcontentsline{toc}{chapter}{Avant-propos} % inclure dans TdM

\section{Problématique}
\label{sec:problematique}
\todo{Section entière à retravailler}


\subsection{Énoncé du problème}
\label{sub:enonce_du_probleme}
\todo{à refaire, étant donné que celle du plan de recherche se retrouve dans
  tout ce chapitre. Peut-être faire un résumé? Prendre les objectifs de
  l'introduction et les rapatrier ici?}

Ce mémoire de maîtrise s'inscrit dans la suite d'un cheminement qui m'a amené à me pencher scientifiquement sur un ensemble plus large, un groupe d'individu auquel je considère moi-même appartenir, la communauté \lgbt{}, plus communément appellée lesbienne, gaie, bisexuelle et trans, à laquelle j'ajoute les individus queers et les asexuels/alliés.
Nous voulons ici présenter les différentes façons qu'ont ces communautés d'utiliser l'espace au Québec et comment elles marquent les territoires qu'elles investissent pour montrer leur présence et caractériser les lieux.
Ainsi, nous essayerons également de nous intéresser aux raisons pour lesquelles ces orientations sexuelles et ces identités de genre ont modelées l'espaces qu'elles fréquentent et pourquoi on ne peut pas parler d'un simple groupe souvent considéré comme homogène.
Plutôt, comme nous le verrons, si les façons de marquer l'espaces sont particulièrement variées, les groupes qui sont représentés le sont tout autant et ces derniers ne cohabitent pas nécessairement.

Ce mémoire fait suite à un essai qui a été produit afin de recenser l'ensemble des lieux \lgbt{} dans la ville de Québec.
Nous voulions d'abord débuter notre travail de recherche en nous intéressant à une ville importante mais ne présentant pas les caractéristiques des métropoles souvent abordées dans les études \lgbt{}.

\section*{Objectifs et hypothèse}
Ce mémoire vise rendre compte de notre recherche visant à dresser un portrait des géosymboles des espaces \emph{queers} en milieux urbains, plus particulièrement ceux des villes de Montréal et de Québec, mais aussi des villes de plus petites envergure.
Pour arriver à atteindre ce but général, nous nous sommes également fixé des objectifs plus spécifiques.
D'abord, cette recherche a permis de développer une méthodologie permettant l'identification et le catalogage des géosymboles marquants les espaces \emph{queers}, en tenant compte du caractère éphémère ou permanent, matériel ou immatériel de ceux-ci.
Ensuite, nous avons pris à tâche de rechercher et répertorier les géosymboles \emph{queers} en tenant en compte de l'environnement et de la forme de ceux-ci dans une variété relative de milieux urbains (métropolitains et non métropolitains) pour par après les géolocaliser.
Nous avons aussi voulu dresser les différences et ressemblances entre les différentes expressions spatiales et symboliques de la diversité sexuelle, notamment selon le genre et l'orientation sexuelle.
Enfin, nous avons cherché à faire progresser la connaissance portant sur les espaces \lgbt{} urbains de la province du Québec en prenant en premier comme terrain d'étude les villes de Québec et de Montréal puis les villes de taille inférieures selon la présence d'une communauté \lgbt{}.
La figure~\ref{fig:arrondissementsmtl} montre l'emplacement de ces deux arrondissements et nous verrons dans les chapitres suivants où ces espaces ont été situés en complément des données accumulées dans cette recherche \todo{approfondir le contexte historique?}

\begin{figure}[ht]
 \centering
 \includegraphics[width=1\textwidth]{arrondissementsmtl}
 \caption[Arrondissements ciblés: ville de Montréal]{Arrondissements ciblés pour la collecte de données: ville de Montréal}\label{fig:arrondissementsmtl}
\end{figure}


\subsection*{Question de recherche} % (fold)
\label{sub:hypothese}
\todo{À simplifier}
%L'ensemble des objectifs précédents ont servi à répondre à la question de recherche que nous exposerons dans les prochaines lignes.
Pour répondre à ces différents objectifs, nous avons articulé la question de recherche suivante: comment les communautés \lgbt{} articulent-elles une relation particulière avec l'espace, et comment celles-ci s'affichent-elles dans celui-ci?
Nous supposons que les groupes et individus \lgbt{} occupent l'espace d'une façon particulière et que cette manière d'occuper l'espace sert à se regrouper autour d'une identité partagée et de l'afficher.
Cette occupation s'exprime par une variété de géosymboles signifiant la différence ou l'inclusion des individus selon cette identité.
L'existence de ces espaces, et des géosymboles les identifiant est permanente ou temporaire, selon les modalités particulières de l'espace.
Ces géosymboles marquants les espaces des communautés \lgbt{} sont récents; leur apparition serait survenue à partir de la \emph{révolution sexuelle}, une période durant laquelle les normes sexuelles se seraient en apparence assouplies.
Plus précisément, pour les communautés \lgbt{}, ce changement débuterait à partir des émeutes de Stonewall aux États-Unis et se serait répandu ailleurs en occident et donc au Québec.
%AJOUT-----------------------------
Ces géosymboles se seraient propagés d'abord dans les milieux urbains selon des paramètres propres à l'histoire des géosymboles, du milieu urbain, de la taille de ce milieu et de sa place dans la hiérarchie urbaine.

Pour conclure, le prochain chapitre servira à confirmer ces choix d'objectifs de recherche par un approfondissement de la problématique de recherche.
