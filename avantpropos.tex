%!TEX root = memoire-maitrise.tex
\chapter*{Avant-propos}         % ne pas numéroter
\phantomsection\addcontentsline{toc}{chapter}{Avant-propos} % inclure dans TdM

\section{Problématique}
\label{sec:problematique}
\todo{Section entière à retravailler}


\subsection{Énoncé du problème}
\label{sub:enonce_du_probleme} 
\todo{à refaire, étant donné que celle du plan de recherche se retrouve dans
  tout ce chapitre. Peut-être faire un résumé? Prendre les objectifs de
  l'introduction et les rapatrier ici?}

Ce mémoire de maîtrise s'inscrit dans la suite d'un cheminement qui m'a amené à
me pencher scientifiquement sur un ensemble plus large, un groupe d'individu
auquel je considère moi-même appartenir, la communauté \lgbt{}, plus communément
appellée lesbienne, gaie, bisexuelle et trans, à laquelle j'ajoute les individus
queers et les asexuels/alliés. Nous nous pencherons dans ce mémoire
sur les raisons pour lesquelles ces orientations et identités sont choisies par
les individus, mais un fait demeure: la liste n'est pas exhaustive. D'autres
acronymes sont parfois utilisés pour rendre compte également des individus
intersexuées ou encore des individus appartenant aux nations autochtones dont
l'identité sexuelle nommée \emph{two-spirit} s'inscrit dans un cadre culturel
différent des européens et de leurs descendants. Ce mémoire fait donc suite à un
essai que j'ai produit à la toute fin de mon baccalauréat, alors que je
constatais, par mes lectures et mes connaissances acquises durant certains cours
de premier cycle, la possibilité de lier la diversité sexuelle à l'analyse
géographique de l'espace. Ce premier travail s'intéressa à la ville de Québec,
une première. En effet, en aucun endroit dans la littérature scientifique en
géographie on se penche directement sur une autre ville que Montréal, autant en
Français qu'en Anglais.

Certaines précautions méritent d'être soulevées avant la poursuite de ce texte.
Si la recherche d'objectivité et d'effacement de l'auteur sont deux principes
importants dans la recherche scientifique au sens large, on ne peut nier le
contexte dans lequel nous-mêmes nous évoluons et quel regard nous sommes portés
à avoir sur les choses. Je suis moi-même né blanc en terre d'Amérique, un
territoire nommé par ses habitants les plus anciens comme la terre de la tortue.
Descendant d'ancêtres colonisateurs, j'habite une terre uxsurpée à une multitude
de nations qui n'ont qu'offert l'hospitalité à des frères et sœurs d'ailleurs.
Cette terre prise, elle fut transformée par une multitude de familles prises
dans un carcan de contraintes sociales et économiques qui menèrent au fil du
siècle à événementiel du capitalisme et par l'exploitation de gens venus
d'ailleurs, réfugiés de guerres et d'autres fuyant les épidémies. À ce titre, on
peut penser aux descendants des populations chinoises, irlandaises, africaines,
etc.

%%% Local Variables:
%%% mode: latex
%%% TeX-master: "memoire-maitrise"
%%% End:
