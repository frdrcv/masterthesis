%!TEX root = ../../memoire-maitrise.tex

\chapter{Le paysage queer d'aujourd'hui}
\label{cha:le_paysage_queer_d_aujourd_hui}
Comme nous avons pu le voir dans notre analyse de données, le paysage géosymboles des communautés \lgbt{} varient énormément, que ce soit selon les axes du genre ou de l'orientation sexuelle, ou encore d'autres qu'on ne lit pas automatiment à la sexualité, comme la classe sociale ou l'ethnicité.
Ces variations débordant le cadre de la sexualité, elles confirment les propos de \todo{faire une citation adéquate de Jackson et Son}, comme quoi nous ne pouvons réduire notre analyse de l'hétérosexualité au simple cadre de la sexualité, celle-ci se manifestant dans l'ensemble des sphères de la société, au même titre que le capitalisme ou l'hégémonie blanche, par exemple.


Nous l'avons souligné déjà à quelques reprises dans ce mémoire, mais plusieurs limitent s'imposent à l'étendue des résultats et à l'analyse dont nous faisons du territoire des communautés \lgbt{} du Québec.

Y'a-t-il une place à la critique des différentes manifestations dont nous avons fait la revue?
Il ne s'agit pas du but premier de notre travail.
Par contre, nous pouvons constater certaines des tendances que nous avons traitées dans le premier chapitre, à savoir que les enjeux d'hétéronormativité semblent se manifester.

\section{Types de symboles}
\label{sec:types_de_symboles}


\section{Sens et utilité}
\label{sec:sens_et_utilit_}
\begin{quotation}
Ces relations se construisent comme une appropriation symbolique de l'espace, sous l'effet de forces qui tantôt unissent, tantôt opposent les acteurs sociaux. 
D'où l'idée qu'il existe, dans une société ou un milieu donné, plusieurs « types » et plusieurs « niveaux » de territorialités, celles-ci pouvant être symétriques ou non, selon la nature des échanges qui s'établissent dans le système (simples relations bilatérales ou coûts supérieurs à consentir qui mettent en danger la structure de ce système).\citep[41]{Courville1991}
\end{quotation}

\section{Types d'espaces rencontrés}
\label{sec:types_d_espaces_rencontr_s}


\section{Pistes de recherche futures}
\label{sec:pistes_de_recherches}

Suite aux différents résultats de ce mémoire ainsi que les conclusions soulevées par les différentes approches méthodologiques sur lesquelles s'assoient ce travail, il apparait maintenant nécessaire de poursuivre le travail auprès des différents groupes rencontrés sur le terrain.
En effet, le portrait dressés restent fortement influencées par ma perspective personnelle du chercheur, autant comme géographe que comme membre de la communauté LGBTQ, avec mes a priori et une volonté de demeurer objectif qui est nécessairement partielle.

Ce travail auprès de la population pourrait prendre diverses formes. 
D'abord, il est envisageable de maintenant prendre contact avec avec certains des organismes treprésent de ces groupes, que ce soit le GRIS-Montréal ou Qouleur par exemple et leur offrir, en plus des conclusions soulevées par ce mémoire, une possibilité de poursuivre la recherche en tentant d'appronfondir l'analyse de l'occupation de l'espace urbain par la population qu'ils représentent \todo{reformuler}.

Cet approdondissement pourrait prendre la forme du cartographie participative des espaces \lgbt{} par la population et pour celles-ci.
En effet, au-delà des cartographies qu'on retrouve à l'intérieur du Fugues, dans le domaine de la recherche \todo{trouver la citation de Podmore pour sa   cartographie} ou exceptionnelles dans le cadre de certains événements~\parencite{Pervers/Cite2015}, aucun outil ne centralise l'ensemble de ces connaissances. 
Comme nous le soulevons dans cette recherche, les espaces \lgbt{} dans les villes de Montréal ou de Québec sont multiples et méritent, en concordance avec la volonté de certains groupes comme ceux de la Dyke March, une meilleure visibilité.
Cette visibilité pourrait potentiellement offrir aux individus d'orientation ou de genre variés de retrouvés les gens qui leur ressemble et obtenir des ressources adaptées à ceux et celles-ci, que ce soit des lieux de socialisation comme les bars réputés sécuritaires ou des cliniques offrant des soins particuliers.

Également, un espace qui a très peu été touché par cette recherche est l'ensemble des villes et villages où s'organisent ou vivent des individus des minorités sexuelles. 


Un travail de plus grande ampleur au niveau géohistorique ouvrirait la possibilité à une étude plus approfondie des géosymboles, mais, en l'absence de ces données, il est difficile de décrire plus particulièrement les autres villes Québecoises possédant une communauté de minorités sexuelles. 
On peut toutefois nommer les villes de Rimouski, Gatineau, Saguenay et Trois-Rivières comme candidates à une analyse plus approfondie. 
Ces villes, par leur inscription au sein d'une structure régionale urbanisée et par leur proximité aux villes importantes de l'est du Canada. 
En effet, comme il le sera décrit dans les chapitres suivants, des données ont été recensés dans ces diverses villes, que ce soient des géosymboles ou du moins, des adresses et des contacts prouvant l'existence de telles communautés.

%%% Local Variables:
%%% mode: latex
%%% TeX-master: "../../memoire-maitrise"
%%% End:
