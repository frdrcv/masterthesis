%!TEX root = ../../memoire-maitrise.tex

\chapter{Éléments conceptuels et problématique}
\label{cha:elements_conceptuels_et_problematique}

\chapterprecishere{\textquote{J’ai abandonné depuis longtemps l’idée qu’une vérité immanente se trouve dans la sexualité, qu’elle soit marquée par le péché ou l’émancipation. J’ai aussi abandonné l’idée qu’il existe quelque chose qu’on appelle \enquote{la sexualité}. Il existe plutôt des sexualités multiples, des sexualités dominantes et des sexualités marginalisées.} \par\raggedleft--- \textup{Anne Archet}, Sexe et liberté}

Avant d'aborder directement le sujet de la recherche, nous débuterons avec un présentation de notre cadre théorique qui permettera de contextualiser notre recherche tout en mettant les bases qui servirons à l'analyse de nos résultats à la toute fin de ce mémoire.
Ce chapitre servira donc à approfondir d'une part les concepts que nous avons utilisés tout au long de notre démarche et qui ont servi à développer la problématique.
D'autre part, nous tenterons ici de faire un tour d'horizon de la littérature dans le domaine de la géographie sexuelle mais aussi dans d'autres champs du savoir.
Cette recherche s’est appuyée sur différents domaines de la littérature scientifique: nous avons souhaité nous inscrire d'emblée dans le champ des études gaies et lesbiennes et queers pour sa filiation à l'étude de la diversité sexuelle, laquelle est centrale à notre recherche.
Les études queers forment un champ pluridisciplinaire, aujourd'hui en croissance à l'intérieur des sciences sociales mais dont les origines se situent plutôt dans les \emph{humanities}.
Trouvant un appui fort dans le corpus de la \anglais{French Theory}~\citep[et principalement Foucault, voir][11]{Kemp2009}, la théorie queer s'est développée par un ensemble d'auteurs de disciplines aussi diverses que la littérature, la philosophie ou les sciences sociales.
Cette dernière tire ses racines d'une volonté de dépasser un essentialisme dans laquel vaquait une partie importante du monde académique s'intéressant au genre et à l'orientation sexuelle~\citep[27--28]{Green2007}.
Par contre, contrairement à d'autres approches sociologiques, la théorie queer aborde les normes comme objet plutôt que de s'attarder directement aux identités sexuelles. 
Son approche vise donc la déconstruction des catégories sociales émergents de ces normes sociales, surtout à l'aide d'une série de concepts novateurs.
Les concepts de négativité, de performativité et d'intersectionnalité sont particulièrement utilisés pour traiter des phénomènes de marginalisation, bien que certains ont pris origine ailleurs, notamment dans les travaux traitant du post-structuralisme pour l'intersectionnalité ou dans les études féministes, à l'époque ou Butler a proposé le concept de performativité.
Les concepts d'hétéronormativité et, plus récemment, d'homonormativité, viennent compléter ses outils théoriques, en remplaçant les concepts d'homophobie et d'hétérosexisme.
Pour la perspective queer, les violences faites contre les minorités sexuelles ne sont pas le fait d'actes isolés ou même d'une violence institutionnelle; celles-ci prennent racines à même les discours et prennent effets dans toutes les structures et relations de la société~\citep{BastienCharlebois2011}.

Afin de rester proche des enjeux spatiaux entourant notre sujet, nous nous avons rattaché plus précisément à la géographie queer, laquelle s'est développée dans les dernières années grâce à de nombreux autrices et auteurs~\citep[sans être exhaustif, nous pouvons citer][]{Bell1995,Bell2004,Bell1995b,Bell1995a,Bell1991,Brown2003,Knopp2004,Knopp1992,Knopp1990,Knopp2003,Lauria1985,Oswin2006,Oswin2004}.
De cette géographie particulière, nous retiendrons principalement le concept d'espace de citoyenneté sexuelle qui permet de spatialiser les questions de sexualité et remettre en question l'ignorance continue de la géographie pour les questions sexuelles.
Bien que cette géographie s'intéressant aux questions sexuelles se développe encore depuis plusieurs années, nous nous avons pas limité à ce seul champ de la recherche.
En considérant l'ampleur du travail, autant en termes de territoire que nous avons souhaité couvrir qu'envers les concepts que nous souhaitions aborder, nous avons priorisé une approche multidisciplinaire.
Nous expliquons cette décision par le partage fréquent des concepts utilisés entre ces disciplines et l'apport que qu'ensemble ces dernières peuvent apporter à notre recherche; les concepts d'identité, de territoire ou de symboles ont tous été abordés, autant en géographie qu'en anthropologie ou en sociologie, bien qu'en soit, ceux-ci peuvent sembler en contradiction avec le point de vue épistémologique queer voulant déconstruire le concept d'identité.
Dans cet esprit, nous avons également exploré le champ de la sémiotique, en mettant l'accent encore une fois sur ses liens avec la géographie pour mieux appuyer certains emprunts que la géographie a pu faire dans ce champ du savoir, surtout au niveau des symboles.
Bien que les concepts que nous expliqueront dans les prochaines pages peuvent s'appliquer à une multitude de sujets d'étude, nous contextualiserons tout au long leur mise en applications pour terminer par une synthèse qui mettra de l'avant une lecture centrée sur les identités sexuelles.

\section{Le regard anthropologique sur la culture}
\label{sec:le_regard_anthropologique_sur_la_culture}

Nous aborderons la culture comme premier concept.
Polysémique selon les disciplines et le contexte, nous voulons décrire ce concept et ses différents usages pour arriver à comprendre comment celui-ci s'articule dans le domaine de la géographie sexuelle.
Plus encore, nous voulons également l'approfondir dans un contexte de regard sur soi au sein des minorités sexuelles.
En effet, nous le verrons plus loin, plusieurs positions sont débattues dans la communauté \lgbt{}, une s'incarnant dans une identité forte et une autre dans une position négative et antinormative.
S'ajoute alors la question de déterminer exactement s'il peut exister une culture propre aux communautés \lgbt{}.

Pour débuter notre analyse de la culture, nous nous pencherons sur la définition proposée par l'anthropologue Clifford Geertz.
Pour lui, tel qu'amené dans~\citetitle{Geertz1972}, la culture: \blockquote[{\cite[21]{Geertz1972}}][.]{\textelp{} désigne un modèle de significations incarnées dans des symboles qui sont transmis à travers l'histoire, un système de conceptions héritées qui s'expriment symboliquement, et au moyen desquelles les hommes [\latin{sic}] communiquent, perpétuent et développent leur connaissance de la vie et leurs attitudes devant elle}.
Geertz inscrit cette définition dans une critique plus large des travaux précédents en anthropologie religieuse.
Cette sous-discipline de l'anthropologie stagnerait au point de vue théorique en axant constamment son analyse sur une version trop fonctionnelle de la culture ou axée sur l'expérience (mystique dans le cas de la religion).
Ce choix d'auteur prendrait autant une inspiration sociologique que anthropologique (en se basant également sur les travaux de Durkheim et sur ceux de Malinowski par exemple~\citep[20]{Geertz1972}).
Ce texte vise donc à proposer de nouvelles bases théoriques sur lesquelles l'analyse anthropologique pourrait s'approfondir et continuer à évoluer.
Geertz précède alors son analyse des systèmes religieux par une définition renouvelée de ce en quoi consiste la culture ou plutôt le système culturel.
Pour lui, ces systèmes religieux offrent un sens au quotidien et ne sont pas nécessaire subordonné aux fonctions sociales; elles interagissent avec elles.
Si certaines perspectives anthropologiques ont voulu voir la religion comme une réponse à l'angoisse de la mort, Geertz nous rappelle que celle-ci offre toute une variété de réponse~\citep[][37]{Geertz1972}.
En fait, c'est dans le chaos du quotidien que la religion offre des réponses, réponses qui sont rapidement remplacées par des explications plus plausibles, ou près du sens commun, lorsque ces dernières sont proposées~\citep[][39]{Geertz1972}.

Par cette lecture, la culture apparait comme un système abstrait, persistant dans le temps au-delà de la vie des individus composant la société.
Par contre, la définition avancée par Geertz ne laisse pas sous-entendre que nous avons affaire à une entité réifiée ou superorganique, pour reprendre les termes de \citet{Duncan1980}.
Les individus formant la société héritent donc des connaissances offertes par la culture pour arriver à comprendre le monde où les symboles qui s'y trouvent portent des significations propres.
Dans les sociétés analysées par l'anthropologue, c'était essentiellement la religion qui jouait ce rôle dans le cadre plus large de la culture.
Nous croyons par contre qu'un tel système religieux, avec ces différents symboles pour décrire le réel et lui donner un sens, peut aussi être repris dans des contextes culturels variés n'ayant pas nécessairement de liens avec la religion, sinon l'usage de symboles.

En reprenant les termes de~\citet{Langer1962}, les symboles sont pour Geertz\todo{à étoffer} dans son analyse sémiotique de la culture: \textquote{\textelp{} tout objet, acte, événement, propriété ou relation qui sert de véhicule à un concept --- le concept est la \textquote{signification du symbole }~\citep[23--24]{Geertz1972}}.
Ainsi, un élément important de la culture peut être convoyé par un symbole tel un message entre deux individus communiquant, et ce même symbole peut prendre plusieurs formes.
Pour Langer, chaque signe est automatiquement lié à un élément de la réalité~\citep[62]{Langer1962}.
Il est important de souligner que les objets en soi que l'on pourrait assimiler à des symboles demeurent ce qu'ils sont matériellement; Geertz prend l'exemple d'une maison qui, si elle peut consister en un objet concret sans significations autres que sa matérialité, peut également jouer le rôle d'un symbole propre dans le contexte relatif à la culture.
Ce caractère varierait selon le regard qu'on lui pose en tant qu'observateur appartenant à une culture spécifique.
Autrement dit, au-delà de sa matérialité, sa forme, sa position ou sa composition, la maison peut porter le témoignage d'un fait culturel particulier.
On pourrait extrapoler en considérant que cette maison informe le public sur le statut social de la personne ou plus simplement informer sur sa fonction, celle d'abriter un individu ou une famille.
Sa fabrication, suivant une forme architecturale particulière, son état actuel, son positionnement dans l'espace, sont des signes qui prennent ensemble un sens particulier sous la forme du symbole qui informe l'observateur.

Cette manière de donner forme aux choses matérielles ou abstraites, de leur octroyer une signification sous la forme de symbole est selon Geertz le fait des programmes fournis par les modèles culturels~\citep[25]{Geertz1972}.
Ces modèles fonctionnent en deux temps: d'abord, ils créent les symboles en se basant sur le réel, en prenant assise sur les structures non symboliques déjà existantes, ce que Geertz nomme des \emph{modèles de}.
L'autre forme de modèle, les \emph{modèles pour} agissent plutôt en orientant les structures non symboliques et en créant des liens entre elles qui n'existent pas nécessairement au préalable~\citep[26--27]{Geertz1972}.
Ces deux phénomènes rappellent en fait les deux manières qu'ont les \emph{modèles de} de donner sens aux symboles qu'ils contiennent; ils dirigent la compréhension des symboles en calquant ceux-ci sur le réel non symbolique et en liant ensemble les éléments composant ce réel~\todo{à revérifier}.

Les \emph{modèles de} et \emph{modèles pour} sont en fait un seul et même modèle culturel où s'articulent les symboles; en effet, un même groupe culturel articule une lecture particulière des structures non symboliques et lui en imposent une selon les deux facettes du modèle culturel.
Par contre, ceux-ci, tels que décrits par Geertz s'inscrivent dans un contexte culturel réputé homogène.
En effet, les différentes analyses faites par l'auteur portent sur les religions \todo{expliquer davantage} (l'objet d'analyse) de façon singulière ou encore sur la place de la religion dans la société de façon générale, sans privilégier un contexte en particulier.
Il s'agit d'une des limites à prendre en compte dans la suite du présent texte; en effet, Geertz n'entreprend pas dans cette recherche de traiter des effets propres aux mélanges culturels.
Nous entendons par ceci par exemple des interactions naissantes de l'immigration avec la société d'accueil ou encore de la genèse de nouveaux phénomènes culturels, comme dans le cas qui nous intéresse les minorités \lgbt{} au sein d'ensembles culturels plus larges.

On trouve tout de même dans son texte des éléments orientant la relation particulière d'un individu et de la religion qui peut servir d'introduction à la suite de ce travail sur l'identité.
En effet, Geertz, plus loin dans son texte, traite des dispositions propres à l'individu s'insérant dans un contexte culturel particulier.
La culture transmet ces dispositions à effectuer certaines activités qui permettent à l'individu de s'identifier à la culture dont il fait partie au-delà des fonctions premières et des motivations derrières la pratique en question.
Le principe d'identité se comprend aisément dans le contexte religieux décrit dans le texte: l'individu religieux pratique dans sa vie la prière et d'autres activités religieuses plus fréquemment selon l'intensité de son sentiment religieux.
Nous voulons surtout souligner ici le fait qu'on parle principalement de la probabilité d'un acte de survenir: l'acte en soi n'est pas nécessaire.
En société, par exemple, on s'attend à ce que l'individu réputé religieux agisse selon certaines dispositions propres au mode de vie religieux~\citep[28--30]{Geertz1972}.
Dans le contexte des identités \lgbt{}, on ne peut s'arrêter seulement aux pratiques pour traiter d'une potentielle identité.

\todo{À lier avec Geertz}
Plutôt, face à un ensemble de structure symboliques où l'hétérosexualité est la norme, les individus \lgbt{} pourraient avoir à créer de nouvelles structures symboliques correspondant à leur situation.
Plus précisément, les individus \lgbt{}, à leur naissance, s'insèrent de facto dans un contexte culturel dont les symboles s'apparentent très peu aux identités sexuelles non hétérosexuelles, celles-ci portant plutôt des significations négatives.
En effet, dans le contexte occidental, les individus sont considérés comme hétérosexuels par défaut, à un point tel qu'il ne s'agit pas, en général, d'une identité particulière, mais d'une norme, cette dernière étant souvent surpassée par des identités nationales, régionales, \emph{ethniques}, etc.\@ ou par des conceptions plus individuelles de l'identité, axées sur des éléments comme les loisirs, la carrière, etc.
Le désir sexuel envers des individus qui ne sont pas considérés comme des partenaires éligibles crée une rupture à laquelle le modèle culturel dominant offre des réponses négatives par ce que socialement l'on conçoit comme de l'homophobie, de la violence, etc.
Même chose en ce qui concerne l'identité de genre pour les personnes trans ou intersexes; le genre dans les modèles culturels occidentaux sont vus comme empreint d'une essence propre par l'existence d'un sexe, masculin ou féminin.
Face à ceci, les individus \lgbt{} ont mis en place de nouveaux espaces dans lesquels leurs relations comme symboles ont d'autres significations.
Nous ne répondrons pas immédiatement à ce problème; nous allons plutôt maintenant nous pencher sur l'identité comme concept et comment celui-ci a été manié et pensé par la théorie queer, pour ensuite revenir sur la place de l'identité chez les groupes \lgbt{}.
\todo{vérifier l'enchaînement des parties}

\section{Le sujet et l'identité}
\todo{Revoir le titre de la section}
\label{sec:sujet_et_identite}

Durant les dernières décennies en occident et ailleurs dans le monde, de nombreux groupes identitaires semblent avoir fait surface, surtout dans les milieux urbains.
On peut penser notamment aux groupes ethniques nés de l'immigration, aux groupes d'intérêts envers des objets culturels particuliers (genres musicaux, dessins animés, cinéma), etc.
Plus près de notre recherche, nous pouvons nommer également les groupes formés par l'identification à des pratiques sexuelles différentes ou une non-coïncidence du genre de la personne avec celui fixé à la naissance.
Dans ce mémoire, nous tenterons d'abord de traiter des concepts de culture et d'identité en relation avec les individus appartenant au spectre \lgbt{}.
Nous souhaitons aborder les enjeux particuliers que pose l'étude géographique de cette partie de la population, en considérant le contexte historique récent.
Ce cadre s'inscrirait dans les rassemblements politiques et l'organisation communautaire auxquels plusieurs des individus du spectre \lgbt{} ont participé au cours des dernières décennies.
On peut penser notamment aux émeutes de Stonewall et à la crise du \sida\ des années 80 jusqu'à aujourd'hui\footnote{Les émeutes de Stonewall sont un événement particulièrement important dans l'histoire des communautés \lgbt{} occidentales. 
  S'ayant déroulées en juin 1969, ces émeutes ont éclatées en réactions au harcèlement policiers dont étaient la cible les espaces fréquentés par les minorités sexuelles, à l'époque une pratique courante. 
  Plutôt que de plier aux arrestations de certains de membres de la communauté, les clients du Stonewall Inn dans le Greenwich Village ont décidé de résister. 
  Le lendemain, diverses manifestations ont eu lieu à New-York pour combattre la répression vécue par les minorités sexuelle, celles-ci menant à la formation de divers groupes politiques \lgbt{} et aux marches de la fierté gay.}.
En effet, nous tenterons de comprendre si la communauté \lgbt{} forme une culture en soi au sein d'un groupe culturel plus large et si l'orientation sexuelle peut être considérée comme une forme d'identité à l'intérieur de ce groupe culturel particulier.

Nous commencerons notre démarche par l'étude du concept de culture en nous basant sur les travaux de Clifford Geertz portant sur ce concept.
Bien que le texte que nous analyserons s'inscrit principalement dans une étude de la religion comme phénomène social, nous retiendrons essentiellement les éléments théoriques qui pourront nous permettre de traiter des communautés \lgbt{} par la suite.
Nous introduirons ensuite le concept d'identité par les travaux de Stuart Hall\todo{revoir la formulation, vu qu'on traite de Geertz précédemment}.
Par les deux définitions qui se dégageront du survol de ces textes, nous pourrons d'un côté comprendre l'évolution généalogique du concept d'identité tout en définissant la culture et voir comment le premier concept s'articule avec le second.
Nous nous pencherons sur les enjeux entourant les populations \lgbt{} du point de vue de la définition: avons-nous affaire à un groupe culturel distinctif ou plutôt une série d'individus ne possédant en commun que des pratiques sexuelles similaires et une volonté relative d'intégration sociale commune?
Nous tenterons ensuite de nous pencher sur les enjeux géographiques particuliers à prendre en compte pour l'analyse de la population \lgbt{}, autant d'un point de vue de la pratique sexuelle que sous la forme d'un groupe culturel.
Enfin, nous verrons comment ces différents points de vue se sont manifestés au sein de ce champ disciplinaire.

\section{Le regard sociohistorique sur l'identité}
\label{sec:le_regard_sociohistoirique_sur_l_identit_}

À partir de de la définition du concept de culture que nous avons articulée dans la dernière partie, nous allons maintenant traiter du concept d'identité tel que défini par~\citet{Hall1996a} dans l'introduction de \citetitle{Hall1996c}.
Ce dernier propose une généalogie simplifiée du concept d'identité dans les sociétés modernes.
Pour ce faire, il remontera l'histoire des derniers siècles en tentant d'identifier sommairement différentes transformations socio-économiques qui auraient participer à une redéfinition du concept d'identité.
Ces différentes transformations pour l'auteur s'inscriveraient en partie dans la suite de divers courants de pensée ayant permis une redéfinition de l'être humain, principalement la psychiatrie et les critiques subséquentes.
Si Hall, à l'origine, souhaitait comprendre s'il existe aujourd'hui une crise de l'identité, cette généalogie nous permet de rendre compte de l'évolution de l'identité individuelle, et également de comprendre l'apparition des identités de genres et identités sexuelles qui sont aujourd'hui répandues en occident\footnote{Nous souhaitons souligner que ces identités ont souvent été comparées à d'autres formes identitaires que l'on peut retrouver dans de nombreuses sociétés non-occidentales.
  Nous les considérons pour la suite du texte comme distinctes bien que, comme nous le verrons dans la section résultats, ces dernières existent belles et bien, dans le contexte d'une société avec un passé colonial comme le Québec.}.

Le premier type identitaire selon Hall est le sujet des lumières qui apparait à partir du \siecle{16}.
Sans trop détailler sa production, Hall souligne qu'il s'agit du sujet né des idées des lumières.
On a affaire à un sujet dont la capacité principale est d'être et de penser en reprenant les idées de Rousseau.
Rattaché à la nation, son identité est peu développée et les conflits de classes ne sont pas encore présents de façon claire.
On a affaire à un individu dont les caractéristiques, comme le statut social, sont conçues comme immuables et banales comparativement aux autres membres de la société~\citeyearpar[596]{Hall1996a}.

Le sujet sociologique, deuxième évolution du concept d'identité, nait en même temps que la sociologie devient une discipline autonome séparée du domaine de l'économie, des sciences politiques, mais surtout de la psychologie.
En effet, selon Hall, celui-ci apparait lorsque la psychologie et la psychanalyse prennent de l'importance dans le domaine de la recherche.
Ces deux champs concentrent leur analyse sur l'individu en son for intérieur et ses relations particulières avec son environnement immédiat pendant la constitution de sa personnalité, à savoir les membres de sa famille et surtout ses parents.
Le domaine du social devient l'apanage de la sociologie.
L'identité agit ici comme l'intermédiaire entre cet individu réputé unique par sa psychologie personnelle et le monde social.
Ce dernier ne sera pas simplement considéré comme une foule d'individus, mais plutôt un ensemble de structures économiques et sociales.
Ces structures en soi influencent crucialement la place de l'individu dans la société, sachant que ce dernier participe et est influencé par celles-ci.
Ce sujet sociologique est relié aux autres individus; Hall le lie autant aux: \textquote[{\citeyear[597]{Hall1996a}}][]{les valeurs, les significations et les symboles --- la culture --- du monde qu'il ou elle habite}.
Ce sujet apparait aussi avec la constitution de l'individualisme.
Il possède également des caractéristiques particulières très peu développées, alors que son essence se compare à celle de ses concitoyens au sein de la nation et des individus appartenant à la même classe sociale.
En somme, les systèmes symboliques prennent une importance qui dépasse les individus bien qu'on commence à voir apparaitre une complexification des identités possibles à la suite du bouleversement des institutions réputées immuables par le renversement bourgeois des traditions et des systèmes de royauté à partir de la fin du \siecle{18} et qui se poursuivra durant le \siecle{19}.

Le sujet postmoderne serait le sujet le plus récent et celui qui cadrerait à notre époque, à partir de la fin du \siecle{20}.
Selon Hall, ce sujet aurait pris la place du sujet sociologique à la suite d'une déstabilisation de l'identité portée par le sujet moderne et donc sa fragmentation au fil des générations.
Hall recense cinq causes à cette déstabilisation qui se retrouvent dans divers travaux sur la société et l'individu.
La première de ces causes provient des travaux de la théorie marxiste.
En effet, dans celle-ci, la faculté d'action (\anglais{agency}) est questionnée par la reconnaissance des structures sociales et économiques.
Cet individu seul ne possède d'ailleurs plus d'essence propre: plutôt, il acquiert de l'extérieur nombre de ses caractéristiques individuelles, bien souvent par la classe sociale de laquelle il est issu~\citeyearpar[606]{Hall1996a}.

Le deuxième décentrement proviendrait de la décomposition psychologique de l'individu, notamment dans les travaux de Freud et plus tard de Lacan.
Cette décomposition remet radicalement selon l'auteur la position de Rousseau sur la Raison individuelle.
L'individu ne contrôlerait plus maintenant l'entièreté de sa vie, mais possèderait en lui des pulsions aussi variées qu'incomprises et une partie de lui-même, héritée socialement, le dépasse et influence sa propre vie.
Les autres individus autour de lui, la famille d'abord, jouent des rôles particuliers dans sa vie.
La reconnaissance de leur existence nous amènent donc à ne plus voir l'individu comme une construction autonome, mais plutôt en relation avec les autres, selon des versions symboliques que ceux-ci représentent pour l'adulte à venir~\citeyearpar[ 607--608]{Hall1996a}.

La troisième cassure selon Hall s'est opérée par les travaux de Ferdinand de Saussure sur le langage.
Selon lui, la langue et les mots ne sont en aucun cas possédés par les individus, au contraire.
Plutôt, les individus s'insèrent dans un système complexe de symboles.
Ils utilisent les mots pour transmettre des significations, des messages aux autres.
Par contre, ces individus ne peuvent être certains de la réception des messages transmis étant donné l'évolution rapide du sens des mots.
Le langage dépasse donc chaque personne et forme un fait de société qui les précède.
La langue prend d'ailleurs place dans la psyché de chaque individu et structure bon nombre de ses pensées~\citeyearpar[608--609]{Hall1996a}

Les travaux de Foucault sur le pouvoir disciplinaire consistent en la quatrième cassure.
Principalement dans \citetitle{Foucault2004a}~\citeyearpar{Foucault2004a}, Foucault dresse une généalogie des moyens disciplinaires utilisés par les sociétés occidentales pour faire justice en société.
On y apprend que continuellement, les techniques disciplinaires vont devenir de plus en plus douces.
Politiquement, les sociétés vont passer d'un pouvoir strictement extérieur à l'individu, incarné dans la royauté d'abord, à un système judiciaire graduellement distant dans la société pour prendre en même temps place à l'intérieur de l'individu.
Par des dispositifs de surveillance toujours plus élaborés, non seulement les criminels seront observés, mais également le reste de la société, dans les écoles et les hôpitaux par exemple grâce aux technologies développées dans les centres pénitenciers, les prisons, les cachots.
Ces institutions, par leur pouvoir grandissant et leur capacité à surveiller, en viennent à acquérir suffisamment de connaissances sur les individus pour devenir des agents normatifs puissants~\citep[608--609]{Hall1996a}.

La cinquième et dernière cassure découle de la place croissante qu'a jouée le féminisme dans le monde occidental.
Hall considère que ce sont les mouvements politiques et intellectuels qui ont rendu possible cette cassure, en s'interrogeant sur les rapports de genres entre les individus d'abord.
Ceci ouvrit la porte à la contestation de nombreux groupes marginalisés ou dont les idées politiques remettaient en question le système social et politique au-delà de la lutte des classes.
Ce mouvement a donc permis la mise en place d'identités politiques correspondant à ces nouveaux mouvements de contestation et ouvert la porte à diverses formes de contestation sociale et politique~\citeyearpar[610]{Hall1996a}.

Comme on peut le voir, le sujet sociologique et le sujet postmoderne sont particulièrement similaires sur plusieurs points.
Plusieurs des facteurs ayant mis en place le sujet sociologique en sont venus à le déstabiliser avec les décennies, quoique Hall ne conçoit pas une généalogie claire et datée de ces déstabilisations; il faut préférablement s'appuyer sur les généalogies internes à chacun des travaux ou événements et leur moment d'apparition dans l'histoire moderne.
On doit plutôt concevoir ceux-ci d'une part comme des facteurs de déstabilisation et des œuvres les expliquant, qui ont amené la mise en place du sujet postmoderne.
Celui-ci, dépassé par le langage, l'économie ou les rapports de pouvoirs est de plus en plus individualisé par la société tout en y étant enchevêtré.
En ceci, comparativement au sujet des lumières, il n'est pas nécessairement raisonnable et son expérience individuelle est beaucoup plus importante dans sa constitution identitaire.
Au-delà de la complexification des rapports sociaux, cette individualisation a ouvert de nouvelles voies pour l'individu, lui permettant d'arriver à coordonner sa place en société.
Pour Hall, ce sujet: \foreigntextquote{english}[{\citeyear[598]{Hall1996a}}][]{\textelp{} assumes different identities at different times, identities which are not unified around a coherent \emph{self}. Within us are contradictory identities pulling in different directions, so that our identifications are continuously being shifted about. If we feel we have a unified identity from birth to death, it is only because we construct a comforting story or \enquote{narrative of the self} about ourselves}.
Nous pouvons croire que cette \emph{narration du soi} s'incarne non pas dans une rationalité objective et indépendante comme pour le sujet des lumières, mais plutôt dans une rationalité subjective.
Cette rationalité répondrait donc à deux défis: d'abord, s'adapter à son environnement immédiat, spatial ou temporel, puis assembler en soi un récit qui arrive à surmonter certaines contradictions inhérentes aux multiples identités individuelles.

Ce que nous appelons la rationalité subjective n'est pas une construction stable et encore moins indépendante de l'individu.
Au-delà des contradictions internes, les autres individus peuvent également contredire ce récit et ces identités.
La politique pour Hall jouerait ce rôle: \foreigntextquote{english}[{\citeyear[610]{Hall1996a}}][]{Since identity shifts according to how the subject is addressed or represented, identification is not automatic, but can be won or lost. It has become politicized. This is sometimes described as a shift from a politics of (class) identity to a politics \emph{difference}}.
Ces changements importants chez l'individu ouvrent la porte à d'autres formes d'identités jusque là improbables.

Nous pouvons conclure cette partie en soulignant que ces bouleversements et les événements historiques du dernier siècle ont mené l'éclosion des identités \lgbt{}, par le traitement de l'orientation sexuelle par le système de santé, par la psychanalyse et une distance du pouvoir.
L'ensemble de ces facteurs auraient permis l'apparition de groupes d'affinités autour de la question sexuelle.
La reconnaissance de ces nombreuses identités dans la société et de l'intersection de celles-ci dans les individus doit être prise en compte dans une analyse géoculturelle de la société et des groupes culturels.
En effet, l'individu \emph{geertzien} que nous avons traité plus tôt dans le chapitre ne dépendrait pas d'un seul système de symboles, mais bien à plusieurs, dont la géographie et la temporalité évolueraient.
L'incompréhension envers des contradictions ne créerait plus \textquote[]{une angoisse très forte dès qu'il sent que ces symboles peuvent ne pas pouvoir répondre à tel ou tel aspect de l'expérience}~\citep[33]{Geertz1972}, mais une adaptation et un changement d'identité.
Ainsi, on arrive d'une certaine façon à dépasser certaines des limites inhérentes à la vision \emph{geertzienne} de la culture dans laquelle l'attachement à une identité comme la religion pour un individu passait principalement par la pratique.
L'identité se construirait plutôt dans l'interaction avec autrui et la médiation de ces identités avec les forces politiques les régissant ou les contredisant.
Néanmoins, l'usage du terme d'identité n'est pas nécessairement aujourd'hui l'apanage des chercheurs et certains contextes, principalement dans le domaine du politique, permettent une \emph{autoréflexivité} sur ce statut d'identité.

\section{Diversité sexuelle et identité}
\label{sec:diversit_sexuelle_et_identit_}

Comme nous l'avons vu dans la section précédente, l'identité de minorité sexuelle peut être considérée comme une création récente propre au contexte postmoderne des dernières décennies.
Ce constat s'explique notamment par le contexte historique actuel dans lequel s'inscrit cette identité ainsi que par la correspondance à certaines caractéristiques proposées par Hall.
Ces caractéristiques peuvent s'apparenter à une identité dans laquelle les individus s'identifient en parallèle à d'autres formes identitaires, comme la communauté ethnique, nationale ou encore de genre, par exemple.
Nous introduirons pour la suite le travail de~\citet{Sinfield1996}, qui, dans \citetitle{Sinfield1996}, traite des difficultés et contradictions propres à l'usage de l'identité chez les communautés \lgbt{}.
Ce dernier met en rapport l'identité et les phénomènes historiques récents de libération sexuelle en occident.
Ce rapport complexe viendrait du développement d'un discours de plus en plus critique en ce qui concerne l'essentialisation des communautés sexuelles.
D'abord dans les mouvements féministes, ce discours s'est concrétisé dans les premiers travaux liés à la théorie queer qui tentèrent de dépasser la \emph{pathologisation} de la sexualité comme on la retrouvait dans les travaux en psychanalyse.
En effet, le texte de Sinfield s'intéresse plus particulièrement aux considérations stratégiques et historiques entourant la mise en place d'une identité homosexuelle pour la communauté elle-même et du point de vue des penseurs de l'identité sexuelle, notamment Foucault.

\subsection{Identités minoritaires et caractère universel de la sexualité}
\label{sub:minorit_s_et_universel}

Le texte de Sinfield débute par la présentation de deux manières de voir l'homosexualité dans la littérature scientifique et comme identité.
Le premier consisterait en un point de vue de \emph{minoritarisation} ou de marginalisation qui conçoit l'homosexuel, gay ou lesbienne, comme un groupe d'individus ayant un style de vie particulier, tel un groupe ethnique.
Le second point de vue serait celui de l'universalisation.
Celui-ci verrait dans l'homosexualité un comportement potentiel chez tous les individus: tous peuvent à un moment ou à un autre commettre un acte homosexuel.
Il n'y a pas lieu de parler d'identité ou de culture comme on traiterait d'un groupe ethnique~\citep[271]{Sinfield1996}.

Le point de vue \emph{minoritarisant} considère les individus à la sexualité déviante, les personnes homosexuelles dans ce cas-ci, comme des groupes identitaires particuliers.
Il va donc à l'encontre du point de vue constructiviste répandu dans les études sur le queer et l'identité sexuelle, inspirées des travaux \citet{Foucault2011}, de \citet{Rubin2010} et de \citet{Butler2007}
Au contraire, le point de vue d'universalisation coïncide avec le champ de pensée constructiviste; en effet, on voit la sexualité comme une donnée variable chez les individus dont la position sociale, l'éducation et l'environnement porteront un effet prépondérant et dont le sens prendra une valeur différente selon la culture traitée.
Contrairement au point de vue \emph{minoritarisant}, le point de vue universalisant voit l'homosexualité comme une attitude, une pratique sexuelle possible pour chaque individu, peu importe la culture.
Cette dernière déterminera si la pratique homosexuelle est tolérée, encouragée ou discriminée et marginalisée~\citep[271]{Sinfield1996}.

Selon Sinfield, les gais et les lesbiennes ont historiquement pris une position stratégique les rapprochant du point de vue \emph{minoritarisant}.
En effet, ces communautés ont emprunté une dynamique de revendication et de lutte sociale similaire à celles des groupes ethniques, notamment des mouvements pour les droits civiques afro-américains~\citep[271]{Sinfield1996}.
Sinfield nomme cette stratégie le \textquote{cadre de l'ethnicité-et-des-droits} (\anglais{ethnicity-and-rights} dans le texte).
Le développement de ce cadre, au-delà de la simple imitation des groupes ethniques, s'est effectué dans le cadre de l'État de droit.
Dans celui-ci, pour améliorer leur position sociale et réduire la marginalisation, les individus doivent, pour reprendre les termes de \citet{Sinfield1996}, \foreigntextquote{english}[{\citeyear[272]{Sinfield1996}}][]{\textelp{} to  compartmentalize their complex subjectivities in order to \emph{make a claim} (envers le pouvoir)}.
Cette compartimentation de la subjectivité individuelle amène les individus touchés à présenter une caractéristique particulière d'eux-mêmes et donc à vivre un rapprochement avec les autres individus touchés qui se reconnaissent dans cette identité potentielle.
Ce cadre stratégique ne laisse pas entendre qu'il n'existait pas de groupes d'individus gais et lesbiens avant que ceux-ci revendiquent des droits, selon Sinfield.
Ces revendications ont plutôt amené ces groupes à se voir: \textquote{\textelp{} as gay in  the terms of a discourse of ethnicity-and-rights}~\citep[272]{Sinfield1996} et que ces regroupements par affinités se sont mutés en groupe identitaires avec un poids politique.
Sinfield souligne plusieurs problèmes dans la poursuite de ce cadre; d'abord, cette nouvelle identité et cette genèse culturelle peuvent entrer en contradiction avec les autres identités assumées par les individus y prenant part.
Elle désengage également le reste de la société à poser une réflexion profonde sur la sexualité.
C'est ce que propose la pensée \emph{universalisante} reprise par certains groupes plus radicaux (dont le mouvement queer, qui par définition vise une refonte des normes sur la sexualité et le genre plutôt que l'acquisition de droits)~\citep[273]{Sinfield1996}.

Plus loin dans son texte, Sinfield explique les différences culturelles dans lesquelles a eu lieu le développement de mouvements de contestation gais et lesbiens.
Plus particulièrement, Sinfield s'intéresse aux différences entre les États-Unis et la Grande-Bretagne.
Pour l'auteur, le cadre de l'ethnicité-et-des-droits se traduit de différentes manières selon la région étudiée.
En Grande-Bretagne, la concession de droit s'inscrit dans la suite de l'État-providence, où l'état anglais concède des acquis supplémentaires dans la perspective d'assurer à tous les citoyens un mode de vie décent.
Aux États-Unis, on s'inspire plutôt des valeurs traditionnelles américaines qui s'orientent surtout vers la liberté aux individus~\citep[274]{Sinfield1996}.
En cherchant à obtenir cette liberté offerte par la société américaine, les groupes ethniques s'appuient du même coup sur ce que Sinfield nomme le mythe de la pluralité américaine.
Celui-ci laisse entendre que chaque groupe culturel équivaut à d'autres et peut revendiquer un accès égal aux mêmes ressources que les autres dans un cadre compétitif.
C'est ce mode stratégique qui se serait par la suite répandu dans les autres mouvements de contestations et qui aurait mis au premier plan le modèle de l'ethnicité-et-des-droits.

\subsection{Diaspora et hybridité}
\label{sub:diaspora_et_hybridite}

Pour comprendre la multiplicité des origines différentes --- pour les individus s'identifiant à l'identité homosexuelle ou lesbienne --- Sinfield propose de concevoir cette identité comme nécessairement hybride.
Cette hybridité se compare de façon analogue celle imposée aux individus appartenant à une diaspora.
On définit une diaspora par l'ensemble des individus se retrouvant géographiquement à une certaine distance du lieu d'où provient leur culture d'appartenance, par immigration ou par déterritorialisation.
À titre d'exemple, on peut notamment penser à la diaspora juive ou encore la population afro-américaine.
L'existence de diasporas pour l'auteur montre la résilience qu'ont les individus à résister à l'assujettissement de leur identité par leur milieu d'accueil et à conserver leur culture.
Structurellement, la: \foreignblockquote{english}[{\cite[278]{Sinfield1996}}][.]{\enquote{Diaspora} \textelp{} usually invokes a true point of origin, and an authentic line --- hereditary and/or historical --- back to that. However, diasporic Black culture, Hall says, is defined \enquote{not by essence or purity, but by the recognition of a necessary heterogeneity and diversity; by a conception of \enquote{identity} which lives with and through, not despite, difference; by hybridity}}

Cette hybridité peut donc être conçue comme participant à une forme d'ethnogenèse tout en possédant un potentiel politique: au lieu de répondre à certains archétypes que la société d'accueil impose sur l'identité des groupes culturels provenant d'une diaspora, ceux-ci peuvent participer à la conception de leur identité en reprenant certains traits culturels:
\foreignblockquote{english}[{\cite[277]{Sinfield1996}}][.]{Stuart Hall traces two phases in self-awareness among British Black people. In Do the first, \enquote{Black} is the organizing principle: instead of colluding with hegemonic versions of themselves, Blacks seek to make their own images, to represent themselves. In the second phase (which Hall says does not displace the first) it is recognized that representation is formative --- active, constitutive --- rather than mimetic}.
Néanmoins, dans le cas de la culture afro-américaine, nous nous retrouvons dans un contexte où ce concept de culture est en concurrence avec celui de la race selon l'auteur, où une certaine \emph{essentialisation} par le racisme maintient cette version hégémonique d'eux-mêmes.

Pour comprendre qu'il existerait une culture née par l'hybridité chez les individus gais et lesbiens, l'auteur considère que l'on doit se baser sur l'histoire des individus plutôt que s'attarder seulement à l'Histoire au sens large des sociétés.
En effet: \foreignblockquote{english}[{\cite[280]{Sinfield1996}}][.]{\textelp{} for lesbians and gay men the diasporic sense of separation and loss, so far from affording a principle of coherence for our subcultures, may actually attach to aspects of the (heterosexual) culture of our childhood, where we are no longer \enquote{at home}. Instead of dispersing, we assemble.

The hybridity of our subcultures derives not from the loss of even a mythical unity, but from the difficulty we experience in envisioning ourselves beyond the framework of normative heterosexism --- the \emph{straightgeist} \textelp{}}
Dans ce contexte, on peut dénoter que l'auteur souligne une des particularités de la culture dominante: son caractère essentiellement hétérosexuel au niveau des normes, ou hétéronormatif (voir partie~\ref{sub:enjeux_g_ographiques_du_recours_l_identit_}).
Le départ de la culture hétérosexuelle ou \anglais{straightgeist} à laquelle tous les individus de la culture homosexuelle doivent répondre est partielle; à tout moment, les gays et lesbiennes pour ne nommer que ceux-ci doivent composer avec le reste de la culture hétérosexuelle dans les autres sphères de leur vie, que ce soit à l'école, au travail ou dans l'espace public.
C'est en raison de cette négociation inévitable avec la culture dominante que pourrait se justifier le caractère hybride de la culture homosexuelle.
Les objets culturels, les pratiques culturelles et sociales s'inscrivent dans cette culture hybride et peuvent donc ou non être compris par la culture dominante.

Pour conclure cette partie, notons que le trait commun partagé par les gays et lesbiennes dans le cadre de l'analyse par l'ethnicité-et-des-droits est l'altérité vécue par les individus non hétérosexuels: \foreignblockquote{english}[{\cite[289]{Sinfield1996}}][.]{Our apparent unity is founded in the shared condition of being not-heterosexual --- compare \enquote{people of colour}, whose collocation derives from being not-white}.
Étant donné cet accent mis sur la discordance à une norme, la communauté \lgbt\ est nécessairement très large et diverse.
Sinfield hésite donc à parler ici d'une culture en soi; on propose plutôt l'usage du concept de sous-culture qui rendrait mieux ce caractère de diversité et qui reconnaîtrait le côté construit et récent de celle-ci:
\foreignblockquote{english}[{\cite[289]{Sinfield1996}}][.]{It is to protect my argument from the disadvantages of the ethnicity model that I have been insisting on \enquote{subculture}, as opposed to \enquote{identity} or \enquote{community}: I envisage it as retaining a strong sense of diversity, of provisionality, of constructedness}.

% , Hall \& Gay introduisent le concept d'identité pour traiter des groupes
% sociaux et culturels qui s'opposerait à l'ancien sujet moderne. À partir de ce
% concept, il devient a priori possible de traiter de groupes ou communautés comme
% les homosexuels, bisexuels, trans- et queers dans le contexte culturel précisé
% précédemment.

\subsection{Enjeux géographiques du recours à l'identité}
\label{sub:enjeux_g_ographiques_du_recours_l_identit_}

%Les textes suivant
% permettraient de situer l'usage de la culture dans un contexte précis. Par
% exemple, \textquote{The Location of Culture: The Urban Culturalist Perspective}
% propose l'étude culturelle des phénomènes urbains, alors que les études urbaines
% utilisent normalement des méthodes quantitatives pour traiter des mêmes
% questions (Borer, 2006). Les parties précédentes se sont principalement
% intéressées à l'analyse générale des concepts d'identité, de culture et
% d'identité sexuelle en demeurant essentiellement dans un contexte sociologique.
% Pour la poursuite de ce texte, nous nous intéresserons plus particulièrement au
% domaine spatial de ces concepts en tentant d'apposer une regard géographique sur
% le culture. %, plus particulièrement par un regard sur la ville comme espace
% culturel. Celle-ci, en plus d'être un des milieux les plus populeux qu'on
% retrouve dans plusieurs des sociétés humaines, sinon la totalité de nos jours,
% permet de comprendre les enjeux entourant la mixité sociale et culturelle

%Pour cette partie, nous pencherons sur deux textes de Michael Borer, à savoir \textquote{The Location of Culture: The Urban Culturalist Perspective} (2006) et \textquote{From Collective Memory to Collective Imagination} (2010).~\citep{Borer2006}
%Le deuxième texte de Borer, \textquote{From Collective Memory to Collective
%Imagination >> propose l'analyse spatiotemporelle des phénomènes culturels en
%milieu urbain, un point de vue méthodologique qui rejoint celui de Larry Knopp.~\citep{Borer2010}

En géographie queer, on retrouve les deux paradigmes soulevés à la partie~\ref{sub:minorit_s_et_universel}, à savoir un partage entre une analyse autour de la diversité comme construction sociale et une autre centrée sur l'identité gaie, lesbiennes, bisexuelle ou trans-.
Plus particulièrement, la pensée géographique peut se pencher sur les espaces occupés par les gays et lesbiennes, ou plutôt s'intéresser au caractère normatif des espaces.
Les premiers travaux en géographie sexuelle se sont principalement attardés au premier point de vue.
À l'inverse, durant les vingt années précédentes, on remet en question le point de vue \emph{minoritarisant} qu'on retrouve toujours dans certains textes comme celui de~\citet{Sinfield1996} pour plutôt se pencher sur les normes sociales et leurs rapports avec l'espace.
C'est ce dernier point de vue que défend et explique Natalie~\citet{Oswin2008} dans l'article \citetitle{Oswin2008} que nous traiterons dans la suite de ce texte.

Dans ce texte, Oswin s'oppose à l'idée que les espaces queers puissent former des lieux en opposition totale avec les normes de la société dominante.
Plutôt, la recherche récente en géographie sexuelle a permis de rendre compte que c'est par la présence d'individus dont la manière de performer le genre ou l'identité sexuelle ne correspondent pas aux normes sexuelles dominantes que les espaces de la société apparaissent comme hétéronormatifs.
Cette hétéronormativité agit par des jeux de pouvoir s'y établissant, entre individus et envers eux-mêmes.
Ce jeu de pouvoir s'instaurerait de plusieurs manières: par la marginalisation (violence homophobe, exclusion), par une présence accrue du pouvoir policier à proximité des espaces queers ou par le refus des instances gouvernementales de répondre aux demandes des populations \lgbt{} (durant la crise du \sida, par exemple).

Ce jeu de pouvoir sur les normes sociales se manifeste particulièrement dans les espaces réputés pour être occupés par des individus appartenant au spectre \lgbt{} où plusieurs visions de l'homosexualité se confrontent.
En effet, comparativement à l'idée d'une culture homosexuelle uniforme et partagée par certains membres de la communauté gaie et lesbienne, les auteurs en études queers et en géographie queer ont plutôt montré que plusieurs groupes luttent selon deux types d'enjeux.
D'abord, certains supportent l'idée que la communauté devrait travailler vers un élargissement des normes qui mènerait à terme à une plus grande inclusion sociale.
Le point de vue \emph{assimilationniste} au \emph{libérationnisme} dont les représentantes et représentants souhaitent plutôt remettre en question certaines normes qu'elles et ils considèrent empruntées à la culture hétérosexuelle et répliquées à l'intérieur même des espaces queers, un phénomène couvert par le concept d'homonormativité.
Oswin va s'appuyer sur une définition de Lisa Duggan, pour qui l'homonormativité est:
\foreigntextquote{english}[{\cite[50]{Duggan2003}}][]{A politics that does not contest dominant heteronormative assumptions and institutions, but upholds and sustains them, while promising the possibility of a demobilized gay constituency and a privatized, depoliticized gay culture anchored in domesticity and consumption}.

Un autre point important du texte d'Oswin est la réinterprétation du sens des multiples identités que peut posséder un individu du spectre \lgbt{}.
Au lieu de se baser sur l'hybridité, ces identités sont plutôt conçues comme des sources d'oppression, en ce qui concerne la racisation, la classe sociale ou le genre par exemple.
Dans de nombreux espaces queers, il a été remarqué que souvent le pouvoir était détenu par des individus dits privilégiés sur d'autres bases identitaires que la simple orientation sexuelle.
Également, au sein même des communautés gaies, d'autres formes d'identité sexuelle ou de genre sont mises de côté, comme la bisexualité ou le vécu trans~\citep[93]{Oswin2008}.
L'auteur met l'accent sur l'importance de reconnaître le potentiel qu'ont les chercheures et chercheurs de réifier les communautés des milieux qu'elles et ils étudient.
Celles et ceux-ci peuvent notamment recréer certaines hiérarchies en omettant les inégalités sociales entre individus d'une même communauté ou les enjeux de racisation, par exemple.

Pour la suite, nous nous intéresserons donc plus particulièrement aux méthodes offertes par la géographie pour traiter efficacement des questions de diversité sexuelle et d'espace tout en étant de minimiser le risque de réification.
Larry~\citet{Knopp2004}, un des premiers chercheurs à lier les études queers à la géographie culturelle, se penche, dans \citetitle{Knopp2004} sur la théorie de l'acteur-réseau.
Cette dernière pourrait permettre méthodologiquement de dépasser certaines limites de l'identité comme concept pour traiter des populations \lgbt{} tout en s’efforçant de fuir certains déterminismes en géographie culturelle~\citep{Knopp2004}.
En effet, au lieu de s'appuyer sur un point de vue \emph{minoritarisant} des groupes et communautés \lgbt{}, Knopp reconnaît d'emblée le potentiel qu'ont ces groupes d'affecter les structures sociales de pouvoir.
Ce texte s'inscrit donc ainsi moins dans l'étude d'un groupe culturel particulier; on vise plutôt l'analyse des normes sociales d'un ensemble culturel par les conflits amenés.
Knopp utilise dans son texte des termes similaires à ceux de \citet{Sinfield1996}, à savoir que les groupes queers seraient entre autres hybrides au point de vue identitaire et que leur présence sociale prendrait la forme d'une diaspora.
En effet, si les espaces queers sont les espaces vers lesquels se dirigent les membres de ces communautés dans le texte de \citet{Sinfield1996}, Knopp considère plutôt que ce sont les déplacements spatiaux et temporels qui sont formateurs des identités queers.
La géographie aurait alors le potentiel de rendre compte de ces déplacements par les théories \emph{non représentationnelles}\footnote{Ensemble de travaux visant à surmonter la confrontation existant entre des théories axées sur l'identité et celles portées vers des explications dites matérialistes visant à expliquer les phénomènes sociaux.}.

Ces déplacements ont en effet un sens particulier: \foreignblockquote{english}[{\cite[123]{Knopp2004}}][.]{For gays, lesbians, bisexuals, transgenders, and other queers, as for other oppressed groups, this means seeking people, places, relationships, and ways of being that provide the physical and emotional security, the wholeness as individuals and as collectivities, and the solidarity that are denied us in a heterosexist world}
Au lieu d'être les héritiers directs d'une culture queer, on doit plutôt concevoir que les individus queers possèdent bel et bien la culture dominante, mais que leur intégration sociale passe par d'autres trajets que ceux proposés normalement par celle-ci.

Knopp avance même que ces déplacements ont une importance assez forte pour être génératrice d'une ontologie particulière:
\foreignblockquote{english}[{\cite[123]{Knopp2004}}][.]{It is also about testing, exploring, and experimenting with alternative ways of \emph{being}, in contexts that are unencumbered by the expectations of tight-knit family, kinship, or community relationships—no matter how accepting these might be perceived to be}
En même temps que les individus \lgbt{} quittent le contexte familial hétérosexuel comme début de parcours et principal lieu d'acquisition de culture, ils transportent avec eux des éléments de cette culture, la transformant au gré de leurs expériences.
Pour Knopp, l'expérience queer en soi provoque la constitution de nouvelles données culturelles, spatiales et participe donc à cette hybridité de l'identité queer.
\foreignblockquote{english}[{\cite[130]{Knopp2004}}][.]{As queer bodies and subjectivities circulate through (and constitute) time and space, they leave legacies, absorb others, and mutate. They spread information, values, and culture, and constitute barriers to such spreads at the same time. This is diffusion par \emph{excellence}}

Si ces caractéristiques sont particulièrement importantes, Knopp souligne tout de même que ces processus formateurs dans le domaine identitaire ne sont pas l'apanage des individus \lgbt{}, mais peuvent être considérés comme des expériences probables pour tous les individus; elles ne sont pas essentielles à l'expérience \lgbt{}.
Néanmoins, le contexte normatif entourant la sexualité pousse tout de même les populations \lgbt{} à ces parcours de façon plus particulière.
Les chemins de vie empruntés ressembleraient à des migrations vers l'acceptation sociale de l'identité sexuelle, qu'elle passe par l'anonymat ou pas l'inclusion au sein d'un espace \lgbt{}.
% Knopp appuie cette position par plusieurs travaux en géographie queer qui reconnaissent l'importance du parcours et du déplacement spatio-temporel chez les individus \lgbt{}~\citep[123]{Knopp2004}.

\subsection{Synthèse}
\label{sub:synth_se}

Au cours de ce chapitre, nous avons d'abord traité de la question de la culture d'un point de vue anthropologique, à l'aide de Geertz\todo{à bonifier!}.
Pour ce dernier, la culture s'articule autour des symboles et ces derniers permettent la transmission d'informations informant les individus de leur contenu et de leur sens.
Si des systèmes culturels permettent de répondre à des angoisses viscérales, face à la mort principalement, l'analyse geertzienne permet également de comprendre que différents systèmes culturels répondent à différentes interrogations et donnent un sens particulier à chaque sphère de la vie.
Si une culture est bel et bien un système de pratiques et de conceptions héritées, on pourrait croire qu'il existe une culture \lgbt{}.
En effet, si la culture hégémonique en occident permet d'expliquer l'hétérosexualité autour des symboles du mariage, de la reproduction ou de la division sexuelle du travail, cette dernière a été à tout le moins muette, au pire violente face à des comportements qu'elle ne pouvait expliquer.
La politisation des individus marginalisés sur la base de leurs pratiques sexuelles et leur rassemblement autour d'identités partagées aurait permis la construction de nouveaux symboles permettant d'expliquer ces pratiques sexuelles.
On constate aujourd'hui qu'il existe des pratiques particulières des communautés gaies et lesbiennes, notamment par l'existence de milieux de vie particuliers, de pratiques sociales et sexuelles différentes des hétérosexuels et d'une histoire particulière à cette communauté (qui demeure tout de même imbriquée dans celle de la société majoritaire).
Le contexte social, politique et intellectuel du dernier siècle semble avoir été propice à l'irruption des communautés gaies et lesbiennes, comme nous l'apprend la partie sur le texte de Hall.
En effet, si, des lumières jusqu'à la modernité, il était difficile d'imaginer l'existence d'identités basées sur la sexualité, différents décentrements et déstabilisations du sujet moderne ont amené cette possibilité.
Elle s'est concrétisée par une remise en question de l'uniformité relative des individus au sein d'une même culture.
Est-ce toutefois suffisant pour dire qu'il existerait aujourd'hui des cultures basées sur des identités?
En ce qui concerne la question de l'orientation sexuelle du moins, la réponse n'est pas claire.
On remet aussi en question l'idée de s'attarder à l'identité dans un contexte où ces individus, ceux de la diversité sexuelle, sont souvent victimes de marginalisation dans le contexte culturel occidental.
Leur présence exprime des instabilités propres au nouveau sujet postmoderne vivant dans un contexte culturel aux normes qui témoignent de jeux de pouvoir bien présents, et ce pour les individus de la diversité sexuelle et le reste de la société.
Au-delà des enjeux stratégiques liés à l'étude des individus \lgbt{}, dans une perspective de changement social, on pourrait croire que ces déstabilisations continuent d'avoir lieu.
La critique des normes sociales et des enjeux de pouvoir demeure l'avenue privilégiée, dans un contexte de l'étude de la culture occidentale.

Nous pouvons conclure en soulignant que les liens entre identité et culture ne sont pas encore clairement définis; l'identité comme concept laisse croire qu'il est nécessaire d'appartenir à une culture particulière pour se lier à une identité précise.
Pourtant, malgré l'ambiguïté entourant l'idée de culture ou de sous-culture \lgbt{}, on peut difficilement remettre en question l'idée d'identité, sachant l'existence de groupes sociaux, culturels et d'espaces \lgbt{} dans la société occidentale.

La question du pouvoir reste à creuser plus en profondeur comme souligné par Oswin dans son texte.
Celle-ci propose également l'étude de l'hétérosexualité, une avenue intéressante pour comprendre si cette identité sexuelle existe comme celle des autres formes d'orientation sexuelle.
On pourrait ainsi comprendre les effets probables de la normalisation de la sexualité à une plus grande échelle au sein de la culture occidentale~\citep[100]{Oswin2008}.
Enfin, il apparait également pertinent d’entamer une étude auprès des groupes \lgbt{} pour savoir qu'elle est la place de la communauté dans leur identité.
Un tel travail permettrait d'apprendre l'existence, selon eux, des éléments nécessaires pour parler d'une culture queer, que ce soit dans histoire ou dans les significations particulières des éléments et pratiques composant les espaces et la temporalité queer.

\subsection{La diversité sexuelle dans les sciences sociales et la géographie}
\label{ssub:la_diversit_sexuelle_dans_les_sciences_sociales_et_la_g_ographie}

Nous travaillerons dans cette étude à partir de l'identité sexuelle et du genre.
\citet{Foucault2011} décrit dans son œuvre \citetitle{Foucault2011} le processus par lequel les relations entre individus de même sexe sont devenues, par la médecine notamment, une forme de trouble de la sexualité jusqu'à constituer aujourd'hui, après bien des luttes sociales, une identité particulière~\citep{Foucault2011}.
Plusieurs autres auteurs, en études féministes en particulier, avec Gayle Rubin et Judith Butler~\citep[98]{Marcus2005}, ont permis de montrer que le genre consiste aussi en un construit social et une performance, contredisant que celui-ci forme l'essence de l'individu~\citep{Butler2007}.
On remet alors en question l'idée de nature en sexualité ainsi que la possibilité que les comportements dits anormaux, comme l'homosexualité, la bisexualité ou un genre non binaire, soient en fait des comportements sociaux qui ne concordent tout simplement pas avec les normes.

Ce travail théorique fut plus tardivement repris par la géographie avec une certaine justification.
À la même époque que les études sur le genre avaient lieu et sortaient du seul cadre de la psychanalyse~\citep{Rubin2011a,Rubin2011}, les luttes sociales entreprises par les gais et lesbiennes prenaient place dans l'espace public~\citep[422-427]{Spencer2005}.
Dans les grandes villes des États-Unis d'abord, puis progressivement d'Europe et du Canada, on a assisté à l'apparition des villages gais.
Ces nouveaux espaces urbains ont été au départ utilisés pour la protection par l'anonymat, puis, avec ces luttes, le partage d'un mode de vie et de lutte politique.
Relativement invisibles, les espaces habités par les gais et lesbiennes acquièrent une visibilité supplémentaire en même temps que l'on commença à s'intéresser aux liens entre l'espace et la sexualité dans un contexte ou l'identité sexuelle semblait détenir une ontologie propre.

Pour certains auteurs, ce développement identitaire prend une forme ontologique.
L'ontologie désigne le savoir représentant une vision particulière de l'univers.
Son usage serait de plus en plus complexifié par l'apparition récente d'une multitude de nouvelles identités, par la politisation de plusieurs pans de population vivant la marginalisation et par une remise en question de la culture dominante~\citep[122]{Knopp2004}.
Le queer pourrait désigner l'ontologie propre au spectre de la diversité sexuelle apte à dépasser les normes séparant chacune des identités du spectre \lgbt{} et pouvant rendre compte d'une certaine cohésion qui rejoint historiquement les individus de ces communautés~\citep[122]{Knopp2004}.

Larry Knopp propose d'ailleurs la théorie de l'acteur-réseau pour parvenir à identifier et analyser toutes les composantes spatiales du queer comme le lieu et le mouvement et recréer une certaine cohésion de la théorie queer face aux divers mouvements qui la façonnent, comme le constructivisme, le matérialisme.
Elle permettrait également d'échapper à la notion du territoire pour plutôt utiliser une vision plus relativiste de l'espace étant donné que les éléments d'un réseau peuvent croiser d'autres réseaux et porter plusieurs sens.
Des approches où l'espace peut prendre diverses formes et qualités~\citep{DiMeo1998} seraient beaucoup plus compatibles avec la théorie de l'acteur-réseau.

La géographie culturelle se penche donc depuis peu à ces nouveaux groupes sociaux, surtout avec l'avènement des approches postmoderne.
Celles-ci, dont la théorie queer fait partie, s'intéressent de nouveau à l'individu en prenant tout de même assise sur les travaux ultérieurs, notamment le structuralisme et la théorie critique.
Une des premières œuvres à avoir marqué le commencement de la géographie queer est~\citetitle{Bell1995b} de~\citet{Bell1995b} qui amena une sélection de textes, montrant la pertinence et les possibilités multiples de cette nouvelle branche de la géographie.
Néanmoins, dans plusieurs études, l'homme gai a été mis en priorité comme objet où on s'attarda à l'usage de certaines méthodes de géographie, comme le recours au territoire~\citep{Podmore2001,Oswin2008}.
On travaillait donc à comprendre la progression des villages gais, souvent menée par des hommes.
Les femmes lesbiennes et les individus caractérisés par d'autres formes de sexualités moins étudiées ont été mis de côté.
L'argument derrière cette division est celles-ci sont globalement moins visibles et présentes dans l'espace public et que l'on devrait s'attarder au ménage et à la maison pour traiter de leurs situations~\citep[333-334]{Podmore2001}.
Il s'agit, selon Podmore, dans l'article \citetitle{Podmore2001} plutôt d'un problème méthodologique alors que les concepts utilisés actuellement en géographie ne suffisent plus pour étudier les nouvelles formes identitaires.
Le recours concept de territoire pour s'intéresser aux communautés sexuelles apparait donc problématique: non seulement on exclut les femmes, mais également d'autres groupes encore moins visibles, comme les trans et les bisexuels.
Il convient alors, en reprenant la position de Natalie Oswin, de traiter d'espaces queers plutôt que d'espaces dits \lgbt.
Cette nuance vise à éviter une essentialisation de l'autre sans mettre en priorité certains genres, utiliser des dichotomies fautives entre l'homosexuel face au \anglais{straight} et considérer \latin{de facto} un milieu queer comme un espace de résistance de dissidence~\citep{Oswin2008}.
En effet, comme l'a montré Nathaniel M. \citet{Lewis2011} dans son article \citetitle{Lewis2011}, les hommes gais, même sans espaces précis, héritent des normes de leur environnement sur l'identité sexuelle.
Ces derniers peuvent également induire un effet sur ces normes, l'auteur prenant exemple sur les familles homoparentales dans les banlieues.
Ces dernières offrent au chercheur un regard différent sur la définition d'un ménage de classe moyenne, particulièrement au sein d'un espace comme un voisinage~\citep[304]{Lewis2011}.

\subsection{La sémiotique}
\label{ssub:la_semiotique}
\todo{à revoir}

La sémiotique prend ses racines dans les travaux de plusieurs auteurs, principalement Ferdinand de Saussure qui a également donné naissance au courant du structuralisme~\citep{Noth1995}.
Dans ses travaux sur la religion, Clifford Geertz considère que le géosymbole est
On retrouve aujourd'hui toute une variété d'auteurs et de mouvements, autant en musique qu'en théologie, mais dans le cas présent, nous nous intéresserons principalement aux auteurs en géographie.
Au sein de ceux-ci, on peut nommer d'abord Joël de Bonnemaison qui démontra l'importance des géosymboles comme moyen utilisé par les groupes ethniques pour s'ancrer à l'espace habité qui deviendra le territoire par des itinéraires, des formes et des tracés dans le paysage~\citep{Bonnemaison1981}.
Mario Bédard va poursuivre la réflexion et inclure tout comportement culturel particulier rattaché à l'espace, et créer une typologie complexe des différents géosymboles que l'on retrouve dans le territoire~\citep{Bedard2002}.
Jérôme Monnet quant à lui traite plutôt des processus conscients de création de symboles, et des relations particulières qui lient les symboles à l'espace, au pouvoir et à l'identité, surtout dans le contexte occidental et capitaliste~\citep{Monnet1998}.
L'idée de contexte se joue d'ailleurs à plusieurs échelles, et il apparait important selon l'auteur de ne pas limiter l'étude d'un symbole seul.
Il faudrait également considérer l'environnement du symbole, sachant que la présence d'un symbole d'un type peut amener la propagation de symboles similaires, comme dans le cas d'un centre commercial qui provoque l'arrivée d'autres commerces, comme symboles marchands~\citep[7-8]{Monnet1998}.
Il apparait donc possible, à l'aide des travaux effectués en sémiotique et en géographie culturelle, d'arriver à décrire l'espace occupé par les groupes et individus appartenant au spectre de la diversité sexuelle
Nous devons par contre tenir compte de certaines difficultés, comme le recours au territoire (qui peut masquer espaces dispersés ou en réseaux) ou encore à la matérialité des symboles, qui peuvent ne pas être suffisants comme appuis pour décrire une réalité spatiale soit temporaire, soit mobile.

\citet[105--109]{Rose2012} résume les différents moyens de faire de la sémiologie une méthode et une analyse de la société.
On retrouverait deux courants principaux dans le domaine de la recherche en sémiologie qui déborde les champs disciplinaires traditionnels comme l'anthropologie et la géographie.
Le premier courant serait les chercheurs pratiquant la sémiologie dite
Le deuxième courant, nommé sémiologie sociale (\anglais{social semiotics}), consisterait en une analyse des faits sociaux.

Dans le chapitre suivant, nous reviendrons sur la sémiotique, mais du point de vue de la méthode.
Nous décrirons alors comment nous pourrons, à partir d'une déconstruction des images collectées, obtenir plus d'information sur le sens des géosymboles évoqués et ce qu'ils traduisent comme relation territoriale entre les espaces des villes traitées et les sous-groupes du spectre \lgbt{}.

%\paragraph{Signifiant, Signifié et Référent}
Selon \citet[113]{Rose2012}, le signe est le principal concept que l'on retrouve en sémiotique.
Il consiste en différentes informations qui circuleraient dans un système de communication, que ce soit entre deux individus ou un individu et son environnement.
Ce concept, dans le modèle de Saussure, est divisé et deux parties pour rendre compte de la relation existant entre un individu dans un système de communication et les objets ou idées auxquelles il réfère.
Ce premier élément correspond au signifié qui consiste en l'objet ou l'idée auquel on octroie un signe.
Dans un contexte de communication ou non, le signifié en soi existe toujours.
Bien que certains signifiés dépendent de l'être humain pour exister, comme la philosophie ou les mathématiques ou des objets manufacturés, en général ces signifiés existent sans dépendance à un système de communication auquel on ferait référence.
Autrement dit, si c'est l'être humain dans sa communication qui permet une telle caractérisation d'un objet par le signifié, il demeure que ce dernier peut exister en soi.
Le signifiant quant à lui ne peut se séparer de ce système de communication.
En effet, ce signifiant forme en quelque sorte l'élément qui permet d'évoquer chez la personne jouant le rôle de communicateur l'idée rattachée au signifié.
Le mot \emph{roche} dans un tel système constitue un signifiant lié à la substance minérale séparée d'un substrat rocheux plus important qui est ici le signifié dans sa forme la plus générale et dont la forme peut différer selon les occasions et les contextes.
Enfin, ce signifiant n'est pas qu'un mot prononcé ou lu, il peut également consister en un média au sein duquel l'idée d'un signifié est transmise, comme une image, une vidéo, un poème, etc.

%\paragraph{Signe}
Dans l'ensemble, cette définition du signe correspond aux théories de Saussure.
Elle présente rapidement des limites: si un signifiant peut prendre plusieurs formes, comment peut-on comparer facilement ceux-ci entre eux?
Si les individus utilisent divers signifiants pour communiquer, certains demandent un bagage culturel particulier pour être compris alors que d'autres sont plus universels.
Les travaux de Pearson permettent conceptuellement de dépasser ces limites.
Nous allons enrichir cet usage du signe en concordance avec la théorie développée par Pearson.
D'après lui, le signe peut prendre plusieurs formes selon le média dans lequel on le trouve et le degré d'abstraction de différenciation qui peut exister entre le signifié et le signifiant.
Ainsi, différents termes ont été développés pour différencier les différents signes.
Parmi ceux-ci, on retrouve d'abord l'icône, qui consiste en une quasi concordance entre le signifié et le signifiant.
Dans un dessin d'enfant, selon les cultures, un arbre est souvent dessiné avec un tronc brun, quelques branches et des feuilles vertes.
Cette image, sans recouper l'immense diversité des arbres, correspond tout de même à plusieurs types d'arbres que l'on retrouve dans un climat tempéré.
Par contre, ce même arbre pourrait figurer sur un roman ou un livre traitant de la vie, de la longévité ou encore de la généalogie dans un contexte familial: dans ce cas-ci, l'arbre est lié de façon abstraite à ces concepts.
On a donc affaire à un symbole, un signe dont la relation entre le signifiant et le signifié est définie de façon arbitraire et relative à un contexte culturel particulier.
Le troisième type de signe est l'index dans lequel le signifiant et le signifié ne sont a priori pas liés, de façon comparable au symbole, mais qu'en plus, le signifiant n'a pas de sens en soi sans le signifié.
En effet, dans les cas précédents, surtout dans des cas plus visuels, on ne peut séparer le signifiant du signifié, ou du moins, on peut trouver d'autres types de relations.
La longévité peut être signifiée par des signifiants différents, comme une personne très âgée, alors que le signifiant d'arbre rappelle le signifié \enquote{arbre}, sans mise en contexte.
Dans l'index, la personne en communication doit avoir connaissance du lien entre le signifiant et le signifié.
On peut penser par exemple aux feux de signalisations: les cercles rouges, jaunes et verts ont été désignés par les signifiants des différents types de restrictions ou de non restrictions au déplacement en automobile.
Par contre, sans connaissance préalable des codes de la route, un individu ne peut comprendre ce lien de façon instinctive.
Un autre exemple: un mot dans une langue.
Si l'individu qui communique n'est pas locuteur de cette langue ou ne possède pas les connaissances pour saisir de quoi il s'agit, l'ensemble des caractères ne peut rien évoquer du signifié qu'il désigne.

On le constate rapidement, ce modèle ne porte \latin{a priori} rien de sens géographique en soi.
Il ne permet pas facilement d'articuler les nuances propres à la culture et à l'identité d'un locuteur au-delà de la connaissance ou de l'ignorance d'un signe prenant la forme d'un index.
C'est ici que nous allons nous arrêter pour traiter des notions d'espaces et de territoire.
Nous nous attarderons ensuite à la synthèse à l'aide du concept de géosymbole tel qu'articulé par \citet{Bonnemaison1981}.

\section{Espaces et territoires}
\label{sec:espaces_et_territoires}

Nous poursuivrons avec ces deux concepts les plus géographiques, traités ensemble.
En effet, notre démarche s'inscrit dans une certaine critique du concept de territoire et en contrepartie nous proposons le concept d'espace comme base à une analyse de multiples communautés.
Ces deux concepts, quoique semblables à première vue, prennent en géographie culturelle des sens différents.
Le concept d'espace témoigne d'une conception objective ou empirique de la réalité en s'intéressant principalement aux questions de distances entre les objets et leurs position géographique exacte.
Principalement utilisé en mathématiques, en physique, et dans l'usage courant d'un point de vue pratique ou technique, l'espace se définit comme une construction intellectuelle neutre dont les propriétés sont quantifiables \citep[99]{DiMeo1998}.
On peut penser aux distances en kilomètres pour un voyageur entre deux villes, le calcul de la taille pour la construction d'une maison, ou encore du volume pour la quantification d'un liquide; ces données forment ce que Di Méo nomme les propriétés spatiales des objets, et s'insèrent dans la géographie euclidienne propre à la perception humaine.
Mais nous allons le voir, ces calculs d'apparence banale semblent participer aujourd'hui à un point de vue particulier sur le monde qu'on décrira comme désenchanté.

L'utilisation d'un tel concept dans le cadre de la géographie culturelle peut de prime abord paraître contradictoire; en effet, ce champ de la géographie s'intéresse surtout aux populations et à leurs territoires, selon leurs pratiques et leurs spécificités identitaires ou ethniques.
L'espace prendrait plutôt sa pertinence ailleurs, dans le contexte de la géographie sociale par exemple, lorsque l'on s'intéresse aux enjeux propres aux déplacements.
On peut penser aussi au champ de la géographie physique ou de la biogéographie qui traitent bien souvent les substrats rocheux ou le monde du vivant comme faisant partie d'ensembles comme les \todo{trouver le mot manquant} comme des données quantifiables.
Par contre, durant le dernier siècle, la géographie culturelle, telle que pratiquée par des géographes du Nouveau\missref{} et de l'Ancien Monde\missref{}, portait son regard sur des populations des pays colonisés ou en voie de décolonisation en utilisant plutôt le concept de territoire.
Particulièrement, nous pensons à la géographie tropicale et son point de vue basé sur l'altérité entre les régions nordiques, dites normales, et les régions du sud~\citep[493]{Power2009} et également les travaux plus anciens en géographie régionale française s'intéressant strictement aux régions~\citep[31]{Courville1991}.
Ceux-ci travaillaient bien souvent sur des populations plutôt restreintes pensées comme des ethnies \todo{à retravailler}.
Ces dernières semblaient à première vue vivre dans des milieux suffisamment isolés pour que les interactions interculturelles doivent être considérées comme quasi inexistantes.
De cette façon, cette géographie pouvait restreindre les caractéristiques de la population comme des faits uniques plutôt que l'objet d'interactions avec d'autres populations destinées à évoluer, avec ou sans la présence des schémas de colonisation~\citep[79--80]{DiMeo2007}.

Dans ce contexte, le territoire est perçu comme un plan de la réalité distinct se superposant à l'espace: il s'agirait d'un ensemble d'éléments abstraits et matériels permettant à la population d'ancrer son identité, son histoire comme son futur.
On retrouve cette définition chez plusieurs auteurs, notamment Bonnemaison pour qui le territoire :
\blockquote[{\cite[253]{Bonnemaison1981}}][.]{n'est pas forcément clos, il n'est pas toujours un tissu spatial uni, il n'induit pas non plus un comportement nécessairement stable}, mais serait plutôt \textquote{un ensemble de lieux hiérarchisés, connectés à un réseau d'itinéraires}.
Di Méo nous offre également une définition similaire, pour qui:
\blockquote[{\cite[76]{DiMeo2007}}][.]{L’assise territoriale, campée sur un réseau de lieux et d’objets géographiques, constitué en éléments patrimoniaux visibles, renforce l’image identitaire de toute collectivité. Elle lui dresse une scène et la pourvoit d’un contexte discursif de justification particulièrement efficace en ville où des lieux très denses, soigneusement et anciennement dénommés, s’inscrivent dans une totalité territoriale représentée, à la fois symbolique et fonctionnelle}. \todo{à approfondir}

Le désenchantement du monde tel que décrit précédemment est décrit dans les travaux de Max Weber comme le phénomène par lequel les explications reliées au mystère, la religion ou la superstition perdent leur place face à celles qu'offre la rationalité scientifique.
On peut donc comprendre que c'est l'avancement scientifique et surtout sa méthode et ses découvertes qui affectera la société en mettant en danger la place de la religion et du mythe dans la société.
Enlevant une part importante du divin dans l'explication des phénomènes terrestres, par la théorie de l'évolution par exemple, le phénomène va prendre de l'ampleur par l'arrivée du capitalisme au \siecle{19}.
Par sa capacité à produire des marchandises en masse sans l'apport individuel de l'ouvrier --- sa subjectivité --- ce nouveau système social gérant le travail et les échanges rendra l'économie plus anonyme et aliénante.
Combinés, ces effets --- rationalité et aliénation --- éloigneront progressivement l'individu de sa capacité à expliquer le monde et à s'y conforter, perdant son pouvoir sur le matériel comme sur l'abstrait.

Ce développement affectera grandement les milieux de vie dans lesquels l'industrie s'implantera, principalement les villes.
Toujours plus demandantes en main-d’œuvre et offrant une quantité importante de marchandises à consommer, ces dernières prendront une taille de plus en plus conséquente alors que les quartiers évolueront pour nourrir cette industrie.
Des milieux villageois aux villes orientées vers les pouvoirs politiques, les sociétés se développeront désormais sur des nœuds urbains que l'urbanisme tentera de rationaliser par la poussée des champs architecturaux nouveaux, notamment par les travaux de Le Corbusier.
De la naissance du mouvement fonctionnaliste en architecture, les bâtiments posséderont maintenant des fonctions et les villes seront pensées comme des machines dont l'efficacité doit être maximisée\missref{}.
On peut souligner ici que l'avènement du capitalisme mettra en place un point de vue rationaliste de l'espace des villes, alors qu'on peut de moins en moins considérer celui-ci comme un territoire où les individus trouveront un sens à leur existence\todo{Reformuler la conclusion.
Également, voir à ce que la fin du paragraphe précédent ne répète pas inutilement celle-ci}.

Nous ne croyons pas qu'il s'agit ici d'un effet ayant pris emprise strictement en occident, alors que le capitalisme et d'une certaine façon la rationalité scientifique comme discours se sont étendus à grande échelle.
Néanmoins, malgré des effets sociaux très larges, nous considérons tout de même que les lieux touchés par ces effets ont des particularités qui leur sont propres du point de vue de l'organisation spatiale.
La réponse sociale à ces effets se conjugue à des effets politiques particuliers, que ce soit le colonialisme, les régimes politiques en vigueur, etc.
Ainsi, nous reconnaissons l'importance de l'Histoire dans les processus sociaux régionaux ailleurs dans le monde.
Les différences spatiales contribueraient donc aux développements de différences socioculturelles quant à l'emprise \todo{à compléter}

Nous voulons montrer que dans les villes, l'espace prend une place particulière après ce que nous considérons comme un recul important de la territorialisation dans un sens traditionnel.
Nous considérons que ceci pourrait se reporter à une vision étroite du concept d'ethnie.
De plus, nous le verrons dans la section \todo{à  compléter}
Ces effets transparaissent encore aujourd'hui à l'échelle des villes.
Après un développement certain des villes autour des industries, l'avènement d'un commerce mondial de plus en plus flexible et des entreprises toujours plus mobiles ont rendu les vieux centres urbains --- où s'étaient d'abord développées ces dernières --- moins intéressants.
Cette perte d'intérêt se retrouve autant pour les industries elles-mêmes, que pour les classes plus aisées bénéficiant du capital ou encore la classe moyenne en création.
Après avoir perdu en valeur, ces centres sont réhabités par les classes dominantes et rénovées selon les moyens des investisseurs accaparant ces espaces par le phénomène décrit par de nombreux chercheurs sous le terme de gentrification or d'embourgeoisement.

Néanmoins, nous n'assistons pas qu'à une simple \emph{désertification} des espaces dans lesquels les classes privilégiées prennent tous les espaces vacants et les individus appartenant aux classes inférieures sont laissés pour compte.
Bardet note que dans la modernité tardive, divers espaces dits anomiques mettent en danger les territoires de-delà des effets de la gentrification;  ce sont les non-lieux comme les routes, les corridors, les terrains vagues qui n'appartiennent à personne et qui ne de différencient pas d'un lieu à l'autre~\citep[22--23]{Bardet2012}.
Plusieurs cas de reterritorialisation de tels espaces ont tout de même été identifiés dans la littérature, par une diversité d'acteurs, avec ou sans succès \citep{Hatvany2005}\todo{trouver d'autres références}.
D'ailleurs, certains géographes ont relevé que cet espace présente tout de même des caractéristiques particulières.
Pour Courville notamment: \blockquote[{\cite[41]{Courville1991}}][.]{l'espace devient un médiateur du rapport entre individus, groupes et collectivités, un produit social à analyser comme tel, au milieu des pouvoirs et des rapports sociaux qui le structurent et l'organisent}.

C'est donc dire que cet espace urbain, selon son évolution, est maintenant le théâtre d'interactions multiples et enchevêtrées.
Des groupes comme les individus du spectre \lgbt{} ont vécu ces diverses interactions, en participant ou en subissant les phénomènes d’ embourgeoisement~\autocite{Podmore2001,Giraud2014,Hogan2005}.
D'autres ont plutôt participé à des luttes d'ordre économique ou politique sans qu'ils ne soient directement impliqués~\autocite{Kelliher2014}.
\todo{trouver des références liées aux luttes anticapitalistes, Angleterre et autres}

Leur présence dans ce jeu d'interactions offre à ces individus la possibilité de réaliser des rencontres qui mènent à une construction identitaire.
Comme souligné par Di Méo: \blockquote[{\cite[81]{DiMeo2007}}][.]{\textelp{} la ville fournit un potentiel privilégié d’outils de recentrage pour toute identité individuelle. Par sa variété intrinsèque et par les innombrables repères sensibles et vécus qu’elle étale, par les \emph{affordances} (emphase de l'auteur) qu’elle sème dans le champ des perceptions individuelles, la ville file une trame dont ses habitants se servent sans restriction pour tisser et inventer leur propre identité}.

Ainsi, on arrive à relier d'une certaine façon les travaux de sociologie sur le sujet et l'identité tels que décrits à la Section~\ref{sec:sujet_et_identite}.
Ces nouvelles identités, naissant en partie des luttes pour la reconnaissance et d'une certaine distance avec les identités nationales, trouvent également un lieu pour l'organisation dans les espaces urbains.

% \todo{passer la hache?} \note{Le terme territoire n'est pas utilisé car il
% apparait complexe de déterminer un territoire selon la définition typiquement
% utilisée en géographie culturelle; dans la partie~\ref{sub:enonce_du_probleme},
% la question de territorialité sera approfondie. Parmi les espaces urbains
% envisagés, on peut compter prioritairement ceux des villes de Montréal et Québec
% (voir figure~\ref{fig:carte_quebec}), et hypothétiquement ceux de villes de plus
% petite envergure selon la première partie de la collecte de données (voir à ce
% propos la Section~\ref{sec:source_des_donn_es}).}

% \note{concept on l'a vu peut présenter des lacunes importantes si elle tend à
% une réification des groupes culturels} \todo{insérer à quelque part?}

\section{Le géosymbole comme marqueur spatial}
\label{sec:le_symbole_comme_marqueur_spatial}

Nous retiendrons, avec les quelques nuances soulevées dans la section précédente, le concept de territoire pour décrire l'espace investi par un groupe culturel particulier.
Pour établir un lien entre cette notion culturelle pour présenter l'espace et les façons dont les individus communiquent entre eux, en demeurant sensible aux questions identitaires, nous proposons l'usage du géosymbole.
Peu répandu dans la géographie anglo-saxonne, ce concept reste intéressant pour l'analyse des groupes culturels contemporains.

La culture n'est plus considérée en géographie culturelle comme un tout monolithique; il s'agit plutôt d'une multitude de visions du monde différentes ou divergentes.
Chez les premiers auteurs en géographie culturelle, la région, le paysage et le territoire ont été des concepts particulièrement importants pour décrire les relations entre l'espace et la culture~\citep{Bonnemaison1981,Monnet1998,DiMeo1998,}.
Ceux-ci tendent par contre dans certains cas à proposer une vision incomplète de la culture, en mettant l'accent sur la vision de la culture dominante / hégémonique d'un espace~\citep[11-12]{Duncan1993}, ou en offrant une vision déformée des groupes culturels minoritaires ou marginalisés.

La culture occidentale par exemple n'est donc, en Amérique du Nord, qu'une des multiples cultures qui se partagent l'espace et avec laquelle il y a négociation~\citep[11]{Duncan1993}.
Contrairement aux anciennes perspectives en géographie culturelle qui définissaient la culture comme une entité à part, homogène et où régnait une apparence de consensus~\citep{Duncan1980}, la réalité semble s'éloigner de cette image.
C'est surtout lorsque l'on s'intéresse aux faits politiques, socio-économiques et d'immigration que les sociétés apparaissent beaucoup plus diversifiées que ce qu'elles semblent l'être.
Il faut donc se distancer de cette perspective dite \enquote{superorganique}, selon \citet[198]{Duncan1980}, qui mène à la réification de la culture; on pourrait même dans ce cas-ci étendre cette précaution aux groupes culturels plus restreints, comme les membres de la diversité sexuelle.
En effet, depuis les dernières décennies, la diversité sexuelle engendre plusieurs nouvelles visions du monde par la résistance à l'hétérosexisme qu'on peut voir dans les différentes luttes gaies et lesbiennes.
Occupant une myriade d'espaces difficiles à situer précisément comparativement à la culture dominante, il apparait inapproprié d'utiliser les concepts spatiaux traditionnels propres à la géographie culturelle comme le territoire pour parvenir à reconnaître et étudier ces groupes.

Il convient donc, pour l'étude géographique des minorités sexuelles, d'avoir recours à certains concepts moins rattachés à la matérialité de l'espace.
En demeurant dans le champ de la géographie culturelle, nous proposons l'usage du géosymbole comme outil conceptuel pour situer les groupes culturels et comprendre leur relation avec l'espace.
En effet, les géosymboles consistent selon Bonnemaison en: \blockquote[{\cite[256]{Bonnemaison1981}}][]{\textelp{} un lieu, un itinéraire, une étendue qui, pour des raisons religieuses, politiques ou  culturelles prend aux yeux de certains peuples et groupes ethniques, une  dimension symbolique qui les conforte dans leur identité}.
En nous réappropriant cette définition dans le contexte des minorités sexuelles, nous pouvons affirmer que nous traitons ici d'un groupe culturel dont la formation s'apparente de certaines façons à un groupe ethnique~\citep{Sinfield1996}, mais sans nécessairement être attaché à territoire au sens traditionnel comme soulevé précédemment.
La sémiotique, l'étude plus large des symboles, apparait pertinente.
Nous considérons que les géosymboles sont une forme particulière de symboles à dimension spatiale et que selon les auteurs, ces derniers peuvent prendre plusieurs formes différentes, matérielles ou non tout en demeurant attachés à l'espace~\citep{Bonnemaison1981,Bedard2002}.
Nous soulevons donc ici une des limites conceptuelles du géosymbole décrit par Bonnemaison (1981): nous allons utiliser ce concept sans la facette territoriale, mais plutôt en se rattachant seulement à l'espace et à la temporalité de son occupation.

Ces géosymboles, bien que permettant de s'intéresser à la culture, n'effacent pas nécessairement les aspects sociaux d'un groupe.
Des concepts plus proches de la géographie sociale comme les classes sociales ou la racisation~\citep{Bonniol2005} demeurent essentiels à la compréhension des espaces gais et lesbiens~\citep[93]{Oswin2008}.
Ils pourraient permettre de comprendre la position de certains géosymboles et leur raison d'être, en montrant les clivages propres à un groupe culturel précis, comme le spectre \lgbt{}.

Par ailleurs, ce qui pourrait apparaitre comme un groupe homogène, les \emph{gais}, forme tout un spectre.
Sa compréhension demande la prise en compte de facteurs sociaux pour comprendre les variances et les divisions, le genre en premier lieu, mais également l'appartenance à une certaine classe sociale.
Cette classe peut être plus ou moins prompte à vouloir s'accorder ou rejeter les normes sociales hétérosexuelles et à générer des espaces particuliers ou à transformer certains espaces conçus comme normaux~\citep{Lewis2011}.
En effet, le contexte actuel dans la littérature pousse à reconnaître que l'espace en général facilite les relations hétérosexuelles au détriment des autres relations, et donc, des identités qui y sont rattachées~\citep{Brown2003}.
Il est ainsi important de s'intéresser aux différents espaces qui existent en marge ou en parallèle et de se questionner sur l'utilité de ces espaces pour les individus impliqués.

Alors que le territoire n'arrive pas à bien définir la spatialité des individus de la diversité sexuelle, le recours à la sémiotique et aux géosymboles pourrait permettre de voir émerger certaines formes spatiales.
On peut croire également que celle-ci offrirait la possibilité de montrer la diversité des lieux, leur position ne permettant pas une compréhension plus poussée de l'utilisation de l'espace par les communautés \lgbt{}.
À notre connaissance, aucun travail ne traite spécifiquement de la sémiotique en géographie queer et il s'agit d'une lacune par rapport au potentiel que contient cette approche dans l'étude de l'espace et au sein de la géographie culturelle.
L'étude des géosymboles permettrait d'atteindre un savoir difficile ou impossible à obtenir par une approche plus matérialiste s'arrêtant au territoire et qui ne prendrait pas en compte l'existence de réseaux dans l'espace.
D'emblée, nous pouvons déjà considérer que des événements portent en eux des marques particulières, des logos ou des apparences qui entrent dans la définition du géosymbole.
À titre d'exemple, les Fiertés gaies, les manifestations politiques, les centres communautaires offrant des services aux individus porteurs du \vih, les bars lesbiens ou encore des lieux de dragues dans des espaces publics sont tous des d'espaces \lgbt{} pouvant utiliser des symboles pour marquer leur présence.

\blockquote[{\cite[108]{DiMeo1998}}][.]{\textelp{} le territoire multidimensionnel participe de trois ordres distincts. Il s'inscrit en premier lieu dans l'ordre de la matérialité, de la réalité concrète de cette terre d'où le terme tire son origine. Il relève en deuxième lieu de la psyché individuelle.
Sur ce plan, la territorialité s'identifie pour partie à un rapport a priori, émotionnel et présocial de l'homme à la terre. Il participe en troisième lieu de l'ordre des représentations collectives, sociales et culturelles. Elles lui confèrent tout son sens et se régénèrent, en retour, au contact de l'univers symbolique dont il fournit l'assise référentielle}.

\section{La diversité sexuelle en géographie}
\label{sec:la_diversit_sexuelle_en_g_ographie}
\todo{à garder ou pas?}

Les travaux arrimant l'ensemble de ces courants théoriques, l'étude des géosymboles avec un accent fort sur la sémiotique en relation avec des groupes dont l'identité se superpose à des ensembles culturels plus grands ne sont pas nombreux.
C'est encore plus vrai en ce qui concerne le cas plus spécifique des identités \lgbt{}.
Ceux-ci existent tout de même et il nous apparait important de nommer brièvement ceux qui existent dans le but précis de définir les angles morts où la recherche a lieu d'être.

L'ouvrage principal à rendre explicite l'usage des symboles par les communautés \lgbt{} sur lequel nous nous appuyons est \citetitle{Giraud2014} de \citet{Giraud2014} qui s'est plus particulièrement intéressé aux cas du Village gai de Montréal et du Marais de Paris selon une démarche comparative.
Plus près de notre champ d'intérêt, une section du livre s'est penchée sur les symboles utilisés par les communautés gaies des deux villes.
On y apprend comment cette visibilité a contribué à une certaine territorialisation des groupes gais.
Choisissant une perspective historique et centrée sur un seul groupe, les hommes homosexuels, la recherche reste muette sur les autres groupes \lgbt{} sachant que ceux-ci interagissaient avec ces hommes et partageaient certains espaces et une histoire interreliée~\citep{Remiggi2000,Demczuk1998,Podmore2001,Higgins1997,Higgins1999}.

Nous nous appuierons donc dans cette recherche sur une variété de travaux pour couvrir plus largement les communautés \lgbt{} québécoises.
Nous avons pensé notamment aux travaux de Julie Podmore, de Frank Remiggi et de Ross Higgins qui figurent parmi les auteurs principaux à traiter de la diversité sexuelle au Québec en géographie ou en anthropologie.
\todo{à compléter}

\section{Vers une vision hétérogène de l'identité, de l'espace et de l'essence du symbole}
\label{sec:vers_une_vision_h_t_rog_ne_de_l_identit_de_l_espace_et_de_l_essence_du_symbole}
\todo{à revoir}

\foreignblockquote{english}[
{\cite[tel que cité dans][97]{Oswin2008}}
][.]{As Elspeth Probyn has stated, sexual spaces \foreigntextquote{english}[{\citeyear[10]{Probyn1996}}][.]{are delineated through coincidence and not through exclusion}. Rather than clinging to the fiction that we can locate queer spaces that exist in coherent opposition to heterosexual spaces, we need to intensify examinations of what comes together in processes of sexualization}

%%% Local Variables:
%%% mode: latex
%%% TeX-master: "../../memoire-maitrise"
%%% End:
