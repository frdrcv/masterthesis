%!TEX root = ../../memoire-maitrise.tex
\chapter{Éléments conceptuels et problématique}
\label{cha:elements_conceptuels_et_problematique}

\chapterprecishere{\textquote{J’ai abandonné depuis longtemps l’idée qu’une
    vérité immanente se trouve dans la sexualité, qu’elle soit marquée par le
    péché ou l’émancipation. J’ai aussi abandonné l’idée qu’il existe quelque
    chose qu’on appelle \enquote{la sexualité}. Il existe plutôt des sexualités
    multiples, des sexualités dominantes et des sexualités
    marginalisées.} \par\raggedleft--- \textup{Anne Archet}, Sexe et liberté}

Ce chapitre servira à approfondir d'une part les concepts que nous avons utilisés dans cette recherche et à faire un tour d'horizon de l'état de la littérature dans le domaine de la géographie sexuelle pour développer la problématique.

\section{Revue de la littérature et principaux concepts}
\label{sec:revue_de_la_litterature_et_principaux_concepts} 

\todo{À utiliser ailleurs peut-être, Caroline recommande de commencer au paragraphe suivant. Également, il faudrait simplifier}
Étant donné la largeur relative de la recherche, nous avons cherché des définitions et des approfondissements aux concepts abordés d'une façon multidisciplinaire, sachant que certains de ces concepts ont été utilisés autant en géographie qu'en anthropologie et en sociologie, principalement ceux tournant autour de l'identité et de la symbolique. 
Afin de garder le focus sur notre sujet de recherche, sois la géosymbolique des espaces et territoires des minorités sexuelles, nous tenterons de contextualiser tout au long de notre texte les liens qu'ont ces concepts avec le sujet et quels sont les limites épistémologiques à lesquelles nous nous frotterons. 

Ce travail de recherche s’appuiera sur différents domaines de la littérature scientifique: d'abord, sur les étude \qus\ et plus particulièrement la géographie \qu\ et sur la sémiotique, en mettant l'accent encore une fois sur les liens avec la géographie. 

\todo{À déplacer?}Plus particulièrement, les études \qus\ sont un champ pluridisciplinaire des sciences sociales. 
Trouvant son origine d'abord dans le corpus de la \anglais{French Theory}, la théorie \qu\ est développée par un ensemble d'auteurs attachés au post-structuralisme dans des disciplines aussi diverses que la littérature, la philosophie ou les sciences sociales. 
Le terme \qu, d'abord utilisé par les mouvements sociaux \lgbt{} lors de la crise du \sida, est une réappropriation d'une insulte dirigée vers les homosexuels~\citep{Laprade2014}.
La théorie \qu\ traite particulièrement des normes sociales comme objet plutôt que sur des identités sexuelles particulières; le concept de négativité, de performativité et l'intersectionnalité sont particulièrement utilisés pour traiter des phénomènes de marginalisation.

\subsection{Le regard anthropologique sur la culture}
\label{subsec:le_regard_anthropologique_sur_la_culture} 
Le premier concept que nous aborderons est celui de la culture. 
Celle-ci étant polysémique selon le contexte et l'usage, nous voulons décrire rapidement ces différents usages pour arriver à comprendre comment cell-ci s'articule dans le domaine de la géographie sexuelle mais également dans un contexte de regard sur soi au sein des minorités sexuelles. 
En effet, nous le verrons plus loin, plusieurs positions sont débattues dans la communauté \lgbt{}, une s'inscrivant dans une identité forte et une autre dans une position négative et anti-normative.

% Dans le cadre de ce travail de recherche, nous nous inscrivons dans une 
% définition sémiotique de la culture inspirée des travaux structuralistes et
% post-structuralistes en anthropologie. Pour d'abord arriver à bien comprendre
% cette définition de la culture, nous allons nous pencher plus particulièrement
% sur le texte \textquote{La religion comme système culturel} de \citet{Geertz1972}.
% Malgré que le sujet principal du texte de Geertz est la religion et consiste en
% son analyse, nous pouvons y voir ici une description avancée d'un système
% culturel particulier dans lequel nous retrouvons les bases pouvant servir à la
% description d'autres systèmes culturels, comme celui de la communauté \lgbt{}.

Dans \citetitle{Geertz1972}, Geertz s'inscrit dans une critique des travaux précédents en anthropologie religieuse, laquelle comme sous-discipline de l'anthropologie serait stagnante au niveau théorique en citant continuellement les mêmes auteurs dont on remarque autant l'inspiration sociologique que anthropologique (en se basant autant sur les travaux de Durkheim que sur ceux de Malinowski par exemple~\citep[20]{Geertz1972}). 
Ce texte vise donc à proposer de nouvelles bases théoriques sur lesquelles l'analyse anthropologique pourrait s'approfondir et continuer à évoluer. 
Geertz précède donc son analyse des systèmes religieux\todo{À étoffert, que sont-ils?} par une définition renouvelée de ce qu'est la culture ou plutôt le système culturel. 
Pour lui: \blockquote[{\cite[21]{Geertz1972}}][.]{\textelp{} il désigne
  un modèle de significations incarnées dans des symboles qui sont transmis à
  travers l'histoire, un système de conceptions héritées qui s'expriment
  symboliquement, et au moyen desquelles les hommes [\latin{sic}] communiquent,
  perpétuent et développent leur connaissance de la vie et leurs attitudes
  devant elle}.

Si la culture apparaît bel et bien comme un système abstrait persistant dans le temps au-delà de la vie des individus composant la société, la définition offerte par Geertz ne laisse pas sous-entendre que nous avons affaire à un entité réifiée ou superorganique pour reprendre les termes de \citet{Duncan1980}. 
Les individus composant la société héritent donc des connaissances offertes par la culture pour arriver à comprendre le monde où les symboles qui s'y trouvent portent des significations particulières.

Les symboles plus particulièrement sont pour Geertz dans son modèle\todo{à étoffer}, en reprenant les termes de~\cite{Langer1962}: \textquote{\textelp{} tout objet, acte,
  événement, propriété ou relation qui sert de véhicule à un concept --- le
  concept est la \textquote{signification du symbole }~\citep[
  23--24]{Geertz1972}}. 
Il est important de souligner que les objets en soi que l'on pourrait assimiler à des symboles demeurent ce qu'ils sont matériellement; Geertz prend l'exemple d'une maison qui, si celle-ci peut être un objet concret sans significations autres que sa matérialité, peut également jouer ou non le rôle d'un symbole particulier selon le regard qu'on lui pose en tant qu'être humain appartenant à une culture particulière. 
Autrement dit, au-delà de sa matérialité, sa forme, sa position ou sa composition la maison peut être le témoignage d'un fait culturel particulier. 
On pourrait extrapoler en considérant que cette maison informe le public sur le statut social de la personne. 
\todo{à   reformuler:} Elle hériterait donc dans sa forme d'une forme architecturale propre à la culture dans laquelle elle s'inscrit.

Cette manière de donner forme aux choses matérielles ou abstraites, de leur donner une signification sous la forme de symbole est selon Geertz le fait des programmes fournis par les modèles culturels~\citep[25]{Geertz1972}. 
Ces modèles agissent en deux temps: d'abord, ils créent les symboles en se basant sur le réel, en prenant assise sur les structures non symboliques déjà existantes, ce que Geertz nomment des \emph{modèles de}. 
L'autre forme de modèle, les \emph{modèles pour} agissent plutôt en orientant les structures non symboliques et en créant des liens entre elles qui n'existent pas nécessairement au préalable~\citep[26--27]{Geertz1972}. 
Ces deux phénomènes sont en fait les deux manières qu'ont les modèles de donner sens aux symboles qu'ils contiennent; ils\todo{à revérifier} orientent la compréhension des symboles en calquant ceux-ci sur le réel non-symbolique et en liant ensemble les éléments composant ce réel.

Ces modèles culturels \todo{à préciser, réitérer on ne sait pas de quoi il s'agit}tels que décrits par Geertz s'inscrivent dans une contexte culturel réputé homogène. 
En effet, les différentes analyses faites par l'auteur traitent des religions todo{expliquer d'avantage} (l'objet d'analyse) de façon singulière ou encore sur la place de la religion dans la société de façon générale, sans traiter d'un contexte en particulier. 
Il s'agit d'une des limites à prendre en compte dans la suite du présent texte; en effet, Geertz n'entreprend pas dans cette recherche de traiter des effets propres aux mélanges culturels qui surviennent par exemple avec l'immigration ou la genèse de nouveaux phénomènes culturels, comme dans le cas qui nous intéresse les minorités \lgbt{}.

On trouve tout de même dans son texte des éléments orientant la relation particulière d'un individu et de la religion qui peut servir d'introduction à la suite de ce travail sur l'identité. 
En effet, Geertz, plus loin dans son texte, traite des dispositions propres à l'individu s'insérant dans un contexte culturel particulier. 
La culture transmets ces dispositions à effectuer certaines activités qui permettent à l'individu de s'identifier à la culture dont il fait partie au-delà des fonctions premières et des motivations derrières la pratique en question. 
Le principe d'identité est simple à comprendre dans le contexte religieux décrit dans le texte: l'individu religieux pratique dans sa vie la prière et d'autres activités religieuses selon une probabilité plus élevé avec l'intensité de son sentiment religieux. 
Il est important de souligner ici qu'on parle principalement de la probabilité d'un acte de survenir: l'acte en soi n'est pas une nécessité. 
En société, par exemple, on s'attend à ce que l'individu réputé religieux agissent selon certaines dispositions propre au mode de vie religieux~\citep[28--30]{Geertz1972}. 
Dans le contexte des identités \lgbt{}, on ne peut s'arrêter seulement au pratiques pour traiter d'une potentielle identité.

\todo{À lier avec Geertz}
Les individus \lgbt{}, à leur naissance, s'insèrent de facto dans un contexte culturel dont les symboles n'ont que peu a voir avec les identités sexuelles non-hétérosexuelles. 
En effet, dans le contexte occidental, les individus sont considérés comme hétérosexuels par défaut, à un point tel qu'il ne s'agit pas, en général, d'une identité particulière mais d'une norme, cette dernière étant escamotée\todo{revoir le terme?} par des identités nationales, régionales, \emph{ethniques}, etc.
Nous ne répondrons par immédiatement à ce problème; nous allons plutôt maintenant nous pencher sur l'identité comme concept et comment celui-ci a été manié et pensé par la théorie queer, pour ensuite revenir sur la place de l'identité chez les groupes \lgbt{}.
\todo{vérifier l'enchaînement des parties}

\subsection{Le sujet et identité}
\todo{Revoir le titre de la section}
\label{subsec:sujet_et_identité} Durant les dernières décennies en occident et ailleurs dans le monde, de nombreux groupes identitaires semblent avoir fait surface, surtout dans les milieux urbains. 
On peut penser notamment aux groupes ethniques nés de l'immigration, aux groupes d'intérêts envers des objets culturels particulier (genres musicaux, dessins animés, cinéma) et également aux groupes nés de l'identification à des pratiques sexuelles différentes où une non-coïncidence du genre de la personne avec celui fixé à la naissance. 
Le but de ce mémoire est de traiter des concepts de culture et d'identité en lien avec les individus appartenant au spectre \lgbt{} et d'aborder les enjeux particulier que pose l'étude géographique de cette partie de la population, en considérant le contexte historique récent de rassemblement politique et d'organisation communautaire auquel plusieurs des individus du spectre \lgbt{} ont participé au cours des dernières décennies, des émeutes de Stonewall à la crise du \sida\ des années 80 jusqu'à aujourd'hui\todo{Si on ne traite pas de Stonewall plus loin dans le texte, décrire l'événement ici}. 
En effet, nous tenterons de comprendre si la communauté \lgbt{} forme une culture en soi au sein d'un groupe culturel plus large et si l'orientation sexuelle peut être considérée comme une forme particulière d'identité au sein de ce groupe culturel particulier. 
% Étant donné
% le contexte dans lequel s'inscrit ce travail, nous tenterons, au su des
% conclusions des analyses précédentes, de tenter de traiter des enjeux
% géographiques entourant l'analyse de ce groupe culturel, ou des individus
% rattachés à ces ou cette identité(s).

Nous débuterons notre démarche par l'étude du concept de culture d'un point de vue symbolique en se basant sur les travaux récents de Clifford Geertz pour ensuite introduire le concept d'identité par les travaux de Stuart Hall\todo{revoir la formulation, vu qu'on traite de Geertz précédemment}. 
Par les deux définitions qui se dégageront du survol de ces textes, nous pourrons d'un côté comprendre l'évolution du concept d'identité d'un point de vue généalogique tout en ayant une définition de la culture et voir comment le premier concept s'articule avec le second. 
Nous nous pencherons ensuite sur les enjeux entourant les populations \lgbt{} du point de vue de la définition: avons-nous affaire à un groupe culturel particulier ou plutôt une série d'individus ne possédant en commun que des pratiques sexuelles similaires et une volonté relative d'intégration sociale commune? 
Enfin, nous tenterons de nous pencher sur les enjeux géographiques particuliers à prendre en compte pour l'analyse de la population \lgbt{}, autant d'un point de vue de la pratique sexuelle que sous la forme d'un groupe culturel et comment ces différents points de vue se sont manifestés au sein de ce champs disciplinaire.

% Mon projet de recherche dans le cadre de mon mémoire porte plus
% particulièrement sur une analyse des symboles \lgbt{} ou queers qu'on retrouve
% dans les espaces urbains. Ces symboles devraient permettre d'identifier les
% différents espaces queers dans les territoires connus pour abriter une
% population non-hétérosexuelle importante – comme le Village gai à Montréal –
% mais également ceux qu'on retrouve ailleurs ou temporairement et dont
% l'existence est peu connue en-dehors des groupes \lgbt{}.

\subsection{Le regard socio-historique sur l'identité}
\label{sec:le_regard_sociohistoirique_sur_l_identit_} Maintenant que nous avons une définition du concept de culture, nous allons traiter du concept d'identité tel que défini par~\citet{Hall1996a} dans le texte \citetitle{Hall1996a}. 
Hall propose une généalogie simplifiée du concept d'identité en lien avec les différentes étapes socio-économiques des derniers siècles et donc des courants de pensées y étant liés\todo{trop vague}. 
Cette généalogie permet de rendre compte de l'évolution de l'identité individuelle, en montrant que celle-ci est passée selon trois stades différents.

Le premier type identitaire selon \citeauthor{Hall1996a} est le sujet des lumières qui apparait à partir du \siecle{16}. 
Sans trop entrer dans détails de sa production, Hall souligne qu'il s'agit du sujet né des idées des lumières. 
On a affaire à un sujet dont la capacité principale est d'être et de penser en reprenant les idées de Rousseau dans ce cas-ci. 
Rattaché à la nation, son identité est peu développée et les conflits de classes ne sont pas encore clairement présents: on a affaire à un individu dont les caractéristiques, comme le statut social, sont conçues comme immuables et banales comparativement aux autres membres de la société~\citeyearpar[596]{Hall1996a}.

Le sujet sociologique, deuxième évolution du concept d'identité, nait en même temps que la sociologie devient une discipline autonome séparée du domaine de l'économie, des sciences politiques mais surtout de la psychologie. 
En effet, selon Hall, celui-ci apparait lorsque la psychologie et la psychanalyse prennent de l'importance et concentrent leur analyse sur l'individu dans son plus fort intérieur et ses relations particulières avec son environnement immédiat lors de la constitution de sa personnalité, à savoir les membres de sa famille et surtout ses parents. 
Le domaine du social devient l'apanage de la sociologie et l'identité agit ici comme l'intermédiaire entre cet individu réputé particulier par sa psychologie personnelle en relation avec le monde social, considéré non pas comme une foule d'individu mais plutôt un ensemble de structures économiques et sociales ayant une influence cruciale sur la place de l'individu dans la société. 
Ce sujet sociologique est donc en relation avec les autre individus ainsi qu'avec \textquote[{\citeyear[597]{Hall1996a}}][]{les valeurs, les
  significations et les symboles --- la culture --- du monde qu'il ou elle
  habite}. 
Ce sujet apparait également avec la constitution de l'individualisme. 
Il possède également des caractéristiques particulières très peu développées, alors que son essence est similaire à celle de ses concitoyens au sein de la nation et des individus appartenant à la même classe sociale. 
En somme, les systèmes symboliques prennent une importance qui dépasse les individus malgré qu'on commence à voir apparaître une complexification des identités possibles suite au bouleversement des institutions réputées immuables par le renversement bourgeois des traditions et des systèmes de royauté. \todo{Dater!}

Le sujet postmoderne serait le sujet le plus récent et celui cadrerait à notre époque, à partir de la fin du \siecle{20}. 
Selon \citeauthor{Hall1996a}, ce sujet aurait pris la place du sujet sociologique suite à une déstabilisation de l'identité portée par le sujet moderne et donc sa fragmentation au fil des générations. 
Hall recense cinq causes à cette déstabilisation qui se retrouvent dans divers travaux sur la société et l'individu. 
La première de ces causes provient des travaux de la théorie marxiste. 
En effet, dans celle-ci, la faculté d'action (\anglais{agency}) est remise en question par la reconnaissance de structures sociales et économiques ayant un impact particulièrement important sur le devenir de l'individu. 
Cet individu seul n'a d'ailleurs plus d'essence particulière: plutôt, il acquiert de l'extérieur nombreuse de ses caractéristiques particulières, bien souvent par la classe sociale de laquelle il est issu~\citeyearpar[606]{Hall1996a}.

Le deuxième décentrement proviendrait de la décomposition psychologique de l'individu, notamment dans les travaux de Freud et plus tard de Lacan. 
Cette décomposition remet radicalement son l'auteur la position de Rousseau sur la Raison individuelle: l'individu n'est maintenant plus maître de sa vie, mais possède en lui des pulsions aussi variées qu'incomprises et une partie de lui-même, héritée socialement, le dépasse et a une influence très importante sur sa propre vie. 
Les autres autour de lui, la famille d'abord, ont des rôles particulier dans sa vie qui l'amènent maintenant à ne plus se construire lui-même de façon autonome mais plutôt en relation avec les autres, selon des versions symboliques que ceux-ci représentent pour l'adulte à venir~\citeyearpar[ 607--608]{Hall1996a}.

La troisième cassure selon Hall est celle opérée par les travaux de Ferdinand de Saussure sur le langage. 
Selon lui, la langue et les mots ne sont en aucun cas la possession des individus, au contraire. 
Plutôt, les individus s'insèrent dans un système complexe de symboles et utilisent les mots pour transmettre des significations, des messages aux autres sans toutefois être certains de la réception de ceux-ci étant donné l'évolution rapide du sens des mots. 
Le langage dépasse donc chaque personne et est un fait de société qui les précède. 
La langue prend d'ailleurs place dans la psyché de chaque individu et structure bon nombre de ses pensées~\citeyearpar[608--609]{Hall1996a}

Les travaux de \citeauthor{Foucault2004a} sur le pouvoir disciplinaire consistent en la quatrième cassure. 
Principalement dans \citetitle{Foucault2004a}~\citeyearpar{Foucault2004a}, \citeauthor{Foucault2004a} dresse une généalogie des moyens disciplinaires utilisés par les sociétés occidentales pour faire justice en société. 
On y apprend que continuellement, les techniques disciplinaires vont devenir de plus en plus douces et passer d'un pouvoir strictement extérieur à l'individu, incarné dans la royauté d'abord, à un système judiciaire de plus en plus distant dans la société pour prendre en même temps place à l'intérieur de l'individu. 
Par des dispositifs de surveillance toujours plus élaborés, non seulement les criminels seront surveillés, mais également le reste de la société, dans les écoles et les hôpitaux par exemple grâce aux technologies développées dans les centres pénitenciers, les prisons, les cachots. 
Ces institutions, par leur pouvoir grandissant et leur capacité à surveiller, en viennent à acquérir suffisamment de connaissances sur les individus pour devenir des agent normatifs puissants~\citep[608--609]{Hall1996a}.

La cinquième et dernière cassure vient de la place grandissante qu'a joué le féminisme dans le monde occidental. 
Hall considère que ce sont les mouvements politiques et intellectuels qui ont permis cette cassure, en remettant en question les rapports de genres entre les individus d'abord et en ouvrant la porte à la contestation de nombreux groupes marginalisés ou dont les idées politiques remettaient en question le système social et politique au-delà de la lutte des classes. 
Ce mouvement permit la mise en place des identités politiques pour chacun de ces nouveaux mouvements de contestation et permit de nouvelles formes de contestation sociales et politiques~\citeyearpar[610]{Hall1996a}.


Comme on peut le voir, le sujet sociologique et le sujet post-moderne sont
particulièrement similaires sur plusieurs points: plusieurs des facteurs ayant
mis en place le sujet sociologique en sont venu à le déstabiliser avec les
décennies, quoique Hall ne fait pas une généalogie claire et datée de ces
déstabilisations; il faut plutôt s'appuyer sur les généalogies internes à chacun
des travaux ou évènements et leur moment d'apparition dans l'histoire moderne.
On doit plutôt concevoir ceux-ci d'une part comme des facteurs de
déstabilisation et des œuvres les expliquant, qui ont amené la mise en place du
sujet post-moderne. Celui-ci, dépassé par le langage, l'économie ou les rapports
de pouvoirs est de plus en plus individualisé par la société tout en y étant
enchevêtré. En ceci, comparativement au sujet des lumières, il n'est pas
nécessairement raisonnable et son expérience individuelle est beaucoup plus
importante dans sa constitution identitaire. Au-delà de la complexification des
rapports sociaux, cette individualisation a ouvert de nouvelles voies pour
l'individu lui permettant d'arriver à coordonner sa place en société. Pour
\citeauthor{Hall1996a}, ce sujet:
\foreigntextquote{english}[{\citeyear[598]{Hall1996a}}][]{\textelp{} assumes
  different identities at different times, identities which are not unified
  around a coherent \emph{self}. Within us are contradictory identities pulling
  in different directions, so that our identifications are continuously being
  shifted about. If we feel we have a unified identity from birth to death, it
  is only because we construct a comforting story or ``narrative of the self''
  about ourselves}. 
Nous pouvons croire que cette \emph{narration du soi} s'incarne non pas dans une rationalité objective et indépendante comme pour le sujet des lumières, mais plutôt dans une rationalité subjective. 
Cette rationalité subjective répondrait donc à deux défis: d'abord s'adapter à son environnement immédiat, spatial ou temporel, puis assembler en soi un récit qui arrive à surmonter certaines contradictions inhérentes aux multiples identités individuelles. 

Ce que nous appellons la rationalité subjective n'est pas une contruction stable et encore moins indépendante de l'individu.
Au-delà des contradictions internes, les autres individus peuvent également contredire ce récit et identités.
La politique pour Hall jouerait ce rôle:
\foreigntextquote{english}[{\citeyear[610]{Hall1996a}}][]{Since
  identity shifts according to how the subject is addressed or represented,
  identification is not automatic, but can be won or lost. It has become
  politicized. This is sometimes described as a shift from a politics of (class)
  identity to a politics \emph{difference}}. 
Ces changements importants chez l'individu ouvrent la porte à d'autres formes d'identités jusque là improbables.

Nous pouvons conclure cette partie en soulignant que ces chamboulements et les évènements historiques du dernier siècle ont ainsi permis l'éclosion des identités \lgbt{}, notamment par le traitement qu'a subi l'orientation sexuelle par le système de santé, par la psychanalyse et une distance du pouvoir qui permis l'apparition de groupes d'affinités autour de la question sexuelle. 
La reconnaissance de ces nombreuses identités dans la société et de l'intersection de celles-ci dans les individus doit être prise en compte dans une analyse géoculturelle de la société et des groupes culturels.
En effet, l'individu geerztien que nous avons traité plus tôt dans le chapitre ne serait plus dépendant à un seul système de symboles mais bien à plusieurs dont la géographie et la temporalité serait changeante.
L'incompréhension envers des contradictions ne créerait plus \textquote[]{une angoisse très forte dès qu'il sent que ces symboles peuvent ne pas pouvoir répondre à tel ou tel aspect de l'expérience}~\citep[33]{Geertz1972}, mais une adaptation et un changement d'identité.
Ainsi, on arrive d'une certaine façon à dépasser certaines des limites inhérentes à la vision \emph{geertzienne} de la culture dans laquelle l'attachement à une identité comme la religion pour un individu passait principalement par la pratique.
L'identité se construirait plutôt dans l'interaction avec autrui et la médiation de ces identités avec les forces politiques les régissant ou les contredisants.
Néanmoins, l'usage du terme d'identité n'est pas nécessairement aujourd'hui l'apanage des chercheurs et certains contextes, principalement dans le domaine du politique, permettent une auto-réflexivité sur ce statut d'identité.

\subsection{Diversité sexuelle et identité}
\label{sec:diversit_sexuelle_et_identit_} Comme nous l'avons vu dans la section précédente, l'identité de minorité sexuelle peut être considérée comme une création récente propre au contexte post-moderne des dernières décennies. 
Ce constat s'explique notamment par le contexte historique récent dans lequel s'inscrit cette identité ainsi que par la correspondance à certaines caractéristiques proposées par Hall, à savoir qu'il s'agit d'une identité dans laquelle les individus s'inscrivent en parallèle à d'autres formes identitaires, comme la communauté ethnique, nationale ou encore de genre, par exemple.
Nous introduiront pour la suire le travail de~\citet{Sinfield1996}, qui, dans \citetitle{Sinfield1996}, traite des difficultés et contradictions propres à l'usage de l'identité chez les communautés \lgbt{}.
Ce dernier met en rapport l'identité et les phénomènes historiques récents de libération sexuelle en occident.
Ce rapport complexe viendrait du développement d'un regard critique vis-à-vis l'essentialisation des communautés sexuelles au sein de certains des premiers travaux liés à la théorie queer qui tentèrent de dépasser la \emph{pathologisation} de la sexualité comme on la retrouvait dans les travaux en psychanalyse. 
En effet, le texte de Sinfield s'intéresse plus particulièrement aux considération stratégiques et historiques entourant la mise en place d'une identité homosexuelle pour la communauté elle-même et du point de vue des penseurs de l'identité sexuelle, notamment Foucault.

\subsubsection{Identités minoritaires et caractère universel de la sexualité}
\label{sub:minorit_s_et_universel}
Le texte de Sinfield débute par la présentation de deux manières de voir l'homosexualité dans la littérature scientifique et comme identité.
Le premier consisterait en un point de vue de \emph{minoritarisation} ou de marginalisation qui voit l'homosexuel, gay ou lesbienne, comme un groupe d'individus ayant un style de vie particulier, tel un groupe ethnique.
Le second point de vue serait celui de l'universalisation qui voit dans l'homosexualité un comportement potentiel chez tout et un chacun: tous peuvent à un moment ou à un autre commettre un acte homosexuel et il n'y a pas lieu de parler d'identité ou de culture comme on traiterait d'un groupe ethnique~\citep[271]{Sinfield1996}.

Le point de vue \emph{minoritarisant} va à l'encontre du point de vue constructiviste répandu dans les études sur le queer et l'identité sexuelle, inspirées des travaux \citet{Foucault2011}, de \citet{Rubin2010} et de \citet{Butler2007} en considérant les individus à la sexualité déviante, les homosexuel-e-s dans ce cas-ci, comme des groupes identitaires particuliers. 
Au contraire, le point de vue d'universalisation est en congruence avec le champs de pensée constructiviste; en effet, on voit la sexualité comme une donnée variable chez les individus dont la position sociale, l'éducation et l'environnement auront un effet prépondérant et dont le sens prendra une valeur différente selon la culture traitée. 
Contrairement au point de vue \emph{minoritarisant}, le point de vue universalisant voit l'homosexualité comme une attitude, une pratique sexuelle possible pour chaque individu, peu importe la culture. 
Cette dernière déterminera si la pratique homosexuelle est tolérée, encouragée ou discriminée et marginalisée~\citep[271]{Sinfield1996}.

Selon Sinfield, les gais et les lesbiennes ont historiquement pris une position stratégique les rapprochant du point de vue \emph{minoritarisant} en empruntant une dynamique de revendication et de lutte sociale similaire à celles des groupes ethniques, notamment des mouvements pour les droits civiques afro-américains~\citep[271]{Sinfield1996}. 
Sinfield nomme cette stratégie le \textquote{cadre de l'ethnicité-et-des-droits} (\anglais{ethnicity-and-rights} dans le texte). 
Le développement de ce cadre, au-delà de la simple imitation des groupes ethniques, s'est fait dans le cadre de l'État de droit où, pour améliorer leur position sociale et réduire la marginalisation, les individus doivent, pour reprendre les termes de \citet{Sinfield1996}, \foreigntextquote{english}[{\citeyear[272]{Sinfield1996}}][]{\textelp{} to   compartmentalize their complex subjectivities in order to \emph{make a claim} (envers le pouvoir)}.
Cette compartimentation de la subjectivité individuelle amène les individus touchés à mettre de l'avant une caractéristique d'eux-mêmes et donc à vivre une rapprochement avec les autres individus touchés qui se reconnaissent dans cette identité potentielle. 
Ce cadre stratégique ne laisse pas entendre qu'il n'existait pas de groupes d'individus gays et lesbiens avant que ceux-ci revendiquent des droits selon Sinfield, mais bien que ces revendications ont amené ces groupes à se voir: \textquote{\textelp{} as gay in   the terms of a discourse of ethnicity-and-rights} ~\citep[272]{Sinfield1996} et que ces regroupement par affinités se sont mutés en groupe identitaires avec un poids politique. 
Sinfield souligne plusieurs problèmes dans la poursuite de ce cadre; d'abord cette nouvelle identité et cette genèse culturelle peut entrer en contradiction avec les autres identités assumées par les individus y prenant part. 
Elle désengage également le reste de la société de poser une réflexion profonde sur la sexualité comme le propose la pensée \emph{universalisante} qui fut assumée en partie par certains groupes plus radicaux (dont le mouvement queer, qui par définition vise une refonte radicale des normes sur la sexualité et le genre plutôt que l'acquisition de droits)~\citep[273]{Sinfield1996}.

Plus loin dans son texte, Sinfield explique les différences culturelles dans lesquelles ont eu lieu le développement de mouvements de contestations pour l'amélioration des conditions de vie des gays et lesbiennes, plus particulièrement entre les États-Unis et la Grande-Bretagne. 
Pour l'auteur, le cadre de l'ethnicité-et-des-droits se traduit de différentes manières selon la région étudiée: en Grande-Bretagne, la concession de droit s'inscrit dans la suite de l'État-providence, où l'état anglais concède des acquis supplémentaires dans la perspective d'assurer à tous les citoyens un mode de vie décent. 
Aux États-Unis, on s'inspire plutôt des valeurs traditionnelles américaines qui s'orientent surtout vers la liberté aux individus~\citep[274]{Sinfield1996}. 
En cherchant à obtenir cette liberté offerte par la société américaine, les groupes ethniques s'appuient du même coup sur ce que Sinfield nomme le mythe de la pluralité américaine, qui laisse entendre que chaque groupe culturel est égal et peut revendiquer un accès égal aux mêmes ressources que les autres dans un cadre compétitif. 
C'est ce mode stratégique qui se serait par la suite répandu dans les autres mouvements de contestations et qui aurait stratégiquement mis à l'avant-plan le modèle de l'ethnicité-et-des-droits.

\subsubsection{Diaspora et hybridité}
\label{sub:diaspora_et_hybridit_} Pour faire face au problème de la multiplicité des origines des individus appartenant à l'identité homosexuelle ou lesbienne, Sinfield propose de concevoir cette identité comme nécessairement hybride. 
Cette hybridité est conceptuellement analogue à l'hybridité qui s'impose aux individus appartenant à une diaspora, donc qui se retrouvent géographiquement à distance du lieux d'origine de leur culture d'appartenance. 
À titre d'exemple, on peut notamment penser à la diaspora juive ou encore la population afro-américaine.
Cette hybridité est pour l'auteur un moyen qu'ont ces individus de résister d'une part à l'assujettissement de leur identité avec leur milieu d'accueil et d'autre part conserver une partie de leur culture: \foreignblockquote{english}[{\cite[278]{Sinfield1996}}][.]{`Diaspora' \textelp{}   usually invokes a true point of origin, and an authentic line --- hereditary   and/or historical --- back to that. However, diasporic Black culture, Hall   says, is defined `not by essence or purity, but by the recognition of a   necessary heterogeneity and diversity; by a conception of ``identity'' which   lives with and through, not despite, difference; by hybridity'}

Cette hybridité peut donc être conçue comme participant à une forme d'ethnogenèse tout en possédant un potentiel politique: au lieu de répondre à certains archétypes que la société d'accueil impose sur l'identité des groupes culturels provenant d'une diaspora, ceux-ci peuvent participer à la conception de leur identité en reprenant certains traits culturels:
\foreignblockquote{english}[{\cite[277]{Sinfield1996}}][.]{Stuart Hall traces
  two phases in self-awareness among British Black people. In Do the first,
  `Black' is the organizing principle: instead of colluding with hegemonic
  versions of themselves, Blacks seek to make their own images, to represent
  themselves. In the second phase (which Hall says does not displace the first)
  it is recognized that representation is formative --- active, constitutive ---
  rather than mimetic}.
Néanmoins, dans le cas de la culture afro-américaine, nous nous retrouvons dans un contexte où ce concept de culture est en compétition avec celui de la race selon l'auteur, où une certaine \emph{essentialisation} par le racisme maintient cette version hégémonique d'eux-mêmes.

Pour comprendre qu'il existerait une culture née par l'hybridité chez les individus gais et lesbiennes, l'auteur considère que l'on doit se baser sur l'histoire des individus plutôt que s'attarder seulement à l'Histoire au sens large des sociétés. 
En effet:
\foreignblockquote{english}[{\cite[280]{Sinfield1996}}][.]{\textelp{} for lesbians
  and gay men the diasporic sense of separation and loss, so far from affording
  a principle of coherence for our subcultures, may actually attach to aspects
  of the (heterosexual) culture of our childhood, where we are no longer `at
  home'. Instead of dispersing, we assemble.

  The hybridity of our subcultures derives not from the loss of even a mythical
  unity, but from the difficulty we experience in envisioning ourselves beyond
  the framework of normative heterosexism --- the \emph{straightgeist} \textelp{}}
Dans ce contexte, on peut dénoter que l'auteur souligne une des particularités de la culture dominante: son caractère essentiellement hétérosexuel au niveau des normes, ou hétéronormatif (voir partie~\ref{sec:enjeux_g_ographiques_du_recours_l_identit_}). 
Le départ de la culture hétérosexuelle ou \anglais{straightgeist} à laquelle tous les individus de la culture homosexuelle doivent répondre est partielle; à tout moment, les gays et lesbiennes pour ne nommer que ceux-ci doivent composer avec le reste de la culture hétérosexuelle dans les autres sphères de leur vie, que ce soit à l'école, au travail ou dans l'espace public. 
C'est en raison de cette négociation inévitable avec la culture dominante que pourrait se justifier le caractère hybride de la culture homosexuelle. 
Les objets culturels, les pratiques culturelles et sociales s'inscrivent dans cette culture hybride et peuvent donc ou non être comprises par la culture dominante.

Pour conclure cette partie, notons que le trait commun partagé par les gays et lesbiennes dans le cadre de l'analyse par l'ethnicité-et-les-droits est l'altérité vécue par les individus non-hétérosexuels:
\foreignblockquote{english}[{\cite[289]{Sinfield1996}}][.]{Our apparent unity is
  founded in the shared condition of being not-heterosexual --- compare `people
  of colour', whose collocation derives from being not-white}. 
Étant donné cette emphase sur la non-correspondance à une norme, la communauté gaie et lesbienne, et on pourrait rajouter bisexuelle et trans-, est nécessairement très large et diverse. 
Sinfield hésite donc à parler ici d'une culture en soi; on propose plutôt l'usage du concept de sous-culture qui rendrait mieux ce caractère de diversité et reconnaître le côté construit et récente de celle-ci:
\foreignblockquote{english}[{\cite[289]{Sinfield1996}}][.]{It is to protect my
  argument from the disadvantages of the ethnicity model that I have been
  insisting on `subculture', as opposed to `identity' or `community': I envisage
  it as retaining a strong sense of diversity, of provisionality, of
  constructedness}.

% , Hall \& Gay introduisent le concept d'identité pour traiter des groupes
% sociaux et culturels qui s'opposerait à l'ancien sujet moderne. À partir de ce
% concept, il devient a priori possible de traiter de groupes ou communautés comme
% les homosexuels, bisexuels, trans- et queers dans le contexte culturel précisé
% précédemment.


\subsubsection{Enjeux géographiques du recours à l'identité}
\label{sec:enjeux_g_ographiques_du_recours_l_identit_} 
%Les textes suivant
% permettraient de situer l'usage de la culture dans un contexte précis. Par
% exemple, \textquote{The Location of Culture: The Urban Culturalist Perspective}
% propose l'étude culturelle des phénomènes urbains, alors que les études urbaines
% utilisent normalement des méthodes quantitatives pour traiter des mêmes
% questions (Borer, 2006). Les parties précédentes se sont principalement
% intéressées à l'analyse générale des concepts d'identité, de culture et
% d'identité sexuelle en demeurant essentiellement dans un contexte sociologique.
% Pour la poursuite de ce texte, nous nous intéresserons plus particulièrement au
% domaine spatial de ces concepts en tentant d'apposer une regard géographique sur
% le culture. %, plus particulièrement par un regard sur la ville comme espace
% culturel. Celle-ci, en plus d'être un des milieux les plus populeux qu'on
% retrouve dans plusieurs des sociétés humaines, sinon la totalité de nos jours,
% permet de comprendre les enjeux entourant la mixité sociale et culturelle

%Pour cette partie, nous pencherons sur deux textes de Michael Borer, à savoir \textquote{The Location of Culture: The Urban Culturalist Perspective} (2006) et \textquote{From Collective Memory to Collective Imagination} (2010).~\citep{Borer2006}
%Le deuxième texte de Borer, \textquote{From Collective Memory to Collective
%Imagination >> propose l'analyse spatiotemporelle des phénomènes culturels en
%milieu urbain, un point de vue méthodologique qui rejoint celui de Larry Knopp.~\citep{Borer2010}

En géographie \qu{} on retrouve les deux paradigmes soulevés à la partie~\ref{sub:minorit_s_et_universel}, à savoir un partage entre une analyse autour de la diversité comme construction sociale et une autre centrée sur l'identité gaie, lesbiennes, bisexuelle ou trans-. 
Plus particulièrement, la pensée géographique peut se pencher sur les espaces occupés par les gays et lesbiennes, ou plutôt s'intéresser au caractère normatif des espaces. 
Les premiers travaux en géographie sexuelle se sont principalement attardés au premier point de vue alors que récemment, durant les vingts dernières années, on remet en question le point de vue \emph{minoritarisant} qu'on retrouve toujours dans certains textes comme celui de~\citet{Sinfield1996} pour plutôt se pencher sur les normes sociales et leurs rapports avec l'espace. 
C'est ce dernier point de vue que défend et explique Natalie~\citet{Oswin2008} dans l'article \citetitle{Oswin2008} que nous traiterons dans la suite de ce texte.

Dans ce texte, Oswin s'oppose à l'idée que les espaces queers puissent être des lieux en opposition totale avec les normes de la société dominante. 
Plutôt, la recherche récente en géographie sexuelle a permis de rendre compte que c'est par la présence d'individus dont les actes ou la manière de performer le genre ou l'identité sexuelle ne correspond pas aux normes sexuelles dominantes que les espaces de la société apparaissent comme hétéronormatifs par le jeu de pouvoir qui s'y établit. 
Ce jeu de pouvoir s'établirait de plusieurs manières, par exemple par la marginalisation (violence homophobe, exclusion), par une présence accrue du pouvoir policier à proximité des espaces queers ou pas le refus des instances gouvernementales de répondre aux demandes des populations \lgbt{} (durant la crise du \sida, par exemple).

Ce jeu de pouvoir sur les normes sociales se manifeste particulièrement dans les espaces réputés occupés par des individus appartenant au spectre \lgbt{} où plusieurs visions de l'homosexualité se confrontent entre elles. 
En effet, comparativement à l'idée d'une culture homosexuelle uniforme et partagée par certains membres de la communautés gaie et lesbienne, les auteurs en études queers et en géographie queer ont plutôt montré que plusieurs groupes luttent selon deux types d'enjeux.
D'abord, soit pour un élargissement des normes vers une plus grande inclusion sociale, le point de vue \emph{assimilationniste}, sois par celui des \emph{libérationnistes} qui souhaitent plutôt remettre en question certaines normes qu'ils considèrent empruntées à la culture hétérosexuelle et répliquées à l'intérieur même des espaces queers, un phénomène couvert par le concept d'homonormativité. 
Oswin définie cette dernière à partir d'une citation de Lisa Duggan, où l'homonormativité est: 
\foreignblockquote{english}[{\cite[tel que cité
  dans][92]{Oswin2008}}][]{\foreigntextquote{english}[{\cite[50]{Duggan2003}}][]{A
    politics that does not contest dominant heteronormative assumptions and
    institutions, but upholds and sustains them, while promising the
    possibility of a demobilized gay constituency and a privatized,
    depoliticized gay culture anchored in domesticity and consumption}}.

Un autre point important du texte d'Oswin est la réinterprétation du sens des multiples identités que peut posséder un individu du spectre \lgbt{}. 
Au lieu de se baser sur l'hybridité, ces identités sont plutôt conçues comme des sources d'oppression, au niveau de la racisation, de la classe sociale ou du genre par exemple. 
Dans de nombreux espaces queers, il a été remarqué que souvent le pouvoir était détenu par des individus dits privilégiés sur d'autres bases identitaires que la simple orientation sexuelle. 
Souvent également, au sein même des communautés gaies, d'autres forme d'identité sexuelle ou de genre sont mises de côté, comme la bisexualité ou les individus trans-~\citep[93]{Oswin2008}.
Il est donc pour l'auteur important de reconnaître le potentiel qu'ont les chercheurs de réifier les communautés des milieux qu'ils étudient, notamment en recréant certains hiérarchie en omettant de reconnaître les inégalités sociales.

Pour la suite, nous nous intéresserons plus particulièrement aux méthodes offertes par la géographie pour traiter efficacement des questions de diversité sexuelle et d'espace. 
Larry~\citet{Knopp2004}, un des premiers chercheurs à lier les études queers à la géographie culturelle, traite dans \citetitle{Knopp2004} de la théorie de l'Acteur-Réseau qui pourrait permettre méthodologiquement de dépasser certaines limites de l'identité comme concept pour traiter des populations \lgbt{} tout en s’efforçant de fuir certains déterminismes en géographie culturelle~\citep{Knopp2004}. 
En effet, plutôt que s'appuyer sur un point de vue \emph{minoritarisant} des groupes et communautés \lgbt{}, Knopp reconnait d'emblée le potentiel qu'ont ces groupes d'avoir un effet sur les structures sociales de pouvoir. 
Ce texte s'inscrit donc ainsi moins dans l'étude d'un groupe culturel particulier que dans une analyse des conflits vis-à-vis les normes sociales d'un ensemble culturel. 
Knopp utilise dans son texte des termes similaires à ceux de \citet{Sinfield1996}, à savoir que les groupes queers seraient entre autres hybrides au niveau identitaire et que leur présence sociale prendrait la forme d'une diaspora. 
En effet, si les espaces queers sont les espaces vers lesquels se dirigent les membres de ces communautés dans le texte de \citet{Sinfield1996}, Knopp considère plutôt que ce sont les déplacements dans le temps et l'espace qui sont formateurs des identités queers et que la géographie a le potentiel de rendre compte de ces déplacements par les théories \emph{non-représentationnelles}\footnote{dont nous ne traiterons pas ici}\todo{Compléter la note de bas de pages à propos de ces théories non-représentationnelles}.

Ces déplacements ont en effet un sens particulier:
\foreignblockquote{english}[{\cite[123]{Knopp2004}}][.]{For gays, lesbians,
  bisexuals, transgenders, and other queers, as for other oppressed groups, this
  means seeking people, places, relationships, and ways of being that provide
  the physical and emotional security, the wholeness as individuals and as
  collectivities, and the solidarity that are denied us in a heterosexist world}
Plutôt que d'être les héritiers directs d'une culture \qu{}, on doit plutôt concevoir que les individus \qus{} possèdent bel et bien la culture dominante, mais que leur intégration sociale passe par d'autres trajets que ceux proposés normalement par la culture dominante.

Knopp avance même que ces déplacements ont une importance assez forte pour être génératrice d'une ontologie particulière:
\foreignblockquote{english}[{\cite[123]{Knopp2004}}][.]{It is also about
  testing, exploring, and experimenting with alternative ways of \emph{being},
  in contexts that are unencumbered by the expectations of tight-knit family,
  kinship, or community relationships—no matter how accepting these might be
  perceived to be} 
En même temps que les individus \lgbt{} quittent le contexte familial hétérosexuel comme début de parcours et principal lieu d'acquisition de culture, ils transportent avec eux des éléments de cette culture et la transforment au gré de leurs expériences. 
Pour Knopp, l'expérience queer en soi provoque la constitution de nouvelles données culturelles, spatiales et participe donc à cette hybridité de l'identité queer.
\foreignblockquote{english}[{\cite[130]{Knopp2004}}][.]{As queer bodies and
  subjectivities circulate through (and constitute) time and space, they leave
  legacies, absorb others, and mutate. They spread information, values, and
  culture, and constitute barriers to such spreads at the same time. This is
  diffusion par \emph{excellence}}

Si ces caractéristiques sont particulièrement importantes, Knopp souligne tout de même que ces processus formateurs au niveau identitaire ne sont pas l'apanage des individus \lgbt{} mais peuvent être considérés comme des expérience probable pour tous les individus; elles ne sont pas essentielles à l'expérience \lgbt{}.
Néanmoins, le contexte normatif entourant la sexualité pousse tout de même les populations \lgbt{} à ces parcours de façon plus particulière: les chemins de vie empruntés ressembleraient à des migrations vers l'acceptation sociale de l'identité sexuelle, qu'elle passe par l'anonymat ou pas l'inclusion au sein d'un espace \lgbt{}. 
Knopp appuie cette caractéristique par plusieurs travaux en géographie queer qui reconnaissent l'importance du parcours et du déplacement spatio-temporel chez les individus \lgbt{}~\citep[123]{Knopp2004}.


% j'envisage dans mon travail d'utiliser le texte \textquote{Critical geographies and the
% uses of sexuality: deconstructing queer space}~\citep{Oswin2008} qui apporte
% certaines critiques sur l'usage de concepts géographiques que l'on retrouve dans
% les travaux portant sur le queer. Celui-ci devrait permettre d'éviter certains
% raccourcis qui pourraient subvenir en tentant d'utiliser les textes précédents
% dans un contexte ethnographique en gardant certaines avancées propres au point
% de vue queer, comme la fluidité des identités, leurs superpositions, etc. 

\subsubsection{Synthèse}
\label{sec:synth_se} 
% Cette dernière partie tentera de proposer une synthèse des parties précédentes en lien avec la culture ou les identités de la diversité sexuelle. 
Au cours de ce chapitre, nous avons d'abord traité de la question de la culture d'un point de vue anthropologique, à partir de Clifford Geertz\todo{À bonifier!}.
Si une culture est bel et bien un système de pratiques et de conceptions héritées, on pourrait croire qu'il existe une culture \lgbt{} étant donné les pratiques particulières des communautés gaies et lesbiennes, notamment par l'existence de milieux de vie particuliers, de pratiques sociales et sexuelles différentes des hétérosexuels et d'une histoire particulière à cette communauté (qui demeure tout de même imbriquée dans celle de la société majoritaire). 
Le contexte social, politique et intellectuel du dernier siècle semble avoir été propice à l'éruption des communautés gaies et lesbiennes, comme nous l'apprend la partie sur le texte de Hall. 
En effet, si, des lumières jusqu'à la modernité, il était difficile d'imaginer l'existence d'identités basées sur la sexualité, différents décentrements et déstabilisations du sujet moderne ont amené cette possibilité par une remise en question de l'uniformité relative des individus au sein d'une même culture. 
Est-ce toutefois suffisant pour dire qu'il existerait aujourd'hui des cultures basées sur des identités? 
En ce qui concerne la question de l'orientation sexuelle du moins, la réponse n'est claire et l'on remet même en question l'idée de s'attarder à l'identité dans un contexte où ces individus, les individus de la diversité sexuelle, sont souvent victime de marginalisation dans le contexte culturel occidental. 
Leur présence témoigne des instabilités propres au nouveau sujet post-moderne qui vit dans un contexte culturel aux normes témoignant de jeux de pouvoirs bien présents, autant pour les individus de la diversité sexuelle que pour le reste de la société. 
Au-delà des enjeux stratégique liés à l'étude des individus \lgbt{}, dans une perspective de changement social, on pourrait croire que ces déstabilisations continuent d'avoir lieu et que la critique des normes sociales et des enjeux de pouvoir demeure l'avenue à privilégier, dans un contexte de l'étude de la culture occidentale.

Nous pouvons conclure en soulignant que les liens entre identité et culture ne sont pas encore clairement définis; l'identité comme concept laisse croire qu'il est nécessaire d'appartenir à une culture précise pour se lier à une identité précise. 
Pourtant, malgré l'ambiguïté entourant l'idée de culture ou de sous-culture \lgbt{}, on peut difficilement remettre en question l'idée d'identité, sachant l'existence de groupes sociaux, culturels et d'espaces \lgbt{} dans la société occidentale.

La question du pouvoir reste à creuser plus en profondeur, tel que souligné pas Oswin dans son texte. 
Celle-ci propose également l'étude de l'hétérosexualité, une avenue intéressante pour voir si cette identité sexuelle existe comme celle des autres formes d'orientation sexuelles et peut-être ainsi arriver à voir les effets probables de la normalisation de la sexualité à une plus grande échelle au sein de la culture occidentale~\citep[100]{Oswin2008}. 
Enfin, il serait également pertinent d’entamer une étude auprès des groupes \lgbt{} pour savoir qu'elle est la place de la communauté dans leur identité et s'il existe, selon eux, les éléments nécessaires pour parler d'une culture queer, que ce soit au niveau historique ou au niveau des significations particulières des éléments et pratiques composant les espaces et la temporalité queer.


\subsubsection{La diversité sexuelle dans les sciences sociales et la
géographie}
\label{ssub:la_diversit_sexuelle_dans_les_sciences_sociales_et_la_g_ographie}
Nous travaillerons dans cette étude à partir de l'identité sexuelle et du genre.
\citet{Foucault2011} décrit dans son œuvre \citetitle{Foucault2011} le processus par lequel les relations entre individus de même sexe sont devenues par la médecine notamment, une forme de trouble de la sexualité jusqu'à être aujourd'hui, après bien des luttes sociales, une identité particulière~\citep{Foucault2011}. 
Plusieurs autres auteurs, en études féministes notamment avec Gayle Rubin et Judith Butler~\citep[98]{Marcus2005}, ont permis de montrer que le genre est aussi un construit social ou plutôt une performance qui n'est pas l'essence de l'individu~\citep{Butler2007}. 
On remet ainsi en question l'idée de nature en sexualité ainsi que la possibilité que les comportements dits anormaux, comme l'homosexualité, la bisexualité ou un genre non-binaire, soient en fait des comportements sociaux qui ne concordent tout simplement pas avec les normes sociales.

Ce travail théorique fut plus tardivement repris par la géographie avec une certaine justification: à la même époque que les études sur le genre avaient lieu et sortaient du seul cadre de la psychanalyse~\citep{Rubin2011a,Rubin2011}, les luttes sociales entreprises par les gais et lesbiennes prenaient place dans l'espace public~\citep[422-427]{Spencer2005}. 
Dans les grandes villes des États-Unis d'abord, puis progressivement d'Europe et du Canada, on a assisté à l'apparition des villages gais, lieux d'abord utilisés pour la protection par l'anonymat, puis, avec ces luttes, le partage d'un mode de vie et de lutte politique. 
Relativement invisibles, les espaces urbains habités par les gais et lesbiennes acquièrent une visibilité supplémentaire en même temps que l'on commença à s'intéresser aux liens entre l'espace et la sexualité dans un contexte ou l'identité sexuelle semblait détenir une ontologie particulière.

Pour certains auteurs, ce développement identitaire prend une forme ontologique.
L'ontologie, terme utilisé pour désigner le savoir représentant une vision particulière de l'univers, est un concept dont l'usage serait de plus en plus complexifié par l'apparition récente d'une multitude de nouvelles identités, par la politisation de plusieurs pans de population vivant la marginalisation et par une remise en question de la culture dominante~\citep[122]{Knopp2004}. 
Le \qu\ pourrait désigner l'ontologie propre au spectre de la diversité sexuelle étant apte à dépasser les normes séparant chacune des identités du spectre \lgbt{} et pouvant rendre compte d'une certaine cohésion qui rejoint historiquement les individus de ces communautés~\citep[122]{Knopp2004}.

Larry Knopp propose d'ailleurs la théorie de l'Acteur-Réseau pour parvenir à identifier et analyser toutes les composantes spatiales du \qu\ comme le lieu et le mouvement et recréer une certaine cohésion de la théorie \qu\ face aux divers mouvements qui la façonnent, comme le constructivisme, le matérialisme. 
Elle permettrait également d'échapper à la notion du territoire pour plutôt utiliser une vision plus relativiste de l'espace étant donné que les éléments d'un réseau peuvent croiser d'autres réseaux et être polysémiques. 
Des approches où l'espace peut prendre diverses formes et qualités~\citep{DiMeo1998} seraient beaucoup plus compatibles avec la théorie de l'Acteur-Réseau.

La géographie culturelle s'intéresse donc depuis peu à ces nouveaux groupes sociaux, surtout avec l'avènement des approches postmodernes dont la théorie \qu\ fait partie qui s'intéressent de nouveau à l'individu, en prenant tout de même assise sur les travaux ultérieurs, notamment le structuralisme et la théorie critique. 
Une des premières œuvres à avoir marqué le commencement de la géographie \qu\ est~\citetitle{Bell1994} de~\citet{Bell1994} qui amena une sélection de textes, montrant la pertinence et les possibilités multiples de cette nouvelle branche de la géographie. 
Néanmoins, dans plusieurs études, l'homme gai fut priorisé comme objet où on s'attarda à l'usage de certaines méthodes de géographie, comme le recours au territoire~\citep{Podmore2001,Oswin2008}. 
On travaillait donc à comprendre la progression des villages gais, souvent menée par des hommes, alors que les femmes lesbiennes et autres formes de sexualités moins étudiées furent mises de côté par l'argument que celles-ci sont souvent moins visibles et présentes dans l'espace public, et que l'on devrait s'attarder au ménage et à la maison pour traiter de leurs situations~\citep[333-334]{Podmore2001}. 
Il s'agit, selon Podmore dans l'article \citetitle{Podmore2001} plutôt d'un problème méthodologique alors que les concepts utilisés actuellement en géographie ne suffisent plus pour étudier les nouvelles formes identitaires. 
Le recours au territoire comme concept géographique pour s'intéresser aux communautés sexuelles apparait donc problématique: non seulement on exclut les femmes, mais également d'autres groupes encore moins visibles, comme les transgenres et les bisexuels. 
Il convient donc, en reprenant la position de Natalie Oswin, de traiter d'espaces \qus\ pour éviter une essentialisation de l'autre, de ne pas prioriser certains genres, d'utiliser des dichotomies fautives entre l'homosexuel face au \anglais{straight} et considérer \latin{de facto} un milieu \qu\ comme un espace de résistance de dissidence~\citep{Oswin2008}. 
En effet, comme l'a montré Nathaniel M. \citet{Lewis2011} dans son article \citetitle{Lewis2011}, les hommes gais, même sans espaces précis, héritent des normes de leur environnement sur l'identité sexuelle, mais peuvent également avoir un effet sur ces normes, l'auteur prenant exemple sur les familles homoparentales dans les banlieues qui offrent un regard différent sur la définition d'un ménage de classe moyenne au sein du voisinage~\citep[304]{Lewis2011}.

\subsubsection{La sémiotique}
\label{ssub:la_semiotique} \todo{à revoir} La sémiotique prend ses racines dans les travaux de plusieurs auteurs, principalement Ferdinand de Saussure qui a également donné naissance au courant du structuralisme~\citep{Noth1995}. 
Dans ses travaux sur la religion, Clifford Geertz considère que le géosymbole est On retrouve aujourd'hui toute une variété d'auteurs et de mouvements, autant en musique qu'en théologie, mais dans le cas présent, nous nous intéresserons principalement aux auteurs en géographie. 
Au sein de ceux-ci, on peut nommer d'abord Joël de Bonnemaison qui démontra l'importance des géosymboles comme moyen utilisé par les groupes ethniques pour s'ancrer à l'espace habité qui deviendra le territoire par des itinéraires, des formes et des tracés dans le paysage~\citep{Bonnemaison1981}. 
Mario Bédard va poursuivre la réflexion et inclure tout comportement culturel particulier rattaché à l'espace, et créer une typologie complexe des différents géosymboles que l'on retrouve dans le territoire~\citep{Bedard2002}. 
Jérôme Monnet quant à lui traite plutôt des processus conscients de création de symboles, et des relations particulières qui lient les symboles à l'espace, au pouvoir et à l'identité, particulièrement dans le contexte occidental et capitaliste~\citep{Monnet1998}. 
L'idée de contexte se joue d'ailleurs à plusieurs échelles, et il apparait important selon l'auteur de ne pas limiter l'étude d'un symbole seul, mais considérer son environnement, sachant que la présence d'un symbole d'un type peut amener la propagation de symboles similaires, comme dans le cas d'un centre d'achat qui provoque l'arrivée d'autres commerces, comme symboles marchands~\citep[7-8]{Monnet1998}.
Il apparait donc possible, à l'aide des travaux effectués en sémiotique et en géographie culturelle, d'arriver à décrire l'espace occupé par les groupes et individus appartenant au spectre de la diversité sexuelle en tenant compte de certaines difficultés, comme le recours au territoire (qui peut masquer espaces dispersés ou en réseaux) ou encore à la matérialité des symboles, qui peuvent ne pas être suffisants comme appuis pour décrire une réalité spatiale soit temporaire, soit mobile.

\citet[105--109]{Rose2012} fait une synthèse des différents moyens de faire de la sémiologie une méthode et une analyse de la société. 
On retrouverait deux courants principaux dans le domaine de la recherche en sémiologie qui déborde les champs disciplinaires traditionnels comme l'anthropologie et la géographie.
Le premier courant serait les chercheurs pratiquant la sémiologie dite  
Le deuxième courant, nommé sémiologie sociale (\anglais{social semiotics}), consisterait en une analyse des faits sociaux

Dans le chapitre suivant, nous reviendrons sur la sémiotique mais du point de vue de la méthode. 
Nous décrirons alors comme nous pourrons, à partir d'une déconstruction des images collectées, obtenir plus d'information sur le sens des géosymboles évoqués et ce qu'ils traduisent comme relation territoriales entre les espaces des villes traitées et les sous-groupes du spectre \lgbt{}.

%\paragraph{Signifiant, Signifié et Référent}
Selon \citet[113][]{Rose2012}, le signe est le principal concept que l'on retrouve en sémiotique. 
Il consiste en les différentes informations qui circuleraient dans un système de communication, que ce soit entre deux individus ou un individu et sont environnement. 
Ce concept, dans le modèle de Saussure, est divisé et deux parties pour rendre compte de la relation existant entre un individu dans un système de communication et les objets ou idées à lesquelles il réfère. 
Ce premier élément est le signifié qui consiste en l'objet ou l'idée auquel on octroie un signe. 
Dans un contexte de communication ou non, le signifié en soi existe toujours. 
Bien que certains signifiés dépendent de l'être humain pour exister, comme philosophie ou les mathématiques ou des objets manufacturés, en général ces signifiés ont une existence propre non-dépendante d'un système de communication dans lequel on ferait référence. 
Autrement dit, si c'est l'être humain dans sa communication qui permet une telle caractérisation d'un objet par le signifié, il demeure que ce dernier peut exister en soi. 
Le signifiant quant à lui ne peut exister qu'au sein de ce système de communication. 
En effet, ce signifiant est en quelque sorte l'élément qui permet d'évoquer chez la personne jouant le rôle de communicateur l'idée liée au signifié. 
Le mot \emph{roche} dans un tel système est un signifiant lié à la substance minérale séparée d'une substrat rocheux plus important qui est ici le signifié dans sa forme la plus générale et dont la forme peut différer selon les occasions et les contextes. 
Enfin, ce signifiant n'est pas qu'un mot prononcé ou lu, il peut également consister en tout média dans lequel l'idée d'un signifié est transmise, comme une image, un vidéo, un poème, etc.

%\paragraph{Signe}
Dans l'ensemble cette définition du signe correspond aux théories de Saussure.
Elle présente rapidement des limites: si un signifiant peut prendre plusieurs formes, comment peut-on comparer facilement ceux-ci entre eux? 
Si les individus utilisent divers signifiants pour communiquer, certains demandent un bagage culturel particulier pour être compris alors que d'autres sont plus universels.
Les travaux de Pearsons permettent conceptuellement de dépasser ces limites.
Nous allons enrichir cet usage du signe en concordance avec la théorie développée par Pearson. 
Selon lui, le signe peut prendre plusieurs formes selon média dans lequel on le trouve et le degré d'abstraction de différenciation qui peut exister entre le signifié et le signifiant. 
Ainsi, différents termes ont été développés pour différencier les différentes signes.
Parmi ceux-ci, on retrouve d'abord l'icône, qui consiste en une quasi-concordance entre le signifié et le signifiant. 
Dans un dessin d'enfant, selon les cultures, un arbre est souvent dessiné avec un tronc brun, quelques branches et des feuilles vertes. 
Cette image, sans correspondre à l'immense diversité des arbres, correspond tout de même à plusieurs types d'arbres que l'on retrouve dans un climat tempéré. 
Par contre, ce même arbre pourrait figurer sur un roman ou un livre traitant de la vie, de la longévité ou encore de la généalogie dans un contexte familial: dans ce cas-ci, l'arbre est lié de façon abstraite à ces concepts. 
On a donc affaire à un symbole, un signe dont la relation entre le signifiant et le signifié est définie de façon arbitraire et relative à un contexte culturel particulier. 
Le troisième type de signe est l'index dans lequel le signifiant et le signifié n'ont a priori pas de liens, de façon similaire au symbole, mais qu'en plus, le signifiant n'a pas de sens en soi sans le signifié. 
En effet, dans les cas précédents, surtout dans des cas plus visuels, on ne peut séparer le signifiant du signifié, ou du moins, on peut trouver d'autres types de relations. 
La longévité peut est signifiée par d'autres signifiants comme une personne très âgée, et le signifiant d'arbre rappelle très rapidement le signifié d'arbre sans mise en contexte. 
Dans l'index, la personne en communication doit avoir connaissance du lien entre le signifiant et le signifié. 
On peut penser par exemple aux feux de signalisations : les cercles rouges, jaunes et verts ont été désignés par être les signifiants des différents types de restrictions ou de non restrictions au déplacement en automobile, mais sans connaissance préalable des codes de la route, un individu ne peut comprendre ce lien de façon instinctive. 
Un autre exemple: un mot dans une langue. 
Si l'individu qui communique n'est pas locuteur de cette langue ou ne possède pas les connaissances pour comprendre quoi il s'agit, l'ensemble des caractères ne peut rien évoquer du signifié qu'il désigne.

On le constate rapidement, ce modèle n'a \latin{a priori} rien de très géographique en soi et ne permet pas facilement d'articuler les nuances propres à la culture et à l'identité d'un locuteur au-delà de la connaissance ou de l'ignorance d'un signe prenant la forme d'un index. 
C'est ici que nous allons nous arrêter pour traiter des notions d'espaces et de territoire, avant de faire la synthèse grâce au concept de géosymbole tel qu'articulé par \citet{Bonnemaison1981}.


\subsection{Espaces et territoires}
\label{sec:espaces_et_territoires} Nous poursuivrons avec ces deux concepts les plus géographiques, traités ensemble car notre démarche s'inscrit dans une certaine critique du concept de territoire qui demande une part d'approfondissements, \todo{repasser sur cette phrase}sans pour autant entrer dans une épistémologie trop approfondie car il ne s'agit pas ici du but de cette recherche. 
Ces deux concepts, quoique semblable à première vue, prennent en géographie culturelle des sens différents. 
Le concept d'espace d'abord se veut être un point de vue rationnel sur les questions de distances entre les objets de ce monde et leurs positionnement \todo{à reformuler}. 
Utilisé en mathématiques, en physique, et dans l'usage courant d'un point de vue pratique ou technique, l'espace est par définition une construction intellectuelle volontairement neutre dont les propriétés sont quantifiables \missref{Di Méo?}.
On peut penser aux distances en kilomètres pour un voyageur entre deux villes, le calcul de la taille pour la construction d'une maison, ou encore du volume pour la quantification d'un liquide. 
Mais nous allons le voir, ces calculs d'apparence banales semblent participer aujourd'hui à un point de vue particulier sur le monde qu'on décrira comme désenchanté.

L'utilisation d'un tel concept dans le cadre de la géographie culturelle peut à prime abord paraître contradictoire; en effet, ce champs de la géographie travaille plutôt sur les populations et les territoires, selon leurs pratiques culturelles et leurs spécificités identitaires ou ethniques. 
L'espace prendrait plutôt sa pertinence au sein de cette discipline dans le contexte de la géographie sociale par exemple, lorsque l'on s'intéresse aux enjeux propres aux déplacements, ou encore en géographie physique ou biologique qui traitent bien souvent les substrats physiques ou le monde du vivant dans des ensembles comme les \todo{trouver le mot manquant} comme des données quantifiables. 
La géographie culturelle par contre, durant le dernier siècle, telle que pratiquée par des géographes du nouveau/missref et de l'ancien monde/missref, travaillaient sur des populations des pays colonisés ou en voie de décolonisation en utilisant plutôt le concept de territoire. 
Nous pensons plus particulièrement à la géographie tropicale et son point de vue basé sur l'altérité entre les régions nordiques dites normales et les régions du sud~\citep[493]{Power2009} et également les travaux plus anciens en géographie régionale française s'intéressant strictement aux régions~\citep[31]{Courville1991}. 
Ceux-ci travaillaient bien souvent sur des populations plutôt restreintes pensées comme des ethnies \todo{à retravailler}.
Ces dernières semblaient à première vue vivre dans des milieux suffisamment isolées pour que les interactions inter culturelles doivent être considérées comme quasi-inexistantes et de cette façon, restreindre les caractéristiques de la population comme des faits uniques plutôt que l'objet d'interactions avec d'autres populations destinées à évoluer, avec ou sans la présence des schémas de colonisation~\citep[79--80]{DiMeo2007}.

Dans ce contexte, le territoire est perçu comme un autre plan de la réalité se superposant à l'espace: il s'agirait d'un ensemble d'éléments abstraits et matériels permettant à la population d'ancrer son identité, son histoire comme son futur. 
On retrouve cette définition chez plusieurs auteurs, notamment chez~\citeauthor{Bonnemaison1981} chez qui le territoire:
\blockquote[{\cite[253]{Bonnemaison1981}}][.]{Les sociétés humaines ont une
  conception différente du territoire. Il n'est pas forcément clos, il n'est pas
  toujours un tissu spatial uni, il n'induit pas non plus un comportement
  nécessairement stable}. 
\citeauthor{DiMeo2007} nous offre également une définition du territoire, pour qui:
\blockquote[{\cite[76]{DiMeo2007}}][.]{L’assise territoriale, campée sur un
  réseau de lieux et d’objets géographiques, constitués en éléments patrimoniaux
  visibles, renforce l’image identitaire de toute collectivité. Elle lui dresse
  une scène et la pourvoit d’un contexte discursif de justification
  particulièrement efficace en ville où des lieux très denses, soigneusement et
  anciennement dénommés, s’inscrivent dans une totalité territoriale
  représentée, à la fois symbolique et fonctionnelle}. \todo{à approfondir}

Le désenchantement du monde tel que nommé précédemment est décrit dans les travaux de Max Weber comme le phénomène par lequel les explications reliés au mystère, la religion ou la superstition perdent leur place face à celles qu'offrent la rationalité scientifique. 
On peut donc comprendre que c'est l'avancement scientifique et surtout sa méthode et ses découvertes qui aura un effet social qui mettra en danger la place de la religion et du mythe dans la société. 
Enlevant une part importante du divin dans l'explication des phénomènes terrestres, par la théorie de l'évolution par exemple, le phénomène va prendre de l'ampleur par l'arrivée du capitalisme au \siecle{19} qui, par sa capacité à produire des marchandises en masse sans l'apport individuel de l'ouvrier --- sa subjectivité --- rendra l'économie de plus en plus anonyme et aliénante. 
Combiné, ces des effets --- rationalité et aliénation --- éloigneront de plus en plus l'individu de sa capacité à expliquer le monde et à s'y conforter, perdant son pouvoir sur la matériel comme sur l'abstrait.

Ce développement aura un effet certain sur les milieux de vie dans lesquels l'industrie s'implantera, principalement les villes: de plus en plus demandantes en main-d’œuvre et offrant une quantité importantes de marchandises à consommer, ces dernières prendront une taille de plus en plus conséquente alors que les quartiers se développeront pour nourrir cette industrie. 
Des milieux villageois aux villes orientées vers les pouvoirs politiques, les sociétés se développeront désormais sur des nœuds urbains que l'urbanisme tentera de rationaliser par la poussée des champs architecturaux nouveaux, notamment par les travaux du Corbusier. 
De la naissance du mouvement fonctionnaliste en architecture, les bâtiment auront maintenant des fonctions et les villes seront pensées comme des machines dont l'efficacité doit être maximisée\missref{}. 
On peut noter ici que l'avènement du capitalisme mettra en place un point de vue rationaliste du l'espace des villes, alors qu'on peut de moins en moins considérer celui-ci comme un territoire où les individus trouveront un sens à leur existence\todo{Reformuler la conclusion. 
Également, voir à ce que la fin du paragraphe précédent ne répète pas inutilement celle-ci}.

Nous ne croyons pas qu'il s'agit ici d'un effet ayant pris emprise strictement en occident, alors que le capitalisme et d'une façon la rationalité scientifique s'est étendue à grande échelle. 
Néanmoins, malgré des effets sociaux très larges, nous considérons tout de même que les lieux touchés par ces effets ont des particularités qui leurs sont propres au niveau de l'organisation spatiale mais surtout, de la réponse sociale à ces effets qui se conjuguent à des effets politiques particulier, que ce soit le colonialisme, les régimes politiques en vigueur, etc. 
Ainsi, nous reconnaissons l'importance de l'Histoire dans les processus sociaux régionaux de part le monde. 
Les différences spatiales contriburaient donc aux développements de différences socio-culturelles quant à l'emprise \todo{à compléter}


Notre point ici est de montrer que dans les villes, l'espace prend une place particulière après ce que nous voyons comme un recul important de la territorialisation dans un sens traditionnel pouvant se reporter à une vision étroite du concept d'ethnie. 
De plus, nous le verrons dans la section \todo{à   compléter} Ces effets sont encore visible aujourd'hui à l'échelle même des villes, alors qu'après un développement certains des villes autour des industries, l'avènement d'un commerce mondial de plus en plus flexible et des entreprises toujours plus mobiles ont rendu les vieux centres urbains où s'étaient d'abord développées ces dernières moins intéressants, que ce soit pour elles-mêmes que pour les classes plus aisées. 
Après avoir perdu en valeur, ces centres sont de nouveaux réhabités par les classes aisées et reconditionnés selon les moyens des investisseurs s'accaparant ces espaces par le phénomènes décrits par de nombreux chercheurs sous le terme de gentrification or d'embourgeoisement.

Néanmoins, nous n'assistons pas qu'à une simple \emph{désertification} des espaces dans lesquels les classes prévilégiées prennent tous les espaces vacants et les individus appartenant aux classes inférieures sont laissées pour compte.
Plusieurs cas de reterritorialisation ont été identifiés dans la littérature, par une diversité d'acteurs, avec ou sans succès \citet{Hatvany2005}\todo{trouver d'autres références}. 
D'ailleurs certains géographes ont relevé que cet espace présente tout de même des caractéristiques particulières. 
Pour \citeauthor{Courville1991} notamment:
\blockquote[{\cite[41]{Courville1991}}][.]{l'espace devient un médiateur du
  rapport entre individus, groupes et collectivités, un produit social à
  analyser comme tel, au milieu des pouvoirs et des rapports sociaux qui le
  structurent et l'organisent}.

C'est donc dire que cet espace urbain, selon son évolution, est maintenant le théâtre d'interactions multiples et enchevêtrées. 
Des groupes comme les individus du spectre \lgbt{} ont d'ailleurs fait l'expérience de ces multiples interactions, en participant ou en subissant les phénomènes de gentrification~\autocite{Podmore2001,Giraud2014,Hogan2005} ou en se mêlant à des luttes d'ordre économiques et politiques auxquelles ils ne font pas partie a priori~\autocite{Kelliher2014} \todo{trouver des références liées aux luttes   anti-capitalistes, angleterre et autres}

Leur présence dans ce jeu d'interactions offrent à ces individus la possibilité de réaliser des rencontres qui mènent à une construction identitaire. 
Comme le souligne~\citeauthor{DiMeo2007}: \blockquote[{\cite[81]{DiMeo2007}}][.]{\textelp{}
  la ville fournit un potentiel privilégié d’outils de recentrage pour toute
  identité individuelle. Par sa variété intrinsèque et par les innombrables
  repères sensibles et vécus qu’elle étale, par les \emph{affordances} (emphase
  de l'auteur) qu’elle sème dans le champ des perceptions individuelles, la
  ville file une trame dont ses habitants se servent sans restriction pour
  tisser et inventer leur propre identité}.

Ainsi, on arrive à faire le lien d'une certaine façon avec les travaux de sociologie sur le sujet et l'identité tel que décrit à la Section~\ref{subsec:sujet_et_identité}. 
Ces nouvelles identités, naissants en partie des luttes pour la reconnaissance et d'une certaine distance avec les identités nationales, trouvent également un lieu pour l'organisation dans les espaces urbains.

% \todo{passer la hache?} \note{Le terme territoire n'est pas utilisé car il
% apparait complexe de déterminer un territoire selon la définition typiquement
% utilisée en géographie culturelle; dans la partie~\ref{sub:enonce_du_probleme},
% la question de territorialité sera approfondie. Parmi les espaces urbains
% envisagés, on peut compter prioritairement ceux des villes de Montréal et Québec
% (voir figure~\ref{fig:carte_quebec}), et hypothétiquement ceux de villes de plus
% petite envergure selon la première partie de la collecte de données (voir à ce
% propos la Section~\ref{sec:source_des_donn_es}).}

% \note{concept on l'a vu peut présenter des lacunes importantes si elle tend à
% une réification des groupes culturels} \todo{insérer à quelque part?}



\subsection{Le géosymbole comme marqueur spatial}
\label{sec:le_symbole_comme_marqueur_spatial} Nous retiendrons, avec les quelques nuances soulevées dans la section précédente, le concept de territoire pour décrire l'espace investi par un groupe culturel particulier. 
Pour faire le pont entre cette notion culturelle de décrire l'espace et les façons dont les individus communiquent entre eux, en demeurant sensible aux questions identitaires, nous proposons l'usage du géosymbole. 
Peu répandu dans la géographie anglo-saxonne, ce concept demeure intéressant pour l'analyse des groupes culturels contemporain.

La culture n'est plus considérée en géographie culturelle comme un tout monolithique; il s'agit plutôt d'une multitude de visions du monde différentes ou divergentes. 
La région, le paysage et le territoire furent des concepts particulièrement importants chez les premiers auteurs en géographie culturelle pour décrire les relations entre l'espace et la culture~\citep{Bonnemaison1981,Monnet1998,DiMeo1998,}, mais ceux-ci tendent dans certains cas à offrir une vision incomplète de la culture, en mettant l'emphase sur la vision de la culture dominante / hégémonique d'un espace~\citep[11-12]{Duncan1993}, ou en offrant une vision déformée des groupes culturels minoritaires ou marginalisés.

La culture occidentale par exemple n'est donc, en Amérique du Nord, qu'une des multiples cultures qui se partagent l'espace et avec laquelle il y a négociation~\citep[11]{Duncan1993}. 
Contrairement aux anciennes perspectives en géographie culturelle qui faisaient de la culture une entité à part, homogène et où régnait une apparence de consensus~\citep{Duncan1980}, la réalité semble s'éloigner de cette image, surtout lorsque l'on s'intéresse aux faits politiques, socio-économiques et d'immigration qui montrent que les sociétés sont beaucoup plus diversifiées que ce qu'elles semblent être. 
Il faut donc s'éloigner de cette perspective dite superorganique selon Duncan qui mène à la réification de la culture; on pourrait même dans ce cas ci étendre cette précaution aux groupes culturels plus restreints, comme les membres de la diversité sexuelle. 
En effet, depuis les dernières décennies, la diversité sexuelle est féconde de plusieurs nouvelles visions du monde par la résistance à l'hétérosexisme qu'on peut voir dans les différentes luttes gaies et lesbiennes.
Occupant une myriade d'espaces difficiles à situer précisément comparativement à la culture dominante, il apparaît inapproprié d'utiliser les concepts spatiaux traditionnels propres à la géographie culturelle comme le territoire pour parvenir à reconnaître et étudier ces groupes.

Il convient donc, pour l'étude géographique des minorités sexuelles, d'utiliser certains concepts moins rattachés à la matérialité de l'espace. 
En demeurant dans le champs de la géographie culturelle, nous proposons l'usage du géosymbole comme outil conceptuel pour situer les groupes culturels et comprendre leur relation avec l'espace. 
En effet, les géosymboles consistent selon Bonnemaison en: \blockquote[{\cite[256]{Bonnemaison1981}}][.]{\textelp{} un lieu, un   itinéraire, une étendue qui, pour des raisons religieuses, politiques ou   culturelles prend aux yeux de certains peuples et groupes ethniques, une   dimension symbolique qui les conforte dans leur identité }. 
En reprenant cette définition dans le contexte des minorités sexuelles, nous pouvons affirmer que nous traitons ici d'un groupe culturel dont la formation s'apparente de certaines façons à un groupe ethnique~\citep{Sinfield1996} mais sans nécessairement être attaché à territoire au sens traditionnel comme soulevé précédemment. 
La sémiotique, sois l'étude plus large des symboles, apparaît donc pertinente car nous considérons que les géosymboles sont une forme particulière de symboles à dimension spatiale et que selon les auteurs, ces derniers peuvent prendre plusieurs formes différentes, matérielles ou non tout en demeurant attaché à l'espace~\citep{Bonnemaison1981,Bedard2002}. 
Nous soulevons donc ici une des limites conceptuelles du géosymbole décrit par Bonnemaison (1981): nous allons utiliser ce concept sans la facette territoriale mais plutôt en se rattachant seulement à l'espace et à la temporalité de son occupation.

Ces géosymboles, bien que permettant de s'intéresser à la culture, ne font pas nécessairement une ombre sur les aspects sociaux d'un groupe; des concepts plus proches de la géographie sociale comme les classes sociales ou la racisation~\citep{Bonniol2005} demeurent nécessaires à la compréhension des espaces gais et lesbiens~\citep[93]{Oswin2008} et pourraient permettre de comprendre la position de certains géosymboles et leur raison d'être, en montrant les clivages propres à un groupe culturel précis, comme le spectre \lgbt{}.

Par ailleurs, ce qui pourrait apparaître comme un groupe homogène, les gais, est en fait tout un spectre qui demande la prise en compte de facteurs sociaux pour comprendre les variances et les divisions, le genre en premier lieu mais également l'appartenance à une certaine classe plus ou moins prompte à vouloir s'accorder ou rejeter les normes sociales hétérosexuelles et à être générateurs d'espaces particuliers ou à transformer certains espaces conçus comme normaux~\citep{Lewis2011}. 
En effet, dans un contexte où l'on reconnaît que l'espace en général facilite les relations hétérosexuelles au détriment des autres relations et donc des identités qui y sont rattachées~\citep{Brown2003}, il est important de s'intéresser aux autres espaces qui existent en marge ou en parallèle et de se questionner sur l'utilité de ces espaces pour les individus impliqués.

Alors que le territoire n'arrive pas à bien définir la spatialité des individus de la diversité sexuelle, le recours à la sémiotique et aux géosymboles pourraient permettre de voir émerger certaines formes spatiales ou du moins montrer la diversité des lieux et leur position et permettre une compréhension plus poussée de l'utilisation de l'espace des communautés \lgbt{}. 
À notre connaissance, aucun travail ne traite spécifiquement de la sémiotique en géographie \qu\ et il s'agit d'une lacune vis-à-vis le potentiel que contient cette approche dans l'étude de l'espace et au sein de la géographie culturelle.
L'étude des géosymboles permettrait d'atteindre un savoir difficile ou impossible à obtenir par une approche plus matérialiste s'arrêtant au territoire sans prendre en compte l'existence de réseaux. 
D'emblée, nous pouvons déjà considérer que des évènements comme les \anglais{gay pride} (ou Fierté gaie), les manifestations politiques pour les droits gays et lesbiens, les centres communautaires offrant des services aux individus porteur du \vih, les bars lesbiens ou encore des lieux de dragues dans des espaces publics sont tous des exemples d'espaces \lgbt{} portant en eux des marques particulières, des logos ou des apparences qui entrent dans la définition du géosymbole.

\blockquote[{\cite[108]{DiMeo1998}}][.]{\textelp{} le territoire multidimensionnel
  participe de trois ordres distincts. Il s'inscrit en premier lieu dans l'ordre
  de la matérialité, de la réalité concrète de cette terre d'où le terme tire
  son origine. Il relève en deuxième lieu de la psyché individuelle. Sur ce
  plan, la territorialité s'identifie pour partie à un rapport a priori,
  émotionnel et présocial de l'homme à la terre. Il participe en troisième lieu
  de l'ordre des représentations collectives, sociales et culturelles. Elles lui
  confèrent tout son sens et se régénèrent, en retour, au contact de l'univers
  symbolique dont il fournit l'assise référentielle}.



\subsection{La diversité sexuelle en géographie}
\label{sec:la_diversit_sexuelle_en_g_ographie} \todo{à garder ou pas?} 
Les travaux arrimant l'ensemble de ces courants théoriques, l'étude des géosymboles avec un accent fort sur la sémiotique en relation avec des groupes dont l'identité se superpose à des ensemble culturels plus grand ne sont pas légions, surtout en ce qui concerne le cas plus spécifique des identités \lgbt{}. 
Ceux-ci sont tout de même existants et il nous apparaît important de nommer brièvement ceux qui existent dans le but précis de définir les angles morts où la recherche a lieu d'être.

L'ouvrage principal à faire la lumière sur l'usage des symboles par les communauté \lgbt{} sur lequel nous nous appuyons est le livre \citetitle{Giraud2014} de \citet{Giraud2014} qui s'est plus particulièrement penché sur les cas du Village gai de Montréal et du Marais de Paris selon une démarche comparative. 
Plus près de nos intérêts, une section du livre s'est plus particulièrement penchée sur les symboles utilisés par les communautés gaies des deux villes. 
On y apprend plus particulièrement comment cette visibilité a contribué à une certaine territorialisation des groupes gais. 
Choisissant une perspective historique et centrée sur un seul groupe, les hommes gais, la recherche reste muette sur les autres groupes \lgbt{}, sachant que ceux-ci interagissaient avec ces hommes et partageaient certaines espaces et une histoire inter-reliée~\citep{Remiggi2000,Demczuk1998,Podmore2001,Higgins1997,Higgins1999}.

Nous nous appuyerons donc dans cette recherche sur d'autres travaux pour couvrir plus largement les communautés \lgbt{} québecoises. 
Nous avons penser notamment aux de Julie Podmore, de Frank Remiggi et de Ross Higgins qui figurent parmi les auteurs principaux à traiter de la diversité sexuelle au Québec en géographie ou en anthropologie.
\todo{à compléter}

\subsection{Vers une vision hétérogène de l'identité, de l'espace et de
l'essence du symbole}
\label{sec:vers_une_vision_h_t_rog_ne_de_l_identit_de_l_espace_et_de_l_essence_du_symbole}
\todo{à revoir} \foreignblockquote{english}[{\cite[tel que cité
  dans][97]{Oswin2008}}][.]{As Elspeth Probyn has stated, sexual spaces
  \foreigntextquote{english}[{\citeyear[10]{Probyn1996}}][.]{are delineated
    through coincidence and not through exclusion}. Rather than clinging to the
  fiction that we can locate queer spaces that exist in coherent opposition to
  heterosexual spaces, we need to intensify examinations of what comes together
  in processes of sexualization}

%%% Local Variables:
%%% mode: latex
%%% TeX-master: "../../memoire-maitrise"
%%% End:
