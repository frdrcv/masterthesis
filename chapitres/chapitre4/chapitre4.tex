%!TEX root = ../../memoire-maitrise.tex

\chapter{De l'inclusion à la différence par le symbole}
\label{cha:de_l_inclusion_la_diff_rence_par_le_symbole}

% \chapterprecishere{\textquote{Everyone needs a place. It shouldn't be inside of someone else.” \par\raggedleft--- \textup{Richard Siken}, Crush}
Après avoir présenté les résultats de la collecte de données dans le chapitre précédent, nous nous procèderons dans celui-ci au portrait d'ensemble soulevé par les données.
Pour chaque géosymbole analysé précédemment, nous avons offert une brève analyse de la sémiotique pour faire ressortir les messages que ceux-ci portent, les codes.
Également, l'approche géographique nous a poussé à localiser ces symboles et nous avons ainsi pu décrire du même coup l'étendue et l'emplacement des géosymboles rencontrés.
Ces informations, combinées ensemble, dressent plusieurs portraits simultanés de la panoplie des symboles rencontrés.
Par contre, nous croyons qu'il est nécessaire pour la suite, conformément aux points soulevés au chapitre 2, de faire les liens entre ces géosymboles pour arriver, par la suite, à faire une analyse d'ensemble.
Cette analyse sera donc une synthèse de notre démarche précédente.
Elle pourra par la suite servir à faire la comparaison avec d'autres territoires, dans d'autres villes ailleurs qu'au Québec dans une perspective de comparaison.

\todo{Fouiller \cite{Fyfe1988}, contenu pertinent selon \cite[11]{Rose2012}}

\subsection{Communauté imaginée}
\label{sub:communaut_imagin_e}
% Pourra être déplacé ailleurs
Devant ces nombreuses dissensions et l'absence apparente d'échanges au sein de ce qu'on nomme communauté~\lgbt{} ou gaie et lesbienne, on peut s'interroger sur ce qui en est de cette communauté, s'il en est vraiment une.
S'intéressant d'abord au nationalisme, le texte de Benedict Anderson offre une analyse intéressante de la question de nation, alors que celles-ci représentent bien souvent de grands ensembles liés par peu de choses, sinon un lieu de naissance commun et une culture commune.
\begin{quote}	
societies are sociological entities of such firm and stable reality that their members (A and D) can even be described as passing each other on the street, without ever becoming acquainted, and still be connected.37 \citep{Anderson 1983}
\end{quote}

\begin{quote}
  That all these acts are performed at the same clocked, calendrical time, but   by actors who may be largely unaware of one another, shows the novelty of this imagined world conjured up by the author in his readers' minds.38
  \citep{Anderson 1983}
\end{quote}

\begin{quote}
	
It should suffice to note that right from the start the image (wholly new to Filipino writing) of a dinner-party being discussed by hundreds of unnamed people, who do not know each other, in quite different parts of Manila, in a particular month of a particular decade, immediately conjures up the imagined community.
[\ldots]

Notice too the tone. 
While Rizal has not the faintest idea of his [28] readers' individual identities, he writes to them with an ironical intimacy, as though their relationships with each other are not in the smallest degree problematic.43
\end{quote}


\begin{figure}[ht]
	\centering
	\includegraphics[width=16cm]{fig4.jpg}
	\caption{Fête Arc-en-ciel de Québec: marche de solidarité et journée
    communautaire\todo{corriger les étiquettes de noms sans accents}}
	\label{fig:figure4}
\end{figure}

\begin{figure}[ht]
	\centering
	\includegraphics[width=16cm]{fig3.jpg}
	\caption[]{Marche Dyke : trajet et moments importants de la manifestation}
	\label{fig:figure3}
\end{figure}

% subsection communaut_imagin_e (end)

\section{Premiers symboles liés à la diversité sexuelle}
\label{sec:premiers_symboles_li_s_la_diversit_sexuelle}

% section premiers_symboles_li_s_la_diversit_sexuelle (end)

\section{Symboles politiques et identitaires}
\label{sec:symboles_politiques_et_identitaire}
Certains événements historiques récents semblent laisser croire à une moins grande nécessité d'un mouvement politique au sein des communautés \lgbt{}.
En effet, le \sida{} n'est plus une urgence à traiter dans les communautés \lgbt{} comme dans la population générale, plusieurs lois ont été votées visant un arrêt de l'homophobie dans les écoles ou la discrimination basée sur l'orientation sexuelle dans différentes sphères de la société.

Nos données montrent par contre que des symboles politiques existent encore aujourd'hui.
Différentes causes sont soutenues selon les groupes étudiés, alors que ces messages politiques s'opposent à l'occasion.
C'est le cas par exemple de Pervers/Cité qui, dans son nom, récupère le nom d'une institution importante de la communauté \lgbt{} pour la critiquer et du même coup, s'en prendre aux normes sociales entourant la sexualité en effectuant une réappropriation du terme \emph{pervers}. 

Les événements entourant la Fierté Trans nous apparaissent être ceux mettant en scène les géosymboles les plus clairement revendicatifs.
Partageant avec la Marche Dyke certains médias comme les banderoles, pancartes et occupations de la rue sous la forme d'une manifestation, la communauté trans au moment de la collecte de données attendait l'instauration d'une loi concernant le statut de personne trans.
Ainsi, divers géosymboles rencontrés visaient ces changements légaux en cours, mais pas nécessairement tous.
Plusieurs d'entre eux semblaient plutôt envoyer un message sur la place publique vis-à-vis de la légitimité des identités de genre trans ou genderqueer.
L'usage de chaînes ou le fait de cacher les mamelons d'un homme trans par exemple montre l'usage de la politique dans un but de changement social en dehors du cadre légal vers, plutôt, un assouplissement des normes sociales visant la société en général.

\section{Une variété de groupes et de symboles}
\label{sec:une_variete_de_groupes_et_de_symboles}
Nous pouvons constater avec les résultats qu'avec une plus grande agglomération apparait une plus grande diversité de lieux et de groupes.
Si la ville de Montréal est reconnue pour son Village gai, une observation plus minutieuse des médias \lgbt montre qu'un grand nombre de groupes se sont organisés dans ce territoire ou en périphérie.
Nous avons déjà fait démonstration de cette diversité dans le chapitre précédent en séparant les géosymboles de la ville de Montréal en différentes catégories selon les groupes et événements rencontrés.

Pour la communauté trans, il existe un groupe organisant la Fierté Trans, événement comportant une manifestation et une soirée spectacle, et un autre visant une portion de cette communauté, les personnes racisées, par le festival Qouleur.
Les groupes queers s'investissent notamment dans Pervers/Cité, le Salon du livre Queer entre les couvertures et dans certains contingents de la marche de la Fierté. 
Après notre collecte, nous avons également pu apprendre l'existence d'autres événements prenant place dans les lieux nommés dans la section précédente; par contre, débordant notre collecte, nous ne nous attarderons pas sur ceux-ci.

Les communautés lesbiennes montréalaises semblent se séparer selon l'identification comme dyke, bien qu'il soit possible que, en dehors des événements que nous avons abordés, ces communautés soient moins divisées qu'il n'y parait par cette distinction.
En effet, nous avons remarqué le faible nombre de lieux spécifiquement pour femmes lesbiennes.
Les événements étant plus nombreux, souvent apolitiques si nous ne tenons pas compte des marches organisées en août, les différentes femmes lesbiennes doivent se rejoindre et se croiser au sein de tels événements.
On peut également supposer, à la suite de la division historique amenée par \citet{Giraud2014}, que la division commune soit plutôt d'ordre de l'âge, les différentes générations de femmes lesbiennes ne partagent pas nécessairement les mêmes intérêts.
Nos données étant limitées en ce sens, plus de travail serait à faire avec les médias lesbiens.

Les communautés gaies masculines semblent quant à elles se séparer dans une très grande variété d'événements très publicisés, avec des moyens variant également.
Certains espaces arrivent difficilement à s'afficher avec évidence, par manque d'expérience ou de moyens financiers, alors que d'autres arrivent à utiliser une variété d'outils importante, en faisant appel à des publicités de haute de qualité, en utilisant un grand nombre de médias, ou en occupant des espaces difficilement abordables par des groupes possédant moins de ressources.

\section{De la nature pour hommes}
\label{sec:de_la_nature_pour_hommes}
Avec une population dépassant toute autre agglomération au Québec, le grand Montréal doit posséder par défaut une population \lgbt{} plus importante qu'ailleurs.
Avec le grand degré d'organisation que nous avons constaté à la suite d'autres chercheurs, on remarque que la population masculine, après avoir construit un territoire important qu'est le Village gai, a également investi d'autres espaces secondaires.
La ville offrant anonymat et possibilités pour des rencontres, elle demeure limitante quant aux rencontres possibles dans les espaces publics.

Comme nous l'avons soulevé au chapitre précédent, plusieurs espaces ont été investis dans la couronne montréalaise.
Les campings par exemple semblent compléter en partie les avantages offerts par la Village gai pour les hommes.
Dans de tels lieux, les espaces ouverts, comparables aux espaces publics urbains, sont beaucoup moins surveillés, les campings étant \latin{de facto} des espaces privés.
Selon les symboles utilisés, sois des corps masculins nus, la sexualité est mise au premier plan.
La pratique du nudisme est également encouragée selon les campings.
Dans de tels cas, on s'attend à ce que ce soit essentiellement des hommes qui fréquentent ces lieux.

La proximité de Montréal semble témoigner de la nécessité d'une population importante à proximité pour assurer la pérennité de ceux-ci.
Cette pérennité semble surtout être lieux au nombre de personnes potentielles intéressées à investir les lieux; les régions moins urbanisées ont souvent accès beaucoup plus facilement à des espaces non urbanisés et à l'apparence naturelle.
Nous utilisons le terme naturel en référence au code soulevé au chapitre précédent; en effet, ce caractère naturel est à mettre en opposition à l'urbain plutôt qu'à l'habituelle notion de culture.
Les espaces ruraux sont tout aussi \emph{non naturels} que les espaces urbains; par contre, la présence d'un couvert végétal important porte en soi une signification d'une moins grande trace anthropique.

Certains auteurs ont abordé le rapprochement de certains groupes \lgbt{} avec les milieux ruraux.
C'est le cas dans \citetitle{Bell1995a} de \citeauthor{Bell1995a}, cet article abordant les imaginaires de la campagne queer.
Selon le genre et l'identité, certains groupes rechercheraient, dans certains romans ou en regardant des photographies d'époques, différents éléments dans une territorialisation de la campagne.
Les paysages ruraux, avec leur aspect naturel, peuvent évoquer une forme de sensualité par l'usage de métaphore portant sur la vie sauvage ou le secret dans l'immensité des paysages~\citep[114]{Bell1995a}.
Pour les femmes, les espaces ruraux possèdent la qualité d'être des lieux où il serait possible d'échapper à la domination hétérosexuelle et masculine des espaces urbains.
La nature serait donc un lieu vierge où il serait possible de reconstruire des microsociétés égalitaires, des utopies.

Ce sont en général des éléments que nous avons retrouvés dans les publicités des campings en nature, mais l'absence de données pour les femmes ne peut nous permettre d'affirmer l'existence de ce genre d'utopies au Québec.
En même temps, le sens politique d'un tel projet se marie mal avec le média qu'est la publicité; la visibilité d'un tel lieu ne serait pas un but recherché, au contraire.

\section{Des régions peu organisées}
\label{sec:des_regions_peu_organisees}
Le niveau d'organisation spatiale des espaces en région semble moins développé que dans les agglomérations de Québec ou de Montréal.
En effet, la durée de vie des lieux est souvent basse pour les espaces commerciaux alors que les espaces communautaires, s'ils persistent dans le temps, sont souvent peu visibles.
Malgré tout, le magazine Fugues nous a permis de constater l'existence de tels lieux, ceux-ci ayant été dénombrés à la fin du chapitre précédent.

Historiquement, les espaces ruraux n'ont pas eu la même utilité que les espaces urbains pour les individus \lgbt{}, principalement au niveau politique.
Avec la dispersion spatiale, des valeurs réputées plus conservatrices au sein de la population générale et le faible nombre de ressources ont rendu l'organisation de communauté \lgbt{} plus complexe.
Comme nous l'avons souligné dans la section précédente, les espaces utopiques où des communautés \lgbt{} auraient pu subsister n'auraient pas nécessairement besoin d'une visibilité, rendant difficile la possibilité de les situer par le seul usage de médias visuels.


% section symboles_politiques_et_identitaire (end)

% chapter de_l_inclusion_la_diff_rence_par_le_symbole (end)
%%% Local Variables:
%%% mode: latex
%%% TeX-master: "../../memoire-maitrise"
%%% End:
