%!TEX root = ../../memoire-maitrise.tex
\chapter{De l'inclusion à la différence par le symbole}
\label{cha:de_l_inclusion_la_diff_rence_par_le_symbole}

% \chapterprecishere{\textquote{Everyone needs a place. It shouldn't be inside of someone else.” \par\raggedleft--- \textup{Richard Siken}, Crush}
Après avoir présenté les résultats de la collecte de données dans le chapitre précédent, nous procèderons dans celui-ci au portrait d'ensemble soulevé par ces dernières.
Pour chaque géosymbole analysé précédemment, nous avons offert une brève analyse sémiotique pour faire ressortir les messages que ceux-ci portent, les codes.
L'approche géographique nous a poussé à localiser ces symboles et nous avons ainsi pu décrire du même coup l'étendue et l'emplacement de l'ensemble
des géosymboles rencontrés.
Ces informations, mises ensemble, vont nous permettre, pour la suite, de dresser un portrait d'ensemble de la panoplie des symboles rencontrés au Québec et des diverses communautés qui sont représentées.
Conformément aux points soulevés au chapitre 2, nous établirons les liens entre ces géosymboles.
%Cette analyse sera donc une synthèse de notre démarche précédente.
Elle pourra subséquemment servir pour la comparaison d'autres territoires, dans d'autres villes ailleurs qu'au Québec dans une perspective de facilitation d'autres analyses et comprendre si les populations \lgbt{} hors Québec sont similaires a celles d'ici.

\todo{Fouiller~\cite{Fyfe1988}, contenu pertinent selon~\cite[11]{Rose2012}}
\begin{quote}
  To understand a visualisation is thus to inquire into its provenance and into the social work that it does.
  It is to note its principles of exclusion and inclusion, to detect the roles that it makes available, to understand the way in which they are distributed, and to decode the hierarchies and differences that it naturalises~\citep[1]{Fyfe1988}.
\end{quote}

\section{Communauté imaginée}
\label{sub:communaut_imagin_e}
% Pourra être déplacé ailleurs

Le premier constat que l'on peut établir à la suite de la revue des résultats collectés est bien la diversité des identités \lgbt{} et de l'inexistence d'une communauté unie sur la question de l'orientation sexuelle \emph{et} du genre par la diversité des géosymboles employés.
Au-delà de la distinction spatiale qui pourrait permettre d'expliquer l'existence de plusieurs groupes sur de trop grands territoires, de mêmes espaces plus restreints accueillent en leur sein plusieurs groupes, lorsque par exemple simultanément, dans le village, la Fierté Trans a lieu en réaction à Fierté Montréal.
Également, des moments très circonscrits dans une année, à une échelle de jours, sont occupés par plusieurs événements, parfois simultanément.
Alors que plusieurs d'entre eux sont superposés, mais inscrits dans un même événement plus large, une information révélée par des codes similaires ou une direction assumée par les mêmes individus ou organisations, plusieurs autres sont organisés par des groupes différents. 
Ces derniers utilisent alors des codes différents, bien que s'intéressant à des groupes similaires.

Devant ces nombreuses dissensions et l'absence apparente d'échanges au sein de ce qu'on nomme communauté~\lgbt{} ou gaie et lesbienne, on peut s'interroger sur ce qui en est de cette communauté, s'il en est vraiment une.
Cette superposition d'événements pourrait s'expliquer par le caractère hégémonique d'une culture~\lgbt{} dans  laquelle les événements superposés se produisent que par l'utilisation de codes communs et une démultiplication des sous-groupes qu'elle rend possible.
S'intéressant au nationalisme, le texte de Benedict Anderson offre une analyse intéressante de la question de la nation qui pourrait s'accorder avec la notion de culture que nous avons présenté au premier chapitre. 
La nation pour Anderson de grands ensembles liés par peu d'éléments, sinon un lieu de naissance commun et une culture commune seulement.
Ce lieu de naissance serait moins pertinent dans le cas de la culture~\lgbt{}, sachant que les individus~\lgbt{} ont souvent à migrer, comme nous l'avons expliqué à la Section~\ref{sec:diversit_sexuelle_et_identit_}.
Plutôt, l'identité sexuelle jouerait ce rôle de liant entre les individus~\lgbt{} dans des espaces comme le Village gai.
Comme le souligne Anderson:
\begin{quote}
  [\ldots] Societies are sociological entities of such firm and stable reality that their members [\ldots] can even be described as passing each other on the street, without ever becoming acquainted, and still be connected.~\citep[25]{Anderson2006}
\end{quote}

\begin{quote}
  That all these acts are performed at the same clocked, calendrical time, but by actors who may be largely unaware of one another, shows the novelty of this imagined world conjured up by the author in his readers’ minds.
  \citep[26]{Anderson2006}
\end{quote}

%\begin{quote}
%  It should suffice to note that right from the start the image (wholly new to Filipino writing) of a dinner-party being discussed by hundreds of unnamed people, who do not know each other, in quite different parts of Manila, in a particular month of a particular decade, immediately conjures up the imagined community.
%  [\ldots]
%
%  Notice too the tone. While Rizal has not the faintest idea of his [28] readers’ individual identities, he writes to them with an ironical intimacy, as though their relationships with each other are not in the smallest degree problematic.43
%\end{quote}

\section{Premiers symboles liés à la diversité sexuelle}
\label{sec:premiers_symboles_li_s_la_diversit_sexuelle}

Comme nous pouvons l'observer par la richesse des résultats, plusieurs symboles désignent les communautés \lgbt{}.
Nous reviendrons dans cette section sur les principaux symboles de ces communautés pour nous intéresser au contexte historique dans lequel ils ont surgi.
Le but ici est principalement de rendre compte de l'âge assez ancien de ces derniers et de comprendre qu'à la base, ceux-ci se sont imposés avant d'être récupérés par la suite dans un mouvement politique défini.

Le symbole le plus fréquemment utilisé de nos jours pour représenter les communautés \lgbt{} est l'arc-en-ciel, celui-ci apparaissant surtout sous la forme d'un drapeau.
En général, s'il n'est pas sous cette forme, il reprend tout de même la forme rectangulaire qui rappelle le drapeau.
Comme on a pu le constater dans les résultats, ce symbole est abondamment utilisé, que ce soit durant les événements correspondant à la Fierté comme la Fête Arc-en-ciel ou dans des espaces permanents, comme le Village gai.
Par contre, comme nous l'avons vu dans le chapitre précédent, plusieurs autres drapeaux sont également ut
\blockquote{[\ldots] the flags serve as a kind of visual text to be read, and a particular kind of text, too. 
These aren’t, after all, just flags—these are pride flags, suggesting not simply an identity but also a kind of emotive investment akin to patriotism.  
But they also reference all the texts that came before: the chats, the erotica, the legends, the traditions. 
In other words, pride flags don’t represent the beginnings of an imagined community; they emerge only after other texts, other discourses, have already formed the community.  
These flags don’t create the identities they represent; they represent identities that have been created~\citep[350]{Barrios2004}.}

Nous continuerons avec un symbole aujourd'hui moins utilisé, mais dont l'origine historique est plus ancienne et controversée.
Il s'agit du triangle utilisé par divers mouvements, quoique son usage soit principalement repris par les groupes les plus radicaux en ce qui concerne notre collecte de données, à quelques exceptions près.
Ce symbole provient de l'Allemagne nazie, peu avant les premières purges basées sur l'ethnicité, et qui également cibleront les individus par leur appartenance politique, leur état de santé, et dans notre cas, leur orientation sexuelle.
En effet, le triangle était un symbole utilisé pour catégoriser rapidement les individus désignés par ces caractéristiques.
Découpés dans du tissu, les triangles étaient cousus sur les vêtements portés par les individus arrêtés par les forces allemandes.
Tout un code de couleurs servait à différencier les triangles entre eux, et la superposition de deux triangles inversés désignait de plus les personnes juives, le motif rappelant l'étoile de David.
Pour les femmes lesbiennes, le triangle noir était le symbole les signifiant, bien que ce code ait regroupé également les personnes ayant des problèmes de santé mentale ou désignée comme asociale.
En ce qui concerne les hommes homosexuels, le rose aurait été la couleur utilisée, bien que selon les historiens, cette division n'est pas clairement délimitée \citep[334]{Jensen2002}.
Aujourd'hui, ce serait le triangle rose qui serait le plus utilisé, depuis son apparition au milieu des années 1970 aux États-Unis~\citep[328]{Jensen2002}.

Plusieurs décennies se sont donc déroulées avant que le symbole réapparaisse pour désigner les femmes et hommes homosexuels.
Une des manifestations les plus répandues de sa réapparition est la récupération qui a eu lieu durant la crise du \vih{} au milieu des années 80.
C'est durant la crise, quand la communauté \lgbt{} a été la plus touchée par la maladie que les différentes organisations Act-Up se sont formées, d'abord aux États-Unis puis outre-mer en France.
Cette organisation s'est attaquée très tôt aux élites politiques, considérées comme inactives et laissant la pandémie faire des dommages au sein d'une population longtemps marginalisée, malgré les avancées politiques précédant de peu l'arrivée due \vih{}.

Aujourd'hui, nous avons pu le voir, ces symboles sont toujours utilisés, quoiqu’inégalement.
On retrouve par exemple le triangle à de nombreuses reprises que

\section{Symboles politiques et identitaires}
\label{sec:symboles_politiques_et_identitaire}

Certains événements historiques récents semblent laisser croire, à un premier abord, à une moins grande nécessité d'un mouvement politique au sein des communautés \lgbt{}.
Au Canada comme dans plusieurs autres pays en occident, le \sida{} n'apparait plus comme une urgence politique pour les individus \lgbt{} ou même dans la population générale.
Avec l'institutionnalisation de la lutte à la pandémie, les mouvements politiques ayant émergé de la crise due \vih{} semblent être moins présents, alors qu'au Québec, plusieurs organismes communautaires prennent en charge la prévention par le sécurisexe.
On peut penser aux organismes \miels{} dans la ville de Québec ou à ZÉRO à Montréal.
Parallèlement, plusieurs lois ont été votées pour combattre l'homophobie dans les écoles ou la discrimination basée sur l'orientation sexuelle dans différentes sphères de la société.
Ces différentes luttes continuent à prendre place, mais une opposition claire aux institutions gouvernementales ou un activisme plus militant semblent avoir laissé la place au travail d'organismes communautaires en ce qui concerne la communauté \lgbt{} générale, ou aux communautés gaies et lesbiennes plus particulièrement.

Nos données nous montrent par contre que certains groupes s'appuient toujours sur le militantisme pour faire avancer certaines causes politiques.
Ceci est d'autant visible par l'utilisation de symboles politiques particuliers à la communauté \lgbt{} ou à certains de ses sous-groupes.
%Différentes causes sont soutenues selon les groupes étudiés, alors que ces messages politiques s'opposent à l'occasion.
C'est le cas par exemple de Pervers/Cité.
Son nom en soi fait office de symbole politique en prenant un sens subversif ou caricatural par rapport au nom de l'ancien festival Divers/Cité.
En effet, Pervers/Cité récupère celui d'une institution qui a été particulièrement importante pour la communauté \lgbt{}, Divers/Cité ayant été une des premières organisations à planifier une marche de la fierté qui n'était pas une manifestation.\missref{}
Du même coup, l'organisation queer semble critiquer les normes sociales entourant la sexualité par une réappropriation du terme \emph{pervers}.
Le terme pervers est connoté négativement; comme le souligne Rubin, le terme de pervers désigne les individus dont la sexualité est hors-norme.
Bien que Rubin s'intéresse à plusieurs axes de divergences, nous retiendrons surtout ici que le terme pervers, selon les époques et le climat politique, peut rassembler tout individu n'étant pas dans une relation stable, monosexuelle, hétérosexuelle, ou dans le cadre du mariage, entre autres caractéristiques\footnote{Pour Rubin, en plus des axes du genre ou de la classe sociale, une hiérarchie existe également s'articulant autour de la sexualité. 
  Cette hiérarchie place en haut la sexualité hétérosexuelle que nous avons nommée, et exclut à divers degrés toutes les autres. 
  Avec les luttes sociales, certaines se rapprochent de plus en plus de la normalité, alors que d'autres sont reléguées aux marges les plus lointaines et sont vues encore plus négativement. 
  On compte parmi celles-ci les relations intergénérationnelles, le travestisme, ou le sexe rémunéré.
}.
Cette récupération du terme pervers parait calquer celle qui a été faite il y a quelques décennies pour le terme queer, qui désignait à l'époque toute personne homosexuelle.
Si l'homosexualité devient de plus en plus normale, le terme pervers semble qualifier un groupe qui n'atteindra jamais le statut de normalité.
Le terme divers dans le nom Divers/Cité, par contre, compte plutôt sur la volonté de reconnaître politiquement la diversité des sexualités, bien que pour Pervers/Cité, ce but n'ait pas été atteint et que le passage d'un organisme de lutte politique à un organisme organisant une fêter apolitique est un échec en soi.

Les événements entourant la Fierté Trans figurent parmi ceux mettant en scène les géosymboles les plus revendicatifs après Pervers/Cité.
Partageant avec la Marche Dyke l'usage de certains médias comme les banderoles, pancartes ou partageant des stratégies visuelles comme l'occupation de la rue par la forme d'une manifestation, la communauté trans au moment de la collecte de données attendait l'instauration d'une loi concernant le statut de personne trans.
Ainsi, divers géosymboles rencontrés visaient ces changements légaux en cours, mais pas nécessairement tous.
Plusieurs d'entre eux semblaient plutôt envoyer un message sur la place publique vis-à-vis de la légitimité des identités trans ou non binaires dans le genre.
L'utilisation de chaînes ou le fait de cacher les mamelons d'un homme trans par exemple montre l'usage politique d'un symbole dans un but de changement social en dehors du cadre légal. Plus précisément, ici, cet usage semble proposer un assouplissement des normes sociales visant la société en général plutôt qu'être un message dirigé aux institutions sociales.

Par contre, contrairement à la proposition de \citet{Probyn1996}, ces espaces sexuels temporaires semblent se créer moins par coïncidence que par exclusion.
Si la marche trans se voulait inclusive et n'excluait pas les individus cisgenres, la marche dyke motivait son existence par une volonté politique de rassembler des individus partageant une identité commune basée sur l'attraction envers des femmes par des femmes.
La coïncidence ici existe donc par le rassemblement de femmes de diverses orientations sexuelles, certaines étant racisées et appartenant probablement à diverses couches de la société, étudiante ou non, bien que ce dernier point resterait à vérifier.

\section{Une variété de groupes et de symboles}
\label{sec:une_variete_de_groupes_et_de_symboles}

Nous pouvons constater avec les résultats qu'avec une plus grande agglomération apparait une plus grande diversité de lieux et de groupes.
Si la ville de Montréal est reconnue pour son Village gai, une observation plus minutieuse des médias \lgbt{} montre qu'un grand nombre d'individus se sont organisés entre eux dans ce territoire ou en périphérie.
Nous avons déjà souligné cette diversité dans le chapitre précédent en séparant les géosymboles de la ville de Montréal en différentes catégories selon les gens et événements rencontrés.

Pour la communauté trans, il existe un groupe affinitaire derrière la Fierté Trans, événement comportant une manifestation et une soirée spectacle, et un autre visant une portion de cette communauté, les personnes racisées, par le festival Qouleur.
Les personnes queers s'investissent notamment dans Pervers/Cité, le Salon du livre Queer entre les couvertures et dans certains contingents de la marche de la Fierté.
Après notre collecte, nous avons également pu apprendre l'existence d'autres événements prenant place dans les lieux nommés dans la section précédente; par contre, débordant le cadre que nous nous sommes fixé, nous ne nous attarderons pas sur ceux-ci.

Les communautés lesbiennes montréalaises semblent se séparer selon l'identification comme dyke, bien qu'il soit possible que, en dehors des faits marquants que nous avons abordés, ces communautés soient moins divisées qu'il n'y parait par cette distinction.
En effet, nous avons remarqué le faible nombre de lieux spécifiquement pour lesbiennes.
Les événements sont plus nombreux et souvent apolitiques si nous ne tenons pas compte des marches organisées en août; ces femmes ne s'identifiant pas comme hétérosexuelles doivent se tenir informées et se rassembler par elles-mêmes lorsque les situations se présentent.
On peut également supposer, à la suite de la division historique amenée par \citet{Giraud2014}, que la séparation soit plutôt de l'ordre de l'âge, les différentes générations de lesbiennes ne partagent pas nécessairement les mêmes intérêts.
Nos données étant limitées en ce sens, plus de travail doit être effectué avec les médias lesbiens.
Par contre, nous pouvons souligner qu'en plus de cette analyse géospatiale, les géosymboles lesbiens peuvent nous informer également sur la sexualité lesbienne de façon indirecte.
Comme nous l'avons remarqué à la Section~\ref{sec:qu_bec_spectre_large}, les codes recensés dans les publicités des événements lesbiens misent peu sur une objectivation du corps féminin.
Cette absence se voit aussi dans les autres symboles, et montre que l'intérêt suscité par les événements ne nécessite pas une sexualisation des corps.

Les communautés gaies masculines semblent quant à elles s'investir dans une très grande variété d'événements, souvent très publicisés.
Les moyens mis en place varient également, surtout si l'on se penche à l'affiliation de ces événements à certains sous-groupes liés à d'autres identités, comme queer, par exemple.
Certains espaces arrivent difficilement à s'afficher avec évidence, par manque d'expérience ou de capacité financière des personnes qui les maintiennent, alors que d'autres utilisent une variété d'outils importante.
Parmi ces outils, nous pensons à la production de publicités professionnelles, à l'utilisation d'un grand nombre de médias, ou encore en occupant des espaces difficilement abordables par des groupes possédant moins de ressources, comme des locaux commerciaux sur des artères urbaines importantes.

\section{Des communautés à la mixité variable}
\label{sec:des_communautes_a_la_mixite_variable}

La mixité des événements et des lieux \lgbt{} n'est pas une caractéristique commune, comme nous avons pu le constater dans la présentation de nos résultats.
En effet, diverses raisons poussent certains organisateurs d'événements à limiter l'accès, ou pas, à certains groupes.
Nous nous attarderons donc dans cette section à ce phénomène pour comprendre à quel point certains espaces, en plus de viser certains groupes précis au sein de la communauté \lgbt{}, empêchent certaines autres personnes d'y accéder.
Bien que cette notion n'ait pas de liens directs avec les géosymboles des communautés \lgbt{}, nous croyons qu'une analyse de cette notion de pouvoir permet de nuancer l'usage des géosymboles comme simples marqueurs identitaires, mais peut également exclure certaines personnes.
Bien que l'exclusion soit vécue au sein des communautés \lgbt{} par rapport au reste de la société, cette exclusion dans la non-mixité peut servir des buts positifs.

À l'observation des résultats, on peut s'apercevoir que cette division s'effectue souvent sur la base du genre surtout lorsque les espaces sont mis en place sur la base de l'orientation sexuelle.
Dans certains cas par contre, cette non-mixité s'articule plutôt par une notion plus large qu'on peut assimiler au \emph{vécu de marginalisation}.
Ce \emph{vécu de marginalisation} se situe principalement à l'intersection du genre ou de l'orientation sexuelle et du vécu de personne racisée; il s'agit, dans nos résultats du principal motif à la non-mixité.
Pour ces groupes \lgbt{}, cette intersection signifie que l'exclusion s'effectue par le regroupement de personnes touchées autant par la suprématie blanche que par l'hétérosexisme.
Nous avons choisi le terme de suprématie blanche par son association fréquente à la notion de racisme.
Dans un cas comme celui du Québec, cette suprématie blanche peut s'exprimer par divers acteurs et sous diverses formes de racisme.
Ceci est d'autant plus visible que les discours des groupes rassemblant des personnes racisées et \lgbt{} visent en particulier certains groupes dans la définition de leurs événements.
Par exemple, en général les événements organisés par l'organisme Qouleur ciblent les personnes autochtones et les personnes racisées, souvent en spécifiant les personnes arabes, musulmanes ou noires.
Au Québec, la question du racisme envers les personnes autochtones a marqué l'histoire et est encore aujourd'hui l'objet de diverses recherches, sachant que cette exclusion a toujours lieu \missref{}.
Le racisme envers les personnes noires est également ancien; on peut remonter à la colonisation et au commerce triangulaire le début de cette exclusion sociale qui perdure, elle aussi, encore aujourd'hui \missref{}.
En ce qui concerne les individus d'origine arabe ou dont la religion est l'Islam, l'exclusion semble a priori plus récente, et l'on peut croire qu'elle prend ses sources dans les premiers cas d'immigration de réfugiés à la suite des conflits en Europe de l'Est et au Moyen-Orient.
Si le racisme envers les personnes migrantes n'est pas une notion nouvelle au Québec, elle a prise de l'ampleur dans l'actualité à la suite du débat sur la laïcité de l'état et de la place des accommodements raisonnables.
Il est donc imaginable que les individus \lgbt{} appartenant à l'un ou l'autre de ces groupes subissent plus durement l'exclusion sociale.
Un organisme comme Qouleur offre alors, par la non-mixité, un espace pour les individus au croisement de ces intersections;  cette exclusion pourrait permettre d'éviter des comportements d'exclusion basée sur l'un ou l'autre des axes d'exclusion dans un cas de non-mixité.
Cette mixité pourrait en fait ouvrir la porte à un racisme de la part de la communauté \lgbt{}, ou à une forme d'hétérosexisme dans un espace appartenant aux personnes racisées.
On retrouve ici le phénomène d'hybridité; autant le vécu de racisation, d'orientation sexuelle ou de genre participe à la constitution de cette communauté.
Il est donc possible de croire qu'un écart culturel important nuirait à l'idée même de communauté unie, sachant la diversité de la communauté \lgbt{}.

Dans les cas moins politisés, la non-mixité semble viser particulièrement à rassembler \emph{exclusivement}, durant une soirée, des partenaires sexuels potentiels.
À première vue, ces événements n'aspirent pas à d'autres buts que de permettre à des hommes ou des femmes réputés homosexuels d'être en compagnie d'individus semblables et de pouvoir trouver des partenaires pour relations sexuelles ou amoureuses, en plus d'offrir un espace de sociabilité sécuritaire.
Étant donné la restriction basée exclusivement sur le genre, on comprend que l'orientation sexuelle se caractérise dans cette explication comme un élément facilitant les relations intimes, bien que des femmes et des hommes s'identifiant comme hétérosexuels fréquentent certains lieux non mixtes.
En effet, pour ces personnes, l'orientation apparait comme un élément plutôt fluide de leur identité; si les personnes homosexuelles sont perçues comme possédant une identité rigide basée sur une orientation clairement définie, plusieurs femmes et certains hommes hétérosexuels se sentent aptes à s'introduire dans les espaces \lgbt{} sans remettre en question la leur.
Cet état de fait est d'autant plus vrai en ce qui concerne les femmes hétérosexuelles, pour qui l'oppression basée sur le genre dans les espaces hétéronormatifs peut les pousser à fréquenter \lgbt{} des lieux moins risqués~\citep[][9]{Bettani2014}.
Nous assistons donc ici à un comportement assimilable au point de vue \emph{minoritarisant} traité au chapitre 1; les femmes militent en faveur de leur identité de genre pour vivre ce rapprochement au lieu \enquote{d'utiliser} l'identité lesbienne comme dans l'exemple des manifestations de la Marche \emph{Dyke}.

Sur une base historique, on peut également voir cette non-mixité comme une caractéristique propre à un certain type d'événement s'amorçant avant une plus grande acceptation sociale des communautés \lgbt{} comme nous l'avons traité à partir du texte de Sinfield.
Pour plusieurs individus \lgbt{} ayant vécu les luttes \lgbt{} des dernières décennies, la non-mixité peut apparaitre comme une nécessité qui s'est doublée d'une habitude.
Sachant que les rencontres entre individus du même genre pouvaient être un risque à leur sécurité, il importait pour les individus appartenant aux  communautés \lgbt{} de se prémunir de lieux non mixtes pour assurer une sécurité minimale, parfois en imitant d'autres formes de lieux non mixtes comme les tavernes pour hommes, prisés par les hommes hétérosexuels.
Un exemple de ceci est le cas du bar Le Drague à Québec, qui a dû, au cours des dernières années, s'ouvrir à une population mixte après avoir été longtemps un lieu de sociabilité réservé exclusivement aux hommes, comme plusieurs débits de boisson dans la capitale ou ailleurs au Québec \missref{}.
Bien qu'aujourd'hui une partie importante des jeunes \lgbt{} commence à fréquenter des lieux mixtes ou des soirées dans des bars n'étant pas réputés comme \lgbt{}, les membres les plus anciens de la communauté peuvent toujours ressentir un besoin de se retrouver avec des individus similaires dans un but de socialisation.

Si dans le cas d'un bar comme Le Drague, la non-mixité était imposée comme un calque à une tradition répandue --- soit l'exclusion des femmes des bars ---, le contrôle de la non-mixité apparait différemment applicable selon les groupes, si nous portons notre attention à l'ensemble des espaces que nous avons traité jusqu'à maintenant.
Si l'on s'intéresse aux espaces et événements non mixtes pour hommes, bien souvent, il s'agit d'événements situés dans des espaces fermés, et donc non publics.
Ceux-ci peuvent dès lors contrôler de façon constante le genre des individus qui peuvent accéder à l'espace, à la différence de certaines activités pour femmes, mais comparables aux groupes donc la non-mixité est basée également sur la racisation.
Bien que pour ces derniers, les événements organisés ne se déroulent pas dans des espaces très prompts à la socialisation, comme les bars, elles et ils choisissent selon notre collecte des locaux peu ou pas passants et l'accès semble basé sur la confiance.
C'est ainsi que s'effectue la non-mixité des événements publics, bien que ces derniers traversent des lieux où des passants sont présents, étant donné la nature des espaces utilisés --- rues, parcs, etc.

\section{De la nature pour hommes}
\label{sec:de_la_nature_pour_hommes}

Avec une population dépassant toute autre agglomération au Québec, le grand Montréal doit posséder par défaut un nombre d'individus \lgbt{} plus important qu'ailleurs.
Avec le grand degré d'organisation que nous avons constaté à la suite d'autres chercheurs, on remarque que la population masculine, après avoir construit un territoire important qu'est le Village gai, a également développé et occupé d'autres lieux secondaires.
La ville offrant anonymat et possibilités pour des rencontres, elle demeure limitante quant à celles qui sont possibles dans les espaces publics.

Comme nous l'avons soulevé au chapitre précédent, plusieurs endroits ont été investis dans la couronne montréalaise.
Les campings par exemple semblent compléter en partie les avantages offerts par la Village gai pour les hommes.
Dans de tels territoires, les espaces ouverts (sans emmurement physique), comparables aux espaces publics urbains, sont beaucoup moins surveillés, de tels lieux de villégiature étant \latin{de facto} privés.
Selon les symboles utilisés, sois des corps masculins nus, la sexualité est mise au premier plan.
La pratique du nudisme est également encouragée selon les campings.
Dans de tels cas, on s'attend à ce que ce soit essentiellement des hommes qui fréquentent ces lieux.

La proximité de Montréal témoigne de la nécessité d'une population importante pour assurer la pérennité de ceux-ci.
Cette pérennité semble surtout être liée au nombre de personnes potentielles intéressées à investir ces territoires; les régions moins urbanisées ont souvent accès beaucoup plus facilement à des espaces non urbanisés et à l'apparence naturelle.
Nous utilisons le terme naturel en référence au code soulevé au chapitre précédent; en effet, ce caractère naturel est opposé ici à la notion d'urbain plutôt qu'à la notion courante de culture.
Les espaces ruraux sont tout aussi \emph{non naturels} que ceux en ville; par contre, la présence d'un couvert végétal important porte en soi une signification d'une moins grande trace anthropique.

Certains auteurs ont abordé le rapprochement de certains groupes \lgbt{} avec les milieux ruraux.
C'est le cas dans \citetitle{Bell1995a} de Bell et Valentine, cet article traitant les imaginaires de la campagne queer.
Selon le genre et l'identité, certains groupes rechercheraient, dans certains romans ou en regardant des photographies d'époques, différents éléments dans une territorialisation de la campagne.
Les paysages ruraux, avec leur aspect naturel, peuvent évoquer une sensualité par l'usage de métaphore sur la vie sauvage ou le secret dans l'immensité des paysages~\citep[114]{Bell1995a}.
Pour les femmes, les espaces ruraux possèdent la qualité d'être des lieux où il serait possible d'échapper à la domination hétérosexuelle et masculine des espaces urbains.
La nature serait donc un territoire vierge où il serait possible de reconstruire des microsociétés égalitaires, des utopies.

Ce sont en général des éléments que nous avons retrouvés dans les publicités des campings en nature, mais l'absence de données pour les femmes ne peut nous permettre d'affirmer l'existence de ce genre d'utopies au Québec avec certitude.
En même temps, le sens politique d'un tel projet se marie mal avec le média qu'est la publicité; la visibilité d'un tel lieu ne serait pas un but recherché, au contraire.
D'ailleurs, les médias \lgbt{} avec lesquels nous avons fait notre collecte de données n'ont pas servi de plate-forme à faire passer des messages politiques clairs, sauf en ce qui concerne les archives de Pervers/Cité et les éléments trouvés sur \emph{Facebook} publiés par le festival Qouleur, la Fierté Trans et la Marche Dyke.

\section{Des régions peu organisées}
\label{sec:des_regions_peu_organisees}

Nous terminerons notre analyse des résultats avec le constat que l'on peut faire sur les espaces régions.
Comme l'indique le titre de cette section, l'organisation spatiale des territoires en région semble moins développée que dans les agglomérations de Québec ou de Montréal que nous avons pu traiter en profondeur.
En effet, la durée de vie des lieux \lgbt{} s'y trouvant est souvent basse pour les espaces commerciaux alors que ceux qui sont communautaires, s'ils persistent dans le temps, sont peu visibles.
Malgré tout, le magazine Fugues nous a permis de constater l'existence de tels \emph{micros}-territoires, ceux-ci ayant été dénombrés à la fin du chapitre précédent.
On peut penser aux organismes communautaires de Sherbrooke et de Rimouski, alors que les villes de Trois-Rivières et de Saguenay ont connu l'existence de quelques bars ou restaurants.

Historiquement, les espaces ruraux n'ont pas eu la même utilité que ceux en ville pour les individus \lgbt{}, principalement au niveau politique.
Avec la dispersion spatiale, des valeurs réputées plus conservatrices au sein de la population s'y trouvant et le faible nombre de ressources semblent avoir complexifié l'organisation de communauté \lgbt{}.
Comme nous l'avons souligné dans la section précédente, les espaces utopiques où des regroupements \lgbt{} auraient pu subsister n'auraient pas nécessairement besoin d'une visibilité, rendant difficile la possibilité de les situer par le seul usage de médias visuels comme nous l'avons fait dans cette recherche.

Moins régional, le cas de la ville de Québec montre également les signes d'une organisation spatiale plus ou moins organisée, surtout selon les sous-groupes ne concernant pas les hommes homosexuels.
Comme nous l'avons vu au chapitre précédent, la communauté lesbienne, active, peine toutefois à trouver des espaces où organiser ses activités, ou, du moins, vise des lieux peu visibles.
Sachant qu'il n'y avait pas de bars strictement lesbiens comme il a pu en avoir à Montréal au moment de la parution de ces publicités~\citep{Podmore2006}, on peut comprendre en partie la nécessité d'investir des lieux alternatifs qui ne sont pas normalement occupés par la communauté \lgbt{}.
Dans ce cas-ci, les organisateurs ne se sont donc pas tournés vers le bar mixte de Québec, Le Drague.
Il est important de souligner également que ces fêtes sont organisées dans le cadre de la Fête Arc-en-ciel de Québec; nous n'avons pas trouvé d'occurrences d'événements similaires durant le reste de l'année dans le journal \emph{Sortie}.
Ces activités consistent fréquemment en des soirées animées par des disk-jockey et quelques fois sont des ateliers de peinture corporelle.
%%% Local Variables:
%%% mode: latex
%%% TeX-master: "../../memoire-maitrise"
%%% End:
