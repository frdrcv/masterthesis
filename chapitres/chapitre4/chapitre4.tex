%!TEX root = ../../memoire-maitrise.tex

\chapter{De l'inclusion à la différence par le symbole}
\label{cha:de_l_inclusion_la_diff_rence_par_le_symbole}

% \chapterprecishere{\textquote{Everyone needs a place. It shouldn't be inside of someone else.” \par\raggedleft--- \textup{Richard Siken}, Crush}
Après avoir présenté les résultats de la collecte de données dans le chapitre précédent, nous nous procèderons dans celui-ci au portrait d'ensemble soulevé par les données.
Pour chaque géosymbole analysé précédemment, nous avons offert une brève analyse du sémiotique pour faire ressortir les messages que ceux-ci portent, les codes.
Également, l'approche géographique nous a poussé à localiser ces symboles et nous avons ainsi pu décrire du même coup l'étendu et la position des géosymboles rencontrés.
Ces informations, combinées ensemble, dressent plusieurs portraits simultanés vis-à-vis la panoplie des symboles rencontrés.
Par contre, nous croyons qu'il est nécessaire pour la suite, conformément aux points soulevés au chapitre 2, de faire les liens entre ces géosymboles pour arriver, par la suite, à faire une analyse d'ensemble.
Cette analyse sera donc une synthèse de notre démarche qui pourra par la suite servir à faire la comparaison avec d'autres territoires ailleurs, dans d'autres villes ailleurs qu'au Québec.



\todo{Fouiller \cite{Fyfe1988}, contenu pertinent selon \cite[11]{Rose2012}}

\subsection{Communauté imaginée}
\label{sub:communaut_imagin_e}
% Pourra être déplacé ailleurs
Devant ces nombreuses dissensions et absences d'échanges au sein ce qu'on nomme communauté~\lgbt{} ou gaie et lesbienne, on peut s'interroger sur ce qui en est de cette communauté, s'il en est vraiment une.
S'intéressant d'abord au nationalisme, le texte de Benedict Anderson offre une analyse intéressante de la question de nation, alors que celles-ci représentent bien souvent de grands ensembles liés par peu de choses, sinon un lieu de naissance commun et une culture commune.
\begin{quote}	
societies are sociological entities of such firm and stable reality that their members (A and D) can even be described as passing each other on the street, without ever becoming acquainted, and still be connected.37 \citep{Anderson1983}
\end{quote}

\begin{quote}
  That all these acts are performed at the same clocked, calendrical time, but   by actors who may be largely unaware of one another, shows the novelty of this imagined world conjured up by the author in his readers' minds.38
  \citep{Anderson1983}
\end{quote}

\begin{quote}
	
It should suffice to note that right from the start the image (wholly new to Filipino writing) of a dinner-party being discussed by hundreds of unnamed people, who do not know each other, in quite different parts of Manila, in a particular month of a particular decade, immediately conjures up the imagined community.
[\ldots]

Notice too the tone. 
While Rizal has not the faintest idea of his [28] readers' individual identities, he writes to them with an ironical intimacy, as though their relationships with each other are not in the smallest degree problematic.43
\end{quote}


\begin{figure}[ht]
	\centering
	\includegraphics[width=16cm]{fig4.jpg}
	\caption{Fête arc-en-ciel de Québec: marche de solidarité et journée
    communautaire\todo{corriger les étiquettes de noms sans accents}}
	\label{fig:figure4}
\end{figure}

\begin{figure}[ht]
	\centering
	\includegraphics[width=16cm]{fig3.jpg}
	\caption[]{Dyke March: trajet et moments importants de la manifestation}
	\label{fig:figure3}
\end{figure}

% subsection communaut_imagin_e (end)

\section{Premiers symboles liés à la diversité sexuelle}
\label{sec:premiers_symboles_li_s_la_diversit_sexuelle}

% section premiers_symboles_li_s_la_diversit_sexuelle (end)

\section{Symboles politiques et identitaire}
\label{sec:symboles_politiques_et_identitaire}

% section symboles_politiques_et_identitaire (end)

% chapter de_l_inclusion_la_diff_rence_par_le_symbole (end)
%%% Local Variables:
%%% mode: latex
%%% TeX-master: "../../memoire-maitrise"
%%% End:
