%!TEX root = ../../memoire-maitrise.tex
\chapter{Méthodologie}
\label{cha:methodologie}

\chapterprecishere{\textquote{ The growth of the pervert population of Brisbane, beautiful capital of Queensland, is astounding, and in the last year hundreds of these queer semi-feminine men have made the city their headquarters.
Now they have evolved into a cult, with two main sects, one on the north and the other on the south side of the town, with the river dividing them. 
And occasionally they meet at queer, indecent, degrading ceremonies when perverted lusts come into full play and shocking rituals are celebrated} \par\raggedleft--- \textup{The Arrow}, le 4 mars 1932}

Ce chapitre se penchera plus particulièrement sur la méthodologie et nous définirons le cadre de l'étude, sois le lieu et la durée dans lesquels a eu lieu la collecte de données. 

% La recherche se déroulera en plusieurs étapes : d'abord, faire
% un portrait de la création et de l'évolution des géosymboles \qus\ ou de la
% diversité sexuelle, dresser un bref portait historique des milieux de la
% diversité sexuelle au Québec et répertorier les géosymboles dans un
% échantillon varié de villes d'une taille minimale (selon les paramètres
% d'apparitions d'une communauté \lgbt).

% Le travail de recherche se terminera par une analyse de ces géosymboles selon
% leur usage et de la trame sémantique sous-jacente à leur déploiement dans
% l'espace, en tentant de comprendre la relation qui existe entre les
% communautés \lgbt{} et leur spatialité.

\section{Lieu d'étude}
\label{sec:lieu_d_tude}
L'espace couvert par cette étude comprend les milieux urbains fréquentés par et pour les individus ou groupes faisant partie du spectre de la diversité sexuelle. 
Cette diversité sexuelle consiste principalement en les individus non hétérosexuels ou dont l'identité de genre ne correspond pas à la norme sociale hétérosexuelle ou cisgenre\footnote{Terme inventé dans la foulée des luttes pour la reconnaissance des personnes trans et de leurs droits. 
Le terme cisgenre vise à remettre en  question la norme sociale sur l'assignation du genre à la naissance.
Il vise donc à montrer que la concordance entre l'identité de genre et le genre assigné à la naissance d'un individu est une identité parmi d'autres. 
Cette concordance, largement répandue dans la société et considérée comme \emph{normale}, serait aussi valide que peut l'être l'identité  trans, dans laquelle le genre assigné ne concorde pas avec l'identité de l'individu.
D'autres exemples d'identités de genre existent, comme les personnes non binaire dans le genre, sans genre, etc., voir~\cite{Barker2015})~\citep[150]{McGeeney2015}.}. 
Cette définition large vise à rassembler en un groupe les personnes homosexuelles, lesbiennes, bisexuelles, trans, queers, ou encore en non-concordance avec le genre (bien que dans l'analyse, ces groupes seront analysés séparément ou conjointement selon le contexte). 
En considérant le temps alloué à la collecte de données, au contenu de certains contenus d'archives et d'inégalités entre certaines identités dans l'ensemble de la société, nous n'avons pas couvert l'ensemble de ce spectre de façon égale. 
Nous reviendrons plus loin dans le chapitre sur les raisons de cette couverture inégale.

Plus spécifiquement, en concordance avec l'hypothèse émise dans le chapitre précédent, nous considérons que cet espace couvre de multiples lieux dispersés parmi les villes québécoises.
Certaines exceptions sont apparues dans notre collecte, sachant que certains lieux en dehors de la province sont publicisés dans les médias québécois, ceux-ci pouvant intéresser un public québécois. 
Ces espaces ont pris dans notre collecte de données la forme de bars, restaurants, commerces, rues, quartiers ou autres en correspondance avec les études déjà effectuées par d'autres chercheurs sur les pratiques d'organisation et de rencontre des minorités sexuelles~\citep{Higgins1999,Hinrichs2012}.

Selon la hiérarchie des villes québécoises, ce projet de recherche se penchera principalement sur deux groupes d'espaces en particulier, sois ceux des villes de Québec et Montréal. 
Celles-ci se démarquent d'une part par leur importance démographique, politique, sociale et culturelle et d'autre part par leur place dans le réseau urbain, ces deux villes occupant une position centrale. 
Nous nous attarderons de façon secondaire sur d'autres villes où il y a une communauté \lgbt{} organisant des activités publicisées, selon notre collecte de données. 
Nous nous pencherons plus particulièrement sur ces différentes villes plus loin dans ce chapitre à la Section~\ref{ssub:autres_villes}.

Ces diverses publicités référant aux activités des minorités sexuelles ont historiquement prises diverses formes, que ce soit sous la forme de brochures ou de magazine culturistes~\citep{Higgins1999}~\footnote{On retrouve d'ailleurs ce   type de documents et plusieurs autres dans les \agq{}.}. 
Étant donné le cadre temporel de cette recherche, nous avons arrêté notre collecte de données sur les médias les plus importants sur le terme de la distribution et de diversité de contenu, soit les revues Fugues, Sortie et les médias sociaux. 

Étant donné le cadre temporel dans lequel s’enchâsse ce mémoire, nous ne reviendrons pas sur l'histoire des communautés \lgbt{} depuis la colonisation, mais nous tenterons tout de même de cerner brièvement le contexte récent des dernières décennies. 
Ce cadre historique s'intéressera principalement au cas de Montréal étant donné le peu d'écrits faits dans le domaine scientifique sur les autres villes québécoises. 
Un travail en archives plus poussé serait à faire pour celles-ci, mais ceci est au-delà du champ couvert par ce mémoire.

\begin{figure}[ht]
	\begin{center}
		\includegraphics[width=18cm]{fig2.png}
	\end{center}
	\caption{Villes pour lesquelles des géosymboles ont été localisés à partir des
    magazines Sortie et Fugues.}\label{fig:carte_quebec}
\end{figure}

\subsubsection{La ville de Montréal}
\label{ssub:montreal}
Anciennement nommée Hochelaga puis Ville-Marie, Montréal est aujourd'hui la métropole de la province de Québec et la deuxième ville en taille au Canada après avoir été la première durant plusieurs décennies. 
\note{À quel point devrais-je décrire l'historique de la ville de Montréal? à compléter}.


Historiquement, dans la métropole, nous pouvons croire qu'il exista pour les minorités sexuelles une forme d'organisation durant pratiquement toute l'histoire de la colonie canadienne-française jusqu’aux années 1950, organisation marquée avant tout par la clandestinité et l'invisibilité~\citep{Higgins1999}. 
Par son statut de métropole très tôt dans l'Histoire, Montréal rayonnait suffisamment pour qu'une partie des individus des minorités sexuelles connaissent ou côtoient d'autres individus semblables de la ville. 
Cette probabilité est allée en grandissant avec l'amélioration des moyens de communication et l'urbanisation progressive des lieux centraux. 
Peu de données par contre permettent de décrire avec précision les contours de ces rassemblements ou de ces rencontres; bien souvent, la connaissance que nous avons de la vie de ces personnes se résume aux faits divers qu'on retrouve dans les \emph{pages jaunes} de l'époque, des médias imprimés spécialisés dans les informations à sensation. 
\todo{Trouver la citation de Higgins sur les gens qui   trouvaient les lieux LGBT rapidement}\citep[]{Higgins1999}. 
On sait que certains cinémas comme le Midway dans les années 1920 par exemple, auraient été un lieu de rencontres entre hommes\citep[30]{Higgins1999} \todo{a compléter}.
\begin{figure}[ht]
	\centering
	\includegraphics[width=15cm]{carto/mtl.png}
	\caption{Arrondissements ciblés pour la collecte de données: ville de
    Montréal}\label{fig:espaces_montreal}
\end{figure}
Aujourd'hui, Montréal est une métropole bien établie à l'échelle du continent nord-américain. 
Possédant de nombreuses institutions postsecondaires, dont quatre universités, elle attire en son sein une population immigrante importante et est réputée à certains égards pour l'ouverture dont elle fait montre envers la communauté \lgbt{}. 
\todo{à terminer}Le Village gai, un des plus développés au monde, l'existence de festivals comme le \anglais{Black \& Blue}, les \anglais{Outgames}, la Fierté gaie, etc.\@est reconnue.

Les arrondissements du Plateau-Mont-Royal et de Ville-Marie ont d'abord été ciblés pour la présence reconnue de plusieurs lieux \qus{}. 
D'abord, le lieu occupé actuellement par le quartier des spectacles a été il y a plusieurs décennies le \anglais{Red Light} du centre-ville et plusieurs cabarets ont été les premiers lieux de rassemblement d'individus \lgbt{}~\citep[198]{Podmore2015}.
Pas vraiment un lieu de communauté et de sécurité, le \anglais{Red Light} et ses cabarets servaient plutôt de lieu de travail pour plusieurs travailleurs et travailleuses du sexe appartenant a la diversité sexuelle. 
La sécurité de ces personnes était en effet minée par la criminalité et une hétérogénéité de la clientèle qui pouvaient être source de violences homophobes et transphobes~\parencite[91]{Higgins1999}. 
De cet espace, une partie de la \textquote{communauté} --- d'abord des investisseurs puis la clientèle --- s'est dirigée vers le quartier sud dans le même arrondissement pour créer ce qui allait devenir le Village gai. 
Une caractéristique intéressante est la proximité de cet espace d'institutions scolaires importantes comme l'\uqam{} et l'Université McGill; ces universités seront occupées durant ces mêmes années par divers groupes étudiants et politiques qui mèneront en partie plusieurs des luttes politiques pour la reconnaissance des droits et de la visibilité des gais et des lesbiennes.

Une autre partie de la clientèle \lgbt{} du \anglais{Red Light} s'est dirigée plus au nord, principalement les femmes. 
Nous savons aujourd'hui que plusieurs lesbiennes ont investi pendant près de deux décennies le Plateau Mont-Royal, principalement autour du boulevard Saint-Laurent~\citep[599]{Podmore2006} étant donné la présence de nombreux bars réservés aux femmes qui existèrent durant les années 80--90. 
Par contre, leur présence aujourd'hui s'est amoindrie; l’embourgeoisement et une certaine compétition avec le Village gai --- dont certains bars sont devenus mixtes et ont réussi à attirer une nouvelle clientèle lesbienne plus jeune --- en seraient en partie la cause~\citep{Podmore2015}.

La figure~\ref{fig:espaces_montreal}\footnote{Afin de faciliter la lecture de la   vue d'ensemble de la ville de Montréal et pour ne pas encombrer la liste des   acronymes au début de ce mémoire, les noms complets des arrondissements de   Montréal se trouvent en annexes.\todo{à faire}} montre la position de ces deux arrondissements et nous verrons dans les chapitres suivants où ces espaces ont été situés en complément des données accumulées dans cette recherche \todo{approfondir le contexte historique?}

\subsubsection{La ville de Québec}
Capitale de la province de Québec, la ville de Québec est également considérée comme la plus ancienne ville fondée par les Européens lors de la colonisation du continent américain. 
Anciennement, cet espace situé au nord de la rive du Saint-Laurent était connu sous le nom de Stadaconé et était un établissement iroquoien~\citep[91]{Dickason1996}. 
La ville de Québec en soi, fondée par des colonisateurs français, devient une possession britannique durant le \siecle{18} tout en demeurant un lieu essentiellement habité par des Canadiens français. 
Une minorité anglophone protestante s'est également établie dans un secteur désigné sous le nom de haute-ville de Québec avec le changement de pouvoir\missref{}.

Si la ville a longtemps été limitée aux quartiers centraux, aujourd'hui, la ville de Québec possède une diversité importante de quartiers différents nés de la fusion d'anciennes municipalités et d'une croissance importante à partir du milieu du \siecle{20}. 
Un des legs de ces fusions de territoires a été le déplacement d'institutions publiques importantes, comme l'Université Laval dans le quartier Sainte-Foy.
Un des quartiers  centraux et anciens, le quartier Saint-Jean-Baptiste, semble avoir été reconnu comme l'espace de vie des minorités sexuelles dans la capitale selon certaines entrevues avec des individus de la communauté gaie actuelle~\citep{CSJB2011}. 
Anciennement ouvrier, il s'agit aujourd'hui d'un quartier morcelé par la rénovation urbaine, la patrimonialisation et par un processus d’ embourgeoisement~\citep{Hatvany2005,Mercier2014}.

Un espace important occupé par la communauté \lgbt{} a été le quartier Saint-Roch qui a également été habité et fréquenté par les minorités sexuelles.
Autre espace morcelé, on le considère aujourd'hui comme le centre-ville de la ville de Québec. 
Plus avancé dans son processus d’embourgeoisement, de nombreux espaces \lgbt{} comme des bars ont existés sur ses rues importantes, principalement la rue de la Couronne et aux environs. 
Aujourd'hui, ses fonctions ont changé et l'on y retrouve principalement des locaux loués par les organismes communautaires appartenant à la communauté \lgbt{} de Québec.

À notre connaissance, pratiquement aucune recherche ne s’ est encore vraiment penchée à la population \lgbt{} de la capitale sauf pour un travail de recherche. 
Un mémoire de baccalauréat s'est intéressé à cette question et a permis de dénombrer plusieurs lieux où la communauté serait active.
Nous ne possédons pas par contre plus d'informations sur l'histoire de ces lieux, un exercice de recherche qui serait important à faire dans le futur. 
Si certains bars de la communauté semblent avoir disparu du paysage de la ville de Québec dans les dernières années dans les espaces décrits précédemment, de nombreux espaces fréquentés par les communautés \lgbt{} existent encore aujourd'hui. 
Certains sont maintenant présents en périphérie du centre-ville avec la dispersion importante des Universités de Québec --- L'Université Laval et l'\uqar{} (qui est plutôt à Lévis) --- et des cégeps. 
Ils n'ont donc pas fait l'objet d'une collecte de données sur le terrain.

Pour l'instant, au su des informations précédentes, nous avons décidé de cibler nos efforts sur le centre-ville de Québec, dans l'arrondissement La-Cité-Limoilou, visible à la figure~\ref{fig:espaces_quebec}. 
Il s'agit de l'espace dans la ville de Québec où l'on retrouve le plus de lieux appartenant à la communauté \lgbt{} locale. 
C'est d'ailleurs l'espace choisi par l'organisation Alliance Arc-en-ciel de Québec pour les célébrations de la Fête Arc-en-ciel, le pendant local des fiertés gaies.

\label{ssub:la_ville_de_quebec}
\begin{figure}[ht]
	\centering
	\includegraphics[width=15cm]{carto/qc.png}
	\caption{Arrondissement ciblé pour la collecte de données: ville de
    Québec}\label{fig:espaces_quebec}
\end{figure}


\subsubsection{Autres villes}
\label{ssub:autres_villes}
D'autres villes ont été envisagées pour ce travail de recherche. 
Par contre, comme il a été mentionné au départ de ce chapitre, l'expérience terrain a été limitée aux deux espaces décrits précédemment. 
Néanmoins, comme souligné précédemment, des données ont été trouvées dans plusieurs autres villes du Québec. 
Nous pensions déjà avant la collecte de données qu'il serait possible de traiter des villes de Rimouski et de Sherbrooke selon nos propres connaissances. 
Plus précisément, nous savions déjà à titre de membre de la communauté \lgbt{} que ces villes possèdent respectivement une communauté plus ou moins active. 
Nous envisagions donc de vérifier si ces villes sont des cas d'exception ou si celles-ci possèdent en raison de leur taille et leur place dans un réseau plus large de villes les caractéristiques nécessaires à l'apparition d'une communauté \lgbt.
L'usage de certains médias de la communauté que nous traiterons à la section \ref{sec:source_des_donn_es} nous a effectivement permis de confirmer cette intuition; nous avons dressé à la figure~\ref{fig:carte_quebec}. 
Selon les résultats de notre analyse de données présentée dans les prochains chapitres, nous arriverons en toute fin à dresser une hiérarchie des villes selon la complexité et la visibilité de leur communauté respective.

\section{Cadre temporel}
\label{sec:cadre_temporel}
En ce qui concerne le cadre temporel, nous abordons l'époque contemporaine en couvrant le \siecle{20} au maximum, sachant que le sujet d'étude est particulièrement récent et que la majeure partie des données proviendront du dernier demi-siècle. 
En effet, la mise en contexte du sujet nécessite une prise en considération de l'évolution historique des communautés formées par les minorités sexuelles. 
ESelon les circonstances historiques décrites dans la littérature~\citep{Spencer2005}, on peut estimer que les géosymboles de la diversité sexuelle actuelle dateraient au maximum des luttes ayant suivies les émeutes de Stonewall aux États-Unis.
Ces communautés par contre existeraient depuis plusieurs décennies avant d'avoir été porteuses d'un discours politique et d'engendrer un mouvement civique de grande ampleur.

Par contre, en ce qui concerne les données collectées pour répondre à la question de recherche, la couverture temporelle est beaucoup plus courte et récente: nous couvrons les dix dernières années pour arriver à dresser un portrait actuel des géosymboles des communautés \lgbt{}. 
Les données étant particulièrement variées dans leur provenance, certaines ont été collectées durant les mois précédents la rédaction du présent mémoire, alors que d'autres proviennent d'archives conservées et couvrant toute cette décennie. 
Nous décrirons plus en profondeur la couverture temporelle de ces données dans les Sections~\ref{sec:source_des_donn_es} traitant des sources de données.

\section{Approche méthodologique}
\label{sec:approche_m_thodologique}
La découverte et l'analyse des géosymboles d'un groupe culturel donné ne s'appuient pas sur une méthodologie particulière sinon une observation approfondie du groupe étudié et de sa relation particulière avec le territoire. 
Nous croyons par contre que certaines approches méthodologiques sont plus appropriées selon les contextes de recherche et les groupes étudiés.

La question de la visibilité et de la présence en plus de l'acceptabilité sociale sont récurrentes dans l'histoire récente des minorités sexuelles en occident. 
Nous croyons donc que s'appuyer sur cette particularité culturelle devrait être un des motifs derrière le choix de l'approche méthodologique adoptée dans ce travail. 
La géographie visuelle semble ici la réponse à cette préoccupation. 
En priorisant les documents visuels comme matériau d'analyse, elle s'inscrit dans le champ plus large de l'analyse qualitative.

Contrairement à d'autres travaux en géographie culturelle, cette recherche va s'appuyer principalement sur l'observation et la recherche dans des archives plutôt que des entrevues avec des individus impliqués dans le sujet de recherche. 
Dans cette section, nous nous intéresserons à cette méthode alternative qui devraient nous permettre de faire ressortir les différents géosymboles qu'on conçoit peupler les espaces urbains.

\subsection{Analyse visuelle}
\label{sub:analyse_visuelle}
L'analyse visuelle des territoires est profondément ancrée dans la discipline de la géographie. 
En effet, pour plusieurs penseurs\missref{}, la géographie, comparativement à d'autres disciplines en sciences humaines, demande du chercheur qu'il se déplace sur son terrain d'étude pour pouvoir se l'approprier visuellement et arriver à en faire une analyse juste. 
Si elles ne sont pas toujours présentes dans les travaux des chercheurs, nombreuses sont les études de cas à intégrer des photographies des espaces étudiés, que ce soit en géographie physique où l'image peut servir à montrer au lecteur les différentes composantes du sous-sol ou pour la géographie humaine, à montrer un paysage ou une organisation spatiale humaine particulière. 
Également, un des outils de prédilection de la géographie est la carte pour la présentation de données, de plus en plus remplacée --- ou améliorée --- par les \sig{} qui remplissent cette fonction en intégrant des éléments d'analyses alimentés par des algorithmes. 
Par contre, si le visuel est aussi important, peu de travaux utilisent la photographie pour une raison autre que la démonstration~\citep[151]{Rose2008}.
Bien que la description peut servir des buts pertinents, comme la démonstration de l'évolution d'un espace dans le temps ou encore pour appuyer un argument, d'autres usages existent\parencite[158]{Rose2008}. 
Nous commencerons donc cette partie de l'approche méthodologique par l'utilisation de l'analyse visuelle dans notre recherche. 
La section suivante se penchera plus particulièrement sur l'usage des \sig{} dans la géographie culturelle.

Une des volontés derrière cette recherche est de poursuivre l'utilisation et l'expérimentation des méthodes visuelles entamées par d'autres chercheurs durant la dernière décennie. 
Au-delà d'un simple renouement avec une pratique traditionnelle, nous considérons qu'il s'agit d'une méthodologie qui a le potentiel de faire le pont avec la théorie géographique, plus particulièrement en géographie culturelle vers une des composantes importantes des groupes culturels minoritaires ou marginalisées, la visibilité. 
De plus, les géosymboles que nous avons traités dans le dernier chapitre ont comme caractéristique d'être des symboles visuels, matériels ou immatériels, qui permettent d'articuler un territoire propre à groupe donné. 
Ainsi, les géosymboles jouent en quelque sorte le rôle des images de l'analyse visuelle. 
\citeauthor{Rose2012} caractérise d'ailleurs les images et les pratiques visuelles d'une façon comparable à la définition que nous avons des géosymboles, à ce savoir que : \foreignblockquote{english}[{\cite[32]{Rose2012}}][.]{\textelp{} the spaces and   practices of display \textelp{are} especially important to bear in mind given   the increasing mobility of images now; images appear and reappear in all sorts of places, and those places, with their particular ways of spectating, mediate the visual effects of those images}.
Repérer ces géosymboles pourrait se faire en travaillant directement avec les populations données, par l'observation et l'entrevue par exemple, méthode prisée dans la plupart des travaux déjà effectués \citep[][pour ne citer que ceux-ci]{Giraud2014, Podmore2015a, Higgins1999}. 
Mais nous pensons que le processus d'analyse du territoire d'un groupe peut se faire par un point de vue inversé, en nous intéressant d'abord aux géosymboles que l'on retrouve préalablement dans un territoire et faire le pont entre ceux-ci et les travaux déjà effectués sur l'histoire, la politique ou la sociologie et surtout la géographie \lgbt. 
Autrement dit, nous envisageons aborder directement le territoire tel qu'il se présente matériellement et dans les médias \lgbt{} et utiliser les méthodes visuelles pour approcher les géosymboles, le tout en nous appuyant sur des technologies comme les applications cellulaires et les \sig.

% La géographie culturelle continua son existence durant ces mêmes décennies
% selon plusieurs sous-disciplines, comme le tropicalisme ou les études plus
% régionales comme au Québec.

Une auteur importante à avoir travaillé sur le domaine de l'analyse visuelle est Gillian~\citet{Rose2008} dont les travaux détaillent plusieurs façons dont les géographes utilisent les images dans leurs analyses. 
En plus de l'usage de description décrit précédemment, les images pourraient également être utilisées comme un outil de représentation, d'évocation ou encore servir de fragment de culture matérielle. 
Ce dernier point est également soutenu par \citet{Frosh2001} pour qui l'image (et plus encore la photographie) est un fragment d'une performance sociale de représentation qui mérite analyse. 
Son pouvoir particulier de représenter des pans de la réalité donne au photographe un pouvoir particulier sur le sujet ainsi que sur l'observateur de la photographie.
Il apparait donc nécessaire dans notre recherche de prendre en compte cette réalité de deux façons : d'abord, reconnaître que les documents visuels analysés tirés des médias sociaux et des archives sont le fait d'individus possédant une certaine vision de la communauté /lgbt{} ou du moins, cherchent à la présenter d'une certaine façon, consciemment ou non. 
Ensuite, utilisant la photographie comme outil pour situer nous-mêmes certains géosymboles, nous avons nous-mêmes un biais envers l'objet étudié, malgré notre volonté de demeurer objectif, surtout dans un contexte où le sujet est social et culturel. 
Nous ne croyons pas par contre qu'il s'agit d'une faiblesse de notre perspective méthodologique, mais plutôt une des particularités des méthodes qualitatives. 
Nous reconnaissons toutefois que dans des recherches subséquentes, il serait également intéressant d'utiliser des photographies produites consciemment dans le cadre d'une recherche, mais par les individus appartenant au groupe culturel visé. 
Cette méthode a également fait ses preuves, notamment dans les travaux de \citet{Kwan2008},~\citet{Moore2008} et de~\citet{Markwell2000}.

En ce qui concerne la collecte de données sur le terrain, notre méthode se rapproche plutôt de l'article de~\citet{Leroy2010} dans lequel la photographie est utilisée pour montrer la diversité des représentations présentes dans la Fierté gaie parisienne. 
Si celui-ci ne décrit pas particulièrement comment l'échantillonnage a été créé, nous croyons qu'il est nécessaire ici de faire cet exercice, tel que souligné par \citet[109]{Rose2012} qui constate que trop souvent, les travaux en sémiotique ne s'attardent pas suffisamment sur le contenu sélectionné, ne mettant l'accent que sur le contenu intéressant à analyser.


% \todo{Parler de Suchar} La méthode de collecte de données s'inspire
% essentiellement de la technique

% \citep{Rose2012} \citep{Rose2008} \citep{Rose2003} \citep{Dorrian2003}
% \citep{Suchar1997} \citep{Frosh2006} \citep{Frosh2001}

L'analyse des données visuelles --- principalement celles qui ont été collectées dans les données d'archives --- va s'inspirer de différentes questions amenées par \citet[157]{Rose2008}, à savoir:
\begin{itemize}
	\item Qui utilise ces photographies, comment et pourquoi?
	\item Est-ce que l'usage de ces photographies à un effet particulier (sur les
    sujets, les auditeurs, etc.)?
	\item Où ces photographies ont-elles été prises?
	\item Est-ce que la localisation du sujet de la photographie est à prendre en
    compte? Comment?
	\item Et enfin, quel est l'impact des photographies sur les lieux où elles ont
    été prises, mais également sur les lieux où elles sont utilisées ou diffusées?
\end{itemize}

Ainsi, nous croyons que les différentes réponses apportées pourront ensemble dresser un portrait de la territorialité des groupes \lgbt{} du point de vue de la visibilité. 
Dans la section suivante, nous verrons comment ces imaginaires visuels pourront être situés dans l'espace, apportant ainsi des nuances ou des approfondissements aux différents portraits que nous dresserons de ces groupes.

\subsection{Géolocalisation du géosymbole}
\label{sub:g_olocalisation_du_sybole}
Ce n'est pas toutes les branches de la géographie qui s'attardent ou qui se sont attardées aux données visuelles, surtout depuis l'avènement de la géographie quantitative, constaté durant la deuxième moitié du \siecle{20} et du bond technologique apporté par l'informatisation. 
Le travail à partir de bases de données statistiques et géoréférencées permit également aux chercheurs de prendre un certain recul vis-à-vis du sujet d'analyse et du même coup prendre une distance avec la partie terrain de la collecte de données. 
On doit tout de même souligner que si ces nouveaux outils informatiques permirent de couvrir des ensembles spatiaux beaucoup plus importants que précédemment, comme on peut le constater dans la branche de la géographie utilisant les méthodes d'analyses spatiales.\note{à garder ou impertinent?}

L'utilisation des \sig{} en géographie culturelle n'est par contre pas encore très répandue; on les retrouve par contre fréquemment utilisés dans le cadre des analyses quantitatives en géographie urbaine, économique et dans les disciplines affiliées comme l'économie \missref{}. 
Une des volontés derrière cette recherche est d'arriver méthodologiquement à faire un pont entre ces techniques propres à la géographie quantitative et celles plutôt utilisées en géographie culturelle dans lesquelles la carte comme résultat d'un \sig{} joue plus souvent le rôle de description d'un espace à analyser. 
Nous croyons comme plusieurs autres auteurs \citep[4]{Elwood2009} que les \sig{} peuvent être un outil pratique à la compilation de données de sources multiples tout en permettant une localisation souvent très précise, surtout lorsque le chercheur travaillant à la collecte a accès à des adresses postales ou encore à un \gps.
Ce but est depuis peu partagé par d'autres chercheurs en géographie culturelle et en géosciences~\citep{Perkins2003,Elwood2011,Elwood2009,Kwan2008,Madden2009,Knigge2006,Jung2010}.


Comme le souligne~\cite{Kwan2008}: \foreignblockquote{english}[{\citeyear[444]{Kwan2008}}][.]{GIS-based data analysis, mapping, and visualization are deployed to complement or triangulate (i.e., verify results using multiple data sources) the knowledge acquired through the qualitative component of the research} 
Il devient ainsi possible d'ajouter certaines couches d'informations à d'autres dans un but d'enrichissement, en superposant une photographie à un lieu géolocalisé sur un \sig{} ou encore positionner une photographie dans l'espace et pouvoir comparer la position de chaque photographie entre elles. 
Dans un contexte multimédia, cette caractéristique peut s'appliquer à plusieurs types de médias, comme la photographie ou l'audio, bien que ce type de données s'intègre mal à l'écriture dans le contexte de la rédaction d'un mémoire ou d'un article scientifique.

En plus de ces caractéristiques \citet{Elwood2011} s'est attardée à la définition de la géovisualisation comme méthode qualitative. 
Dans un but de comparaison dépassant le simple type de données traitées, Elwood souligne que contrairement à l'utilisation classique des \sig{} dans un contexte quantitatif: \foreignblockquote{english}[{\cite{Elwood2011}}][.]{\textelp{} what defines   these approaches as qualitative geovisualization is not absence of numeracy.
  Rather, it is their integration of multiple modes of representation –--  visual, textual, numerical --– and iterative interpretive analysis of these representations to tease out what they reveal about social and material situations. 
Most of these qualitative geovisualization methods emerge from qualitative GIS, but could clearly be applied to georeferenced multimedia drawn from the geoweb}. 
Notre recherche comprendra elle-même une multitude de médias visuels d'origines multiples, dont des données en ligne que l'on pourrait assimilé au \emph{géoweb} comme souligné dans cette citation d'\citeauthor{Elwood2011}. 
L'avantage des \sig{} sera d'incorporer ces images et en même temps de les situer pour mieux rendre compte de la dispersion de ces représentations et en même temps de simplifier la tâche de garder en mémoire la localisation des symboles trouvés par nous-mêmes sur le terrain.

À la suite de l'observation des espaces urbains ciblés et du travail en archives, l'ensemble des données on subit un traitement de géolocalisation pour la plupart et de codage pour la totalité. 
Nous avons décidé de ne pas effectuer de géoréférencement précis pour les symboles accumulés à partir des médias imprimés archivés par manque de qualité dans certaines données. Dans certains cas, des adresses manquantes dans les symboles et des fermetures/dissolutions au fil des années nous ont empêchés de trouver la localisation de chacun des espaces. 
De plus, certains symboles ne concernaient pas nécessairement des espaces en particulier, mais étaient des messages lancés à la communauté par des organisations hors communauté ou dont la localisation n'était pas, ou ne semblait pas \latin{a priori} pertinente. 
Nous pouvons penser par exemple aux messages publiés pas certains syndicats ou paliers de gouvernements nationaux.

Les données que nous avons donc géoréférencées consistent en des photographies prises durant les événements \lgbt{} auxquels nous avons participé et les données collectées sur les réseaux sociaux. 
Le géoréférencement s'est déroulé en plusieurs étapes à l'aide de plusieurs outils et services différents. 
D'abord, à l'aide principalement du logiciel GIS Cloud et accessoirement de Google Photo, nous avons accumulé des photographies des géosymboles dont le géoréférencement a été effectué à l'aide du \gps{} intégré au cellulaire utilisé pour la collecte de données. 
GIS Cloud était la solution retenue au départ et devait servir tout au long de la collecte, de façon exclusive. 
Par contre, l'usage de cette solution s'est avéré moins concluant qu'envisagé au départ. 
D'abord, il faut savoir que, comme spécifié sur la page d'accueil de leur site internet, GIS Cloud consiste en un service de collecte sur le terrain et d’emmagasinage de données, en plus d'être utilisé pour la publication \citep{Cloud2014}. 
Plus spécifiquement, un des avantages de Cloud GIS pour la collecte de données sur le terrain dans le cadre d'une recherche utilisant des méthodes qualitatives est de permettre la construction d'un guide d'entretien similaire à celui utilisé par exemple dans le contexte d'entretiens enregistrés. 
Avant la collecte, il est en effet possible pour le chercheur de construire des questions telles quelles seront utilisées pour la collecte.
Celles-ci peuvent être par exemple des questions fermées ou des questions ouvertes, ainsi que des questions qui sont en fait des objets capturés par le périphérique utilisé. 
Ainsi, dans le contexte de notre recherche, nous avons monté un questionnaire comprenant plusieurs questions sur le contexte de chaque point localisé, visible en annexes à la figure~\ref{ann:cloudgis}.

En pratique, lors du terrain, le logiciel s'est montré gourmand en ressources (avec pour conséquence une faible anatomie de l'appareil) et nécessitant une attention parfois difficilement conciliable avec le déroulement de l'activité en cours. 
Par exemple, dans les cas où nous avons eu à faire l'observation d'espaces comme le Village gai, nous avions tout le temps disponible pour nous arrêter devant un géosymbole potentiel, l'analyser, prendre des notes et faire un bon usage du questionnaire. 
Par contre, lors de manifestations par exemple, durant lesquelles plusieurs individus tenaient des pancartes, qu'un discours était prononcé ou que les individus en-dehors de l'événement réagissait, il devenait difficile de tout prendre en note à l'aide du seul logiciel. 
Par adaptation, il a été nécessaire de modifier notre usage de l'application pour poursuivre la collecte de données de façon efficace. 
D'abord, nous avons décidé de nous servir de l'application Cloud GIS seulement à une reprise, soit à chaque moment explicite d'un événement pour marquer un point dans la base de données. 
Par la suite, il devenait possible de prendre des photos normalement en dehors de l'application et de collecter des notes de terrain par nos propres moyens en tenant de l'heure et du contenu des notes et du géoréférencement des données présentes dans l'application cellulaire. 
Cette méthode s'est avérée plus efficace sachant qu'il était maintenant possible de faire un usage prolongé du cellulaire pour la collecte de données sans avoir à constamment synchroniser nos données collectées avec une base de données.

Lors du traitement des photographies, nous avons remarqué que l'ensemble des photographies prises sur cellulaire étaient déjà géoréférencées par défaut par le système d'exploitation de l'appareil, ce qui nous permit d'accélérer une partie du processus de localisation et d'augmenter la fiabilité du positionnement des données récoltées. 
Finalement, grâce à cette fonction, nous avons obtenu un résultat similaire à ce qui avait été prévu au départ, c'est-à-dire le géoréférencement automatique des données. 
\todo{faire attention,   peut-être fusionner cette partie avec la section plus loin traitant des données, vu qu'une partie des informations se recoupe}

En ce qui concerne les données d'archives, le géoréférencement a été fait de façon manuelle à partir des adresses postales. 
Nous nous doutions que nous n'aurions pas toujours accès aux coordonnées des lieux ciblés dans la collecte de données, mais nous croyions alors qu'il devrait être possible d'obtenir celles-ci à l'aide de certaines sources de données, comme les répertoires de Fugues ou encore Google Maps qui réussissent normalement à garder en mémoire les lieux fermés, mais possédant préalablement une entrée dans leur base de données à l'époque de leur fonctionnement. 
En effet, étant donné la couverture assez large de collecte de données sur dix années, certains établissements sont apparus et disparus, aujourd'hui remplacés par d'autres du même acabit ou d'espaces totalement différents. 
En fait, si certains géosymboles, comme les publicités, risquaient d'avoir des adresses intégrées, sachant qu'une publicité incite normalement le client potentiel à se déplacer sur les lieux du commerce ou du service, d'autres, comme des photos toutes simples, devaient être situées à l'aide du contenu qu'il supporte.  
Dans le cas de documents comme le magazine Fugues, le contenu visuel est pratiquement toujours accompagné de textes comme des articles ou du moins des titres pouvant fournir de genre de données pour la localisation, mais en général nous disposions peu de données réellement utiles à la géolocalisation des données. 
À l'aide des répertoires qui sont compris dans ce média, nous avons tout de même pu situer géographiquement une certaine partie de ces documents.

Dans le même cadre d'idée, la participation à des activités de la communauté \lgbt{} lors des différents événements qui se sont déroulés durant le terrain a permis, durant la planification, d'accéder à des données supplémentaires sur les réseaux sociaux. 
Nous savions préalablement à titre individuel que de nombreux événements sociaux sont organisés aujourd'hui à partir de certains sites web comprenant de telles fonctions. 
Le réseau retenu pour l'analyse est Facebook qui permet l'organisation d'événements et de leur publication auprès d'un grand nombre d'individus. 
Ces événements permettent de situer les différents événements, autant dans le lieu que dans l'espace tout en étant une plateforme pour publier des images porteuses de géosymboles.
\citep{Barkhuus2010} \citep{Boyd2010}

Finalement, nous avons réussi à géoréférencer une majeure partie de ces données variées en un seul \sig{} et ainsi permettre d'offrir une lecture visuelle des multiples espaces \qus{} des espaces urbains.

\section{Source des données}
\label{sec:source_des_donn_es}
Comme énoncé dans la section précédente, nous avons eu recours à diverses sources de données. 
Celles-ci ont été envisagées et retenues dans le but d'arriver à couvrir l'ensemble du spectre des minorités sexuelles, selon le niveau d'activité de chaque groupe ou sous-groupe tout en accumulant selon les nécessités et la découverte de ces sources de données d'autres données. 
Notre processus de collecte de données s'apparente d'une certaine façon aux méthodes de la théorie ancrée, dans lesquelles les chercheurs alimentent d'abord leur travail de théorisation à partir de premières données collectées pour ensuite continuer et améliorer la collecte selon les prémisses soulevées par une première analyse de ces données. 
C'est d'ailleurs cette méthode qui nous a poussés à inclure certaines données non prévues au départ de cette recherche, comme les données trouvées sur les réseaux sociaux. 
Également, choisir dès le départ les sources de données demeurait un choix ardu selon les modalités de notre recherche. 
Couvrir l'ensemble du spectre \lgbt{} ne pouvait se résumer à notre avis à un seul média étant donné le risque de biais dans notre recherche.

Ceux-ci étant en général considéré comme marginalisées, certains de ces le sont plus que d'autres et des enjeux de pouvoir particulier existent entre elles, alors que certaines minorités ont obtenu une plus grande sympathie au sein de la société et disposent ainsi de moyens communicationnels bien différents que d'autres minorités. 
Ainsi, la diversité de média permet d'éviter certains biais que l'on croie exister au sein de certains médias ou certaines manifestations de la présence d'identités particulières. 
Il ne s'agit pas ici de faire une critique du public visé de certains médias, au contraire: certaines manifestations géosymboliques sont le fruit de certains groupes minoritaires pour des raisons particulières, par exemple une meilleure reconnaissance dans la loi qui n'affecte pas d'autres groupes marginalisés. 
Également, l'analyse des communautés \lgbt{} au Québec est amorcée depuis déjà quelques décennies. 
Nous nous servirons de ces nombreux travaux dans nos analyses.

\subsection{Données d'archives}
\label{sub:donn_es_d_archives}
Nous décrirons donc maintenant plus en profondeur ces diverses sources de données. 
Les sources secondaires sont composées, d'une part, de brochures, affiches et magazines gais et lesbiens présentement en circulation au Québec et, d'autre part, des données référencées des \agq{}. 
Ces archives, situées dans la ville de Montréal, ont le mandat de: \blockquote[{\cite{LAGQ2014}}][.]{\textelp{} de recevoir,   conserver et préserver tout document manuscrit, imprimé, visuel, sonore, et   tout objet témoignant de l'histoire des gais et lesbiennes du Québec   (\textsc{Canada})} et sont parmi les seules au Québec à disposer de telles données. 
Étant donné cet isolement institutionnel, les données qu'on peut y trouver sont nombreuses et débordent le cadre de ce travail de recherche. 
Dans ce cas où les données abondent et dont le traitement nécessiterait un temps et un effort qui n'apporterait pas d'informations supplémentaires dans le cadre du mémoire de recherche, nous avons décidé de limiter la collecte de données à un nombre restreint de médias. 
Avec l'aide du personnel et de la professeur Julie Podmore, des sources ont été sélectionnées pour leur pertinence et leur statut récent lors d'une première visite.

Le premier média sélection est le magazine Fugues. 
Celui-ci, selon son site internet, se décrit comme suit: \blockquote[{\cite{LesNitram2015}}][.]{
  Fondé à Montréal par les Éditions Nitram, Fugues est le plus important média gai au Québec. 
Depuis sa fondation en 1984, Fugues jouit d’une notoriété et d’une crédibilité qui n’a cessé de croître au fil des années. 
On y retrouve toute l’actualité gaie d’ici et d’ailleurs, ainsi qu'une foule de rubriques et chroniques. 
En livrant chaque mois un contenu éditorial fiable sur l’actualité et les enjeux de la communauté GLBT, Fugues permet aux gais et aux lesbiennes de la région de Montréal et du reste du Québec,[sic] de rester informé[sic] sur ce qui concerne spécifiquement leurs communautés. 
C’est pourquoi, plusieurs générations de gais et de lesbiennes du Québec apprécient beaucoup ce magazine et lui sont fidèles depuis trente ans} 
Comme on peut le noter dans cette description, le magazine privilégie un point de vue montréalais sans pour autant omettre l'activité des communautés \lgbt{} des autres villes québécoises, ce qui nous permet ainsi de trouver une grande diversité de données et de toucher à de nombreuses villes, plus qu'aucune autre source de données. 
L'ancienneté du média nous permet également de couvrir l'entièreté de l'époque désignée, sois les années entre 2005 et 2015. 
À raison de douze numéros par années, c'est presque 120 numéros qui seront analysés pour la collecte de données. 
Nous avons décidé d'arrêter la collecte au mois d'août 2015, malgré qu'a posteriori, la diversité de données recherchée dans cette recherche a été atteinte plus tôt.

% \begin{quote}
%   Fugues est offert sur différentes plateformes. La version imprimée est
% 	 publiée 12 fois par année et comprend, outre une synthèse de l’information
% 	 et nos suggestions de sorties pour le mois, des analyses, des débats
% 	 d’idées et de nombreuses chroniques.

%   Le site web Fugues.com, quant à lui, suit l’actualité de plus près, et
% 	 propose plusieurs galeries de photos et de vidéos. Des nouvelles viennent
% 	 alimenter quotidiennement le site et ces articles, régulièrement cités dans
% 	 d’autres médias, sont repris sur les réseaux sociaux.~\citep{LesNitram2015}
% \end{quote}

Le deuxième média sélectionné est le journal Sortie:
\blockquote[{\cite{AllianceArc2014}}][.]{ Communautaire et participatif, le journal Sortie est produit par l’Alliance Arc-en-ciel de Québec dans le but d’informer la population sur les réalités et les droits des personnes LGBT+.
  Il a pour mission de traiter des enjeux et des événements en lien avec la lutte contre l’homophobie et la transphobie. 
Cinq éditions paraissent chaque année, chacune imprimée à 10 000 exemplaires couleurs de format tabloïd. 
Elles sont distribuées gratuitement dans plus de 200 points stratégiques de Québec.
  De plus, son édition présente et ses archives se retrouvent en version intégrale sur le présent site web} 
Possédant moins de moyens que le magazine Fugues étant donné la vocation communautaire du journal, il s'agit tout de même d'une des sources les plus complètes que nous traiterons dans cette recherche.
En effet, le journal Sortie s'étend sur sept années, soit de mai 2007 à décembre 2014. 
Le nombre de numéros fluctue d'année en année; nous avons couvert dans cette recherche 32 journaux. 
Par contre, si le journal n'est pas officiellement en arrêt de publication selon le site internet de l'Alliance Arc-en-ciel, aucun numéro n'a été produit durant l'année 2015, dernière année couverte par cette collecte de données. 
Le choix de ce journal est sa position centrale à Québec, faisant contrepoids à la couverture plus montréalaise du magazine Fugues.

Pas un média à proprement parler, il a été convenu durant la sélection des sources de données d'inclure les archives du festival Pervers/Cité. 
Cette décision a été prise pour représenter également les milieux dits alternatifs existants au sein des minorités sexuelles québécoises. 
Ces archives sont les moins volumineuses et possèdent certaines lacunes: nous disposons de données que pour les données 2011, 2014 et 2015. 
Ces données sont également plus variées étant constituées de feuillets d'informations, de cartes et d'affiches qui semblent plutôt être tirées de manifestations politiques idéologiquement similaires, mais sans lien au niveau de l'organisation à proprement parler.

Également, nous souhaitions inclure les archives des festivals de Fierté Montréal et de Divers/Cité, son prédécesseur. 
Par contre, dans les deux cas, les archives possèdent trop peu de données: en ce qui concerne Fierté Montréal, l'événement semble encore trop récent pour que les \agq{} possèdent des documents sur celui-ci. 
Pour Divers/Cité, l'organisation a récemment déclaré faillite et il est attendu que les documents dont disposent les anciens propriétaires soient intégrés dans le futur aux \agq{}~\citep{Cormier2015}. 
Pour l'instant, les \agq{} n'entreposent que quelques affiches et dépliants trop anciens pour être intégrées dans notre documentation. 
Néanmoins, on va le voir dans les prochains chapitres, le magazine Fugues est un des médias principaux où ces deux festivals ont affiché des publicités et programmes.

\subsection{Données collectées sur le terrain}
\label{sub:donnees_collectees_sur_le_terrain}
En parallèle à la consultation et à la collecte des données auprès des sources secondaires, des données primaires seront recueillies sur les espaces permanents de la diversité sexuelle comme le Village gai de Montréal, mais aussi des espaces dits temporaires comme le vieux port de Montréal durant Divers/Cité ou la rue Saint-Jean-Baptiste durant la Fête Arc-en-ciel dans la ville de Québec. 
À l'aide d'internet notamment, il sera possible de retracer une partie des événements publics organisés par les communautés \lgbt{}, sachant que ceux-ci peuvent permettre une mobilisation hors-ligne, du moins en contexte politique et ainsi occuper l'espace~\citep[153-154]{Mercea2011}. 
Ces données doivent comprendre également des photographies prises sur le terrain dans des secteurs des villes reconnus pour abriter des espaces \qus{}. 
Les photographies serviront notamment à capturer la composante visuelle des géosymboles rencontrés pour en faire la recension et servir par la suite de matériel d'analyse. 
Ces photographies seront géoréférencées dans le but d'ajouter une couche d'information spatiale qui devrait faciliter l'analyse géographique.

Il n'est pas prévu dans le cadre de cette recherche de faire des entrevues; néanmoins, il est envisageable que certaines informations soient recueillies à l'occasion auprès de passants ou d'individus impliqués dans les organismes, événements ou espaces identifiés si jamais il devait y avoir un échange fortuit avec ceux-ci. 
Ces informations seront recueillies dans ce que nous nommons un guide de relevé; les données qui y seront consignées un ajout complémentaire à la photographie et à l'identification des géosymboles. 
Il n'est donc pas envisagé de passer devant un comité d'éthique, car il s'agira essentiellement d'observations personnelles sur l'environnement humain et bâti entourant les géosymboles rencontrés.

À l'aide des technologies à notre disposition, il a été possible de procéder à la géolocalisation en temps réel des données collectées. 
Préalablement à la collecte de données, plusieurs applications ont été recherchées et testées pour faciliter la collecte et s'assurer d'avoir facilement accès aux différentes localisations par \gps. 
Plusieurs options sont possibles, notamment Cloud GIS, logiciel propriétaire et dont les options dans la version gratuite sont limitées, et OpenDataKit, logiciel libre complet, mais qui demanderait la mise en place d'un serveur~\citep{OpenDataKit2014}. 
Ces outils permettent l'enregistrement de données multimédias: dans le cas où des données écrites devraient être notées, notamment lors de discussions, on envisage d'utiliser des logiciels de codage réflexif~\footnote{Plus communément appelés en anglais   \cadqas}. 
Il sera possible de travailler soit avec Sonal ou avec \rqda{} selon le type de données collectées. 
Le guide d'entretien pourra également être analysé par la suite à l'aide des logiciels nommés précédemment.

\subsection{Sources secondaires}
\label{sub:sources_secondaires}
Ce travail de recherche s'appuie également sur des travaux déjà effectués sur les espaces \qus{} montréalais. 
En effet, malgré une certaine jeunesse des études sur les communautés \lgbt{} au Québec, plusieurs travaux ont déjà produits dans les dernières décennies. 
On compte parmi ceux-ci le mémoire de \cite{Leznoff1954} qui consiste en une ethnographie des homosexuels durant les années 1950, les divers travaux de l'anthropologue et membre fondateur des \agq{}, Ross Higgins, ainsi que divers articles scientifiques; plusieurs de ces travaux ont été nommés dans le chapitre précédent à la section \ref{sec:la_diversit_sexuelle_en_g_ographie}. 
Nous tenons à souligner que nous nous appuierons grandement sur le travail de \citep{Giraud2014} dans une perspective de continuité et d'approfondissement du travail déjà effectué.

\section{Séjours}
\label{sec:s_jours}
L'ensemble des données sur le terrain ont été collectées durant l'été 2015. 
Une première collecte a d'abord été effectuée durant le mois de juin dans la ville de Québec à l'aide de données déjà connues d'une recherche précédente dans le cadre d'un mémoire de maîtrise sur les divers lieux des minorités sexuelles de la ville~\citep{Vachon2014}. 
Étant donné que nous possédions un lieu de résidence dans la ville de Québec, nous avons pu mettre à l'essai les outils de collecte de données sans avoir d'inquiétudes en ce qui concerne le temps nécessaire. 
De plus, durant cette période d'essai, aucun événement d'importance n'avait lieu en lien avec la communauté LGBT de Québec et l'occupation des lieux correspondait à l'achalandage régulier auquel on peut s'attendre durant l'année, si l'on fait fit d'une probable hausse d'activité lors d'événements d'importance ailleurs dans la ville. 
Durant cette première collecte de données, il est apparu nécessaire de détailler plus en profondeur le questionnaire construit à l'intérieur de l'interface de CloudGIS pour la suite de la collecte. 
Certaines limites quant aux outils utilisés, un cellulaire, nous a poussé à nous assurer d'avoir accès à une grande quantité de données cellulaires pour effectuer le transfert des données géolocalisées quotidiennement une fois débutée la collecte à Montréal.

Cette première partie effectuée, nous avons pu ainsi procéder au mois d'août à la plus grande part de la collecte de données dans la ville de Montréal.
Préalablement au départ, nous avons construit un calendrier en ligne dans lequel nous avons compilé l'ensemble des événements organisés par la communauté \lgbt{} prévus durant mon séjour. 
Une première partie de ces événements ont été trouvés à l'aide des sites internet officiels des organismes organisateurs d'événements comme Fierté Montréal ou Pervers/Cité. 
Étant donné la présence de plus en plus grande de Facebook dans la promotion desdits événements, nous avons également utilisé ce réseau social pour obtenir plus d'informations sur certains événements ou encore découvrir d'autres qui n'ont pas été promus autrement sur internet. 
C'est le cas de la totalité des événements du Festival Qouleurs et des événements politiques. 
Les pages événementielles sur Facebook possédant un grand nombre d'informations tout en présentant des images promotionnelles au potentiel géosymbolique fort, elles ont été retenues dans nos données et feront l'objet dans les chapitres suivants d'une analyse plus poussée.

Notre collecte de données à Montréal a duré au total trois semaines. 
Les premiers jours ont été consacrés à une familiarisation avec les lieux durant lesquels nous avons visité le Village gai et observé les divers événements quotidiens et l'achalandage des lieux. 
À cause d'une hausse très forte d'activité vers la fin de la journée, étant donné la fonction commerciale des lieux, les matins ont été priorisés pour la photographie des géosymboles repérés. 
En collaboration avec les bénévoles des \agq{} pour l'établissement d'un horaire de recherche, nous avons pu faire le travail en archives et couvrir l'ensemble des événements \lgbt{} qui se déroulaient durant le mois. 

Comme il était prévu pour la collecte, durant le mois d'août sont planifiés une grande part des événements \lgbt{} durant l'année. 
Il s'agit d'ailleurs d'une particularité au Québec étant donné qu'ailleurs dans le monde ces événements ont lieu durant le début de l'été. 
Cette décision de produire la fierté annuelle à ce moment a été prise d'abord par Divers/Cité puis par Fierté Montréal et a été reprise également à Québec. 
D'autres événements d'envergure ont d'ailleurs été organisés en parallèle par d'autres groupes de la communauté et il est donc apparu pertinent de profiter de cette opportunité pour planifier la collecte de données à ce moment de l'année sachant quand même que le portrait dressé est partiellement contingent à ce moment de l'année. 

La collecte à Montréal terminée, le retour à Québec nous a permis d'être présents par la suite aux festivités de la Fête Arc-en-ciel. 
Cette dernière partie de la collecte de données a été facilitée par notre participation comme bénévole à l'événement. 
Nous avons donc ainsi eu accès à l'ensemble de l'espace et du temps couvert par les festivités. 
D'autres espaces auraient été intéressants à couvrir, mais par conflit d'horaire il ne n'a pas été possible d'effectuer les déplacements nécessaires. 
C'est le cas des villes nommées à la Section~\ref{ssub:autres_villes}.


%%% Local Variables:
%%% mode: latex
%%% TeX-master: "../../memoire-maitrise"
%%% End:
