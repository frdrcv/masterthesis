%!TEX root = ../../memoire-maitrise.tex

\chapter{De la clandestinité à l'espace public}
\label{cha:de_la_clandestinite_a_l_espace_public}
\todo{titres et sous-titre à revoir : ne concorde pas avec les données}
% \chapterprecishere{\textquote{Everyone needs a place. It shouldn't be inside of someone else.} \par\raggedleft--- \textup{Richard Siken}, Crush}

\section{Spatialité homosexuelle aux débuts de la pathologisation}
\label{sec:spatialit_homosexuelle_aux_d_buts_de_la_pathologisation}

\section{Politisation et place publique}
\label{sec:politisation_et_place_publique}



\subsection{Mouvement des droits civiques}
\label{sub:mouvement_des_droits_civiques}

\section{Institutionnalisation de la diversité sexuelle}
\label{sec:institutionnalisation_de_la_diversit_sexuelle}

% section institutionnalisation_de_la_diversit_sexuelle (end)
% chapter de_la_clandestinite_a_l_espace_public (end)

\blockquote[{\cite{Pervers/Cite2015}}][.]{Organisé en collaboration,
  Pervers/Cité est un festival d’été visant à faire des liens entre les groupes
  de justice sociale, les communautés queers et les visions radicales de ce
  qu'étaient et devraient être les Fiertés LGBT\@. Dans un climat où prévaut
  l’agenda corporatif gai et l’aseptisation homogénéisé des queers, Pervers/cité
  tâche de fournir des activités critiques et accessibles, destinées à redonner
  une colonne vertébrale au mouvement LGBT}

Le classement des codes en catégorie s'est fait en relation avec les images et textes affiliés avec les codes, car ces images en soi orientaient le sens du code, à défaut d'avoir un code dont le sens serait suffisemment précis pour ne pas porter à confusion. 
Par exemple, le code maquillage a été classé dans la catégorie \code{visuel} car le maquillage qu'on retrouvait dans certaines images servaient en fait de visuel au sein d'une publicité par exemple, plutôt que d'être un maquillage porté par un individu et qui dans ce cas, aurait été instinctivement placé dans une catégorie comme habillement ou dans public ciblé si celui-ci aurait été offert à des enfants.

\section{Québec: spectre large}
\label{sec:qu_bec_spectre_large}
Nous commencerons notre présentation des résultats des médias avec la ville de Québec. 
Comme mentionné dans le chapitre sur la méthodologie, il s'agit de la deuxième ville en importance à être étudiée. 
En plus de figurer dans le magazine Fugues, celle-ci possède son propre média local, le journal Sortie. 
Nous analyserons donc les symboles de la ville dans ces deux médias.

Dans le journal Sortie, c'est vingt et une images qui ont été relevées dont dix-neuf à propos de la ville de Québec. 
Les images utilisées sont particulièrement variées: si nous avons ciblé des publicités, en général elles ne mettent pas en promotion à Québec des produits ou des services d'ordre commercial, malgré que ces dernières existent. 
En effet, en terme de produits, les publicités sont pratiquement absentes, nous n'avons relevés qu'une publicité pour la boutique érotique Chez Priape qui n'a même pas d'enseigne à Québec, celle-ci se situant à Montréal. 
En terme de services, les symboles sont plutôt orientées vers les soirées organisées sois dans les bars de la capitale, le cabaret-bar le Drague et le bar le Saint-Matthew's, ou encore dans les saunas comme L'Hippocampe. 
Ces soirées visent sois un public général, sois un genre en particulier. 
Le Drague dans ses publicités offrent des soirées réccurentes dans le temps en visant des types de soirées basées sur le contenu plutôt que vers une clientèle précise. 
Alors que dans le journal Fugues, nous avons relevé dix-sept images traitant de la capitale. 
Cela se remarque dans les visuels utilisés: on utilise les couleurs de façon variées et les symboles ne semblent pas cibler un genre en particulier. 
Des photographies sont utilisées mais celles-ci mettent essentiellement en avant des personnificateurs féminins pour des spectacles de ce type. 
Un autre cas d'activité mixte est le cas d'une publicité pour une activité de danse, plus précisément du Queer tango, où l'image, si elle met de l'avant un couple masculin, ne semble pas non viser un genre en particulier. 
Il s'agit d'ailleurs de la seule image utilisée dans le contexte de la ville de Québec où le terme queer est utilisé. 
On peut douter ici qu'il s'agit d'un usage du terme dans un cadre politique, plutôt, le terme queer ici semble désigner le caractère négatif et alternatif mis de l'avant dans l'activité du tango, une danse habituellement pratiquée entre un homme et une femme.

Parmi les images d'activités s'adressant spécifiquement à un genre particulier, nous avons d'abord deux activités festives s'adressant aux femmes, visible à la figure~\ref{figs3132}. 
Il s'agit de soirées organisées par le magazine Saphomag, un magasine destinée à une clientèle lesbienne. 
Contrairement aux activités réservées aux hommes dont nous traiterons plus loin dans ce chapitre, celles-ci ne se déroulent pas dans des bars, mais dans des locaux loués à l'église St-Jean-Baptiste dans le quartier du même nom. 
Sachant qu'il n'y a plus de bars strictement lesbiens comme il a pu en avoir à Montréal~\citep{Podmore2006}, on peut comprendre en partie la nécessité de se tourner vers des lieux alternatifs, mais dans ce cas-ci, les organisateurs ne se sont pas tourné vers le bar mixte de Québec, le Drague. 
Il est important de souligner également que ces fêtes sont organisées dans le cadre de la fête arc-en-ciel de Québec; nous n'avons pas trouvé d'occurence de fêtes similaires durant le reste de l'année dans le journal Sortie. 
Ces activités mettent de l'avant des soirées de DJ mais également de peinture corporelle.

\begin{figure}
\centering
\subcaptionbox{Événement de 2007\label{fig31}}
{\includegraphics[width=9cm]{fig31.jpg}}
\subcaptionbox{Événement de 2008 (et événement mixte)\label{fig32}}
{\includegraphics[width=6cm]{fig32.jpg}}
\caption{Événements destinés aux femmes (lesbiennes) : deux éditions
  différentes}\label{figs3132}
\end{figure}

\section{Montréal: diversité des imaginaires}
\label{sec:montr_al_diversit_des_imaginaires}

\subsection{Fétichisme}
Nous retrouvons une bonne part d'imagerie liées au fétichisme sexuel dans les médias étudiés. 
Ce fétichisme est particulièrement apparent dans les espaces orientés vers les hommes gais. 
Nous retrouvons ces images presque essentiellement dans le magazine Fugues et dans une moindre mesure sur quelques événements Facebook, pour des événements touchant encore une fois cette communauté masculine. 
Parmi les bars mettant en avant un tel imaginaire, on peut nommer en ordre d'importance l'Aigle Noir, le Tool et le Drague, à Québec \todo{Vérifier l'ordre réel des bars utilisant ce type d'imagerie}. 
Le Drague fait figure d'exception car il s'agit du seul lieu hors Montréal de ce type, quoique cette impression n'est pas la norme pour cet espace. 
En effet, aujourd'hui le Drague n'utilise plus ce type d'imagerie: il s'agit maintenant d'un lieu essentiellement mixte et peu sexualisé, du moins dans les symboles invoqués. 
Auparavent, et dans les premières années traitées dans notre collecte de données, le troisième étage servait de lieu de drague utilisé par les hommes et strictement pour eux. 

Tel que souligné par \citet{Giraud2013a}, le bar l'Aigle Noir vise une clientèle masculine d'abord et plus particulièrement fétichiste. 
Ceci est d'autant plus visible par le choix du logo et des couleurs utilisées: le noir, couleur fréquemment utilisée pour les vêtements et articles de cuir, on y voit un homme habillé de ce qui semble être du cuir et un aigle. 
Cet aigle, utilisé dans de nombreuses cultures selon des sens variés, peut rappeler l'imagerie militaire, tel qu'utilisé par exemple par l'État américain. 
Cette interprétation cadrerait avec une part importante de la sous-culture fétichiste s'attardant plus particulièrement aux uniformes, dont les uniformes militaires. 
L'Aigle Noir ne fait pas montre de censure vis-à-vis le type d'activité à caractère fétichiste qui s'y produisent: << party bobettes >>, << soirées bulles >>, party latex, ventes << d'esclaves >>, etc. 
Ce type d'activités sont propres au fétichisme: en plus du cuir, d'autres matériaux sont utilisés selon les fétichismes, dont le latex et les sous-vêtements masculins sont également souvent priorisés. 
Les soirées bulles semblent être l'apanage de nombreux autres espaces: en effet, en plus d'être du genre d'activités proposées à l'Aigle Noir, on les retrouve également dans les saunas, autant de Montréal que de Québec. 
Les soirées d'esclaves s'inscrivent plus particulièrement dans le cadre du \bdsm{}\footnote{\citeauthor{Turley2015} définisent le \bdsm{} comme:   \foreignquote{english}{\textelp{} the umbrella term   used to describe a set of consensual sexual practices that usually involve an   eroticised exchange of power and the application or receipt of painful and/or   intense sensations (Barker et al., 2007). 
The range of \bdsm{}-related activities   is wide and complex. 
‘BDSM’ denotes the assorted consensual activities   involved in the experience of participating in \bdsm{}; bondage and discipline   (B\&D), dominance and submission (D/s), and sadism and masochism   (SM)~\citeyearpar[24]{Turley2015}.}}. 
On retrouve de moins en moins cet imaginaire dans les dernières années de notre collecte de données en ce qui concerne particulièrement les données d'archives et celà correspond à quelques discussions que nous avons eu sur le terrain avec certains bénévoles des \agq{}: le bar a évolué dans les dernières années pour satisfaire un plus grand éventail au niveau de la clientèle, tout en demeurant un bar essentiellement pour hommes gais. 

Certains lieux orientés vers les fétichisme continuent toutefois à exister: c'est le cas de soirées organisées au \emph{Bunker}, la section sous-sol du bar \emph{Les Katakombes}, un bar orienté vers la scène métal et rock en-dehors du Village gai. 
Ce sous-sol remplie d'une certaine façon le rôle qu'a joué par le passé le bar l'Aigle Noir, et offrant des activités orientées vers le fétichisme en plus d'être un \emph{Backroom}\footnote{À COMPLÉTER}. 
Encore une fois, l'imagerie fétichiste est mise de l'avant, en misant toujours sur une clientèle masculine. 
Au niveau de l'accessibilité, les lieux diffèrent : si les soirées semblaient gratuites dans l'Aigle Noir, \emph{Le Bunker} n'organise que des soirées épisodiques dont l'entrée est payante, au montant de 30\$. 

Dans l'événementiel, certaines soirées mettent de l'avant des codes similaires.
C'est le cas notemment de certaines soirées organisées dans le cadre de la Fierté 2015, comme la soirée  \emph{BlackNight}.
Dans celle-ci, le noir et le rouge sont utilisés avec comme point focal une photographie d'un homme portant la barbe et ce qui semble être un uniforme de cuir.
Un autre exemple est l'événement \emph{Beardrop édition Montréal} organisé par \emph{Scruff}\footnote{Application de rencontre et drague pour hommmes fonctionnant sur cellulaire à l'aide du \gps{} de l'utilisateur.}.
On retrouve dans l'imagerie utilisée les mêmes codes que ceux nommés précédemment, malgré que le nom de l'événement semble cibler plus particulièrement les hommes entrant dans la catégorie de \emph{bear}.
En fait, dans l'image, on peut voir deux hommes sveltes mais musclés dans une posture s'apparentant à une danse.
Les vêtements, plutôt qu'être des uniformes de style militaire ou policier comme pour la soirée \emph{BlackNight}, sont plutôt des tenues ressemblant à des tenues de travail pour l'un des hommes, et des sous-vêtements pour l'autre.
On retrouve donc qu'une partie des codes visuels permettant d'identifier les hommes typés \emph{bear}; on retrouve pas la carrure habituellement mise de l'avant, sois un surpoids important ou une carrure impressionante.

L'ensemble des exemples nommés précédemment se rapportent à la ville de Montréal.
Peu de lieux ou d'événements mettent vraiment de l'avant le fétichisme dans leur imagerie, sauf à quelques occasions à Québec.
En effet, on peut nommer déjà certaines soirées organisées dans le bar Saint-Matthew's qui mélangent fétichisme et masculinité dans la promotion de leur événements.
Également, comme dans la Fierté montréalaise, la marche pour la diversité sexuelle est un des moments où plusieurs individus décident de mettre de l'avant sur la place publique leur intérêt pour les sexualités alternatives.
On retrouve par exemple plusieurs personnes utilisant les drapeaux cuirs, latex et \bdsm{} ainsi que des costumes se rapportant à ces fétichismes ou à des pratiques affiliées, comme le \emph{Puppy play}\footnote{Jeu de rôle de domination et soumission dans laquelle le soumis joue le rôle d'un chien  et où le dominant joue le rôle de \emph{handler}, à savoir le \emph{propriétaire} du chien. Le tout est souvent appuyé par l'usage de certains accessoires comme des masques, queues en caoutchouc, des harnais, un collier, etc.}.

\subsection{Marches et manifestations}
\label{subsec:label}
\todo{Probablement que ça va changer de place, à voir} Montréal se distingue par l'activité politique de plusieurs groupes \lgbt{}.
Nous traiterons dans cette partie de deux cas en particulier, sois les marches organisées par les communautés lesbiennes et trans, sachant que celles-ci n'ont pas nécessairement de liens direct (on peut croire par contre que certains individus participent aux deux événements, nous reviendrons sur les raisons) mais qu'elles adoptent des stratégies similaires.

\subsubsection{Marche Trans}
\label{subsubsec:marchetrans}
La marche trans s'inscrit plus largement dans le cadre de la Fierté Trans, un événement organisé immédiatement avant la fierté de Fierté Montréal. 
Comme le laisse entendre le nom choisi, l'événement s'adresse plus particulièrement aux personnes trans, quoique ce dernier terme rassemble un grand nombre d'identités entourant le genre. 
Parmi ces identités, on compte les personnes trans\footnote{nous n'utiliserons pas les termes de transexuels ou de   transgenres, ceux-ci n'apparaissant pas dans aucun des documents que nous   avons traités entourant cet événement}, les personnes non-binaires dans le genre\footnote{Nous utiliserons une définition assez large du terme, en   considérant celui-ci comme représentant autant les personnes se disant   non-binaires que les personnes agenres, neutroïdes, demi-hommes, demi-femmes,   au genre fluctuant, etc.\citep[see][]{Barker2015}} et les personnes intersexes. 
Nous appuyons cette définition d'ensemble de trans par l'imagerie utilisée par les organisateurs, plus particulièrement l'affiche de la marche.
Celle-ci montre en effet un grand nombre de symboles de ces différentes identités, comme des drapeaux. 
Chacune des identités nommées précédemment est évoquée par un des drapeaux: \todo{faire la liste des drapeaux et des identités}.

Comme nous l'avons souligné dans le début de cette section, la marche Trans s'inscrit dans la fierté trans qui compte d'autres événements: plus particulièrement, on peut souligner les différentes fêtes et campagnes de financement qui ont tous eu lieu dans une même soirée au café Cléopâtre. 
Cet espace est particulier par sa proximité avec l'histoire des communautés \lgbt{} québecoises et de sa proximité relative avec la communauté trans.

La marche s'est fait un trajet assez linéaire et situé dans les espaces reconnus de la communauté \lgbt{}. 
Tel qu'on peut le voir à la figure \todo{insérer la   figure}, la manifestation commence à proximité du café Cléopâtre, traité précédemment, dans le coeur de l'ancien \anglais{Red Light} montréalais. 
Cette marche se dirige par la suite dans le Village Gai pour terminer dans le Parc Lafontaine, toujours à proximité du Village. 
Plusieurs arrêts ont marqués cette marche. 
D'abord, au tout départ, plusieurs intervenants ont procédé à des discours par rapport aux droits des personnes trans et dans un cas précis, le cas des femmes trans de couleur (tel que décrit par la banderole utilisée).
Quelques organismes étaient présents à la marche et visibles; nous avons remarqué la présence de l'\atq{} et du \rlq{}. 
On peut croire que d'autres organismes ou membres d'organismes étaient également présents, étant donné la présence de ceux-ci à la soirée précédent la marche, tel l'\astteq{}. 
Par la suite, la marche se dirigea vers le Village gai pour un arrêt au parc de l'espoir, un lieu commémoratif aux victimes du \sida{}~\citep{Lafontaine2012} connu pour avoir été le théâtre d'actions politiques par le passé.

La marche trans s'articule autour d'un discours particulier, étant donné le contexte politique dans lequel cette dernière s'inscrit et également de l'actualité québecoise au niveau législatif. 
En s'intéressant au texte d'invitation de la marche Trans tel que publié sur Facebook, on apprend que la marche s'appuie sur plusieurs revendications entourant le changement de statut légal de genre. 
Ces points tournent autour du statut de citoyennté, l'âge, le genre, les exigences médicales (au niveau chirurgical notamment) et du coût des démarches. 
D'autres revendications sont également supportées ayant moins à voir avec le statut légal, sois l'absence de ressources spécialisées en prévention du \vih{} pour les personnes trans et pour faciliter le changement de statut.

On va mieux le comprendre dans la section suivante, la marche trans utilise une stratégie similaire au groupe des personnes \dyke{}, sois l'utilisation de la marche pour s'affiche publiquement. 
Cette marche correspond à première vue à une manifestation politique dans laquelle la visibilité est extrêmement importante.

\subsubsection{Marche dyke}
\label{subsubsec:marchedyke}
La \dm{}, comparativement à la marche trans, ne s'inscrit pas dans un événement plus large. 
En fait, nous constatons que le choix de la date se fait plutôt en réponse à la fierté organisée par Fierté Montréal. 
Nous croyons effectivement que, selon les motifs politiques de l'événement, celle-ci vise à offrir une visibilité à la communauté \dyke{} que l'on ne retrouverait pas dans la Fierté plus traditionnelle, malgré qu'une marche est été aussi organisée pour les femmes, cette dernière étant soutenue par l'organisation de la Fierté. 
Par conflit d'horaire, nous n'avons été présent que pour la marche Dyke. 
Si cette section va principalement traiter de cette dernière, nous nous intéresserons également à la marche lesbienne selon les informations que nous avons pu accumuler sur Facebook et dans la documentation promotionnelle de Fierté Montréal.

La visibilité semble était le but principal derrière la \dm{}: ceci est particulièrement apparente par le choix esthétique de la bannière ornant l'événement Facebook, où on peut y voir des pictogrammes d'oeils ainsi que dans le slogan utilisé. 
On retrouve également cette visibilité portée par la volonté de manifester sa présence dans l'espace public où la visibilité est normalement mobilisée comparativement à l'espace privé qui s'appuie plutôt sur les notions d'intimité et par l'absence de d'observateur (la présence d'un tel observateur possible mais on parle plutôt ici d'intrusion et d'une certaine part de violence \todo{Trouver des références sur l'intimité dans l'espace privé}). 

Pour la suite de notre analyse de la \dm{}, nous nous pencherons sur les publics sollicités, à savoir quels individus sont invités à participer à la marche et vers qui le message de la marche est orientée. 
Pour répondre à ces questions, nous pouvons déjà solliciter le travail de~\cite{Podmore2015a} qui a travaillé sur l'édition 2012 de cette marche. 
S'appuyant sur les travaux déjà effectués sur des éditions d'autres villes d'Amérique du Nord, comme Chicago, on apprend que les marches \dykes{} prennent racines dans une certaine exclusion des femmes des marches qui désiraient s'organiser entre elles et faire valoir leur présence au sein du mouvement principal derrière les fiertés. 
En conséquent, celles-ci décidèrent de créer des leurs propres marches, celles-ci se déroulant quelques jours avant la marche officielle. 
C'est également la stratégie choisie par les organisatrices de la marche de Montréal des dernières années, qui ont toujours placé l'événement quelques jours avant les débuts de la semaine de la fierté, et non seulement avant la marche officielle qui se déroule habituellement dans la dernière fin de semaine de la fierté.

Ce positionnement particulier et cette division du mouvement semble conforter une certaine identité autour du mot \dyke{}, terme que~\citet{Podmore2015a} dans son article \citetitle{Podmore2015a} décrit comme politisé et radicalisé comparativement au terme plus général de lesbienne. 
Le terme \dyke{} d'ailleurs vise à englober une plus vase population que le terme lesbienne, alors que n'importe quelle femme non-hétérosexuelle peut s'identifier avec ce terme et se joindre au mouvement de la marche \dyke{}. 
Cette politisation s'exprime en même temps par une non-mixité qui vise à exclure les hommes de la marche --- exclusion basée sur le respect des principes de l'événement plutôt qu'une exclusion qu'on pourrait qualifier d'agressive. 
Les alliés intéressés par l'évènement mais n'entrant pas dans la catégorie \dyke{} étaient invitéEs à suivre la marche à l'extérieur, en marchant sur les trottoir. 
C'est ce que j'ai du faire, mais j'ai pu constater l'absence d'individus se réclamant ou agissant comme \emph{alliés} (tenant des pancartes ou scandant des slogans en marge de la marche). 
Un seul autre homme était présent. 
En discutant avec cette personne j'ai pu apprendre qu'il était là par curiosité et j'ai du moi-même dû lui expliquer que sa présence dans la marche n'était pas tolérée et quels étaient les principes soutenus par cette dernière.

Le public visé par la marche dans son ensemble est moins facilement définissable. 
En fait, nous pouvons considérer que l'ensemble de la société est visée par le message de la marche, étant donné que la visibilité comme telle s'exprime dans l'espace public, comme je l'ai souligné précédemment. 
Par contre, on ne peut ignorer le fait que la marche s'inscrit dans les pratiques d'autres événements politiques similaires. 
Nous ne pouvons pas clairement savoir s'il s'agit ici d'une tradition ou un message qui demeure encore porté à l'organisation principale de la fierté. 
Également, le choix du parcours peut nous renseigner sur le public ciblé. 
Contrairement à l'édition sur laquelle~\citet{Podmore2015a} a travaillée, la marche de 2015 a complètement évité le Village gai pour commencer plus au nord et terminer dans le quartier du \anglais{Mile End}.

Pour approfondir la question de la visibilité, l'article de \citet{Frosh2006} \citetitle{Frosh2006} offre une vue intéressante sur le partage d'un message --- \anglais{text} dans l'article --- à l'aide d'un média. 
Dans un contexte social où les interactions entre individus dans l'espace public sont réduites au minimum jusqu'à l'indifférence, se rendre visible auprès d'autrui permet d'agir à contre-courant et déranger cette indifférence. 

Pourtant, \citeauthor{Frosh2006} nous apprend que cette indifférence peut être une forme de respect ou d'intégration. 
Les pratiques sociales dans l'espace public qui se fondent sur l'indifférence témoignent chez les individus une forme de respect mutuel qui est bien différente d'une relation basée sur l'altérité et l'incompréhension. 
Si certains auteurs d'après \citeauthor{Frosh2006} considèrent la froideur des relations humaines --- l'inattention civile chez \citeauthor{Goffman1956} --- dans l'espace public comme une preuve d'un espace moralement vide et étranger, on apprend aussi que ce serait plutôt la réaction à la vision d'un autre qui sera le témoignage d'une forme de peur ou d'appréhension vis-à-vis l'autre~\citep[279--280]{Frosh2006}. 
L'action de réclamer cette visibilité dans la \dm{} pourrait être conçue comme une rupture volontaire de ce respect mutuel pour montrer que l'égalité sous-entendue n'est pas concrète et reste à faire. 
Cet acte montrerait dont une limite de l'inattention civile comme concept centré sur la communication et la perception, sois une certaine incapacité à tenir compte des minorités conçues comme invisibles ou, du moins, d'individus se regroupant autour d'une différence commune vis-à-vis la norme, ici hétérosexuelle et même masculine. 
L'appel à la non-mixité, si elle répond à une volonté de se retrouver entre individus partageant une oppression commune comme femme et lesbienne, permet également de d'avoir un contrôle sur l'image qu'elles véhiculent collectivement par rapport à l'observateur qui se trouve souvent être en position de pouvoir\todo{Traiter du   Male gaze avec les   références~\cite{Wood2004,Patterson2002,Skelton2002,Snow1989}}.

\section{Festivités s'étirant sur plusieurs jours}
\label{sec:festivitesplusieursjours}
Si les marches et manifestations sont symboliquement marquantes par l'usage de nombreux géosymboles et par une subversion partielle ou complète de l'espace public, il en demeure pas moins que leur présence est très circonscrite dans le temps.
Nous nous intéresserons dans cette section à des événements temporaires, mais s'étirant dans le temps.
La totalité de ceux-ci consistent en des célébrations de le diversité sexuelle, certaines bien connues du public, d'autres appartenant plutôt à une certaine contre-culture ou orientée vers les individus s'indentifiant spécifiquement à l'identité de genre ou l'orientation sexuelle ciblée.

\subsection{Fierté trans}
\label{subsec:fiertetrans}
Nous avons traités brièvement de la Fierté trans précédemment dans la section précédente à propos de la marche trans.

\subsection{Qouleur}
\label{subsec:qouleur}
Le terme de Qouleur désigne autant le nom de l'événement que le nom du collectif derrière celui-ci.
À première vue, on peut croire que le terme Qouleur peut désigner deux réalités, sois la diversité sexuelle par un lien avec le drapeau arc-en-ciel aux multiples couleurs et sois une désignation donnée aux personnes racisées, personnes de couleur.
Entendu comme un terme moins discriminant que le terme \emph{race}, le terme couleur prend ses origines dans \todo{Trouver l'origine du mot couleur}.
Bien qu'utilisé fréquemment pour désigner les personnes d'origine afro, le terme de couleur dans le nom de l'événement semble plutôt désigner la totalité des individus racisés, sachant que les événements se déroulant dans ce festival ciblent certains groupes précis.
Si certains de ceux-ci sont publics, plusieurs d'entre eux visent pas exemple les individus autochtones par l'appellation de \emph{two-spirits}

\subsection{Pervers/Cité}
\label{subsec:perverscite}
Pervers/Cité est un autre des nombreux événements se déroulant en parallèle à la Fierté Montréal du mois d'août.
L'existence de Pervers/Cité se démarque comme un événement créé en réaction à l'évolution des festivités plus traditionnelles de la communauté gaie montréalaise.
En effet, à la création de Pervers/Cité existait déjà un organisme organisant le fierté gaie montréalaise, Divers/Cité.
Ce dernier, nous y reviendrons plus tard, a d'abord été un événement à tendance communautaire et politique pour devenir une festival plutôt orienté vers la fête avec un gain de popularité des festivités.
Pervers/Cité est né d'une réponse à cette évolution qui a été dénoncée comme marchande et non-rassembleuse pour l'ensemble des minorités sexuelles montréalaises.
Organisée par une faction encore très politisée, Pervers/Cité souhaitaient alors offrir une alternative radicale aux festivités de Divers/Cité, l'événement se déroulant dans les mêmes journées mais à l'extérieur du périmètre occupé par Divers/Cité, sois la rue Sainte-Catherine et plus tard le vieux port de Montréal.
Cette tendance se poursuite aujourd'hui, alors que le festival s'étend sur une vaste territoire, visible à la figure \todo{Mettre la figure}.

\todo{À remplacer par la carte de Pervers/Cité}

\subsubsection{Activités}
\label{subsec:activitesperverscite}
Les activités offertent dans le cadre de Pervers/Cité sont particulièrement variées.
En effet, les sujets couverts semblent parfois ne pas avoir de liens directs: certains traitent plus particulièrement du militantisme, d'autres de la sexualité et certains s'apparent plus à des jeux ou des activités sociales.
En général du moins, l'ensemble de celles-ci touchent des thématiques subversives, sois par l'utilisation particulière de l'espace, sois par les sujets traités.

Parmi les activités tourant autour de la thématique du militantisme, on peut nommer la projection du film \emph{Pride} et de la conférence subséquente sur les liens entre syndicalisme et droits des communautés \lgbt{}.
Ne se déroulant pas nécessairement dans l'espace public, ces deux activités se sont déroulées dans des lieux semi-publics, sois un centre communautaire et un local de l'Université Concordia.
Pour l'occasion étaient invités un des acteurs importants des premières liaisons entre syndicalistes et groupe \lgbt{} en Grande-Bretagne.

Le salon du livre \emph{Queers entre les couvertures} est une autre des activités importantes du festival Pervers/Cité.
Il s'agit également d'une des rares activités à se dérouler dans le village, dans le cas présent sur la rue Amherst dans le Centre communautaire de loisirs Ste-Catherine d'Alexandrie.
Prenant l'ensemble de l'espace loué, le lieu
\subsection{Fierté Montréal 2016}
\label{subsec:fiertemontreal2016}

\subsubsection{Activités}
\label{subsec:activitesfiertemontreal}



\subsection{Fête Arc-en-ciel}
\label{subsec:fetearcenciel}
Dernier événement temporaire traité dans cette section, la fête arc-en-ciel se déroule dans la ville de Québec durant le congé de la fête du travail, sois durant la première fin de semaine du mois de septembre.
Comme pour la Fierté Montréal, dans ce cas-ci aussi nous avons un exemple d'un événement d'envergure ne se déroulant pas durant le début de l'été contrairement aux autres événements ailleurs dans le monde.
Le moment choisi est d'ailleurs particulièrement tard dans la saison estivale, sachant qu'à ce moment de l'année plusieurs personnes ne sont plus en vacances comme il est coutume de le faire durant les mois de juin, juillet et d'août, ou sont de retours à l'école pour les étudiants.
Par contre, on peut croire que cette raison s'explique par une proximité géographique de Montréal qui ferait une trop grande compétition aux festivités de la capitale si jamais les deux événements auraient lieux en même temps.
Cette hypothèse se confirme également par la présence de plusieurs groupes de Québec dans les événements publics de la fierté montréalaise, durant les journées communautaires et le défilé principalement.

La fête Arc-en-ciel s'étire dans un plus courte durée que la fierté montréalaise.
En effet, celle-ci ne dure que 4 jours, sois du jeudi au dimanche, avec une quantité d'activités prévues beaucoup plus importante dans les deux dernières journées, sachant que pour plusieurs il s'agit de journées de congé.
Tout comme la fierté montréalaise, la fête arc-en-ciel occupe symboliquement l'espace public de façon marquée: une partie importante de la rue Saint-Jean-Baptiste et le carré d'Youville sont décorés de nombreuses décorations, fanions, affices et autres aux couleurs de l'arc-en-ciel.
Également, durant le samedi et le dimanche, la rue Saint-Jean-Baptiste sur une partie de sa longueur est fermée à la circulation automobile.
D'autres décorations arc-en-ciel sont alors ajoutées sur la rues et sur les trottoirs, ceux-ci n'étant plus utilisés obligatoirement par les piétons.
La côte Sainte-Geneviève est également décorée tout en étant fermée d'une certaine façon à la circulation piétonnière.
En effet, si celle-ci est toujours fermée à la circulation automobile, durant les deux derniers jours de la fête arc-en-ciel elle devient un espace où sont organisées des activités pour la fête arc-en-ciel et devient également un lieu de restoration.

\subsubsection{Activités}
\label{subsec:activitesfetearcenciel}
Étant donné le cadre temporel plus restreint, moins d'activités se déroulent dans le cadre de la fête arc-en-ciel que dans plusieurs des événements traités précédemment.
Néanmoins, une certaine diversité existe similaire à la fierté montréalaise, sois un mélange de soirée dans les discothèque et bars, des conférences et des spectacles sur la place publique.

Comme dans la fierté montréalaise, on peut noter une certaine volonté de l'organisation de cibler un spectre plus large que la simple communauté gaie masculine.
Une conférence par exemple a eu lieu sur les enjeux vécus par les personnes trans la veille du début de la fête arc-en-ciel et à l'extérieur de celle-ci.
Si la visibilité de cet événement est moindre que plusieurs autres sur le plan physique, on retrouvait quand même l'événement dans la publicité du festival ainsi que sur \emph{Facebook}.

Certains événements ont ciblé plus particulièrement la communauté lesbienne, comme \todo{trouver exemples}.

En ce qui concerne les événements pour hommes gais, on peut nommer les soirées \emph{Boys gone wild}\todo{Vérifier si c'est le bon nom} et \todo{trouver l'autre événement} se déroulant tous les deux au bar \emph{Saint-Matthew's}, un bar pour hommes comme nommé à la section \todo{trouver le numéro de section si déjà écrit}.

Une autre particularité de la fête arc-en-ciel et d'inviter des personnalités connues ou représentatives des minorités sexuelles.
Dans l'édition de 2016, la personne invitée a été Michèle Richard, désignée comme icône sois pour sa popularité chez une partie du spectre \lgbt{}, sois pas son interprétation de la chanson Disco \emph{I will survive} de \todo{Trouver le nom de la chanteuse}.
Si cette chanson traite normalement du sentiment de rupture et de la volonté d'indépendance de la chanteuse, dans ce cas précis, l'expression \emph{I will survive} ou \emph{Je vais survivre} dans l'interprétation de Michèle Richard, fait écho à la résilience des communautés \lgbt depuis sa sortie de la clandestinité et plus particulièrement de la crise du \vih{}.

Enfin, on peut souligner le virage familial mis de l'avant par l'organisation de la fête arc-en-ciel dans la volonté d'organiser un pique-nique.

%%% Local Variables:
%%% mode: latex
%%% TeX-master: "../../memoire-maitrise"
%%% End:
