%!TEX root = ../../memoire-maitrise.tex

\chapter{De la clandestinité à l'espace public}
\label{cha:de_la_clandestinite_a_l_espace_public}
\todo{titres et sous-titre à revoir : ne concorde pas avec les données}
% \chapterprecishere{\textquote{Everyone needs a place. It shouldn't be inside of someone else.} \par\raggedleft--- \textup{Richard Siken}, Crush}

\section{Spatialité homosexuelle aux débuts de la pathologisation}
\label{sec:spatialit_homosexuelle_aux_d_buts_de_la_pathologisation}

\section{Politisation et place publique}
\label{sec:politisation_et_place_publique}



\subsection{Mouvement des droits civiques}
\label{sub:mouvement_des_droits_civiques}

\section{Institutionnalisation de la diversité sexuelle}
\label{sec:institutionnalisation_de_la_diversit_sexuelle}

% section institutionnalisation_de_la_diversit_sexuelle (end)
% chapter de_la_clandestinite_a_l_espace_public (end)

\blockquote[{\cite{Pervers/Cite2015}}][.]{Organisé en collaboration,
  Pervers/Cité est un festival d’été visant à faire des liens entre les groupes
  de justice sociale, les communautés queers et les visions radicales de ce
  qu'étaient et devraient être les Fiertés LGBT\@. Dans un climat où prévaut
  l’agenda corporatif gai et l’aseptisation homogénéisée des queers, Pervers/Cité
  tâche de fournir des activités critiques et accessibles, destinées à redonner
  une colonne vertébrale au mouvement LGBT}.

Le classement des codes en catégorie s'est fait en relation avec les images et textes affiliés avec les codes. Ceci s'explique par le fait que les images en soi orientaient le sens du code, à défaut d'avoir un code dont le sens serait suffisamment précis pour ne pas porter à confusion. 
Par exemple, le code maquillage a été classé dans la catégorie \code{visuel} car le maquillage qu'on retrouvait dans certaines images servait en fait de visuel au sein d'une publicité. Par exemple, plutôt que d'être un maquillage porté par un individu et qui dans ce cas, aurait été instinctivement placé dans une catégorie comme habillement ou dans public ciblé si celui-ci avait été offert à des enfants.

\section{Québec: spectre large}
\label{sec:qu_bec_spectre_large}
Nous commencerons notre présentation des résultats des médias avec la ville de Québec. 
Comme mentionné dans le chapitre sur la méthodologie, il s'agit de la deuxième ville en importance à être étudiée. 
En plus de figurer dans le magazine Fugues, celle-ci possède son propre média local, le journal Sortie. 
Nous analyserons donc les symboles de la ville dans ces deux médias.

Dans le journal Sortie, vingt et une images ont été relevées, dont dix-neuf à propos de la ville de Québec. 
Les images utilisées sont particulièrement variées: si nous avons ciblé des publicités, en général elles ne mettent pas en promotion à Québec des produits ou des services d'ordre commercial, malgré que ces dernières existent. 
En effet, en termes de produits, les publicités sont pratiquement absentes, nous n'avons relevé qu'une publicité pour la boutique érotique Chez Priape qui n'a même pas d'enseigne à Québec, celle-ci se situant à Montréal. 
En termes de services, les symboles sont plutôt orientés vers les soirées organisées sois dans les bars de la capitale, le cabaret-bar Le Drague et le Bar St-Matthew’s, ou encore dans les saunas comme L'Hippocampe. 
Ces soirées visent sois un public général, sois un genre en particulier. 
Le Drague dans ses publicités offre des soirées récurrentes dans le temps en visant des types de soirées basées sur le contenu plutôt que vers une clientèle précise. 
Alors que dans le journal Fugues, nous avons relevé dix-sept images traitant de la capitale. 
Cela se remarque dans les visuels utilisés: on utilise les couleurs de façon variées et les symboles ne semblent pas cibler un genre en particulier. 
Des photographies sont utilisées, mais celles-ci mettent essentiellement en avant des personnificateurs féminins pour des spectacles de ce type. 
Un autre cas d'activité mixte est le cas d'une publicité pour une activité de danse, plus précisément du tango queer, où l'image, si elle montre un couple masculin, ne semble pas non viser un genre en particulier. 
Il s'agit d'ailleurs de la seule image utilisée dans le contexte de la ville de Québec où le terme queer est utilisé.
On peut douter ici qu'il s'agisse d'un usage du terme dans un cadre politique. Plutôt, le terme queer ici semble désigner le caractère négatif et alternatif mis de l'avant dans l'activité du tango, une danse habituellement pratiquée entre un homme et une femme.

Parmi les images d'activités s'adressant spécifiquement à un genre particulier, nous avons d'abord deux activités festives s'adressant aux femmes, visibles à la figure~\ref{figs3132}. 
Il s'agit de soirées organisées par le magazine Saphomag, un magasine destiné à une clientèle lesbienne. 
Contrairement aux activités réservées aux hommes dont nous traiterons plus loin dans ce chapitre, celles-ci ne se déroulent pas dans des bars, mais dans des locaux loués à l'église Saint-Jean-Baptiste dans le quartier du même nom. 
Sachant qu'il n'y a plus de bars strictement lesbiens comme il a pu en avoir à Montréal~\citep{Podmore2006}, on peut comprendre en partie la nécessité de se tourner vers des lieux alternatifs. Dans ce cas-ci, les organisateurs ne se sont pas tournés vers le bar mixte de Québec, Le Drague. 
Il est important de souligner également que ces fêtes sont organisées dans le cadre de la Fête Arc-en-ciel de Québec; nous n'avons pas trouvé d'occurrence de fêtes similaires durant le reste de l'année dans le journal Sortie. 
Ces activités mettent de l'avant des soirées de DJ mais également de peinture corporelle.

\begin{figure}
\centering
\subcaptionbox{Événement de 2007\label{fig31}}
{\includegraphics[width=9cm]{fig31.jpg}}
\subcaptionbox{Événement de 2008 (et événement mixte)\label{fig32}}
{\includegraphics[width=6cm]{fig32.jpg}}
\caption{Événements destinés aux femmes (lesbiennes) : deux éditions
  différentes}\label{figs3132}
\end{figure}

\section{Montréal: diversité des imaginaires}
\label{sec:montr_al_diversit_des_imaginaires}

\subsection{Fétichisme}
Nous retrouvons une bonne part d'imagerie liée au fétichisme sexuel dans les médias étudiés. 
Ce fétichisme est particulièrement apparent dans les espaces orientés vers les hommes gais. 
Nous retrouvons ces images presque essentiellement dans le magazine Fugues et dans une moindre mesure sur quelques événements Facebook, pour des événements touchant encore une fois cette communauté masculine. 
Parmi les bars mettant en avant un tel imaginaire, on peut nommer en ordre d'importance l'Aigle Noir, le Tool et Le Drague, à Québec \todo{Vérifier l'ordre réel des bars utilisant ce type d'imagerie}. 
Le Drague fait figure d'exception, car il s'agit du seul lieu hors Montréal de ce type, quoique cette impression n'est pas la norme pour cet espace. 
En effet, aujourd'hui Le Drague n'utilise plus ce type d'imagerie: il s'agit maintenant d'un lieu essentiellement mixte et peu sexualisé, du moins dans les symboles invoqués. 
Auparavant, et dans les premières années traitées dans notre collecte de données, le troisième étage servait de lieu de drague utilisé par les hommes et strictement pour eux. 

Tel que souligné par \citet{Giraud2013a}, le bar l'Aigle Noir vise une clientèle masculine d'abord et plus particulièrement fétichiste. 
Ceci est d'autant plus visible par le choix du logo et des couleurs utilisées: le noir, couleur fréquemment utilisée pour les vêtements et les articles de cuir, on y voit un homme habillé de ce qui semble être du cuir et un aigle. 
Cet aigle, utilisé dans de nombreuses cultures selon des sens variés, peut rappeler l'imagerie militaire, tel qu'utilisé par exemple par l'État américain. 
Cette interprétation cadrerait avec une part importante de la sous-culture fétichiste s'attardant plus particulièrement aux uniformes, dont les uniformes militaires. 
L'Aigle Noir ne fait pas montre de censure vis-à-vis le type d'activité à caractère fétichiste qui s'y produisent: << party bobettes >>, << soirées bulles >>, party latex, ventes << d'esclaves >>, etc. 
Ce type d'activités sont propres au fétichisme: en plus du cuir, d'autres matériaux sont utilisés selon les fétichismes, dont le latex et les sous-vêtements masculins sont également souvent priorisés. 
Les soirées bulles semblent être l'apanage de nombreux autres espaces: en effet, en plus d'être du genre d'activités proposées à l'Aigle Noir, on les retrouve également dans les saunas, autant de Montréal que de Québec. 
Les soirées d'esclaves s'inscrivent plus particulièrement dans le cadre du \bdsm{}\footnote{\citeauthor{ Turley2015} définisent le \bdsm{} comme: \foreignquote{ english}{\textelp{} the umbrella term used to describe a set of consensual sexual practices that usually involve an eroticised exchange of power and the application or receipt of painful and/or intense sensations (Barker et al., 2007). 
The range of \bdsm{}- related activities is wide and complex. 
“BDSM” denotes the assorted consensual activities involved in the experience of participating in \bdsm{}; bondage and discipline (B\&D), dominance and submission (D/s), and sadism and masochism (SM)~\citeyearpar[24]{Turley2015}.}}. 
On retrouve de moins en moins cet imaginaire dans les dernières années de notre collecte de données en ce qui concerne particulièrement les données d'archives et cela correspond à quelques discussions que nous ayons eu sur le terrain avec certains bénévoles des \agq{}: le bar a évolué dans les dernières années pour satisfaire un plus grand éventail au niveau de la clientèle, tout en demeurant un bar essentiellement pour hommes gais. 

Certains lieux orientés vers le fétichisme continuent toutefois à exister: c'est le cas de soirées organisées au \emph{Bunker}, la section sous-sol du bar \emph{Les Katakombes}, un bar orienté vers la scène métal et rock en dehors du Village gai. 
Ce sous-sol rempli d'une certaine façon le rôle qu'a joué par le passé le bar l'Aigle Noir, et offrant des activités orientées vers le fétichisme en plus d'être un \emph{Backroom}\footnote{À COMPLÉTER}. 
Encore une fois, l'imagerie fétichiste est mise de l'avant, en misant toujours sur une clientèle masculine. 
Au niveau de l'accessibilité, les lieux diffèrent : si les soirées semblaient gratuites dans l'Aigle Noir, \emph{Le Bunker} n'organise que des soirées épisodiques dont l'entrée est payante, au montant de 30\$. 

Dans l'événementiel, certaines soirées mettent de l'avant des codes similaires.
C'est le cas notamment de certaines soirées organisées dans le cadre de la Fierté 2015, comme la soirée  \emph{BlackNight}.
Dans celle-ci, le noir et le rouge sont utilisés avec comme point focal une photographie d'un homme portant la barbe et ce qui semble être un uniforme de cuir.
Un autre exemple est l'événement \emph{Beardrop édition Montréal} organisé par \emph{Scruff}\footnote{Application de rencontre et drague pour hommes fonctionnant sur cellulaire à l'aide du \gps{} de l'utilisateur.}.
On retrouve dans l'imagerie utilisée les mêmes codes que ceux nommés précédemment, malgré que le nom de l'événement semble cibler plus particulièrement les hommes entrant dans la catégorie de \emph{bear}.
En fait, dans l'image, on peut voir deux hommes sveltes, mais musclés dans une posture s'apparentant à une danse.
Les vêtements, plutôt qu'être des uniformes de style militaire ou policier comme pour la soirée \emph{BlackNight}, sont plutôt des tenues ressemblant à des tenues de travail pour l'un des hommes, et des sous-vêtements pour l'autre.
On retrouve donc qu'une partie des codes visuels permettant d'identifier les hommes typés \emph{bear}; on ne retrouve pas la carrure habituellement mise de l'avant, sois un surpoids important ou une carrure impressionnante.

L'ensemble des exemples nommés précédemment se rapportent à la ville de Montréal.
Peu de lieux ou d'événements mettent vraiment de l'avant le fétichisme dans leur imagerie, sauf à quelques occasions à Québec.
En effet, on peut nommer déjà certaines soirées organisées dans le bar St-Matthew's qui mélangent fétichisme et masculinité dans la promotion de leurs événements.
Également, comme dans la Fierté montréalaise, la marche pour la diversité sexuelle est un des moments où plusieurs individus décident de mettre de l'avant sur la place publique leur intérêt pour les sexualités alternatives.
On retrouve par exemple plusieurs personnes utilisant les drapeaux cuirs, latex et \bdsm{} ainsi que des costumes se rapportant à ces fétichismes ou à des pratiques affiliées, comme le \emph{puppy play}\footnote{Jeu de rôle de domination et soumission dans laquelle le soumis joue le rôle d'un chien  et où le dominant joue le rôle de \emph{handler}, à savoir le \emph{propriétaire} du chien. Le tout est souvent appuyé par l'usage de certains accessoires comme des masques, queues en caoutchouc, des harnais, un collier, etc.}.

\subsection{Marches et manifestations}
\label{subsec:label}
\todo{Probablement que ça va changer de place, à voir} Montréal se distingue par l'activité politique de plusieurs groupes \lgbt{}.
Nous traiterons dans cette partie de deux cas en particulier, sois les marches organisées par les communautés lesbiennes et trans. Si celles-ci n'ont pas nécessairement de liens directs (on peut croire par contre que certains individus participent aux deux événements, nous reviendrons sur les raisons), elles adoptent des stratégies similaires.

\subsubsection{Marche Trans}
\label{subsubsec:marchetrans}
La Marche Trans s'inscrit plus largement dans le cadre de la Fierté Trans, un événement organisé immédiatement avant la fierté de Fierté Montréal. 
Comme le laisse entendre le nom choisi, l'événement s'adresse plus particulièrement aux personnes trans, quoique ce dernier terme rassemble un grand nombre d'identités entourant le genre. 
Parmi ces identités, on compte les personnes trans\footnote{nous n'utiliserons pas les termes de transsexuels ou de   transgenres, ceux-ci n'apparaissant pas dans aucun des documents que nous   avons traités entourant cet événement}, les personnes non binaires dans le genre\footnote{Nous utiliserons une définition assez large du terme, en   considérant celui-ci comme représentant autant les personnes se disant   non binaires que les personnes agenres, neutrois, demi-hommes, demi-femmes,   au genre fluctuant, etc.\citep[see][]{Barker2015}} et les personnes intersexes. 
Nous appuyons cette définition d'ensemble de trans par l'imagerie utilisée par les organisateurs, plus particulièrement l'affiche de la marche.
Celle-ci montre en effet un grand nombre de symboles de ces différentes identités, comme des drapeaux. 
Chacune des identités nommées précédemment est évoquée par un des drapeaux: \todo{faire la liste des drapeaux et des identités}.

Comme nous l'avons souligné dans le début de cette section, la Marche Trans s'inscrit dans la Fierté Trans qui compte d'autres événements: plus particulièrement, on peut souligner les différentes fêtes et campagnes de financement qui ont toutes eu lieu dans une même soirée au café Cléopâtre. 
Cet espace est particulier par sa proximité avec l'histoire des communautés \lgbt{} québécoises et de sa proximité relative avec la communauté trans.

La marche s'est fait un trajet assez linéaire et situé dans les espaces reconnus de la communauté \lgbt{}. 
Tel qu'on peut le voir à la figure \todo{insérer la   figure}, la manifestation commence à proximité du café Cléopâtre, traité précédemment, dans le cœur de l'ancien \anglais{Red Light} montréalais. 
Cette marche se dirige par la suite dans le Village gai pour terminer dans le parc La Fontaine, toujours à proximité du Village. 
Plusieurs arrêts ont marqué cette marche. 
D'abord, au tout départ, plusieurs intervenants ont procédé à des discours par rapport aux droits des personnes trans et dans un cas précis, le cas des femmes trans de couleur (tel que décrit par la banderole utilisée).
Quelques organismes étaient présents à la marche et visibles; nous avons remarqué la présence de l'\atq{} et du \rlq{}. 
On peut croire que d'autres organismes ou membres d'organismes étaient également présents, étant donné la présence de ceux-ci à la soirée précédent la marche, tel l'\astteq{}. 
Par la suite, la marche se dirigea vers le Village gai pour un arrêt au parc de l'espoir, un lieu commémoratif aux victimes du \sida{}~\citep{Lafontaine2012} connu pour avoir été le théâtre d'actions politiques par le passé.

La Marche Trans s'articule autour d'un discours particulier, étant donné le contexte politique dans lequel cette dernière s'inscrit et également de l'actualité québécoise au niveau législatif. 
En s'intéressant au texte d'invitation de la Marche Trans tel que publié sur Facebook, on apprend que la marche s'appuie sur plusieurs revendications entourant le changement de statut légal de genre. 
Ces points tournent autour du statut de citoyenneté, l'âge, le genre, les exigences médicales (au niveau chirurgical notamment) et du coût des démarches. 
D'autres revendications sont également supportées ayant moins à voir avec le statut légal, sois l'absence de ressources spécialisées en prévention du \vih{} pour les personnes trans et pour faciliter le changement de statut.

On va mieux le comprendre dans la section suivante, la Marche Trans utilise une stratégie similaire au groupe des personnes \dyke{}, sois l'utilisation de la marche pour s'affiche publiquement. 
Cette marche correspond à première vue à une manifestation politique dans laquelle la visibilité est extrêmement importante.

\subsubsection{Marche dyke}
\label{subsubsec:marchedyke}
La \dm{}, comparativement à la Marche Trans, ne s'inscrit pas dans un événement plus large. 
En fait, nous constatons que le choix de la date se fait plutôt en réponse à la fierté organisée par Fierté Montréal. 
Nous croyons effectivement que, selon les motifs politiques de l'événement, celle-ci vise à offrir une visibilité à la communauté \dyke{} que l'on ne retrouverait pas dans la Fierté plus traditionnelle, malgré qu'une marche est été aussi organisée pour les femmes, cette dernière étant soutenue par l'organisation de la Fierté. 
Par conflit d'horaire, nous n'avons été présents que pour la marche Dyke. 
Si cette section va principalement traiter de cette dernière, nous nous intéresserons également à la marche lesbienne selon les informations que nous avons pu accumuler sur Facebook et dans la documentation promotionnelle de Fierté Montréal.

La visibilité semble était le but principal derrière la \dm{}: ceci est particulièrement apparente par le choix esthétique de la bannière ornant l'événement Facebook, où on peut y voir des pictogrammes d'œil ainsi que dans le slogan utilisé. 
On retrouve également cette visibilité portée par la volonté de manifester sa présence dans l'espace public où la visibilité est normalement mobilisée comparativement à l'espace privé qui s'appuie plutôt sur les notions d'intimité et par l'absence d'observateur (la présence d'un tel observateur possible, mais on parle plutôt ici d'intrusion et d'une certaine part de violence \todo{Trouver des références sur l'intimité dans l'espace privé}). 

Pour la suite de notre analyse de la \dm{}, nous nous pencherons sur les publics sollicités, à savoir quels individus sont invités à participer à la marche et vers qui le message de la marche est orienté. 
Pour répondre à ces questions, nous pouvons déjà nous appuyer sur le travail de~\cite{Podmore2015a} qui a travaillé sur l'édition 2012 de cette marche. 
S'appuyant sur les travaux déjà effectués sur des éditions d'autres villes d'Amérique du Nord, comme Chicago, on apprend que les marches \dykes{} prennent racine dans une certaine exclusion des femmes des fiertés mixtes qui désiraient s'organiser entre elles et faire valoir leur présence au sein du mouvement principal derrière les fiertés. 
Par conséquent, celles-ci décidèrent de créer des leurs propres marches, celles-ci se déroulant quelques jours avant la marche officielle. 
C'est également la stratégie choisie par les organisatrices de la marche de Montréal des dernières années, qui ont toujours placé l'événement quelques jours avant les débuts de la semaine de la fierté. La marche des femmes quant a elle se situe durant la semaine de travail, donc, avant la marche officielle qui se déroule habituellement dans la dernière fin de semaine de la fierté.

Ce positionnement particulier et cette division du mouvement semblent conforter une certaine identité autour du mot \dyke{}, terme que~\citet{Podmore2015a} dans son article \citetitle{Podmore2015a} décrit comme politisé et radicalisé comparativement au terme plus général de lesbienne. 
Le terme \dyke{} d'ailleurs vise à englober une plus vase population que le terme lesbienne, alors que n'importe quelle femme non-hétérosexuelle peut s'identifier avec ce terme et se joindre au mouvement de la marche \dyke{}. 
Cette politisation s'exprime en même temps par une non-mixité qui vise à exclure les hommes de la marche --- exclusion basée sur le respect des principes de l'événement plutôt qu'une exclusion qu'on pourrait qualifier d'agressive. 
Les alliés intéressés par l'évènement, mais n'entrant pas dans la catégorie \dyke{} étaient invitéEs à suivre la marche à l'extérieur, en marchant sur les trottoirs. 
C'est ce que j'ai dû faire, mais j'ai pu constater l'absence d'individus se réclamant ou agissant comme \emph{alliés} (tenant des pancartes ou scandant des slogans en marge de la marche). 
Un seul autre homme était présent. 
En discutant avec cette personne j'ai pu apprendre qu'il était là par curiosité et j'ai du moi-même dû lui expliquer que sa présence dans la marche n'était pas tolérée et quels étaient les principes soutenus par cette dernière.

Le public visé par la marche dans son ensemble est moins facilement définissable. 
En fait, nous pouvons considérer que l'ensemble de la société est visée par le message de la marche, étant donné que la visibilité comme telle s'exprime dans l'espace public, comme je l'ai souligné précédemment. 
Par contre, on ne peut ignorer le fait que la marche s'inscrit dans les pratiques d'autres événements politiques similaires. 
Nous ne pouvons pas clairement savoir s'il s'agit ici d'une tradition ou un message qui demeure encore porté à l'organisation principale de la fierté. 
Également, le choix du parcours peut nous renseigner sur le public ciblé. 
Contrairement à l'édition sur laquelle~\citet{Podmore2015a} a travaillé, la marche de 2015 a complètement évité le Village gai pour commencer plus au nord et terminer dans le quartier du \anglais{Mile-End}.

Pour approfondir la question de la visibilité, l'article de \citet{Frosh2006} \citetitle{Frosh2006} offre une vue intéressante sur le partage d'un message --- \anglais{text} dans l'article --- à l'aide d'un média. 
Dans un contexte social où les interactions entre individus dans l'espace public sont réduites au minimum jusqu'à l'indifférence, se rendre visible auprès d'autrui permet d'agir à contre-courant et déranger cette indifférence. 

Pourtant, \citeauthor{Frosh2006} nous apprend que cette indifférence peut être une forme de respect ou d'intégration. 
Les pratiques sociales dans l'espace public qui se fondent sur l'indifférence témoignent chez les individus une forme de respect mutuel qui est bien différente d'une relation basée sur l'altérité et l'incompréhension. 
Si certains auteurs d'après \citeauthor{Frosh2006} considèrent la froideur des relations humaines --- l'inattention civile chez \citeauthor{Goffman1956} --- dans l'espace public comme une preuve d'un espace moralement vide et étranger, on apprend aussi que ce serait plutôt la réaction à la vision d'un autre qui sera le témoignage d'une forme de peur ou d'appréhension vis-à-vis l'autre~\citep[279--280]{Frosh2006}. 
L'action de réclamer cette visibilité dans la \dm{} pourrait être conçue comme une rupture volontaire de ce respect mutuel pour montrer que l'égalité sous-entendue n'est pas concrète et reste à faire. 
Cet acte montrerait dont une limite de l'inattention civile comme concept centré sur la communication et la perception, sois une certaine incapacité à tenir compte des minorités conçues comme invisibles ou, du moins, d'individus se regroupant autour d'une différence commune vis-à-vis la norme, ici hétérosexuelle et même masculine. 
L'appel à la non-mixité, si elle répond à une volonté de se retrouver entre individus partageant une oppression commune comme femme et lesbienne, permet également d'avoir un contrôle sur l'image qu'elles véhiculent collectivement par rapport à l'observateur qui se trouve souvent être en position de pouvoir\todo{Traiter du   Male gaze avec les   références~\cite{Wood2004,Patterson2002,Skelton2002,Snow1989}}.

\section{Festivités s'étirant sur plusieurs jours}
\label{sec:festivitesplusieursjours}
Si les marches et manifestations sont symboliquement marquantes par l'usage de nombreux géosymboles et par une subversion partielle ou complète de l'espace public, il n’en demeure pas moins que leur présence est très circonscrite dans le temps.
Nous nous intéresserons dans cette section à des événements temporaires, mais s'étirant dans le temps.
La totalité de ceux-ci consiste en des célébrations de la diversité sexuelle, certaines bien connues du public, d'autres appartenant plutôt à une certaine contre-culture ou orientée vers les individus s'identifiant spécifiquement à l'identité de genre ou l'orientation sexuelle ciblée.

\subsection{Fierté Trans}
\label{subsec:fiertetrans}
Nous avons traité brièvement de la Fierté Trans précédemment dans la section précédente à propos de la Marche Trans.
En fait, la marche s'incrit dans l'ensemble de la Fierté Trans, celle-ci étant l'activité principale autour de laquelle tourne le reste des événements.
La Fierté trans, comme la marche dyke, se déroule dans les jours précédents la fierté traditionelle.
Elle débute avec une soirée organisée au cabaret cléo où se déroule une soirée de spectacles variés, tournant principalement autour de performances dragues mettant en scène des artistes trans de la communauté.
Si les performances de dragues sont connues pour être pratiquées par des hommes homosexuels, dans ce cas-ci quelques femmes ont également joué un rôle de drag-king, soit une personnification masculine.
Pour assurer l'accessibilité de l'événement, les organisateurs ont fixé un prix libre à l'entrée, les recettes servant au financement du groupe l'\astteq{}\footnote{Selon leur site Internet,: \textquote{ASTT(e)Q a pour mission de favoriser la santé et le bien-être des personnes trans par l’intermédiaire du soutien par les pairs et de la militance, de l’éducation et de la sensibilisation, de l’empowerment et de la mobilisation.}}

\subsection{Qouleur}
\label{subsec:qouleur}
Le terme de Qouleur désigne autant le nom de l'événement que le nom du collectif derrière celui-ci.
À première vue, on peut croire que le terme Qouleur peut désigner deux réalités, sois la diversité sexuelle par un lien avec le drapeau arc-en-ciel aux multiples couleurs et sois une désignation donnée aux personnes racisées, personnes de couleur.
Entendu comme un terme moins discriminant que le terme \emph{race}, le terme couleur prend ses origines dans \todo{Trouver l'origine du mot couleur}.
Bien qu'utilisé fréquemment pour désigner les personnes d'origine afro, le terme de couleur dans le nom de l'événement semble plutôt désigner la totalité des individus racisés, sachant que les événements se déroulant dans ce festival ciblent certains groupes précis.
Si certains de ceux-ci sont publics, plusieurs d'entre eux visent pas exemple les individus autochtones par l'appellation de \emph{two-spirits}.

Une des activités accessibles à tous était la visite du \mai{}.
Durant celle-ci, il était possible d'accéder en partie à des oeuvres des personnes racisées et trans, ainsi qu'à des oeuvres de personnes en-dehors de cette communauté mais dont les sujets pouvaient être lié au thème de cette édition du festival Qouleur de 2015, l'amour et la rage.
Par contre, si cette exposition était publique, il n'était pas possible d'interagir avec d'autres individus de la communauté; lors de notre visite, en semaine et sur l'heure du midi, personne n'était présente sinon les chargés de l'accueuil du musée.

Parmi les oeuvres, on retrouvait certaines mettant en scène des membres de la communauté.
La plus plus importante, par la taille comme par le positionnement dans le local de l'exposition, était une vidéo projetée sur une toile géante dans lesquelles des personnes interviewées décrivaient leurs expériences de rage et d'amour en tant que personnes trans et racisées.
On pouvait reconnaître là certaines personnes qui participaient également à la Fierté trans.

Comme d'autres événements, certaines activités ont tourné autour de la sexualité.
C'est le cas d'une soirée \bdsm{} pour les personnes trans et queers.
Étant donné le degré d'intimité impliquée dans ce genre d'activité, le lieu n'était pas communiqué sur Facebook et il était nécessaire de prendre contact avec les organisateurs pour être convié à l'événement.
Par manque d'informations, il est difficile de dire si cet évènement était ouvert à toute personne s'identifiant comme trans ou comme queer, ou si il était nécessaire de connaître personellement une des personnes organisatrices de l'événement.
Également, aucun visuel n'était utilisé.

\subsection{Pervers/Cité}
\label{subsec:perverscite}
Pervers/Cité est un autre des nombreux événements se déroulant en parallèle à la Fierté Montréal du mois d'août.
L'existence de Pervers/Cité se démarque comme un événement créé en réaction à l'évolution des festivités plus traditionnelles de la communauté gaie montréalaise.
En effet, à la création de Pervers/Cité existait déjà un organisme organisant la fierté gaie montréalaise, Divers/Cité.
Ce dernier, nous y reviendrons plus tard, a d'abord été un événement à tendance communautaire et politique pour devenir une festival plutôt orienté vers la fête avec un gain de popularité des festivités.
Pervers/Cité est né d'une réponse à cette évolution qui a été dénoncée comme marchande et non-rassembleuse pour l'ensemble des minorités sexuelles montréalaises.
En fait, progressivement, Divers/Cité s'est mis à délaisser le côté militant des festivités de la fierté pour progressivement devenir un festival similaire à d'autres en mettant de l'avant principalement des soirées spectacles, avec des \emph{dj} et autres musiciens liés à la musique électronique.
Organisée par une faction encore très politisée, Pervers/Cité souhaitait alors offrir une alternative radicale aux festivités de Divers/Cité, l'événement se déroulant dans les mêmes journées, mais à l'extérieur du périmètre occupé par Divers/Cité, sois la rue Sainte-Catherine et plus tard le vieux port de Montréal.
Cette tendance se poursuit aujourd'hui, alors que le festival s'étend sur un vaste territoire, visible à la figure \todo{Mettre la figure}.
En général, les espaces utilisés par les organisateurs de Pervers/cité ne sont pas des lieux nécessairement occupés quotidiennement par des individus queers.
Les moyens limités d'un tel type d'organisation, fonctionnant sans financement autre que le bénévolat et les contributions volontaires, impliquent que ceux-ci ont du miser sur les centres communautaires, des locaux d'universités ou des espaces publics à quelques exceptions près pour occuper l'espace.
Par exemple, durant le salon du livre Queer entre les couvertures, une partie des ateliers ont eu lieu dans le local de l'Astérisk dont nous traiterons plus loin.

\begin{figure}[ht]
	\centering
	\includegraphics[width=15cm]{images/fig33.jpg}
	\caption{Carte produite par Pervers/Cité présentant les différents lieux des activités du festival}\label{fig:carte_perverscite}
\end{figure}

Les visuels utilisés dans les publicités et événements facebook n'avaient pas un thème ou une esthétique commune; on peut penser que les activités étaient organisées en soi par une multitude de personnes proposant leur propres matériel publicitaire, mais nous manquons d'informations à ce niveau.
En général par contre, les visuels utilisaient des termes à connotation vuldaire, sinon sexuelles, et souvent en anglais.
Par exemple, dans le cadre d'une projection publique et d'un spectacle, le nom choisi est \emph{Whorelock/up}, pouvant être traduit par \emph{Enfermement de pute}.
Dans un autre cas, l'événement nommé était \anglais{Sick, Sad Summer}, traduisable par \emph{Triste été malade} et dessiné de façon similaire à du sang, sur laquelle est utilisé l'image d'une femme corpulente et partiellement nue sous une averse.
Le nom de Pervers/Cité apparait dans un coin, dont le point sur le \enquote{i} de cité (et tous les autres de l'affiche) est remplacé par un coeur brisé.
Si la première image ne mettait pas de l'avant de code particulier en-dehors du titre, on trouve ici une image plus complexe et codée.
On peut faire le lien entre cette affiche et certaines utilisées par des événements non-queers dans lesquels la chaleur de l'été est mise de l'avant, avec tout un lot de signifiants positifs faisant l'expression du bonheur et du plaisir.

\begin{figure}
\centering
\subcaptionbox{Affiche de l'activité \anglais{Sick, Sad Summer}\label{fig:affiche_sicksadsummer}}
{\includegraphics[width=9cm]{fig34.jpg}}
\subcaptionbox{Événement de 2008 (et événement mixte)\label{fig32}}
{\includegraphics[width=6cm]{fig32.jpg}}
\caption{Exemple d'affiches variées d'activités organisées dans le cadre de Pervers/Cité}\label{figs3132}
\end{figure}

\subsubsection{Activités}
\label{subsec:activitesperverscite}
Les activités offertes dans le cadre de Pervers/Cité sont particulièrement variées.
En effet, les sujets couverts semblent parfois ne pas avoir de liens directs: certains traitent plus particulièrement du militantisme, d'autres de la sexualité et certains s'apparentent plus à des jeux ou des activités sociales.
En général du moins, l'ensemble de celles-ci touchent des thématiques subversives, soit par l'utilisation particulière de l'espace, soit par les sujets traités.

Parmi les activités tourant autour de la thématique du militantisme, on peut nommer la projection du film \emph{Pride} et de la conférence subséquente sur les liens entre syndicalisme et droits des communautés \lgbt{}.
Ne se déroulant pas nécessairement dans l'espace public, ces deux activités se sont déroulées dans des lieux semi-publics, sois un centre communautaire et un local de l'Université Concordia.
Pour l'occasion était invité un des acteurs importants des premières liaisons entre syndicalistes et groupe \lgbt{} en Grande-Bretagne.

Le Salon du livre \emph{Queers entre les couvertures} est une autre des activités importantes du festival Pervers/Cité.
Il s'agit également d'une des rares activités à se dérouler dans le village, dans le cas présent sur la rue Amherst dans le Centre communautaire de loisirs Ste-Catherine d'Alexandrie.
Prenant l'ensemble de l'espace loué, le lieu
\subsection{Fierté Montréal 2016}
\label{subsec:fiertemontreal2016}
La Fierté Montréal propose à Montréal l'une des plus importantes activités \lgbt en terme de quantité d'individus impliqués et de la quantité d'événements isolés se déroulant dans la ville de Montréal.
La grande majorité de ces activités sont organisées dans le Village gai ou dans ses environs.
Une exception d'importance est le lieu choisi pour la marche de la fierté, celle-ci se déroulant sur le boulevard René-Lévesque situé au sud de la rue Sainte-Catherine, l'axe principal du Village gai.
Ce choix est récent en ce qui concerne l'histoire de la communauté \lgbt montréalaise.
Avant, la marche organisée par le festival Divers/Cité se déroulait dans le Village, mais par une augmentation croissante de l'affluence de l'événement, il est devenu nécessaire de modifier le trajet de cette dernière.
Fierté Montréal organisant aujourd'hui les festivités de la fierté, les organisateurs et organisatrices ont choisi le boulevard pour des raisons qui semblent être d'ordre de capacité.
En effet, ce boulevard à quatre voies est beaucoup plus large que la rue Sainte-Catherine, surtout durant la saisons festivale.
Cette dernière devient piétonne durant les festivités et une partie de la surface est occupée par les terrasses des restaurants et des bars.
La présence d'une foule très importante, confirmée par notre présence à l'édition 2016, avec un très grand nombre de chars allégoriques, semble justifier ce choix de localisation.

La journée communautaire est l'une des activités importantes de la semaine de la fierté, quoique étant l'une des moins festives.
En effet, durant celle-ci, l'espace de la rue Sainte-Catherine, demeurant piétonnier, est occupé cette fois par un grand nombre de kiosques et d'installations d'abord destinés aux groupes communautaires gravitant au sein du spectre \lgbt.
On retrouve tout de même une présence commerciale importante, avec plusieurs banques présentes et entreprises offrants des produits destinés, de près ou de loin, à la communauté \lgbt.
Nous pensons surtout aux entreprises Viagra et Trojan qui offrent des produits destinés aux hommes principalement, ces derniers étant historiquement investis dans le Village.
D'autres acteurs privés étaient présents, comme \anglais{General Mills}, qui semblaient mettre l'emphase sur la concordance entre l'esthétique d'un de leurs produits, les céréales \anglais{Lucky Charms}, et l'arc-en-ciel utilisé par les communautés \lgbt.
Comme durant l'ensemble des activités de la semaine organisée par Fierté Montréal, un grand nombre d'institutions bancaires était présent.
On peut noter principalement le cas de la banque TD qui a été commenditaire des festivités, mais les autres banques majeures de la province et les caisses Desjardins étaient présents.

La marche en soi est un événement mettant en scène un grand nombre de symboles très variés, représentant en effet la disparité des organisations s'arrimant de près ou de loins à la culture ou aux buts politiques des communautés \lgbt.
Prenant ses assises dans... \todo{écrire l'histoire des marches de la fierté}.
Les premier instants de la marche marque en sois la volonté des organisateurs de représenter le plus grands nombres d'identités possibles.
Un des premiers groupes à ouvrir la marche est un contingent de personnes portant différents drapeaux représentants les différentes identités affiliées au spectre \lgbt{} sans nécessairement être intégrées explicitement dans l'acronymes.
Les drapeaux identifiés dans ce contingent correspondent aux identités gaies, lesbiennes, bisexuelles, trans, genderqueers, asexuelles, amoureuses du latex, du cuir, alliées.
Il était possible durant le reste de la marche de constater la présence de différents contingents affiliés à ces identités.
En ce qui concerne les identités plus connues publiquement, comme l'identité homosexuelle masculine, celle-ci faisait apparition dans un grand nombre de contingents, souvent sportifs.
Les autres identités étaient moins représentées, comme les personnes trans ou les femmes lesbiennes, ou encore totalement absentes en ce qui concerne les identités genderqueer ou asexuelles, quoiqu'il envisageables que des individus s'identifiant à ces identités participent à d'autres contingents.
En effet, étant les différences entre le genre et l'orientation sexuelle, nous pouvons croire que des personnes trans participent à des groupes affiliés à leur orientation sexuelle par exemple.
Il demeure par contre qu'il existe une très grande inégalité dans la visibilité de ces différentes identités, pour des raisons que nous traiterons au chapitre suivant.

La présence de groupes politiques ou politisé est un autre fait à souligner quant à la mixité des contingents de la marche.
Presque tous les partis politiques fédéraux canadiens et provinciaux québecois avaient une présence dans la marche, à l'exception notable du parti conservateur du Canada.
Des inégalités figuraient en ce qui concerne la visibilité des partis.
Nous avons noté la forte visibilité du parti libéral du Canada qui a utilisé un autobus adapté et un grand nombre d'affiches mettant en avant le logo du parti aux couleurs de l'arc-en-ciel.
Québec Solidaire avait également une présence importante mais comme les autres partis, n'adaptait pas son visuel aux couleurs mises de l'avant dans la marche.
Pour plusieurs de ces partis, les chefs étaient également présents pour saluer la foule.

Tout comme la journée communautaire, les banques étaient présentes et avaient souvent leurs propres contingents aux couleurs de leur logos et de l'arc-en-ciel.
\subsubsection{Activités}
\label{subsec:activitesfiertemontreal}



\subsection{Fête Arc-en-ciel}
\label{subsec:fetearcenciel}
Dernier événement temporaire traité dans cette section, la Fête Arc-en-ciel se déroule dans la ville de Québec durant le congé de la fête du Travail, sois durant la première fin de semaine du mois de septembre.
Comme pour la Fierté Montréal, dans ce cas-ci aussi nous avons un exemple d'un événement d'envergure ne se déroulant pas durant le début de l'été contrairement aux autres événements ailleurs dans le monde.
Le moment choisi est d'ailleurs particulièrement tard dans la saison estivale, sachant qu'à ce moment de l'année plusieurs personnes ne sont plus en vacances comme il est coutume de le faire durant les mois de juin, juillet et d'août, ou sont de retour à l'école pour les étudiants.
Par contre, on peut croire que cette raison s'explique par une proximité géographique de Montréal qui ferait une trop grande compétition aux festivités de la capitale si jamais les deux événements auraient lieu en même temps.
Cette hypothèse se confirme également par la présence de plusieurs groupes de Québec dans les événements publics de la fierté montréalaise, durant les journées communautaires et le défilé principalement.

La Fête Arc-en-ciel s'étire dans une plus courte durée que la fierté montréalaise.
En effet, celle-ci ne dure que 4 jours, sois du jeudi au dimanche, avec une quantité d'activités prévues beaucoup plus importante dans les deux dernières journées, sachant que pour plusieurs il s'agit de journées de congé.
Tout comme la fierté montréalaise, la Fête Arc-en-ciel de Québec occupe symboliquement l'espace public de façon marquée: une partie importante de la rue Saint-Jean-Baptiste et le carré d'Youville sont décorés de nombreux fanions, affiches et ornements aux couleurs de l'arc-en-ciel.
Également, durant le samedi et le dimanche, la rue Saint-Jean-Baptiste sur une partie de sa longueur est fermée à la circulation automobile.
D'autres décorations arc-en-ciel sont alors ajoutées sur les rues et sur les trottoirs, ceux-ci n'étant plus utilisés obligatoirement par les piétons.
La côte Sainte-Geneviève est également décorée tout en étant fermée d'une certaine façon à la circulation piétonnière.
En effet, si celle-ci est toujours fermée à la circulation automobile, durant les deux derniers jours de la Fête Arc-en-ciel elle devient un espace où sont organisées des activités pour la Fête Arc-en-ciel et devient également un lieu de restauration.

\subsubsection{Activités}
\label{subsec:activitesfetearcenciel}
Étant donné le cadre temporel plus restreint, moins d'activités se déroulent dans le cadre de la Fête Arc-en-ciel que dans plusieurs des événements traités précédemment.
Néanmoins, une certaine diversité existe similaire à la fierté montréalaise, sois un mélange de soirées dans les discothèques et bars, des conférences et des spectacles sur la place publique.

Comme dans la fierté montréalaise, on peut noter une certaine volonté de l'organisation de cibler un spectre plus large que la simple communauté gaie masculine.
Une conférence par exemple a eu lieu sur les enjeux vécus par les personnes trans la veille du début de la Fête Arc-en-ciel et à l'extérieur de celle-ci, dans un bar de la basse-ville.
Si la visibilité de cet événement est moindre que plusieurs autres sur le plan physique, on retrouvait quand même l'événement dans la publicité du festival ainsi que sur \emph{Facebook}.

Certains événements ont ciblé plus particulièrement la communauté lesbienne, comme \todo{trouver exemples}.

En ce qui concerne les événements pour hommes gais, on peut nommer les soirées \emph{Boys Gone Wild}\todo{Vérifier si c'est le bon nom} et \todo{trouver l'autre événement} se déroulant tous les deux au bar \emph{Saint-Matthew's}, un bar pour hommes comme nommé à la section \todo{trouver le numéro de section si déjà écrit}.
Ces activités, bien que s'inscrivant dans l'édition 2015 de la fête arc-en-ciel, sont également des activités qui animent la communauté gaie de la capitale durant le reste de l'année.
Ces événements se produisent à raison de trois à quatre fois par années et sont organisées par l'Alliance arc-en-ciel en collaboration avec le bar \emph{Saint-Matthew's}.
Souvent thématiques, ces soirées permettent aux hommes de se rencontrer dans un espace \e ph{de facto} non-mixte sans être affiché comme tel, comparativement aux activités organisées par les communautés \emph{dykes} ou racisées montréalaises.
La non-mixité sert donc ici surtout à organiser des rencontres entre hommes plutôt qu'être une revendication politique.
Les thématiques choisies témoignent de cet apolitisme, par un accent mis sur des fêtes traditionnelles, comme noël, ou certains fétichismes communs à la communauté gaie, comme l'équipement de sport, la masculinité, le cuir, etc.
D'après l'offre de bars et d'autres lieux de sociabilité dans la ville de Québec, ces activités semblent répondre à un manque de bars spécialisés permettant de répondre aux différents goûts et intérêts des hommes gais.
Ceci est d'autant plus plus visible en comparaison avec la ville de Montréal qui possède en son sein un nombre varié de lieux de rencontre.

Ciblant une partie de la communauté plus politisée, le dernier événement a été une soirée \queer, organisée elle aussi en-dehors du secteur principal de la fête arc-en-ciel.
Il s'agissait même de l'activité la plus excentrée, celle-ci s'étant déroulée dans la bar l'Anti, situé dans la limite nord du quartier Saint-Roch, en basse-ville de Québec.
Cet événement mixte s'inscrit dans la suite d'autre événements à thématiques similaires organisées dans le cadre ou non de la fête arc-en-ciel, surtout dans des bars alternatifs mais non affiliés clairement à la communauté \lgbt.
Malgré la charge politique portée par le thème \qu, il ne s'agissait pas en soi d'un événement politique mais d'une soirée mélangeant spectacle et danse.
Le premier spectacle était l'oeuvre d'un collectif \qu de la ville de Rimouski dans lequel le jeu autour du genre était mis de l'avant.
Il s'agissait également pour cette occasion d'une soirée de financement pour le \ggul.
Si plusieurs autres soirées visait un genre en particulier, celle-ci se démarquait surtout par la jeunesse de l'auditoire et la mixité des genres.
Nous pouvoir croire que le terme \qu, utilisé récemment au Québec \todo{voir si je pourrais citer laprade ici}, semble être connu surtout par la jeunesse \lgbt.
Certains participants plus âgés vinrent au début de la soirée participer à la soirée, mais ne restèrent pas en général tout au long de la soirée.

Une autre particularité de la Fête Arc-en-ciel et d'inviter des personnalités connues ou représentatives des minorités sexuelles.
Dans l'édition de 2016, la personne invitée a été Michèle Richard, désignée comme icône sois pour sa popularité chez une partie du spectre \lgbt{}, sois pas son interprétation de la chanson Disco \emph{I will survive} de \todo{Trouver le nom de la chanteuse}.
Si cette chanson traite normalement du sentiment de rupture et de la volonté d'indépendance de la chanteuse, dans ce cas précis, l'expression \emph{I will survive} ou \emph{Je vais survivre} dans l'interprétation de Michèle Richard, fait écho à la résilience des communautés \lgbt depuis sa sortie de la clandestinité et plus particulièrement de la crise du \vih{}.

Enfin, on peut souligner le virage familial mis de l'avant par l'organisation de la Fête Arc-en-ciel dans la volonté d'organiser un pique-nique.
En effet, ce type d'événement semble plutôt être une occasion pour les familles, homoparentales ou non, de se rencontrer et de partager un moment ensemble.
Cette absence de contenu politique ou festif semble témoigner de l'ouverture plus grande de la société face aux enjeux \lgbt, ici la famille.
Un contexte sans revendication dans laquelle les participants pratiquent une activité pouvant être qualifiée de normale --- sois similaire à la norme de la famille hétérosexuelle --- montre plutôt que ces familles peuvent être comme les autres et utiliser l'espace public comme n'importe quel ménage.

La marche en soi est un événement mettant en scène un grand nombre de symboles très variés, représentant en effet la disparité des organisations s'arrimant de près ou de loins à la culture ou aux buts politiques des communautés \lgbt.
En effet, tout comme la marche montréalaise, on peut constater un grand nombre de drapeau et de logos disséminés entre les participants à la marche.
Également, plusieurs personnalités publiques s'affichent comme participants à la marche.
Étant donné le cadre plus réduit de la marche de Québec en comparaison avec celle de Montréal, l'ensemble des participants étaient réunis en une seule congrégation marchant dans les rues de la ville.
En soi, l'ensemble était comparable à une manifestation politique plutôt qu'à un défilé de la fierté plus traditionnel dans lesquelles les groupes sont séparés par thématiques et utilisent des véhicules divers pour occuper l'espace de façon plus visible.
Dans ce contexte, les personnalités connues, souvent des politiciens, se placèrent à l'avant de la marche et ont tenu une banière fournie par l'organisation de l'événement permettant d'identifier facilement le contexte de cette marche.
D'autres individus plus visibles étaient aussi présents, sois des personnes appartenant au spectre \bdsm{}.
Ceux-ci étaient placés directement derrière les personnalités publiques et sont apparues à plusieurs reprises dans les photographies que nous avons prises de l'événement.
Nous avons noté la présence de personnes vêtues de latex, de cuir et également de deux pratiquants du \anglais{puppy play}, par les symboles marquants leurs vêtements mais aussi l'usage de masque à l'effigie de chien.
Ces personnes détonnaient du reste du défilé sachant que les participants n'étaient pas vétus de façon particulière, à l'exception de quelques accessoires à l'effigie du drapeau arc-en-ciel.

Ce format de marche permit de faire participer le public.
En effet, dans un défilé traditionnel, la foule est invitée à occuper les marges de la marche, sois les trottoirs autour des boulevards ou des rues utilisées.
Elles sont donc bien souvent immobiles avec la possibilité de pouvoir voir l'ensemble de la marche.
Dans celle de Québec, l'ensemble de la marche était composée d'individus n'ayant pas de rôle particulier et le public extérieur était surtout composé de passants, de touristes ou encore de consommateurs dans les restaurants et les magasins longeants les rues où passait la marche.
Certains des participants avaient amener avec eux des pancartes affichants des slogans, mais en général moins de symboles étaient convoyés par la marche de Québec que celle de Montréal, en proportion à la quantitié de participants. 

Tout comme le défilé de Montréal, la marche de Québec s'est déroulé en écart du centre géographique des célébrations.
Commençant au Carré d'Youville où était organisé plusieurs spectacles extérieurs, la marche poursuivie sur la rue Saint-Jean-Baptiste mais dans la section opposée au secteur occupé par les kiosques des journées communautaires.
Après avoir passé devant un lot de restaurant et de commerce, elle continua dans un secteur moins occupé du quartier pour se terminer devant le parlement de la province.

\subsubsection{Activités}
\label{subsec:activitesfiertemontreal}



% \subsection{Fête Arc-en-ciel}
% \label{subsec:fetearcenciel}
% Dernier événement temporaire traité dans cette section, la Fête Arc-en-ciel se déroule dans la ville de Québec durant le congé de la fête du Travail, sois durant la première fin de semaine du mois de septembre.
% Comme pour la Fierté Montréal, dans ce cas-ci aussi nous avons un exemple d'un événement d'envergure ne se déroulant pas durant le début de l'été contrairement aux autres événements ailleurs dans le monde.
% Le moment choisi est d'ailleurs particulièrement tard dans la saison estivale, sachant qu'à ce moment de l'année plusieurs personnes ne sont plus en vacances comme il est coutume de le faire durant les mois de juin, juillet et d'août, ou sont de retour à l'école pour les étudiants.
% Par contre, on peut croire que cette raison s'explique par une proximité géographique de Montréal qui ferait une trop grande compétition aux festivités de la capitale si jamais les deux événements auraient lieu en même temps.
% Cette hypothèse se confirme également par la présence de plusieurs groupes de Québec dans les événements publics de la fierté montréalaise, durant les journées communautaires et le défilé principalement.

% La Fête Arc-en-ciel s'étire dans une plus courte durée que la fierté montréalaise.
% En effet, celle-ci ne dure que 4 jours, sois du jeudi au dimanche, avec une quantité d'activités prévues beaucoup plus importante dans les deux dernières journées, sachant que pour plusieurs il s'agit de journées de congé.
% Tout comme la fierté montréalaise, la Fête Arc-en-ciel de Québec occupe symboliquement l'espace public de façon marquée: une partie importante de la rue Saint-Jean-Baptiste et le carré d'Youville sont décorés de nombreux fanions, affiches et ornements aux couleurs de l'arc-en-ciel.
% Également, durant le samedi et le dimanche, la rue Saint-Jean-Baptiste sur une partie de sa longueur est fermée à la circulation automobile.
% D'autres décorations arc-en-ciel sont alors ajoutées sur les rues et sur les trottoirs, ceux-ci n'étant plus utilisés obligatoirement par les piétons.
% La côte Sainte-Geneviève est également décorée tout en étant fermée d'une certaine façon à la circulation piétonnière.
% En effet, si celle-ci est toujours fermée à la circulation automobile, durant les deux derniers jours de la Fête Arc-en-ciel elle devient un espace où sont organisées des activités pour la Fête Arc-en-ciel et devient également un lieu de restauration.

% \subsubsection{Activités}
% \label{subsec:activitesfetearcenciel}
% Étant donné le cadre temporel plus restreint, moins d'activités se déroulent dans le cadre de la Fête Arc-en-ciel que dans plusieurs des événements traités précédemment.
% Néanmoins, une certaine diversité existe similaire à la fierté montréalaise, sois un mélange de soirées dans les discothèques et bars, des conférences et des spectacles sur la place publique.

% Comme dans la fierté montréalaise, on peut noter une certaine volonté de l'organisation de cibler un spectre plus large que la simple communauté gaie masculine.
% Une conférence par exemple a eu lieu sur les enjeux vécus par les personnes trans la veille du début de la Fête Arc-en-ciel et à l'extérieur de celle-ci.
% Si la visibilité de cet événement est moindre que plusieurs autres sur le plan physique, on retrouvait quand même l'événement dans la publicité du festival ainsi que sur \emph{Facebook}.

% Certains événements ont ciblé plus particulièrement la communauté lesbienne, comme \todo{trouver exemples}.

% En ce qui concerne les événements pour hommes gais, on peut nommer les soirées \emph{Boys Gone Wild}\todo{Vérifier si c'est le bon nom} et \todo{trouver l'autre événement} se déroulant tous les deux au bar \emph{Saint-Matthew's}, un bar pour hommes comme nommé à la section \todo{trouver le numéro de section si déjà écrit}.

% Une autre particularité de la Fête Arc-en-ciel et d'inviter des personnalités connues ou représentatives des minorités sexuelles.
% Dans l'édition de 2016, la personne invitée a été Michèle Richard, désignée comme icône sois pour sa popularité chez une partie du spectre \lgbt{}, sois pas son interprétation de la chanson Disco \emph{I will survive} de \todo{Trouver le nom de la chanteuse}.
% Si cette chanson traite normalement du sentiment de rupture et de la volonté d'indépendance de la chanteuse, dans ce cas précis, l'expression \emph{I will survive} ou \emph{Je vais survivre} dans l'interprétation de Michèle Richard, fait écho à la résilience des communautés \lgbt depuis sa sortie de la clandestinité et plus particulièrement de la crise du \vih{}.

% Enfin, on peut souligner le virage familial mis de l'avant par l'organisation de la Fête Arc-en-ciel dans la volonté d'organiser un pique-nique.

%%% Local Variables:
%%% mode: latex
%%% TeX-master: "../../memoire-maitrise"
%%% End:
