\subsection{Fête Arc-en-ciel}
\label{subsec:fetearcenciel}
Dernier événement temporaire traité dans cette section, la Fête Arc-en-ciel se déroule dans la ville de Québec durant le congé de la fête du Travail, sois à la première fin de semaine du mois de septembre.
Comme pour la Fierté Montréal, dans ce cas-ci aussi nous avons un exemple d'un événement d'envergure ne se déroulant pas durant le début de l'été contrairement aux autres événements similaires ailleurs dans le monde.
Contrairement à la Fierté de Montréal, le moment choisi est particulièrement tard dans la saison estivale.
Normalement, les événements festifs cherchant à accueillir de vastes publics visent surtout les mois durant lesquels les personnes ayant des emplois traditionnels ou les étudiants sont en vacances.
Par contre, cette décision pourrait s'expliquer par une proximité géographique de Montréal.
La métropole pourrait en effet faire une trop grande compétition aux festivités de la capitale dans l'éventualité où les deux événements auraient lieu en même temps.
Cette hypothèse se confirme également par la présence de plusieurs groupes de Québec dans les événements publics de la fierté montréalaise, durant les journées communautaires et le défilé principalement.
Aussi, à l'échelle temporelle, la Fête Arc-en-ciel s'étire dans une plus courte durée que la fierté montréalaise.
En effet, celle-ci ne dure que 4 jours, sois du jeudi au dimanche, avec une quantité d'activités prévues beaucoup plus importante dans les deux dernières journées, sachant que pour plusieurs il s'agit de journées de congé.

Tout comme la fierté montréalaise, la Fête Arc-en-ciel de Québec occupe symboliquement l'espace public de façon marquée.
La partie ouest de la rue Saint-Jean-Baptiste et le carré d'Youville ont été décorés de nombreux fanions, affiches et ornements aux couleurs de l'arc-en-ciel.
Également, durant le samedi et le dimanche, la rue Saint-Jean-Baptiste sur une partie de sa longueur est fermée à la circulation automobile.
D'autres décorations arc-en-ciel sont alors ajoutées sur les rues et sur les trottoirs, ceux-ci n'étant plus utilisés obligatoirement par les piétons.
La côte Sainte-Geneviève est également décorée tout en ne permettant pas la circulation piétonnière.
En effet, si celle-ci est toujours fermée à la circulation automobile, durant les deux derniers jours de la Fête Arc-en-ciel elle devient un espace où sont organisées des activités pour la Fête Arc-en-ciel. Parmi ces activités, on dénombre par exemple des \anglais{stands} de photographie, des jeux pour enfants, et également des lieux de restauration.

\subsubsection{Activités}
\label{subsec:activitesfetearcenciel}
Étant donné le cadre temporel plus restreint, moins d'activités se déroulent dans le cadre de la Fête Arc-en-ciel que dans plusieurs des événements traités précédemment.
Néanmoins, une certaine diversité existe analogue à la fierté montréalaise, sois un mélange de soirées dans les discothèques et bars, des conférences et des spectacles sur la place publique.

Comme dans la fierté montréalaise, on peut constater une volonté de l'organisation de cibler un spectre plus large que la simple communauté gaie masculine.
Une conférence par exemple a eu lieu sur les enjeux vécus par les personnes trans la veille du début de la Fête Arc-en-ciel et à l'extérieur de celle-ci, dans un bar de la basse-ville.
Si la visibilité de cet événement est moindre que plusieurs autres sur le plan physique, on retrouvait quand même celui-ci dans la publicité du festival ainsi que sur \emph{Facebook}.

Certains événements ont ciblé plus particulièrement la communauté lesbienne, comme \todo{trouver exemples}.

En ce qui concerne les événements pour hommes gais, on peut nommer les soirées \emph{Boys Gone Wild}\todo{Vérifier si c'est le bon nom} et \todo{trouver l'autre événement} se déroulant tous les deux au Bar \emph{Saint-Matthew's}, un bar pour hommes comme nous l'avons souligné à la section \todo{trouver le numéro de section si déjà écrit}.
Ces activités, bien que s'inscrivant dans l'édition 2015 de la Fête Arc-en-ciel, animent également la communauté gaie de la capitale durant le reste de l'année.
Ces événements se produisent à raison de trois à quatre fois par année et sont organisés par l'Alliance Arc-en-ciel en collaboration avec le Bar \emph{Saint-Matthew's}.
Souvent thématiques, ces soirées permettent aux hommes de se rencontrer dans un espace \emph{de facto} non mixte, sans être affiché comme tel, comparativement aux activités organisées par les communautés montréalaises \emph{dykes} ou racisées.
La non-mixité sert donc ici surtout à rendre possibles des rencontres entre hommes plutôt qu'être une revendication politique.
Les thématiques choisies témoignent de cet apolitisme, par un accent mis sur des fêtes traditionnelles, comme Noël, ou certains fétichismes communs à la communauté gaie, comme l'équipement de sport, la masculinité, le cuir, etc.
D'après l'offre de bars et d'autres lieux de sociabilité dans la ville de Québec, ces activités semblent répondre à un manque de bars spécialisés permettant de satisfaire les différents goûts et intérêts des hommes gais.
Ceci est d'autant plus plus visible en comparaison avec la ville de Montréal qui possède en son sein un nombre varié de lieux de rencontre.

Ciblant une partie de la communauté plus politisée, le dernier événement a été une soirée \qu{, celle-ci étant organisée aussi en dehors du secteur principal de la Fête Arc-en-ciel.
Il s'agissait de l'activité la plus excentrée; celle-ci s'est déroulée dans le bar L'Anti situé dans la partie nord du quartier Saint-Roch, en basse-ville de Québec.
Cet événement mixte s'inscrit dans la suite d'autres à thématiques similaires organisées dans les éditions précédentes de la Fête Arc-en-ciel ou de manière indépendante ailleurs dans l'année.
La soirée queer et ses prédécesseurs se sont surtout déroulées dans des bars alternatifs, mais non affiliés de façon évidente à la communauté \lgbt.
Malgré la charge symbolique portée par le thème \qu, il ne s'agissait pas en soi d'un événement politique, mais d'une soirée mélangeant musique, chant et danse.
Le premier spectacle était l'œuvre d'un collectif \qu rimouskois dans lequel le jeu autour du genre était clairement affiché par l'usage du drag.
Il s'agissait également pour cette occasion d'une activité de financement pour le \ggul.
Si plusieurs autres soirées visaient un genre en particulier, celle-ci se démarquait surtout par la jeunesse de l'auditoire et la mixité des personnes présentes.
Nous pouvons croire que le terme \qu, utilisé récemment au Québec \todo{voir si je pourrais citer Laprade ici}, semble être connu surtout par la jeunesse \lgbt.
Certains participants qui semblaient plus âgés que la moyenne du public sont venus au début de la soirée pour participer, mais ils ne sont pas restés tout au long, en général.

Une autre particularité de la Fête Arc-en-ciel était d'inviter des personnalités connues ou représentatives des minorités sexuelles.
Dans l'édition de 2016, la personne choisie a été Michèle Richard, désignée comme icône sois pour sa popularité chez une partie du spectre \lgbt{, sois pas son interprétation de la chanson disco \emph{I will survive} de \todo{Trouver le nom de la chanteuse}.
Si cette chanson traite normalement du sentiment de rupture et de la volonté d'indépendance de la chanteuse, dans ce cas précis, l'expression \emph{I will survive}\footnote{\emph{Je vais survivre} dans l'interprétation de Michèle Richard}, fait écho à la résilience des communautés \lgbt depuis sa sortie de la clandestinité et plus particulièrement de la crise du \vih{.

Enfin, on peut souligner le virage familial mis de l'avant par l'organisation de la Fête Arc-en-ciel dans la volonté d'organiser un pique-nique.
En effet, ce type d'événement semble plutôt être une occasion pour les familles, homoparentales ou non, de se rencontrer et de partager un moment ensemble.
Cette absence de contenu politique ou festif peut témoigner d'une ouverture perçue plus grande de la société au sujet des enjeux \lgbt, ici la famille, au sein des marcheurs.
Un contexte sans revendications dans lequel les participants pratiquent une activité pouvant être qualifiée de normale --- concordant avec la norme de la famille hétérosexuelle --- montre plutôt que ces familles peuvent être semblables aux autres et utiliser l'espace public comme n'importe quel ménage.

La marche en soi est un événement mettant en scène une quantité et une variété importante de symboles, représentant la disparité des organisations s'arrimant de près ou de loin à la culture ou aux buts politiques des communautés \lgbt.
En effet, tout comme l'équivalent montréalais, on peut constater un grand nombre de drapeau et de logos disséminés entre les participants.
Également, plusieurs personnalités publiques s'affichent comme adhérant à la marche.
Étant donné le cadre plus réduit de la marche de Québec en comparaison avec celle de Montréal, l'ensemble des participants étaient réunis en une seule congrégation marchant dans les rues de la ville.
En soi, le tout était comparable à une manifestation politique plutôt qu'à un défilé de la fierté plus traditionnel.
En effet, dans ceux-ci, les groupes sont séparés par thématiques, ils utilisent des véhicules divers pour occuper l'espace de façon plus évocatrice et structurée et le public est à l'extérieur dans une position d'observation, ce qui n'a pas été le cas ici.
Plutôt, les personnes connues, souvent des politiciens, se placèrent à l'avant de la marche et ont tenu une bannière fournie par l'organisation de l'événement permettant d'identifier facilement le contexte de cette marche.
D'autres individus plus visibles étaient aussi présents, sois des personnes appartenant au spectre \bdsm{.
Ceux-ci étaient localisés directement derrière les personnalités publiques et sont apparus à plusieurs reprises dans les photographies que nous avons prises de l'événement.
Nous avons noté la présence de participants vêtus de latex, de cuir et également de deux pratiquants du \anglais{puppy play}, par les symboles marquant leurs vêtements, mais aussi l'usage de masque à l'effigie de chien.
Ces personnes détonnaient du reste du défilé sachant que les autres n'étaient pas vêtus de façon particulière, à l'exception de quelques accessoires à l'effigie du drapeau arc-en-ciel.

Ce format de marche permit de faire participer le public.
En effet, dans un défilé traditionnel, la foule est invitée à occuper les marges, sois les trottoirs autour des boulevards ou des rues utilisées.
Elle est alors bien souvent immobile, mais les participants ont ainsi la possibilité de voir l'ensemble de la marche.

Dans celle de Québec, la marche était composée d'individus n'ayant pas de rôles particuliers à jouer.
Le public extérieur était donc surtout constitué de passants, de touristes ou encore de consommateurs dans les restaurants et les magasins longeant les rues où passait la marche.
Ce public était plutôt passif, mais à de nombreuses reprises, les personnes à l'avant de la marche lançaient des appels à crier et à applaudir les participants et, surtout, à montrer leur ouverture à la diversité sexuelle.
En général, les gens répondaient à la demande et nous n'avons pas remarqué de réactions négatives ou d'homophobie en réaction à ces appels.
Certains des marcheurs avaient amené avec eux des pancartes affichant des slogans, mais en comparaison moins de symboles étaient convoyés par le défilé de Québec que celui de Montréal, en proportion par rapport à la quantité de participants.

Tout comme le défilé de Montréal, la marche de Québec s'est déroulée en écart du centre géographique des célébrations.
Commençant au Carré d'Youville où étaient organisés plusieurs spectacles extérieurs, la marche s'est poursuivie sur la rue Saint-Jean-Baptiste dans la section opposée au secteur occupé par les kiosques des journées communautaires.
Après être passée devant un lot de restaurants et de commerces, elle a continué dans une partie moins fréquentée du quartier pour se terminer face au parlement provincial.
Ce choix de parcours contraste particulièrement avec celui de Montréal où l'axe choisi, le boulevard René-Lévesque, est beaucoup plus large et offre une plus grande visibilité.
Celui de la marche de Québec, s'il a débuté dans un endroit très passant, s'est poursuivi dans des espaces peu visités et visibles du quartier, au détriment d'une certaine visibilité.
Par contre, ce choix de passage semble avoir permis une plus grande densité de personnes participantes et ainsi, donner l'impression d'une grande participation, un effet plus difficile à atteindre sur un boulevard plus large.

Dernier événement temporaire traité dans cette section, la Fête Arc-en-ciel se déroule dans la ville de Québec durant le congé de la fête du Travail, sois à la première fin de semaine du mois de septembre.
Comme pour la Fierté Montréal, dans ce cas-ci aussi nous avons un exemple d'un événement d'envergure ne se déroulant pas durant le début de l'été contrairement aux autres événements similaires ailleurs dans le monde.
Contrairement à la Fierté de Montréal, le moment choisi est particulièrement tard dans la saison estivale.
Normalement, les événements festifs cherchant à accueillir de vastes publics visent surtout les mois durant lesquels les personnes ayant des emplois traditionnels ou les étudiants sont en vacances.
Par contre, cette décision pourrait s'expliquer par une proximité géographique de Montréal.
La métropole pourrait en effet faire une trop grande compétition aux festivités de la capitale dans l'éventualité où les deux événements auraient lieu en même temps.
Cette hypothèse se confirme également par la présence de plusieurs groupes de Québec dans les événements publics de la fierté montréalaise, durant les journées communautaires et le défilé principalement.
Aussi, à l'échelle temporelle, la Fête Arc-en-ciel s'étire dans une plus courte durée que la fierté montréalaise.
En effet, celle-ci ne dure que 4 jours, sois du jeudi au dimanche, avec une quantité d'activités prévues beaucoup plus importante dans les deux dernières journées, sachant que pour plusieurs il s'agit de journées de congé.

Tout comme la fierté montréalaise, la Fête Arc-en-ciel de Québec occupe symboliquement l'espace public de façon marquée.
La partie ouest de la rue Saint-Jean-Baptiste et le carré d'Youville ont été décorés de nombreux fanions, affiches et ornements aux couleurs de l'arc-en-ciel.
Également, durant le samedi et le dimanche, la rue Saint-Jean-Baptiste sur une partie de sa longueur est fermée à la circulation automobile.
D'autres décorations arc-en-ciel sont alors ajoutées sur les rues et sur les trottoirs, ceux-ci n'étant plus utilisés obligatoirement par les piétons.
La côte Sainte-Geneviève est également décorée tout en ne permettant pas la circulation piétonnière.
En effet, si celle-ci est toujours fermée à la circulation automobile, durant les deux derniers jours de la Fête Arc-en-ciel elle devient un espace où sont organisées des activités pour la Fête Arc-en-ciel. Parmi ces activités, on dénombre par exemple des \anglais{stands} de photographie, des jeux pour enfants, et également des lieux de restauration.

\subsubsection{Activités}
\label{subsec:activitesfetearcenciel}
Étant donné le cadre temporel plus restreint, moins d'activités se déroulent dans le cadre de la Fête Arc-en-ciel que dans plusieurs des événements traités précédemment.
Néanmoins, une certaine diversité existe analogue à la fierté montréalaise, sois un mélange de soirées dans les discothèques et bars, des conférences et des spectacles sur la place publique.

Comme dans la fierté montréalaise, on peut constater une volonté de l'organisation de cibler un spectre plus large que la simple communauté gaie masculine.
Une conférence par exemple a eu lieu sur les enjeux vécus par les personnes trans la veille du début de la Fête Arc-en-ciel et à l'extérieur de celle-ci, dans un bar de la basse-ville.
Si la visibilité de cet événement est moindre que plusieurs autres sur le plan physique, on retrouvait quand même celui-ci dans la publicité du festival ainsi que sur \emph{Facebook}.

Certains événements ont ciblé plus particulièrement la communauté lesbienne, comme \todo{trouver exemples}.

En ce qui concerne les événements pour hommes gais, on peut nommer les soirées \emph{Boys Gone Wild}\todo{Vérifier si c'est le bon nom} et \todo{trouver l'autre événement} se déroulant tous les deux au Bar \emph{Saint-Matthew's}, un bar pour hommes comme nous l'avons souligné à la section \todo{trouver le numéro de section si déjà écrit}.
Ces activités, bien que s'inscrivant dans l'édition 2015 de la Fête Arc-en-ciel, animent également la communauté gaie de la capitale durant le reste de l'année.
Ces événements se produisent à raison de trois à quatre fois par année et sont organisés par l'Alliance Arc-en-ciel en collaboration avec le Bar \emph{Saint-Matthew's}.
Souvent thématiques, ces soirées permettent aux hommes de se rencontrer dans un espace \emph{de facto} non mixte, sans être affiché comme tel, comparativement aux activités organisées par les communautés montréalaises \emph{dykes} ou racisées.
La non-mixité sert donc ici surtout à rendre possibles des rencontres entre hommes plutôt qu'être une revendication politique.
Les thématiques choisies témoignent de cet apolitisme, par un accent mis sur des fêtes traditionnelles, comme Noël, ou certains fétichismes communs à la communauté gaie, comme l'équipement de sport, la masculinité, le cuir, etc.
D'après l'offre de bars et d'autres lieux de sociabilité dans la ville de Québec, ces activités semblent répondre à un manque de bars spécialisés permettant de satisfaire les différents goûts et intérêts des hommes gais.
Ceci est d'autant plus plus visible en comparaison avec la ville de Montréal qui possède en son sein un nombre varié de lieux de rencontre.

Ciblant une partie de la communauté plus politisée, le dernier événement a été une soirée \qu{}, celle-ci étant organisée aussi en dehors du secteur principal de la Fête Arc-en-ciel.
Il s'agissait de l'activité la plus excentrée; celle-ci s'est déroulée dans le bar L'Anti situé dans la partie nord du quartier Saint-Roch, en basse-ville de Québec.
Cet événement mixte s'inscrit dans la suite d'autres à thématiques similaires organisées dans les éditions précédentes de la Fête Arc-en-ciel ou de manière indépendante ailleurs dans l'année.
La soirée queer et ses prédécesseurs se sont surtout déroulées dans des bars alternatifs, mais non affiliés de façon évidente à la communauté \lgbt.
Malgré la charge symbolique portée par le thème \qu, il ne s'agissait pas en soi d'un événement politique, mais d'une soirée mélangeant musique, chant et danse.
Le premier spectacle était l'œuvre d'un collectif \qu rimouskois dans lequel le jeu autour du genre était clairement affiché par l'usage du drag.
Il s'agissait également pour cette occasion d'une activité de financement pour le \ggul.
Si plusieurs autres soirées visaient un genre en particulier, celle-ci se démarquait surtout par la jeunesse de l'auditoire et la mixité des personnes présentes.
Nous pouvons croire que le terme \qu, utilisé récemment au Québec \todo{voir si je pourrais citer Laprade ici}, semble être connu surtout par la jeunesse \lgbt.
Certains participants qui semblaient plus âgés que la moyenne du public sont venus au début de la soirée pour participer, mais ils ne sont pas restés tout au long, en général.

Une autre particularité de la Fête Arc-en-ciel était d'inviter des personnalités connues ou représentatives des minorités sexuelles.
Dans l'édition de 2016, la personne choisie a été Michèle Richard, désignée comme icône sois pour sa popularité chez une partie du spectre \lgbt{}, sois pas son interprétation de la chanson disco \emph{I will survive} de \todo{Trouver le nom de la chanteuse}.
Si cette chanson traite normalement du sentiment de rupture et de la volonté d'indépendance de la chanteuse, dans ce cas précis, l'expression \emph{I will survive}\footnote{\emph{Je vais survivre} dans l'interprétation de Michèle Richard}, fait écho à la résilience des communautés \lgbt depuis sa sortie de la clandestinité et plus particulièrement de la crise du \vih{}.

Enfin, on peut souligner le virage familial mis de l'avant par l'organisation de la Fête Arc-en-ciel dans la volonté d'organiser un pique-nique.
En effet, ce type d'événement semble plutôt être une occasion pour les familles, homoparentales ou non, de se rencontrer et de partager un moment ensemble.
Cette absence de contenu politique ou festif peut témoigner d'une ouverture perçue plus grande de la société au sujet des enjeux \lgbt, ici la famille, au sein des marcheurs.
Un contexte sans revendications dans lequel les participants pratiquent une activité pouvant être qualifiée de normale --- concordant avec la norme de la famille hétérosexuelle --- montre plutôt que ces familles peuvent être semblables aux autres et utiliser l'espace public comme n'importe quel ménage.

La marche en soi est un événement mettant en scène une quantité et une variété importante de symboles, représentant la disparité des organisations s'arrimant de près ou de loin à la culture ou aux buts politiques des communautés \lgbt.
En effet, tout comme l'équivalent montréalais, on peut constater un grand nombre de drapeau et de logos disséminés entre les participants.
Également, plusieurs personnalités publiques s'affichent comme adhérant à la marche.
Étant donné le cadre plus réduit de la marche de Québec en comparaison avec celle de Montréal, l'ensemble des participants étaient réunis en une seule congrégation marchant dans les rues de la ville.
En soi, le tout était comparable à une manifestation politique plutôt qu'à un défilé de la fierté plus traditionnel.
En effet, dans ceux-ci, les groupes sont séparés par thématiques, ils utilisent des véhicules divers pour occuper l'espace de façon plus évocatrice et structurée et le public est à l'extérieur dans une position d'observation, ce qui n'a pas été le cas ici.
Plutôt, les personnes connues, souvent des politiciens, se placèrent à l'avant de la marche et ont tenu une bannière fournie par l'organisation de l'événement permettant d'identifier facilement le contexte de cette marche.
D'autres individus plus visibles étaient aussi présents, sois des personnes appartenant au spectre \bdsm{}.
Ceux-ci étaient localisés directement derrière les personnalités publiques et sont apparus à plusieurs reprises dans les photographies que nous avons prises de l'événement.
Nous avons noté la présence de participants vêtus de latex, de cuir et également de deux pratiquants du \anglais{puppy play}, par les symboles marquant leurs vêtements, mais aussi l'usage de masque à l'effigie de chien.
Ces personnes détonnaient du reste du défilé sachant que les autres n'étaient pas vêtus de façon particulière, à l'exception de quelques accessoires à l'effigie du drapeau arc-en-ciel.

Ce format de marche permit de faire participer le public.
En effet, dans un défilé traditionnel, la foule est invitée à occuper les marges, sois les trottoirs autour des boulevards ou des rues utilisées.
Elle est alors bien souvent immobile, mais les participants ont ainsi la possibilité de voir l'ensemble de la marche.

Dans celle de Québec, la marche était composée d'individus n'ayant pas de rôles particuliers à jouer.
Le public extérieur était donc surtout constitué de passants, de touristes ou encore de consommateurs dans les restaurants et les magasins longeant les rues où passait la marche.
Ce public était plutôt passif, mais à de nombreuses reprises, les personnes à l'avant de la marche lançaient des appels à crier et à applaudir les participants et, surtout, à montrer leur ouverture à la diversité sexuelle.
En général, les gens répondaient à la demande et nous n'avons pas remarqué de réactions négatives ou d'homophobie en réaction à ces appels.
Certains des marcheurs avaient amené avec eux des pancartes affichant des slogans, mais en comparaison moins de symboles étaient convoyés par le défilé de Québec que celui de Montréal, en proportion par rapport à la quantité de participants.

Tout comme le défilé de Montréal, la marche de Québec s'est déroulée en écart du centre géographique des célébrations.
Commençant au Carré d'Youville où étaient organisés plusieurs spectacles extérieurs, la marche s'est poursuivie sur la rue Saint-Jean-Baptiste dans la section opposée au secteur occupé par les kiosques des journées communautaires.
Après être passée devant un lot de restaurants et de commerces, elle a continué dans une partie moins fréquentée du quartier pour se terminer face au parlement provincial.
Ce choix de parcours contraste particulièrement avec celui de Montréal où l'axe choisi, le boulevard René-Lévesque, est beaucoup plus large et offre une plus grande visibilité.
Celui de la marche de Québec, s'il a débuté dans un endroit très passant, s'est poursuivi dans des espaces peu visités et visibles du quartier, au détriment d'une certaine visibilité.
Par contre, ce choix de passage semble avoir permis une plus grande densité de personnes participantes et ainsi, donner l'impression d'une grande participation, un effet plus difficile à atteindre sur un boulevard plus large.
