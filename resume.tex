\chapter*{Résumé}                      % ne pas numéroter
\phantomsection\addcontentsline{toc}{chapter}{Résumé} % inclure dans TdM

\begin{otherlanguage*}{français}
  Ce travail de recherche vise à recenser les géosymboles utilisés par les communautés gaies, lesbiennes, bisexuelles, transgenres, queers et d'autres identités liées à l'orientation sexuelle ou au genre pour marquer leur territoire.
  Cette démarche s'appuie sur les travaux théoriques de nombreux auteurs en géographie culturelle et tente d'établir un lien avec les avancées de la théorie queer et de la géographie sexuelle.
  À l'aide de données collectées dans les villes de Québec et de Montréal durant l'été 2015, des médias Fugues et Sortie, de documents d'archives et des réseaux sociaux, il a été possible d'établir un portait diversifié des différents territoires de ces communautés, surtout en milieu urbain.
  Il ressort que Montréal possède de nombreux territoires, le Village gai n'étant qu'un de ceux-ci parmi d'autres, certains étant éphémères et d'autres plus permanents selon les fonctions.
  D'autres espaces existeraient également en dehors de la métropole, et si l'on retrouve des traces de ceux-ci dans les médias étudiés, cette recherche ouvre la porte à une plus grande exploration des espaces ruraux et des milieux urbains plus modestes.
\end{otherlanguage*}
