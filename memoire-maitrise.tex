%!TEX program = xelatex
%% GABARIT POUR MÉMOIRE STANDARD
%%
%% Consulter la documentation de la classe ulthese pour une
%% description détaillée de la classe, de ce gabarit et des options
%% disponibles.
%%
%% [Ne pas hésiter à supprimer les commentaires après les avoir lus.]
%%
%% Déclaration de la classe avec le type de grade
%%   [l'un de MSc, LLM, MA, MMus, MServSoc, MScGeogr, MATDR]
%% et les langues les plus courantes. Le français sera la langue par
%% défaut du document.
\documentclass[MScGeogr,nobabel,nonatbib,12pt]{ulthese}
  %% Encodage utilisé pour les caractères accentués dans les fichiers
  %% source du document. Les gabarits sont encodés en UTF-8. Inutile avec
  %% XeLaTeX, qui gère Unicode nativement.
  \ifxetex\else \usepackage[utf8]{inputenc} \fi


  %% Charger ici les autres paquetages nécessaires pour le document.
  %% Quelques exemples; décommenter au besoin.
  %\usepackage{amsmath}          % recommandé pour les mathématiques
  %\usepackage{icomma}           % gestion de la virgule dans les nombres

  %% Utilisation d'une autre police de caractères pour le document.
  %% - Sous LaTeX
  %\usepackage{mathpazo}         % texte et mathématiques en Palatino
  %\usepackage{mathptmx}         % texte et mathématiques en Times
  %% - Sous XeLaTeX
  \setmainfont{TeX Gyre Pagella}      % texte en Pagella (Palatino)
  %\setmathfont{TeX Gyre Pagella Math} % mathématiques en Pagella (Palatino)
  %\setmainfont{TeX Gyre Termes}       % texte en Termes (Times)
  %\setmathfont{TeX Gyre Termes Math}  % mathématiques en Termes (Times)

  %% Gestion des hyperliens dans le document. S'assurer que hyperref
  %% est le dernier paquetage chargé.
  \usepackage{hyperref}
  \hypersetup{colorlinks,allcolors=ULlinkcolor}

  %% Options de mise en forme du mode français de babel. Consulter la
  %% documentation du paquetage babel pour les options disponibles.
  %% Désactiver (effacer ou mettre en commentaire) si l'option
  %% 'nobabel' est spécifiée au chargement de la classe.
  %\frenchbsetup{%
  %  CompactItemize=false,         % ne pas compacter les listes
  %  ThinSpaceInFrenchNumbers=true % espace fine dans les nombres
  %}

  %% Style de la bibliographie.
  %\bibliographystyle{}

% \usepackage{fontspec}
% \usepackage{tgpagella}% Latin Modern typeface (font)
% \setmainfont{TeX Gyre Pagella}
% \setmainfont{EBGaramond12-Regular}
% \usepackage{ebgaramond}
% \setmainfont[Ligatures=TeX]{EB Garamond}
% \setmainfont{texgyrepagella}

\usepackage{xunicode} % Pour l'encodage UTF-8
\usepackage{polyglossia} % Équivalent de Babel
\usepackage[]{csquotes}
%\MakeBlockQuote{<}{|}{>}
\setdefaultlanguage{french} % Set default language for the Polyglossia package
\setotherlanguage{english}

%\usepackage{natbib}
% Start of 'ignore natbib' hack
%\let\bibhang\relax
%\let\citename\relax
%\let\bibfont\relax
%\let\citeauthor\relax
%\let\Citeauthor\relax
%\let\citeyear\relax
%\expandafter\let\csname ver@natbib.sty\endcsname\relax
% End of 'ignore natbib' hack


\usepackage[style=authoryear,sortcites,backend=biber,maxbibnames=99,natbib=true]{biblatex}
\setcounter{biburlnumpenalty}{100}
\addbibresource{~/Dropbox/bibliographie/literature_repository/library.bib} % Name of the file (without the .bib extension) to use as source for bibliography references
%\DeclareFieldFormat{url}{\space\url{#1}}
%\DeclareFieldFormat{urldate}{\addcomma\space\bibstring{urlseen}\space#1}
%\setcounter{biburlnumpenalty}{100} % Permet une brisure dans les DOI
\DeclareSourcemap{
  \maps[datatype=bibtex]{
    \map{
      \pertype{article}
       \step[fieldset=issn, null]
       \step[fieldset=url, null]
    }
    \map{
      \pertype{collection}
       \step[fieldset=url, null]
    }
    \map{
      \pertype{book}
       \step[fieldset=url, null]
       \step[fieldset=doi, null]
    }
    \map{
      \pertype{online}
%       \step[fieldset=url, null]
       \step[fieldset=doi, null]
    }
    \map{
      \step[fieldset=language, null]
    }
  }
}


\DefineBibliographyStrings{french}{
  urlseen = {visité le},
  in = {},
  mathesis = {Mém. de maît.}
}

%\defbibfilter{ressources}{ % Créer une catégorie bibliographique pour "outils" (permet d'exclure)
  %type=manual or
  %type=online
%}

  %% Déclarations de la page titre. Remplacer les éléments entre < >.
  %% Supprimer les caractères < >. Couper un long titre ou un long
  %% sous-titre manuellement avec \\.
  \titre{Au-delà du territoire}
  % \titre{Ceci est un exemple de long titre \\
  %   avec saut de ligne manuel}
  \soustitre{Espaces et géosymboles de la diversité sexuelle}
  % \soustitre{Ceci est un exemple de long sous-titre \\
  %   avec saut de ligne manuel}
  \auteur{Frédéric Vachon}
  \programme{Maîtrise en Sciences géographiques}
  \annee{2016}


%% Ajouts personnels -----------------------------------------
% Siècles 

\def\siecle#1{\textsc{\romannumeral #1}\textsuperscript{e}~siècle}
\def\anglais#1{\textit{#1}}
\def\latin#1{\textit{#1}}
\def\code#1{\texttt{#1}}
%--------

% Macros (surtout des acronymes)
\newcommand\agq{\gls{agq}\xspace}
\newcommand\bdsm{\gls{bdsm}\xspace}
\newcommand\atq{\gls{atq}\xspace}
\newcommand\astteq{\gls{astteq}\xspace}
\newcommand\cadqas{\gls{cadqas}\xspace}
\newcommand\ggul{\gls{ggul}\xspace}
\newcommand\gps{\gls{gps}\xspace}
\newcommand\lgbt{\gls{lgbtq}\xspace}
\newcommand\mtl{Montréal\xspace}
\newcommand\qc{Québec\xspace}
\newcommand\dyke{\emph{dyke}\xspace}
\newcommand\dykes{\emph{dykes}\xspace}
\newcommand\Dyke{\emph{Dyke}\xspace}
\newcommand\Dykes{\emph{Dykes}\xspace}
\newcommand\dm{\emph{Dyke March} de Montréal\xspace}
\newcommand\mai{\gls{mai}\xspace}
\newcommand\qu{\emph{queer}\xspace}
\newcommand\qus{\emph{queers}\xspace}
\newcommand\rlq{\gls{rlq}\xspace}
\newcommand\rqda{\gls{rqda}\xspace}
\newcommand\sida{\gls{sida}\xspace}
\newcommand\sig{\gls{sig}\xspace}
\newcommand\vih{\gls{vih}\xspace}
\newcommand\uqam{\gls{uqam}\xspace}
\newcommand\uqar{\gls{uqar}\xspace}
\renewcommand{\arraystretch}{1.5}

% Note importante, en rouge
\newcommand{\note}[1]{\color{red}(#1!)\color{black}}
% Besoin d'une référence
\newcommand{\missref}{\note{[REF]}}
% À faire
\newcommand{\todo}[1]{\textcolor{blue}{[Todo: #1]}}
\usepackage{xspace}


%--------

% Pour l'interligne
\usepackage{setspace} 
%--------

% Gestion des acronymes
%\usepackage{acronym}
\usepackage{silence}
\WarningFilter{glossaries}{Overriding \printglossary}
\WarningFilter{glossaries}{overriding `theglossary'}
\usepackage[acronym,xindy,toc,smallcaps,nomain,nonumberlist]{glossaries}
\usepackage{enumitem}
\renewcommand{\theglossary}{\begin{description}[font=\normalfont]}
\makeglossaries
%!TEX root = memoire-maitrise.tex
% \newacronym{<label>}{\textsc{<abbrv>}}{<full>}
\newacronym{accm}{\textsc{Accm}}{\emph{AIDS Community Care Montreal}}
\newacronym{agq}{\textsc{Agq}}{Archives gaies du Québec}
\newacronym{alccva}{\textsc{Alccva}}{Centre communautaire pour les modes de vie alternatifs}
\newacronym{ascgcn}{\textsc{Ascgcn}}{Association Socio-Culturelle Gaie de la Capitale Nationale}
\newacronym{astteq}{\textsc{Astt}(e)\textsc{q}}{Action Santé Travesti(e)s \& Transsexuel(le)s du Québec}
\newacronym{atq}{\textsc{Atq}}{Aide aux trans du Québec}
\newacronym{bdsm}{\textsc{Bdsm}}{<< Bondage, discipline, domination, soumission, sadisme et masochisme >>}
\newacronym{cadqas}{\textsc{Cadqas}}{\emph{Computer Assisted Qualitative Data AnalysiS}}
\newacronym{ccglm}{\textsc{Ccglm}}{Centre communautaire des gais et lesbiennes de Montréal}
\newacronym{cocqsida}{\textsc{Cocq-Sida}}{Coalition des organismes communautaires québécois de lutte contre le sida}
\newacronym{cpavih}{\textsc{Cpavih}}{Comité des personnes atteintes du VIH du Québec}
\newacronym{csq}{\textsc{Csq}}{Centrale des syndicats du Québec}
%\newacronym{dj}{\textsc{Dj}}{Centrale des syndicats du Québec}
\newacronym{gps}{\textsc{Gps}}{Global Positioning System}
\newacronym{ggul}{\textsc{Ggul}}{Groupe gai de l'Université Laval}
\newacronym{grip}{\textsc{Grip}}{Groupe de recherche à intérêt public}
\newacronym{gris}{\textsc{Gris}}{Groupe régional d'intervention sociale}
\newacronym{itss}{\textsc{Itss}}{Infections transmissibles sexuellement et par le sang}
\newacronym{irc}{\textsc{Irc}}{\emph{Internet Relay Chat}}
\newacronym{lgbtqia2}{\textsc{Lgbtqia2}}{Lesbien(ne)s, gays, bisexuel-le-s, trans--, queer, intersexes, asexuel-le-s, bispirituel-le-s}
\newacronym{lstw}{\textsc{Lstw}}{\emph{Lez spread the word}}
\newacronym{mai}{\textsc{Mai}}{Musée d'arts interculturels}
\newacronym{mcm}{\textsc{Mcm}}{Monsieur Cuir Montréal}
\newacronym{mbr}{\textsc{Mbr}}{\emph{Montreal Bear Rendez-vous}}
\newacronym{miels}{\textsc{Miels}-Québec}{Mouvement D'informations \& D'entraide Dans La Lutte Contre Le Sida}
\newacronym{npd}{\textsc{Npd}}{Nouveau Parti Démocratique du Canada}
\newacronym{pflag}{\textsc{Pflag}}{\emph{Parents and Friends of Lesbians and Gays of Canada}}
\newacronym{rlq}{\textsc{Rlq/Qln}}{Réseau des lesbiennes du Québec}
\newacronym{rqda}{\textsc{Rqda}}{\emph{R's Qualitative Data Analysis}}
\newacronym{sida}{Sida}{Syndrome d'immuno-déficience humaine}
\newacronym{sig}{\textsc{Sig}}{Système d'informations géographiques}
\newacronym{vih}{\textsc{Vih}/Sida}{Virus d'immuno-déficience humaine / Syndrome d'immuno-déficience humaine}
\newacronym{uqam}{\textsc{Uqam}}{Université du Québec à Montréal}
\newacronym{uqar}{\textsc{Uqar}}{Université du Québec à Rimouski}
\newacronym{om}{OM}{Outremont}
\newacronym{ls}{LS}{LaSalle}
\newacronym{mr}{MR}{Mont-Royal}
\newacronym{vm}{VM}{Ville-Marie}
\newacronym{pm}{PM}{Le Plateau-Mont-Royal}
\newacronym{hs}{HS}{Hampstead}
\newacronym{so}{SO}{Le Sud-Ouest}
\newacronym{rp}{RP}{Rivière-des-Prairies-Pointe-aux-Trembles}
\newacronym{lc}{LC}{Lachine}
\newacronym{dv}{DV}{Dorval}
\newacronym{mn}{MN}{Montréal-Nord}
\newacronym{is}{IS}{L'Île-Bizard-Sainte-Geneviève}
\newacronym{kl}{KL}{Kirkland}
\newacronym{do}{DO}{Dollard-des-Ormeaux}
\newacronym{sv}{SV}{Senneville}
\newacronym{ac}{AC}{Ahuntsic-Cartierville}
\newacronym{cl}{CL}{Côte-Saint-Luc}
\newacronym{ln}{LN}{Saint-Léonard}
\newacronym{mo}{MO}{Montréal-Ouest}
\newacronym{pc}{PC}{Pointe-Claire}
\newacronym{id}{ID}{L'Île-Dorval}
\newacronym{mh}{MH}{Mercier-Hochelaga-Maisonneuve}
\newacronym{cn}{CN}{Côte-des-Neiges-Notre-Dame-de-Grâce}
\newacronym{ro}{RO}{Rosemont-La Petite-Patrie}
\newacronym{lr}{LR}{Saint-Laurent}
\newacronym{bf}{BF}{Beaconsfield}
\newacronym{vs}{VS}{Villeray-Saint-Michel-Parc-Extension}
\newacronym{wm}{WM}{Westmount}
\newacronym{me}{ME}{Montréal-Est}
\newacronym{aj}{AJ}{Anjou}
\newacronym{pr}{PR}{Pierrefonds-Roxboro}
\newacronym{bv}{BV}{Sainte-Anne-de-Bellevue}
\newacronym{vd}{VD}{Verdun}
\newacronym{bu}{BU}{Baie-d'Urfé}

%--------

% Définir un dossier par défaut pour les images
\graphicspath{{images/}}
%-----------------------------------------


%\usepackage{float}
%\usepackage[caption = false]{subfig}
\usepackage{subcaption}

% Cases à cocher
\usepackage{wasysym}

\usepackage{longtable}




\begin{document}


\frontmatter                    % pages liminaires

\pagetitre                      % production de la page titre

\chapter*{Résumé}                      % ne pas numéroter
\phantomsection\addcontentsline{toc}{chapter}{Résumé} % inclure dans TdM

\begin{otherlanguage*}{français}
  Ce travail de recherche vise à recenser les géosymboles utilisés par les communautés gaies, lesbiennes, bisexuelles, transgenres, queers et d'autres identités liées à l'orientation sexuelle ou au genre pour marquer leur territoire.
  Cette démarche s'appuie sur les travaux théoriques de nombreux auteurs en géographie culturelle et tente d'établir un lien avec les avancées de la théorie queer et de la géographie sexuelle.
  À l'aide de données collectées dans les villes de Québec et de Montréal durant l'été 2015, des médias Fugues et Sortie, de documents d'archives et des réseaux sociaux, il a été possible d'établir un portait diversifié des différents territoires de ces communautés, surtout en milieu urbain.
  Il ressort que Montréal possède de nombreux territoires, le Village gai n'étant qu'un de ceux-ci parmi d'autres, certains étant éphémères et d'autres plus permanents selon les fonctions.
  D'autres espaces existeraient également en dehors de la métropole, et si l'on retrouve des traces de ceux-ci dans les médias étudiés, cette recherche ouvre la porte à une plus grande exploration des espaces ruraux et des milieux urbains plus modestes.
\end{otherlanguage*}
                % résumé français
\chapter*{Abstract}                      % ne pas numéroter
\phantomsection\addcontentsline{toc}{chapter}{Abstract} % inclure dans TdM

\begin{otherlanguage*}{english}
This research aims to identify the geosymbols used by the gay, lesbian, bisexual, transgender and queer communities as well as those of other identities linked to sexual orientation or gender marking their territories.
  This approach is based on the theoretical works of many authors in cultural geography and try to establish the link with the contributions of queer theory and sexual geography.
  Using data collected in fieldwork in Quebec City and Montreal during the summer of 2015, the \emph{Fugues} and \emph{Sortie} print media, archival data and social networks, we have been able to build a diversified portrayal of the different territories of these communities, especially in urban settings.
  We discovered that Montreal possesses a plurality of these spaces, the gay village being one of many, certain being temporary and other permanent, depending on the need to wish they answer.
  Other spaces also exist outside of the metropolis, and if we find proofs of their existence in the studied media, this research opens the door to a bigger exploration of rural territories and small towns.
\end{otherlanguage*}
              % résumé anglais
\cleardoublepage

\tableofcontents                % production de la TdM
\cleardoublepage

\listoftables                   % production de la liste des tableaux
\cleardoublepage

\listoffigures                  % production de la liste des figures
\cleardoublepage

\printglossary[type=\acronymtype,title=Liste des acronymes,toctitle=Liste des acronymes] % production de la liste des acronymes
\cleardoublepage

\dedicace{Dédicace si désiré}
\cleardoublepage

\epigraphe{Texte de l'épigraphe}{Source ou auteur}
\cleardoublepage

\chapter*{Remerciements}         % ne pas numéroter
\phantomsection\addcontentsline{toc}{chapter}{Remerciements} % inclure dans TdM
J'aimerais d'abord débuter mes remerciement en 
% Caroline Desbiens
% Marie-Hélène Vandersmissen
% Julie Podmore
% Olivier
% Charles
% Joëlle
% Les bénévoles de la fête arc-en-ciel
% Laurence Simard-Gagnon
% Benoît Lalonde
         % remerciements
%!TEX root = memoire-maitrise.tex
\chapter*{Avant-propos}         % ne pas numéroter
\phantomsection\addcontentsline{toc}{chapter}{Avant-propos} % inclure dans TdM

\section{Problématique}
\label{sec:problematique}
\todo{Section entière à retravailler}


\subsection{Énoncé du problème}
\label{sub:enonce_du_probleme}
\todo{à refaire, étant donné que celle du plan de recherche se retrouve dans
  tout ce chapitre. Peut-être faire un résumé? Prendre les objectifs de
  l'introduction et les rapatrier ici?}

Ce mémoire de maîtrise s'inscrit dans la suite d'un cheminement qui m'a amené à me pencher scientifiquement sur un ensemble plus large, un groupe d'individu auquel je considère moi-même appartenir, la communauté \lgbt{}, souvent appellée gaie et lesbienne, à laquelle j'ajoute les individus queers et les asexuels/alliés.
Nous voulons ici présenter les différentes façons qu'ont ces communautés d'utiliser l'espace au Québec et comment elles marquent les territoires qu'elles investissent pour montrer leur présence et surtout, pour construire un espace à leur image.
Ainsi, nous essayerons également de nous intéresser aux raisons pour lesquelles ces orientations sexuelles et ces identités de genre ont modelées les espaces qu'elles fréquentent et pourquoi on ne peut parler d'un simple groupe homogène comme l'acronyme \lgbt{} ou ses nombreuses variantes peuvent le laisser penser.
Plutôt, comme nous le verrons, si les façons de marquer l'espaces sont particulièrement variées, les groupes qui sont représentés le sont tout autant et ces derniers ne cohabitent pas nécessairement.

Ce mémoire fait suite à un essai qui a été produit afin de recenser l'ensemble des lieux \lgbt{} dans la ville de Québec.
Nous voulions d'abord débuter notre travail de recherche en nous intéressant à une ville importante mais ne présentant pas les caractéristiques des métropoles souvent abordées dans les études \lgbt{}.

\section*{Objectifs et hypothèse}
Ce mémoire vise rendre compte de notre recherche visant à dresser un portrait des géosymboles des espaces \emph{queers} en milieux urbains, plus particulièrement ceux des villes de Montréal et de Québec, mais aussi des villes de plus petites envergure.
Pour arriver à atteindre ce but général, nous nous sommes également fixé des objectifs plus spécifiques.
D'abord, cette recherche a permis de développer une méthodologie permettant l'identification et le catalogage des géosymboles marquants les espaces \emph{queers}, en tenant compte du caractère éphémère ou permanent, matériel ou immatériel de ceux-ci.
Ensuite, nous avons pris à tâche de rechercher et répertorier les géosymboles \emph{queers} en tenant en compte de l'environnement et de la forme de ceux-ci dans une variété relative de milieux urbains (métropolitains et non métropolitains) pour par après les géolocaliser.
Nous avons aussi voulu dresser les différences et ressemblances entre les différentes expressions spatiales et symboliques de la diversité sexuelle, notamment selon le genre et l'orientation sexuelle.
Enfin, nous avons cherché à faire progresser la connaissance portant sur les espaces \lgbt{} urbains de la province du Québec en prenant en premier comme terrain d'étude les villes de Québec et de Montréal puis les villes de taille inférieures selon la présence d'une communauté \lgbt{}.
La figure~\ref{fig:arrondissementsmtl} montre l'emplacement de ces deux arrondissements et nous verrons dans les chapitres suivants où ces espaces ont été situés en complément des données accumulées dans cette recherche \todo{approfondir le contexte historique?}

\begin{figure}[ht]
 \centering
 \includegraphics[width=1\textwidth]{arrondissementsmtl}
 \caption[Arrondissements ciblés: ville de Montréal]{Arrondissements ciblés pour la collecte de données: ville de Montréal}\label{fig:arrondissementsmtl}
\end{figure}


\subsection*{Question de recherche}
\label{sub:hypothese}
\todo{À simplifier}
%L'ensemble des objectifs précédents ont servi à répondre à la question de recherche que nous exposerons dans les prochaines lignes.
Pour répondre à ces différents objectifs, nous avons articulé la question de recherche suivante: comment les communautés \lgbt{} articulent-elles une relation particulière avec l'espace, et comment celles-ci s'affichent-elles dans celui-ci?
Nous supposons que les groupes et individus \lgbt{} occupent l'espace d'une façon particulière et que cette manière d'occuper l'espace sert à se regrouper autour d'une identité partagée et de l'afficher.
Cette occupation s'exprime par une variété de géosymboles signifiant la différence ou l'inclusion des individus selon cette identité.
L'existence de ces espaces, et des géosymboles les identifiant est permanente ou temporaire, selon les modalités particulières de l'espace.
Ces géosymboles marquants les espaces des communautés \lgbt{} sont récents; leur apparition serait survenue à partir de la \emph{révolution sexuelle}, une période durant laquelle les normes sexuelles se seraient en apparence assouplies.
Plus précisément, pour les communautés \lgbt{}, ce changement débuterait à partir des émeutes de Stonewall aux États-Unis et se serait répandu ailleurs en occident et donc au Québec.
%AJOUT-----------------------------
Ces géosymboles se seraient propagés d'abord dans les milieux urbains selon des paramètres propres à l'histoire des géosymboles, du milieu urbain, de la taille de ce milieu et de sa place dans la hiérarchie urbaine.

Pour conclure, le prochain chapitre servira à confirmer ces choix d'objectifs de recherche par un approfondissement de la problématique de recherche.
           % avant-propos

\mainmatter                     % corps du document


%!TEX root = memoire-maitrise.tex
\chapter*{Introduction}         % ne pas numéroter
\phantomsection\addcontentsline{toc}{chapter}{Introduction} % inclure dans TdM

%\chapterprecishere{<< Everyone needs a place. It shouldn't be inside of someone else.” \par\raggedleft--- \textup{Richard Siken}, Crush}
Avant d'entâmer mes études universitaires à Québec, j'ai vécu loin des grandes villes; d'abord dans la région de Bellechasse, puis durant une bonne partie de mon enfance et de mon adolescence dans la ville de Rimouski, dans l'est du Québec.
Rimouski n'est pas la ville régionale typique organisée autour d'une industrie particulière; au contraire, il s'agit d'une ville réputée pour ses services et sa proximité avec le fleuve, possédant de nombreuses institutions gouvernementales et d'éducation.
Sans vouloir porter de jugement vis-à-vis ce type de ville, les valeurs étaient plus traditionnelles que dans une ville comme Montréal, mais restait tout de même assez progressives comparativement à l'image que l'on peut se faire des villages et villes industrielles.
Par contre, pour ceux qui débordaient un tant soit peu de la norme, les espaces où se retrouver étaient peu nombreux, à moins que cette déviance soit fondée sur l'attachement à un style musical, comme la culture punk ou metal.
Toutes deux étaient d'ailleurs bien représentées durant mon adolescence; j'y ait d'ailleurs participé.
Par contre, si la déviance était basée sur l'orientation sexuelle\ldots

Adolescent, parler de diversité sexuelle, ça se faisait dans les cours d'éducation sexuelle, quelques fois par années, par un ou une enseignante plus ou moins intéressée.
Même s'il s'agissait d'un début, c'était tout de même mieux à certains égars qu'aujourd'hui; ces cours ont été annulés pour être remplacés par des interludes par-ci par-là dans les cours considérés comme plus sérieux, comme ceux de français, de biologie, etc.
Lorsque venait la puberté et qu'on commençait à se poser des questions sur ses intérêts sexuels, sur nos amours, il y avait peu de place vers qui se tourner.
Il restait l'Internet, avec ses premiers sites de rencontres rudimentaires (mais graphiques) pour hommes de 18 ans et plus et les canaux \irc{} (beaucoup plus austères) où il était possible de publier sa petite annonce, en espérant trouver l'amour ou, plus souvent qu'autrement, une baise d'un soir.
Une fois la décision prise de se rencontrer, plusieurs options s'offraient à nous; on allait chez l'autre s'il avait un appartement, on se retrouvait dans le Tim Hortons du centre-ville pour faire connaissance, ou l'on se retrouvait en-dehors de la ville, en forêt sur le bord de la rivière Rimouski où d'autres se rencontraient aussi, chacun dans sa voiture, les fenêtres embuées.
J'ai appris tardivement qu'il s'organisait des soirées dans une taverne près du cégep, puis dans une autre, après que la précédente ferma ses portes pour être réouvert par le même propriétaire sur une autre rue.
À moins de l'apprendre par Internet, rien n'indiquait clairement que durant cette soirée-là, il était possible de trouver des gens pas tout à fait straights.
Il n'y avait pas vraiment de femmes, lesbiennes ou non, ou d'individus s'identifiant socialement comme trans.
On y voyait par contre tel chauffeur de taxi, ou tel ami du secondaire, et on rencontrait un nombre non-néglieable de nouvelles personnes qu'on ne croyait pas avoir jamais croisée, malgré la taille modeste de la ville.
Enfin, j'ai aussi entendu parler de la discothèque retro du centre-ville où plusieurs personnes non-straights allait danser sans que lieu soit considéré comme le bar gay de ville.
Il faut dire aussi que c'est là que les mineurs allaient essayer d'entrer avant d'avoir atteint la majorité et où certaines personnes plus âgées, femmes ou hommes, s'intéressaient aux jeunes qui y dansaient.

Tous ces lieux, je les ai connus ou fréquentés il y a plusieurs années, en les découvrant par le bouche-à-oreille (numérique comme matériel).
Peu de choses les identifiaient, et le reste de la population rimouskoise pouvaient passer à proximité de ces espaces, le jour comme la nuit, sans nécessairement remarquer qu'il s'agissait d'espaces queers.
Mais quand je suis arrivé à Québec, quand j'ai été à Montréal dans le Village gai, ou plus récemment, quand j'ai été à New-York ou à Philadelphie dans les centres-villes, j'ai pu voir des drapeaux multicolores, ceux qu'on voyaient parfois à la télévision et un peu partout sur Internet en allant sur les bons sites.
Le symbole de l'arc-en-ciel, bien que représentant l'ensemble de la communauté \lgbt{}, m'a toujours apparu représentant la communauté gaie masculine, et parfois aussi les femmes lesbiennes.
Bien que sachant l'existence des personnes trans, intersexes et bisexuelles, celles-ci ne semblaient pas présentes où je posais mon regard.
S'agissait-il d'un phénomène de rareté, ou étais-je tout simplement pas apte à regarder aux endroits, à m'intéresser aux bonnes personnes?
Ou peut-être que je considérais mon prochain comme étant hétérosexuel ou gay, sans autres issues possibles?

L'idée d'écrire ce mémoire m'est venue durant mes études au baccalauréat en géographie.
Domaine réputé s'intéresser à l'espace, souvent selon l'un ou l'autre des axes physiques ou humains, peu laissait envisager qu'il était possible de s'intéresser à des groupes, à des personnes plus marginalisées.
J'ai eu la chance de rencontrer certaines personnes, certains groupes, qui m'ont permi de constater que cette possibilité existait bel et bien.
Il m'est donc apparu possible d'utiliser cette discipline pour répondre à ces interrogations que j'ai cultivé durant bien des années.

Je m'inscris, dans cette recherche, dans le domaine de la géographie culturelle, en mettant de l'avant les travaux déjà effectués auprès des communautés \lgbt{}.
Les intérêts associés à cette recherche sont multiples.
Sans prétendre innover au niveau conceptuel, cette étude pourrait permettre d'abord d'associer un champ de la géographie culturelle, l'étude des géosymboles, à la géographie \emph{queer} ou sexuelle, de développer une méthodologie pour l'étude des géosymboles à la jonction de ces deux champs, et enfin, de mettre à l'étude des espaces, les villes Québecoises, peu traitées encore en études gaies et lesbiennes sauf exceptions \parencite{Chamberland1993a,Podmore2006,Podmore2001,Hebert2012,Hunt2008,Laprade2014}.

Il sera possible de développer une méthodologie pour la reconnaissance, la description et l'usage des géosymboles des espaces \emph{queers} en milieu urbain.
Il apparait important de dépasser le concept de territoire au sein de la
géographie: si les géosymboles demeurent pertinents comme il sera démontré dans la partie~\ref{sec:problematique}, il convient de complexifier l'usage de ceux-ci en géographie en prenant en compte des éléments immatériels ou temporaires dans l'espace, sachant que certains groupes culturels n'ont pas d'assises territoriales stabilisées et tangibles, mais font plutôt usage de l'espace de manière ponctuelle ou subversive \parencite{Talburt2012}.

Le domaine de la géographie sexuelle n'est pas un domaine d'étude courant dans
la recherche francophone en géographie: la majeure partie des travaux sont effectués en Anglais dans des villes Américaines ou en Europe, en Belgique ou en France \parencite{Blidon2010,Blidon2006,Cattan2010,Deligne2006}, quoiqu'on retrouve de plus en plus de travaux de recherches dans d'autre pays développés ou en voie de développement, en Chine ou à Singapour par exemple \parencite{Oswin2014a,Kong2012}.
Cela pose un problème sachant que des différences culturelles majeures existent entre États et régions notamment. 
Il apparaît donc nécessaire d'enrichir le domaine de la géographie sexuelle au Québec, un territoire peu étudié en français notamment en-dehors de la ville de Montréal.


\section*{Objectifs et hypothèse}
Ce mémoire vise rendre compte de notre recherche visant à dresser un portrait des géosymboles des espaces \emph{queers} en milieux urbains, plus particulièrement ceux des villes de Montréal et de Québec, mais aussi des villes de plus petite envergure.
Pour arriver à atteindre cet but général, nous nous avons également fixé des objectifs plus spécifiques.
D'abord, cette recherche a permis de développer une méthodologie pour identifier et répertorier les géosymboles marquant les espaces \emph{queers} en tenant compte du caractère éphémère ou permanent, matériel ou immatériel de ceux-ci. 
Ensuite, nous avons pris à tâche de rechercher et répertorier les géosymboles \emph{queers} en tenant en compte de l'environnement et de la forme de ceux-ci dans une variété relative de milieux urbains (métropolitains et non-métropolitaints) pour ensuite les géolocaliser.
Nous avons aussi voulu dresser les différences et ressemblances entre les différentes expressions spatiales et symboliques de la diversité sexuelle, notamment selon le genre et l'orientation sexuelle.
Enfin, nous avons cherché à faire progresser la connaissance portant sur les espaces \lgbt{} urbains de la province du Québec en prenant d'abord comme terrain d'étude les villes de Québec et de Montréal puis les villes de taille inférieures selon la présence d'une communauté \lgbt{}.
% subsection objectifs_specifiques (end)

\subsection*{Question de recherche} % (fold)
\label{sub:hypothese}
\todo{À simplifier}
L'ensemble des objectifs précédents ont servi à répondre à la question de recherche que nous exposerons dans les prochaines lignes.
Nous supposons que les groupes et individus \lgbt{} occupent l'espace d'une façon particulière selon leur propre identité.
Cette occupation s'exprime par une variété de géosymboles marquant la différence ou l'inclusion des individus selon, encore une fois, cette identité.
L'existence de ces espaces, et des géosymboles les marquant, est permanente ou temporaire, selon les modalités particulières de l'espace.
Ces géosymboles marquants les espaces des communautés \lgbt sont récents; leur apparition serait survenue à partir de la \emph{révolution sexuelle}, une période durant laquelle les normes sexuelles se seraient en apparence assouplies.
Plus précisément, pour les communautés \lgbt{}, ce changement débuterait à partir des émeutes de Stonewall aux États-Unis et se serait répandu ailleurs en occident, dont au Québec.
%AJOUT-----------------------------
Ces géosymboles se seraient répandus d'abord dans les milieux urbains selon des paramètres propres à l'histoire des géosymboles, du milieu urbain, de la taille de ce milieu et de sa place dans la hiérarchie urbaine.

Pour conclure, le prochain chapitre servira à la confirmation de ces choix d'objectifs de recherche par un approfondissement de la problématique de recherche.

%%% Local Variables:
%%% mode: latex
%%% TeX-master: "memoire-maitrise"
%%% End:
          % introduction
%!TEX root = ../../memoire-maitrise.tex
\chapter{Éléments conceptuels et problématique}
\label{cha:elements_conceptuels_et_problematique}

\chapterprecishere{\textquote{J’ai abandonné depuis longtemps l’idée qu’une vérité immanente se trouve dans la sexualité, qu’elle soit marquée par le péché ou l’émancipation. J’ai aussi abandonné l’idée qu’il existe quelque chose qu’on appelle \enquote{la sexualité}. Il existe plutôt des sexualités multiples, des sexualités dominantes et des sexualités marginalisées.} \par\raggedleft--- \textup{Anne Archet}, Sexe et liberté}

Ce chapitre servira à approfondir d'une part les concepts que nous avons utilisés dans cette recherche et à effectuer un tour d'horizon de l'état de la littérature dans le domaine de la géographie sexuelle pour développer la problématique.

%\section{Revue de la littérature et principaux concepts}
%\label{sec:revue_de_la_litterature_et_principaux_concepts}

Ce travail de recherche s’appuiera sur différents domaines de la littérature scientifique: d'abord, sur les études queers et plus précisément la géographie queer et sur la sémiotique, en mettant l'accent encore une fois sur les liens avec la géographie.

Plus particulièrement, les études queers formeraient un champ pluridisciplinaire des sciences sociales.
Trouvant son origine dans le corpus de la \anglais{French Theory}, la théorie queer est développée par un ensemble d'auteurs attachés au post-structuralisme dans des disciplines aussi diverses que la littérature, la philosophie ou les sciences sociales.
Le terme queer, d'abord utilisé par les mouvements sociaux \lgbt{} lors de la crise du \vih{}, est devenu une réappropriation d'une insulte dirigée vers les homosexuels~\citep{Laprade2014}.
La théorie queer aborde les normes  comme objet plutôt que sur des identités sexuelles spécifiques; le concept de négativité, de performativité et l'intersectionnalité sont particulièrement utilisés pour traiter des phénomènes de marginalisation.

Étant donné l'étendue de la recherche, nous avons puisé les textes théoriques qui ont servi à formulation de notre problématique dans de nombreuses disciplines.
Ce choix s'explique par le fait que certains de ces concepts ont été utilisés autant en géographie qu'en anthropologie et en sociologie, principalement ceux tournant autour de l'identité et de la symbolique.
Afin de garder l’accent sur notre sujet de recherche, sois la géosymbolique des espaces et territoires des minorités sexuelles, nous tenterons de contextualiser tout au long de notre texte les liens qu'ont ces concepts avec le sujet.
En même temps, nous soulignerons les limites épistémologiques auxquelles nous nous frotterons.


\section{Le regard anthropologique sur la culture}
\label{subsec:le_regard_anthropologique_sur_la_culture}
Nous aborderons la culture comme premier concept.
Polysémique selon le contexte et l'usage, nous voulons décrire ce concept et ses différents usages pour arriver à comprendre comment celui-ci s'articule dans le domaine de la géographie sexuelle.
Plus encore, nous voulons également l'approfondir dans un contexte de regard sur soi au sein des minorités sexuelles.
En effet, nous le verrons plus loin, plusieurs positions sont débattues dans la communauté \lgbt{}, une s'incarnant dans une identité forte et une autre dans une position négative et anti-normative.

% Dans le cadre de ce travail de recherche, nous nous inscrivons dans une
% définition sémiotique de la culture inspirée des travaux structuralistes et
% post-structuralistes en anthropologie. Pour d'abord arriver à bien comprendre
% cette définition de la culture, nous allons nous pencher plus particulièrement
% sur le texte \textquote{La religion comme système culturel} de \citet{Geertz1972}.
% Malgré que le sujet principal du texte de Geertz est la religion et consiste en
% son analyse, nous pouvons y voir ici une description avancée d'un système
% culturel particulier dans lequel nous retrouvons les bases pouvant servir à la
% description d'autres systèmes culturels, comme celui de la communauté \lgbt{}.

Pour débuter notre analyse de la culture, nous nous pencherons sur la définition proposée par Clifford Geertz.
Pour lui, tel qu'amené dans~\citetitle{Geertz1972}, : \blockquote[{\cite[21]{Geertz1972}}][.]{\textelp{} [elle] désigne un modèle de significations incarnées dans des symboles qui sont transmis à travers l'histoire, un système de conceptions héritées qui s'expriment symboliquement, et au moyen desquelles les hommes [\latin{sic}] communiquent, perpétuent et développent leur connaissance de la vie et leurs attitudes devant elle}.
Geertz s'inscrit cette définition dans une critique plus large des travaux précédents en anthropologie religieuse.
Cette sous-discipline de l'anthropologie stagnerait au point de vue théorique en axant constamment son analyse sur une version trop fonctionnelle de la culture ou axé sur l'expérience (mystique dans le cas de la religion).
Ce choix d'auteur prendrait autant une inspiration sociologique qu'anthropologique (en se basant également sur les travaux de Durkheim et sur ceux de Malinowski par exemple~\citep[20]{Geertz1972}).
Ce texte vise donc à proposer de nouvelles bases théoriques sur lesquelles l'analyse anthropologique pourrait s'approfondir et continuer à évoluer.
Geertz précède alors son analyse des systèmes religieux par une définition renouvelée de ce en quoi consiste la culture ou plutôt le système culturel.
Pour lui, ces systèmes religieux offrent un sens au quotidien des individus et ne sont pas nécessairement subordonnés aux fonctions sociales; ils interagissent avec elles.
Si certaines perspectives anthropologiques ont voulu voir la religion comme une réponse à l'angoisse de la mort, Geertz nous rappelle que celle-ci offre toute une variété de réponses~\citep[][37]{Geertz1972}.
En fait, c'est dans le chaos du quotidien que la religion offre des réponses, réponses qui sont rapidement remplacées par des explications plus plausibles, ou près du sens commun, lorsque ces dernières sont proposées~\citep[][39]{Geertz1972}.

Par cette lecture, la culture apparait bel et bien comme un système abstrait, persistant dans le temps au-delà de la vie des individus composant la société.
Par contre, la définition avancée par Geertz ne laisse pas sous-entendre que nous avons affaire à une entité réifiée ou superorganique, pour reprendre les termes de \citet{Duncan1980}.
Les individus formant la société héritent donc des connaissances offertes par la culture pour arriver à comprendre le monde où les symboles qui s'y trouvent portent des significations propres.
Dans les sociétés analysées par l'anthropologue, c'était essentiellement la religion qui jouait ce rôle dans le cadre plus large de la culture.
Nous croyons par contre qu'un tel système religieux, avec ses différents symboles pour décrire le réel et lui donner un sens, peut aussi être repris dans des contextes culturels variés n'ayant pas nécessairement de liens avec la religion.
La largeur et la flexibilité donnée au concept de symbole, la base des systèmes religieux et de la culture chez Geertz, nous apparait pertinent pour l'analyse des populations \lgbt{}.

En reprenant les termes de~\citet{Langer1962}, les symboles sont pour Geertz dans son analyse sémiotique de la culture: \textquote{\textelp{} tout objet, acte, événement, propriété ou relation qui sert de véhicule à un concept --- le concept est la \textquote{signification du symbole }~\citep[23--24]{Geertz1972}}.
Il est important de souligner que les objets en soi que l'on pourrait assimiler à des symboles demeurent ce qu'ils sont matériellement; Geertz prend l'exemple d'une maison qui, si celle-ci peut consister en un objet concret sans significations autres que sa matérialité, peut également jouer le rôle d'un symbole propre.
Ce caractère varierait selon le regard qu'on lui pose en tant qu'être humain appartenant à une culture spécifique.
Autrement dit, au-delà de sa matérialité, sa forme, sa position ou sa composition, la maison peut porter le témoignage d'un fait culturel particulier.
On pourrait extrapoler en considérant que cette maison informe le public sur le statut social de la personne.
Elle hériterait donc dans sa forme d'une forme architecturale propre à la culture dans laquelle elle s'inscrit.

Cette manière de donner forme aux choses matérielles ou abstraites, de leur octroyer une signification sous la forme de symbole est selon Geertz le fait des programmes fournis par les modèles culturels~\citep[25]{Geertz1972}.
Ces modèles fonctionnent en deux temps: d'abord, ils créent les symboles en se basant sur le réel, en prenant assise sur les structures non symboliques déjà existantes, ce que Geertz nomme des \emph{modèles de}.
L'autre forme de modèle, les \emph{modèles pour} agissent plutôt en orientant les structures non symboliques et en créant des liens entre elles qui n'existent pas nécessairement au préalable~\citep[26--27]{Geertz1972}.
Ces deux phénomènes rappellent en fait les deux manières qu'ont les \emph{modèles de} de donner sens aux symboles qu'ils contiennent; ils dirigent la compréhension des symboles en calquant ceux-ci sur le réel non-symbolique et en liant ensemble les éléments composant ce réel.

Les \emph{modèles de} et \emph{modèles pour} sont en fait un seul et même modèle culturel où s'articulent les symboles; en effet, un même groupe culturel articule une lecture particulière des structures non symboliques et lui en impose une selon les deux facettes du modèle culturel.
Par contre, ceux-ci comme décrit par Geertz s'inscrivent dans un contexte culturel réputé homogène.
En effet, les différentes analyses faites par l'auteur portent sur les religions \todo{expliquer d'avantage} (l'objet d'analyse) de façon singulière (en traitant de la population balinaise en Indonésie) ou encore sur la place de la religion dans la société de façon générale (le protestantisme en occident), sans privilégier un contexte en particulier.
Il s'agit d'une des limites à prendre en compte dans la suite du présent texte; en effet, Geertz n'entreprend pas dans cette recherche de traiter des effets propres aux mélanges culturels.
Nous entendons par ceci par exemple des interactions naissantes de l'immigration avec la société d'accueil ou encore de la genèse de nouveaux phénomènes culturels, comme dans le cas qui nous intéresse les minorités \lgbt{} au sein d'ensembles culturels plus larges.

On trouve tout de même dans son texte des éléments orientant la relation particulière d'un individu et de la religion qui peut servir d'introduction à la suite de ce travail sur l'identité.
En effet, Geertz, plus loin dans son texte, traite des dispositions propres à l'individu s'insérant dans un contexte culturel particulier.
La culture transmet ces dispositions à effectuer certaines activités qui permettent à l'individu de s'identifier à la culture dont il fait partie au-delà des fonctions premières et des motivations derrière la pratique en question.
Le principe d'identité se comprend aisément dans le contexte religieux décrit dans le texte: l'individu religieux pratique dans sa vie la prière et d'autres activités religieuses plus fréquemment selon l'intensité de son sentiment religieux.
Nous voulons surtout souligner ici le fait qu'on parle principalement de la probabilité d'un acte de survenir: l'acte en soi n'est pas nécessaire.
En société, par exemple, on s'attend à ce que l'individu réputé religieux agisse selon certaines dispositions propres au mode de vie religieux~\citep[28--30]{Geertz1972}.
Dans le contexte des identités \lgbt{}, on ne peut s'arrêter seulement aux pratiques pour traiter d'une potentielle identité.

Plutôt, en subissant un ensemble de structure symbolique où l'hétérosexualité est la norme, les individus \lgbt{} pourraient avoir à créer de nouvelles structures symboliques correspondant à leur situation.
Plus précisément, les individus \lgbt{}, à leur naissance, s'insèrent de facto dans un contexte culturel dont les symboles s'apparentent très peu aux identités sexuelles non hétérosexuelles.
En effet, dans le contexte occidental, les individus sont considérés comme hétérosexuels par défaut, à un point tel qu'il ne s'agit pas, en général, d'une identité particulière.
Celle-ci est plutôt une norme souvent surpassée par des identités nationales, régionales, \emph{ethniques}, etc. ou par des conceptions plus individuelles de l'identité, comme les loisirs, la carrière, etc.
La non-correspondance de l'envie sexuelle envers des individus qui ne sont pas considérés comme des partenaires éligibles créent une forme de chaos auquel le modèle culturel dominant offre des réponses négatives par l'homophobie, la violence, etc.
Même chose en ce qui concerne l'identité de genre pour les personnes trans\footnote{Nous utiliserons dans ce mémoire le terme trans pour traiter l'ensemble des identités affiliées au terme trans, soit les personnes non binaires dans le genre, les personnes transgenres, les personnes transsexuelles et possiblement d'autres identités liées à une fluidité dans le genre. Cette décision vise à ne pas exclure d'individus sur une base étymologique et rendre compte de la diversité des expériences~\citep[193]{Nash2011}. Nous reprenons également la pratique courante dans plusieurs travaux académiques qui utilisent le terme trans par défaut~\citep[par exemple, voir][]{Pfeffer2014, Rosenberg2014}} ou intersexes; le genre dans les modèles culturels occidentaux est vu comme empreint d'une essence propre par l'existence d'un sexe, masculin ou féminin.
Si les identités \lgbt{} existent malgré tout aujourd'hui, c'est que l'on peut croire que face à une réalité qui refusait leur existence, le processus de regroupement d'individus semblables et leur apparition dans l'espace public se sont fait par la mise en place de nouvelles structures symboliques.
Nous proposons que ces structures se soient formées à partir des symboles émergents du militantisme et des lieux fréquentés, selon la forme du \emph{modèle pour}.
Ces dernières ont permis, éventuellement, l'émergence d'identités de plus en plus homogènes et reconnaissables.
Nous allons maintenant nous pencher sur l'identité comme concept et comment celui-ci a été manié et pensé par la théorie queer, pour ensuite revenir sur la place de l'identité chez les groupes \lgbt{}.

\section{Le sujet chez Stuart Hall et la formation du concept d'identité}
\label{subsec:sujet_et_identité} Maintenant que nous avons une définition claire de la culture, comme nous l'avons souligné précédemment, nous devons maintenant nous pencher sur le concept d'identité qui semble avoir pris en importance durant les dernières décennies.
En effet, en occident et ailleurs dans le monde, de nombreux groupes identitaires semblent avoir fait surface, surtout dans les milieux urbains.
On peut penser notamment aux groupes ethniques nés de l'immigration, aux groupes d'intérêts envers des objets culturels particuliers (genres musicaux, dessins animés, cinéma), etc.
Plus près de notre recherche, nous pouvons ajouter également les groupes formés par l'identification à des pratiques sexuelles différentes ou une non-coïncidence du genre de la personne avec celui fixé à la naissance, les individus se retrouvant dans le spectre \lgbt{}.
Dans ce mémoire, nous traiterons donc d'abord des concepts de culture et d'identité relatifs aux individus appartenant au spectre \lgbt{}, 
Par contre, comme nous le verrons, certains contextes derrière l'émergence de ces identités sont en tout point similaires à d'autres, comme les objets culturels ou encore l'ethnicité.
Également, nous souhaitons aborder les enjeux particuliers que pose l'étude géographique de cette partie de la population, en considérant le contexte historique récent; nous nous attarderons à cette tâche plus loin dans ce chapitre, après nous être penché sur le concept d'identité.

%Ce cadre s'inscrirait dans les rassemblements politiques et l'organisation communautaire auxquels plusieurs des individus du spectre \lgbt{} ont participé au cours des dernières décennies.
%On peut penser notamment aux émeutes de Stonewall et à la crise du \sida\ des années 80 jusqu'à aujourd'hui.
%\todo{Si on ne traite pas de Stonewall plus loin dans le texte, décrire l'événement ici}.
%En effet, nous tenterons de comprendre si la communauté \lgbt{} forme une culture en soi au sein d'un groupe culturel plus large et si l'orientation sexuelle peut être considérée comme une forme d'identité à l'intérieur de ce groupe culturel particulier.

Nous introduirons d'abord le concept d'identité par les travaux de Stuart Hall.
Par la définition amenée de la culture amenée par Geertz et celle de l'identité d'Hall, nous pourrons d'un côté comprendre l'évolution généalogique du concept d'identité tout en définissant la culture et voir comment le premier concept s'articule avec le second.
Nous nous pencherons ensuite sur les enjeux entourant les populations \lgbt{} du point de vue de la définition: avons-nous affaire à un groupe culturel distinctif ou plutôt une série d'individus ne possédant en commun que des pratiques sexuelles similaires et une volonté relative d'intégration sociale commune?
Sinfield s'est intéressé à la définition de l'homosexualité dans ce contexte et son travail sur le sujet devrait nous permettre de répondre à cette question.
À cela, nous ferons le parallèle avec les autres identités du spectre \lgbt{}.
Nous tenterons ensuite de nous pencher sur les enjeux géographiques particuliers à prendre en compte pour l'analyse de la population \lgbt{}, autant d'un point de vue de la pratique sexuelle que sous la forme d'un groupe culturel.
Enfin, nous verrons comment ces différents points de vue se sont manifestés au sein de la géographie.

\section{Le regard sociohistorique sur l'identité}
\label{sec:le_regard_sociohistoirique_sur_l_identit_}
%Maintenant que nous avons une définition du concept de culture, nous allons traiter du concept d'identité tel que défini par~\citet{Hall1996a} dans le texte \citetitle{Hall1996a}.
Hall propose une généalogie simplifiée du concept d'identité en reprenant l'historique des derniers siècles en occident, en suivant les différentes étapes socio-économiques qui les ont définies et des courants de pensée qui s'y sont liés.
En tant que tel, Hall remonte au \siecle{18} à l'époque des lumières jusqu'aux dernières décennies, avec le passage de la modernité à la postmodernité.
De ces époques, on peut résumer ces étapes socio-économiques à un passage d'une économie féodale à une économie industrielle vers une économie extrêmement diversifiée et globale typique du capitalisme tardif.
Cette généalogie permet de rendre compte de l'évolution de l'identité individuelle, en montrant que celle-ci est passée selon ces trois stades différents.
Au risque de paraître trop simple, ce schéma se complexifie selon les différents facteurs ayant émergé durant chaque étape socio-économique.

Le premier type identitaire selon \citeauthor{Hall1996a} est le sujet des lumières qui apparait à partir du \siecle{16}.
Sans trop détailler sa production, Hall souligne qu'il s'agit du sujet né de l'esprit des idées des lumières.
On a affaire à un sujet dont la capacité principale est d'être et de penser, en reprenant les idées de Rousseau sur la notion d'être et de penser.
Rattachée à la nation, son identité est peu développée et les conflits de classes ne sont pas encore présents de façon claire.
On a affaire à un individu dont les caractéristiques, comme le statut social, sont conçues comme immuables et banales comparativement aux autres membres de la société~\citeyearpar[596]{Hall1996a}.

Le sujet sociologique, deuxième évolution du concept d'identité, nait en même temps que la sociologie devient une discipline autonome séparée du domaine de l'économie, des sciences politiques, mais surtout de la psychologie.
En effet, selon Hall, celui-ci apparait lorsque la psychologie et la psychanalyse prennent de l'importance dans le domaine de la recherche.
Ces deux champs concentrent leur analyse sur l'individu en son for intérieur et ses relations particulières avec son environnement immédiat pendant la constitution de sa personnalité, à savoir les membres de sa famille et surtout ses parents.
Le domaine du social devient l'apanage de la sociologie.
L'identité agit ici comme l'intermédiaire entre cet individu réputé unique par sa psychologie personnelle et le monde social.
Ce dernier ne sera pas simplement considéré comme une foule d'individus, mais plutôt un ensemble de structures économiques et sociales.
Ces structures en soi influencent crucialement la place de l'individu dans la société, sachant que ce dernier participe et est influencé par celles-ci.
Ce sujet sociologique est relié aux autres individus; Hall le lie autant aux: \textquote[{\citeyear[597]{Hall1996a}}][]{les valeurs, les significations et les symboles --- la culture --- du monde qu'il ou elle habite}.
Ce sujet apparait aussi avec la constitution de l'individualisme.
Il possède également des caractéristiques particulières très peu développées, alors que son essence se compare à celle de ses concitoyens au sein de la nation et des individus appartenant à la même classe sociale.
En somme, les systèmes symboliques prennent une importance qui dépasse les individus bien qu'on commence à voir apparaitre une complexification des identités possibles.
Celle-ci débuterait à fin des lumières, autour du début du \siecle{19}, à la suite du bouleversement des institutions réputées immuables par le renversement bourgeois des traditions et des systèmes de royauté.

Le sujet postmoderne serait le sujet le plus récent et celui qui cadrerait à notre époque, à partir de la fin du \siecle{20}.
Selon \citeauthor{Hall1996a}, ce sujet aurait pris la place du sujet sociologique à la suite d'une déstabilisation de l'identité portée par le sujet moderne et donc sa fragmentation au fil des générations.
Hall recense cinq causes à cette déstabilisation qui se retrouvent dans divers travaux sur la société et l'individu.
La première de ces causes provient des travaux de la théorie marxiste.
En effet, dans celle-ci, la faculté d'action (\anglais{agency}) est questionnée par la reconnaissance des structures sociales et économiques.
Cet individu seul ne possède d'ailleurs plus d'essence propre: plutôt, il acquiert de l'extérieur nombre de ses caractéristiques individuelles, bien souvent par la classe sociale de laquelle il est issu~\citeyearpar[606]{Hall1996a}.

Le deuxième décentrement proviendrait de la décomposition psychologique de l'individu, notamment dans les travaux de Freud et plus tard de Lacan.
Cette décomposition remet radicalement selon l'auteur la position de Rousseau sur la Raison individuelle.
L'individu ne contrôlerait plus l'entièreté de sa vie, mais possèderait en lui des pulsions aussi variées qu'incomprises et une partie de lui-même, héritée socialement, le dépasse et influence sa propre vie.
Les autres individus autour de lui, la famille d'abord, jouent des rôles particuliers dans sa vie.
La reconnaissance de leur existence nous amène donc à ne plus voir l'individu comme une construction autonome, mais plutôt en relation avec les autres, selon des versions symboliques que ceux-ci représentent pour l'adulte à venir~\citeyearpar[607--608]{Hall1996a}.
\todo{à revoir}

La troisième cassure selon Hall s'est opérée par l'apport théorique apporté par les travaux de Ferdinand de Saussure sur le langage.
Selon lui, la langue et les mots ne sont en aucun cas possédés par les individus, au contraire.
Plutôt, les individus s'insèrent dans un système complexe de symboles.
Ils utilisent les mots pour transmettre des significations, des messages aux autres.
Par contre, ces individus ne peuvent être certains de la réception des messages transmis étant donné l'évolution rapide du sens des mots.
Le langage dépasse donc chaque personne et forme un fait de société qui les précède.
La langue prend d'ailleurs place dans la psyché de chaque individu et structure bon nombre de ses pensées~\citeyearpar[608--609]{Hall1996a}

Les travaux de Foucault sur le pouvoir disciplinaire consistent en la quatrième cassure.
Principalement dans \citetitle{Foucault2004a}~\citeyearpar{Foucault2004a}, \citeauthor{Foucault2004a} dresse une généalogie des moyens disciplinaires utilisés par les sociétés occidentales pour faire justice en société.
On y apprend que continuellement, les techniques disciplinaires vont devenir de plus en plus douces.
Politiquement, les sociétés vont passer d'un pouvoir strictement extérieur à l'individu, incarné dans la royauté d'abord, à un système judiciaire graduellement distant dans la société pour prendre en même temps place à l'intérieur de l'individu.
Par des dispositifs de surveillance toujours plus élaborés, non seulement les criminels seront observés, mais également le reste de la société, dans les écoles et les hôpitaux par exemple grâce aux technologies développées dans les centres pénitenciers, les prisons, les cachots.
Ces institutions, par leur pouvoir grandissant et leur capacité à surveiller, en viennent à acquérir suffisamment de connaissances sur les individus pour devenir des agents normatifs puissants~\citep[608--609]{Hall1996a}.

La cinquième et dernière cassure découle de la place croissante qu'a jouée le féminisme dans le monde occidental.
Hall considère que ce sont les mouvements politiques et intellectuels qui ont rendu possible cette cassure, en s'interrogeant sur les rapports de genres entre les individus d'abord.
Ceci ouvrit la porte à la contestation de nombreux groupes marginalisés ou dont les idées politiques remettaient en question le système social et politique au-delà de la lutte des classes.
Ce mouvement a donc permis la mise en place d'identités politiques correspondant à ces nouveaux mouvements de contestation et ouvert la porte à diverses formes de contestation sociale et politique~\citeyearpar[610]{Hall1996a}.

Comme on peut le voir, le sujet sociologique et le sujet postmoderne sont particulièrement similaires sur plusieurs points.
Plusieurs des facteurs ayant mis en place le sujet sociologique en sont venus à le déstabiliser avec les décennies.
Hall ne conçoit pas par contre une généalogie claire et datée de ces déstabilisations; il faut préférablement s'appuyer sur les généalogies internes à chacun des travaux ou événements et leur moment d'apparition dans l'histoire moderne.
On doit plutôt concevoir ceux-ci d'une part comme des facteurs de déstabilisation et des œuvres les expliquant, qui ont amené la mise en place du sujet postmoderne.
Celui-ci, dépassé par le langage, l'économie ou les rapports de pouvoirs est de plus en plus individualisé par la société tout en y étant enchevêtré.
En ceci, comparativement au sujet des lumières, il n'est pas nécessairement raisonnable et son expérience individuelle est beaucoup plus importante dans sa constitution identitaire.
Au-delà de la complexification des rapports sociaux, cette individualisation a ouvert de nouvelles voies pour l'individu, lui permettant d'arriver à coordonner sa place en société.
Pour \citeauthor{Hall1996a}, ce sujet: \foreigntextquote{english}[{\citeyear[598]{Hall1996a}}][]{\textelp{} assumes different identities at different times, identities which are not unified around a coherent \emph{self}. Within us are contradictory identities pulling in different directions, so that our identifications are continuously being shifted about. If we feel we have a unified identity from birth to death, it is only because we construct a comforting story or ``narrative of the self'' about ourselves}.
Nous pouvons croire que cette \emph{narration du soi} s'incarne non pas dans une rationalité objective et indépendante comme pour le sujet des lumières, mais plutôt dans une rationalité subjective.
Cette rationalité répondrait donc à deux défis: d'abord, s'adapter à son environnement immédiat, spatial ou temporel, puis assembler en soi un récit qui arrive à surmonter certaines contradictions inhérentes aux multiples identités individuelles.

Ce que nous appelons la rationalité subjective n'est pas une construction stable et encore moins indépendante de l'individu.
Au-delà des contradictions internes, les autres individus peuvent également contredire ce récit et ces identités.
La politique pour Hall jouerait ce rôle: \foreigntextquote{english}[{\citeyear[610]{Hall1996a}}][]{Since identity shifts according to how the subject is addressed or represented, identification is not automatic, but can be won or lost. It has become politicized. This is sometimes described as a shift from a politics of (class) identity to a politics \emph{difference}}.
Ces changements importants chez l'individu ouvrent la porte à d'autres formes d'identités jusque là improbables.

Nous pouvons conclure cette partie en soulignant que ces bouleversements et les événements historiques du dernier siècle ont mené à l'éclosion des identités \lgbt{}, par le traitement de l'orientation sexuelle par le système de santé, par la psychanalyse et une distance du pouvoir.
L'ensemble de ces facteurs auraient permis l'apparition de groupes d'affinités autour de la question sexuelle.
La reconnaissance de ces nombreuses identités dans la société et de l'intersection de celles-ci entre elles chez les individus doit être prise en compte dans une analyse géoculturelle de la société et des groupes culturels.
En effet, l'individu s'inscrivant dans un modèle culturel chez Geertz traité plus tôt dans le chapitre ne dépendrait plus maintenant d'un seul système de symboles, mais bien de plusieurs, dont la géographie et la temporalité évolueraient.
L'incompréhension envers des contradictions ne créerait plus \textquote[]{une angoisse très forte dès qu'il sent que ces symboles peuvent ne pas pouvoir répondre à tel ou tel aspect de l'expérience}~\citep[33]{Geertz1972}, mais une adaptation et une évolution de son identité.
Ainsi, on arrive d'une certaine façon à dépasser certaines des limites inhérentes à la vision \emph{geertzienne} de la culture dans laquelle l'attachement à une identité comme la religion pour un individu passait principalement par la pratique et une réponse au chaos (et non pas seulement dans une angoisse envers la mort).
L'identité se construirait plutôt dans l'interaction avec autrui et dans la médiation de ces identités avec les forces politiques les régissant ou les contredisant.
Néanmoins, l'usage du terme d'identité n'est pas nécessairement aujourd'hui l'apanage des chercheurs et certains contextes, principalement dans le domaine du politique, permettent une \emph{autoréflexivité} sur ce statut d'identité.

\section{Diversité sexuelle et identité}
\label{sec:diversit_sexuelle_et_identit_} Comme nous l'avons vu dans la section précédente, l'identité de minorité sexuelle peut être considérée comme une création récente propre au contexte postmoderne des dernières décennies.
Ce constat s'explique notamment par le contexte historique actuel dans lequel s'inscrit cette identité ainsi que par la correspondance à certaines caractéristiques proposées par Hall.
Ces caractéristiques peuvent s'apparenter à une identité dans laquelle les individus s'identifient en parallèle à d'autres formes identitaires, comme la communauté ethnique, nationale ou encore de genre, par exemple.
Nous introduirons pour la suite le travail de~\citet{Sinfield1996}, qui, dans \citetitle{Sinfield1996}, traite des difficultés et contradictions propres à l'usage de l'identité chez les communautés \lgbt{}.
Ce dernier met en rapport l'identité et les phénomènes historiques récents de libération sexuelle en occident.
Ce rapport complexe viendrait du développement d'un discours de plus en plus critique en ce qui concerne l'essentialisation des communautés sexuelles.
D'abord dans les mouvements féministes, ce discours s'est concrétisé dans les premiers travaux liés à la théorie queer qui tentèrent de dépasser la \emph{pathologisation} de la sexualité comme on la retrouvait dans les travaux en psychanalyse.
En effet, le texte de Sinfield s'intéresse plus particulièrement aux considérations stratégiques et historiques entourant la mise en place d'une identité homosexuelle pour la communauté elle-même et du point de vue des penseurs de l'identité sexuelle, notamment Foucault.

\subsection{Identités minoritaires et caractère universel de la sexualité}
\label{sub:minorit_s_et_universel}
Le texte de Sinfield débute par la présentation de deux manières de voir l'homosexualité dans la littérature scientifique et comme identité.
Le premier consisterait en un point de vue de \emph{minoritarisation} ou de marginalisation qui conçoit l'homosexuel, gay ou lesbienne, comme un groupe d'individus ayant un style de vie particulier, tel un groupe ethnique.
Le second point de vue serait celui de l'universalisation.
Celui-ci verrait dans l'homosexualité un comportement potentiel chez tous les individus: tous peuvent à un moment ou à un autre commettre un acte homosexuel.
Il n'y a pas lieu de parler d'identité ou de culture comme on traiterait d'un groupe ethnique~\citep[271]{Sinfield1996}.

Le point de vue \emph{minoritarisant} considère les individus à la sexualité déviante, les homosexuels hommes et femmes dans ce cas-ci, comme des groupes identitaires particuliers.
Il va donc à l'encontre du point de vue constructiviste répandu dans les études sur le queer et l'identité sexuelle, inspirées des travaux \citet{Foucault2011}, de \citet{Rubin2010} et de \citet{Butler2007}
Au contraire, le point de vue d'universalisation coïncide avec le champ de pensée constructiviste; en effet, on voit la sexualité comme une donnée variable chez les individus dont la position sociale, l'éducation et l'environnement porteront un effet prépondérant et dont le sens prendra une valeur différente selon la culture traitée.
Contrairement au point de vue \emph{minoritarisant}, le point de vue universalisant voit l'homosexualité comme une attitude, une pratique sexuelle possible pour chaque individu, peu importe la culture.
Cette dernière déterminera si la pratique homosexuelle est tolérée, encouragée ou discriminée et marginalisée~\citep[271]{Sinfield1996}.

Selon Sinfield, les gais et les lesbiennes ont historiquement pris une position stratégique les rapprochant du point de vue \emph{minoritarisant}.
En effet, ces communautés ont emprunté une dynamique de revendication et de lutte sociale similaire à celles des groupes ethniques, notamment des mouvements pour les droits civiques afro-américains~\citep[271]{Sinfield1996}.
Sinfield nomme cette stratégie le \textquote{cadre de l'ethnicité-et-des-droits} (\anglais{ethnicity-and-rights} dans le texte).
Le développement de ce cadre, au-delà de la simple imitation des groupes ethniques, s'est effectué dans le contexte de l'État de droit.
Dans celui-ci, pour améliorer leur position sociale et réduire la marginalisation, les individus doivent, pour reprendre les termes de \citet{Sinfield1996}, \foreigntextquote{english}[{\citeyear[272]{Sinfield1996}}][]{\textelp{} compartmentalize their complex subjectivities in order to \emph{make a claim} (envers le pouvoir)}.
Cette compartimentation de la subjectivité individuelle amène les individus touchés à présenter une caractéristique particulière d'eux-mêmes et donc à vivre un rapprochement avec les autres individus touchés qui se reconnaissent dans cette identité potentielle.
Ce cadre stratégique ne laisse pas entendre qu'il n'existait pas de groupes d'individus gais et lesbiens avant que ceux-ci revendiquent des droits, selon Sinfield.
Ces revendications ont plutôt amené ces groupes à se voir: \textquote{\textelp{} as gay in the terms of a discourse of ethnicity-and-rights} ~\citep[272]{Sinfield1996} et que ces regroupements par affinités se sont mutés en groupe identitaires avec un poids politique.
Sinfield souligne plusieurs problèmes dans la poursuite de ce cadre; d'abord, cette nouvelle identité et cette genèse culturelle peuvent entrer en contradiction avec les autres identités assumées par les individus y prenant part.
Elle désengage également le reste de la société à poser une réflexion profonde sur la sexualité.
C'est ce que propose la pensée \emph{universalisante} reprise par certains groupes plus radicaux (dont le mouvement queer, qui par définition vise une refonte des normes sur la sexualité et le genre plutôt que l'acquisition de droits)~\citep[273]{Sinfield1996}.

Plus loin dans son texte, Sinfield explique les différences culturelles dans lesquelles a eu lieu le développement de mouvements de contestation gais et lesbiens.
Plus particulièrement, Sinfield s'intéresse aux différences entre les États-Unis et la Grande-Bretagne.
Pour l'auteur, le cadre de l'ethnicité-et-des-droits se traduit de différentes manières selon la région étudiée.
En Grande-Bretagne, la concession de droit s'inscrit dans la suite de l'État-providence, où l'état anglais concède des acquis supplémentaires dans la perspective d'assurer à tous les citoyens un mode de vie décent.
Aux États-Unis, on s'inspire plutôt des valeurs traditionnelles américaines qui s'orientent surtout vers la liberté aux individus~\citep[274]{Sinfield1996}.
En cherchant à obtenir cette liberté offerte par la société américaine, les groupes ethniques s'appuient du même coup sur ce que Sinfield nomme le mythe de la pluralité américaine.
Celui-ci laisse entendre que chaque groupe culturel équivaut à d'autres et peut revendiquer un accès égal aux mêmes ressources que les autres dans un cadre compétitif.
C'est ce mode stratégique qui se serait par la suite répandu dans les autres mouvements de contestations et qui aurait  mis au premier plan le modèle de l'ethnicité-et-des-droits.

\subsection{Diaspora et hybridité}
\label{sub:diaspora_et_hybridit_} Pour comprendre la multiplicité des origines différentes --- pour les individus s'identifiant à l'identité homosexuelle ou lesbienne --- Sinfield propose de concevoir cette identité comme nécessairement hybride.
Cette hybridité se compare de façon analogue celle imposée aux individus appartenant à une diaspora.
On définit une diaspora par l'ensemble des individus se retrouvant géographiquement à une certaine distance du lieu d'où provient leur culture d'appartenance, par immigration ou par déterritorialisation.
À titre d'exemple, on peut notamment penser à la diaspora juive ou encore la population afro-américaine.
L'existence de diasporas pour l'auteur montre la résilience qu'ont les individus à résister à l'assujettissement de leur identité par leur milieu d'accueil et à conserver leur culture.
Structurellement, la: \foreignblockquote{english}[{\cite[278]{Sinfield1996}}][.]{`Diaspora' \textelp{} usually invokes a true point of origin, and an authentic line --- hereditary and/or historical --- back to that. However, diasporic Black culture, Hall says, is defined `not by essence or purity, but by the recognition of a necessary heterogeneity and diversity; by a conception of ``identity'' which lives with and through, not despite, difference; by hybridity'}

Cette hybridité peut donc être conçue comme participant à une forme d'ethnogenèse tout en possédant un potentiel politique. 
Au lieu de répondre à certains archétypes que la société d'accueil impose sur l'identité des groupes culturels diasporiques, ceux-ci peuvent participer à la conception de leur identité en reprenant certains traits culturels:
\foreignblockquote{english}[{\cite[277]{Sinfield1996}}][.]{Stuart Hall traces two phases in self-awareness among British Black people. In the first, `Black' is the organizing principle: instead of colluding with hegemonic versions of themselves, Blacks seek to make their own images, to represent themselves. In the second phase (which Hall says does not displace the first) it is recognized that representation is formative --- active, constitutive --- rather than mimetic}\citep[Sinfield cite ici][]{Hall1990}\todo{Compléter la référence}.
Néanmoins, dans le cas de la culture afro-américaine, nous nous retrouvons dans un contexte où ce concept de culture est en concurrence avec celui de la race selon l'auteur, où une certaine \emph{essentialisation} par le racisme maintient cette version hégémonique d'eux-mêmes.

Pour comprendre qu'il existerait une culture née par l'hybridité chez les individus gais et lesbiens, l'auteur considère que l'on doit se baser sur l'histoire des individus plutôt que s'attarder seulement à l'Histoire au sens large des sociétés.
En effet: \foreignblockquote{english}[{\cite[280]{Sinfield1996}}][.]{\textelp{} for lesbians and gay men the diasporic sense of separation and loss, so far from affording a principle of coherence for our subcultures, may actually attach to aspects of the (heterosexual) culture of our childhood, where we are no longer `at home'. Instead of dispersing, we assemble.

The hybridity of our subcultures derives not from the loss of even a mythical unity, but from the difficulty we experience in envisioning ourselves beyond the framework of normative heterosexism --- the \emph{straightgeist} \textelp{}}
Dans ce contexte, on peut dénoter que l'auteur souligne une des particularités de la culture dominante: son caractère essentiellement hétérosexuel au niveau des normes, ou hétéronormatif (voir partie~\ref{sec:enjeux_g_ographiques_du_recours_l_identit_}).
Le départ de la culture hétérosexuelle ou \anglais{straightgeist} à laquelle tous les individus de la culture homosexuelle doivent répondre est partiel; à tout moment, les individus \lgbt{} pour ne nommer que ceux-ci doivent composer avec le reste de la culture hétérosexuelle dans les autres sphères de leur vie, que ce soit à l'école, au travail ou dans l'espace public.
C'est en raison de cette négociation inévitable avec la culture dominante que pourrait se justifier le caractère hybride de la culture homosexuelle.
Les objets culturels, les pratiques culturelles et sociales s'inscrivent dans cette culture hybride et peuvent donc ou non être compris par la culture dominante.

Pour conclure cette partie, notons que le trait commun partagé par les gays et lesbiennes dans le cadre de l'analyse par l'ethnicité-et-des-droits est l'altérité vécue par les individus non hétérosexuels: \foreignblockquote{english}[{\cite[289]{Sinfield1996}}][.]{Our apparent unity is founded in the shared condition of being not heterosexual --- compare `people of colour', whose collocation derives from being not-white}.
Étant donné cet accent mis sur la discordance à une norme, la communauté \lgbt\ est nécessairement très large et diverse.
Sinfield hésite donc à parler ici d'une culture en soi; on propose plutôt l'usage du concept de sous-culture qui rendrait mieux ce caractère de diversité et qui reconnaîtrait le côté construit et récent de celle-ci:
\foreignblockquote{english}[{\cite[289]{Sinfield1996}}][.]{It is to protect my argument from the disadvantages of the ethnicity model that I have been insisting on `subculture', as opposed to `identity' or `community': I envisage it as retaining a strong sense of diversity, of provisionality, of constructedness}.

% , Hall \& Gay introduisent le concept d'identité pour traiter des groupes
% sociaux et culturels qui s'opposerait à l'ancien sujet moderne. À partir de ce
% concept, il devient a priori possible de traiter de groupes ou communautés comme
% les homosexuels, bisexuels, trans- et queers dans le contexte culturel précisé
% précédemment.


\subsection{Enjeux géographiques du recours à l'identité}
\label{sec:enjeux_g_ographiques_du_recours_l_identit_}
%Les textes suivant
% permettraient de situer l'usage de la culture dans un contexte précis. Par
% exemple, \textquote{The Location of Culture: The Urban Culturalist Perspective}
% propose l'étude culturelle des phénomènes urbains, alors que les études urbaines
% utilisent normalement des méthodes quantitatives pour traiter des mêmes
% questions (Borer, 2006). Les parties précédentes se sont principalement
% intéressées à l'analyse générale des concepts d'identité, de culture et
% d'identité sexuelle en demeurant essentiellement dans un contexte sociologique.
% Pour la poursuite de ce texte, nous nous intéresserons plus particulièrement au
% domaine spatial de ces concepts en tentant d'apposer une regard géographique sur
% le culture. %, plus particulièrement par un regard sur la ville comme espace
% culturel. Celle-ci, en plus d'être un des milieux les plus populeux qu'on
% retrouve dans plusieurs des sociétés humaines, sinon la totalité de nos jours,
% permet de comprendre les enjeux entourant la mixité sociale et culturelle

%Pour cette partie, nous pencherons sur deux textes de Michael Borer, à savoir \textquote{The Location of Culture: The Urban Culturalist Perspective} (2006) et \textquote{From Collective Memory to Collective Imagination} (2010).~\citep{Borer2006}
%Le deuxième texte de Borer, \textquote{From Collective Memory to Collective
%Imagination >> propose l'analyse spatiotemporelle des phénomènes culturels en
%milieu urbain, un point de vue méthodologique qui rejoint celui de Larry Knopp.~\citep{Borer2010}

En géographie queer{}, on retrouve les deux paradigmes soulevés à la partie~\ref{sub:minorit_s_et_universel}, à savoir un partage entre une analyse autour de la diversité comme construction sociale et une autre centrée sur l'identité gaie, lesbiennes, bisexuelle ou trans.
Plus particulièrement, la pensée géographique peut se pencher sur les espaces occupés par les gays et lesbiennes, ou plutôt s'intéresser au caractère normatif des espaces.
Les premiers travaux en géographie sexuelle se sont principalement attardés au premier point de vue.
À l'inverse, durant les vingt années précédentes, on remet en question le point de vue \emph{minoritarisant} qu'on retrouve toujours dans certains textes comme celui de~\citet{Sinfield1996} pour plutôt se pencher sur les normes sociales et leurs rapports avec l'espace.
C'est ce dernier point de vue que défend et explique Natalie~\citet{Oswin2008} dans l'article \citetitle{Oswin2008} que nous traiterons dans la suite de ce texte.

Dans ce texte, Oswin s'oppose à l'idée que les espaces queers puissent former des lieux en opposition totale avec les normes de la société dominante.
Plutôt, la recherche récente en géographie sexuelle a permis de rendre compte que c'est par la présence d'individus dont la manière de performer le genre ou l'identité sexuelle ne correspondent pas aux normes sexuelles dominantes que les espaces de la société apparaissent comme hétéronormatifs.
Cette hétéronormativité agit par des jeux de pouvoir s'y établissant, entre individus et envers eux-mêmes.
Ce jeu de pouvoir s'instaurerait de plusieurs manières: par la marginalisation (violence homophobe, exclusion), par une présence accrue du pouvoir policier à proximité des espaces queers ou par le refus des instances gouvernementales de répondre aux demandes des populations \lgbt{} (durant la crise du \sida, par exemple).

Ce jeu de pouvoir sur les normes sociales se manifeste particulièrement dans les espaces réputés occupés par des individus appartenant au spectre \lgbt{} où plusieurs visions de l'homosexualité se confrontent.
En effet, comparativement à l'idée d'une culture homosexuelle uniforme et partagée par certains membres de la communauté gaie et lesbienne, les auteurs en études queers et en géographie queer ont plutôt montré que plusieurs groupes luttent selon deux types d'enjeux.
D'abord, certains supportent l'idée que la communauté devrait travailler vers un élargissement des normes qui mènerait à terme à une plus grande inclusion sociale.
Nommé point de vue \emph{assimilationniste}, il s'opposerait à celui des \emph{libérationnistes} qui souhaitent plutôt remettre en question certaines normes qu'ils considèrent comme empruntées à la culture hétérosexuelle et répliquées à l'intérieur même des espaces queers, un phénomène couvert par le concept d'homonormativité.
Oswin définit cette dernière à partir d'une citation de Lisa Duggan, où l'homonormativité est:
\foreignblockquote{english}[{\cite[tel que cité dans][92]{Oswin2008}}][]{\foreigntextquote{english}[{\cite[50]{Duggan2003}}][]{A politics that does not contest dominant heteronormative assumptions and institutions, but upholds and sustains them, while promising the possibility of a demobilized gay constituency and a privatized, depoliticized gay culture anchored in domesticity and consumption}}.

Un autre point important du texte d'Oswin est la réinterprétation du sens des multiples identités que peut posséder un individu du spectre \lgbt{}.
Au lieu de se baser sur l'hybridité, ces identités sont plutôt conçues comme des sources d'oppression, en ce qui concerne la racisation, la classe sociale ou le genre par exemple.
Dans de nombreux espaces queers, il a été remarqué que souvent le pouvoir était détenu par des individus dits privilégiés sur d'autres bases identitaires que la simple orientation sexuelle.
Également, au sein même des communautés gaies, d'autres formes d'identité sexuelle ou de genre sont mises de côté, comme la bisexualité ou les individus trans-~\citep[93]{Oswin2008}.
L'auteur met l'accent sur l'importance de reconnaître le potentiel qu'ont les chercheurs de réifier les communautés des milieux qu'ils étudient. Ceux-ci peuvent notamment recréer certaines hiérarchies en omettant les inégalités sociales entre individus d'une même communauté ou les enjeux de racisation, par exemple.

Pour la suite, nous nous intéresserons plus particulièrement aux méthodes offertes par la géographie pour traiter efficacement des questions de diversité sexuelle et d'espace.
Larry~\citet{Knopp2004}, un des premiers chercheurs à lier les études queers à la géographie culturelle, se penche, dans \citetitle{Knopp2004} sur la théorie de l'acteur-réseau.
Cette dernière pourrait permettre méthodologiquement le dépassement de certaines limites conceptuelles de l'identité, dans notre cas \lgbt{}, tout en s’efforçant de fuir certains déterminismes en géographie culturelle~\citep{Knopp2004}.
En effet, au lieu de s'appuyer sur un point de vue \emph{minoritarisant} des groupes et communautés \lgbt{}, Knopp reconnaît d'emblée le potentiel qu'ont ces groupes d'affecter les structures sociales de pouvoir.
Ce texte s'inscrit donc ainsi moins dans l'étude d'un groupe culturel particulier; on vise plutôt l'analyse des normes sociales d'un ensemble culturel par les conflits amenés.
Knopp utilise dans son texte des termes similaires à ceux de \citet{Sinfield1996}, à savoir que les groupes queers seraient entre autres hybrides au point de vue identitaire et que leur présence sociale prendrait la forme d'une diaspora.
En effet, si les espaces queers sont les espaces vers lesquels se dirigent les membres de ces communautés dans le texte de \citet{Sinfield1996}, Knopp considère plutôt que ce sont les déplacements spatiaux et temporels qui sont formateurs des identités queers.
La géographie aurait alors le potentiel de rendre compte de ces déplacements par les théories \emph{non représentationnelles}\footnote{Bien que l'étude du mouvement au sein des communautés \lgbt{} nous apparait extrêmement pertinente, nous allons plutôt nous arrêter aux conséquences de ces déplacements dans les trames urbaines et rurales. Certaines de nos données pourraient se prêter à une analyse du mouvement dans d'éventuelles recherches subséquentes; nous reviendrons sur ce point en conclusion.}.

Ces déplacements ont en effet un sens particulier: \foreignblockquote{english}[{\cite[123]{Knopp2004}}][.]{For gays, lesbians, bisexuals, transgenders, and other queers, as for other oppressed groups, this means seeking people, places, relationships, and ways of being that provide the physical and emotional security, the wholeness as individuals and as collectivities, and the solidarity that are denied us in a heterosexist world}
Au lieu d'être les héritiers directs d'une culture queer{}, on doit plutôt concevoir que les individus queers{} possèdent bel et bien la culture dominante, mais que leur intégration sociale passe par d'autres trajets que ceux proposés normalement par celle-ci.

Knopp avance même que ces déplacements ont une importance assez forte pour être génératrice d'une ontologie particulière:
\foreignblockquote{english}[{\cite[123]{Knopp2004}}][.]{It is also about testing, exploring, and experimenting with alternative ways of \emph{being}, in contexts that are unencumbered by the expectations of tight-knit family, kinship, or community relationships—no matter how accepting these might be perceived to be}
En même temps que les individus \lgbt{} quittent le contexte familial hétérosexuel comme début de parcours et principal lieu d'acquisition de culture, ils transportent avec eux des éléments de cette culture et la transforment au gré de leurs expériences.
Pour Knopp, l'expérience queer en soi provoque la constitution de nouvelles données culturelles, spatiales et participe donc à cette hybridité de l'identité queer.
\foreignblockquote{english}[{\cite[130]{Knopp2004}}][.]{As queer bodies and subjectivities circulate through (and constitute) time and space, they leave legacies, absorb others, and mutate. They spread information, values, and culture, and constitute barriers to such spreads at the same time. This is diffusion \emph{pari excellence}}

Si ces caractéristiques sont particulièrement importantes, Knopp souligne tout de même que ces processus formateurs dans le domaine identitaire ne sont pas l'apanage des individus \lgbt{}, mais peuvent être considérés comme des expériences probables pour tous les individus; elles ne sont pas essentielles à l'expérience \lgbt{}.
Néanmoins, le contexte normatif entourant la sexualité pousse tout de même les populations \lgbt{} à ces parcours de façon plus particulière.
Les chemins de vie empruntés ressembleraient à des migrations vers l'acceptation sociale de l'identité sexuelle, qu'elle passe par l'anonymat ou pas l'inclusion au sein d'un espace \lgbt{}.
Knopp appuie cette caractéristique par plusieurs travaux en géographie queer qui reconnaissent l'importance du parcours et du déplacement spatio-temporel chez les individus \lgbt{}~\citep[123]{Knopp2004}.

\subsection{Synthèse}
\label{sec:synth_se}
Au cours de ce chapitre, nous avons d'abord traité de la question de la culture d'un point de vue anthropologique, à partir de Clifford Geertz\todo{à bonifier!}.
Si une culture est bel et bien un système de pratiques et de conceptions héritées répondant à certaines absences de réponse dans la réalité, dans le quotidien, on pourrait croire qu'il existe une culture \lgbt{}.
Il existe des pratiques particulières des communautés gaies et lesbiennes, notamment par l'existence de milieux de vie particuliers, de pratiques sociales et sexuelles différentes des hétérosexuels et d'une histoire particulière à cette communauté (qui demeure tout de même imbriquée dans celle de la société majoritaire).
Le contexte social, politique et intellectuel du dernier siècle semble avoir été propice à l'irruption des communautés gaies et lesbiennes, comme nous l'apprend la partie sur le texte de Hall.
En effet, si, des lumières jusqu'à la modernité, il était difficile d'imaginer l'existence d'identités basées sur la sexualité, différents décentrements et déstabilisations du sujet moderne ont amené cette possibilité.
Elle s'est concrétisée par une remise en question de l'uniformité relative des individus au sein d'une même culture.
Est-ce toutefois suffisant pour dire qu'il existerait aujourd'hui des cultures basées sur des identités?
En ce qui concerne la question de l'orientation sexuelle du moins, la réponse n'est pas claire.
On remet aussi en question l'idée de s'attarder à l'identité dans un contexte où ces individus, ceux de la diversité sexuelle, sont souvent victimes de marginalisation dans le contexte culturel occidental.
Leur présence exprime des instabilités propres au nouveau sujet postmoderne vivant dans un contexte culturel aux normes qui témoignent de jeux de pouvoir bien présents, et ce pour les individus de la diversité sexuelle et le reste de la société.
Au-delà des enjeux stratégiques liés à l'étude des individus \lgbt{}, dans une perspective de changement social, on pourrait croire que ces déstabilisations continuent d'avoir lieu.
La critique des normes sociales et des enjeux de pouvoir demeure l'avenue privilégiée, dans un contexte de l'étude de la culture occidentale.

Nous pouvons conclure en soulignant que les liens entre identité et culture ne sont pas encore clairement définis; l'identité comme concept laisse croire qu'il est nécessaire d'appartenir à une culture particulière pour se lier à une identité précise.
Pourtant, malgré l'ambiguïté entourant l'idée de culture ou de sous-culture \lgbt{}, on peut difficilement remettre en question l'idée d'identité, sachant l'existence de groupes sociaux, culturels et d'espaces \lgbt{} dans la société occidentale.

La question du pouvoir reste à creuser plus en profondeur comme souligné par Oswin dans son texte.
Celle-ci propose également l'étude de l'hétérosexualité, une avenue intéressante pour comprendre si cette identité sexuelle existe comme celle des autres formes d'orientation sexuelle.
On pourrait ainsi comprendre les effets probables de la normalisation de la sexualité à une plus grande échelle au sein de la culture occidentale~\citep[100]{Oswin2008}.
Enfin, il apparait également pertinent d’entamer une étude auprès des groupes \lgbt{} pour savoir qu'elle est la place de la communauté dans leur identité.
Un tel travail permettrait d'apprendre l'existence, selon eux, des éléments nécessaires pour parler d'une culture queer, que ce soit dans histoire ou dans les significations particulières des éléments et pratiques composant les espaces et la temporalité queer.

\subsection{La diversité sexuelle dans les sciences sociales et la géographie}
\label{ssub:la_diversit_sexuelle_dans_les_sciences_sociales_et_la_g_ographie}
Nous travaillerons dans cette étude à partir de l'identité sexuelle et du genre.
\citet{Foucault2011} décrit dans son œuvre \citetitle{Foucault2011} le processus par lequel les relations entre individus de même sexe sont devenues, par la médecine notamment, une forme de trouble de la sexualité jusqu'à constituer aujourd'hui, après bien des luttes sociales, une identité particulière~\citep{Foucault2011}.
Plusieurs autres auteurs, en études féministes en particulier, avec Gayle Rubin et Judith Butler~\citep[98]{Marcus2005}, ont permis de montrer que le genre consiste aussi en un construit social et une performance, contredisant que celui-ci forme l'essence de l'individu~\citep{Butler2007}.
On remet alors en question l'idée de nature en sexualité ainsi que la possibilité que les comportements dits anormaux, comme l'homosexualité, la bisexualité ou un genre non binaire, soient en fait des comportements sociaux qui ne concordent tout simplement pas avec les normes.

Ce travail théorique a été plus tardivement repris par la géographie.
À la même époque que les études sur le genre avaient lieu et sortaient du seul cadre de la psychanalyse~\citep{Rubin2011a,Rubin2011}, les luttes sociales entreprises par les gais et lesbiennes prenaient place dans l'espace public~\citep[422-427]{Spencer2005}.
D'abord dans le bar Stonewall à New York, où des femmes trans organisèrent une résistance à des descentes policières répétées visant à harceler les individus \lgbt{} qui fréquentaient les lieux, le mouvement a pris la forme du mouvement des droits civiques dans les années 1970.
Cette série d'événements s'est fait parallèlement à d'autres luttes, dont le mouvement de libération afroaméricain \todo{revoir avec Internet} et le mouvement féministe de la deuxième vague..
Dans les grandes villes des États-Unis d'abord, puis progressivement d'Europe et du Canada, on a assisté à l'apparition des villages gais qui se sont créés suite à l'achat et la mise en place de bars et commerces tenus ou fréquentés par des individus \lgbt{}
Ces lieux sont principalement apparus dans les anciens \anglais{Red Light}, ou dans d'anciens quartiers ouvriers, pratiquement toujours au centre des villes.
Ces nouveaux espaces urbains ont été au départ utilisés pour la protection qu'ils apportaient par l'anonymat, puis, avec ces luttes, le partage d'un mode de vie et de lutte politique.
Relativement invisibles, les espaces  habités par les gais et lesbiennes acquièrent une visibilité supplémentaire en même temps que l'on commença à s'intéresser aux liens entre l'espace et la sexualité dans un contexte ou l'identité sexuelle semblait détenir une ontologie propre.

Pour certains auteurs, ce développement identitaire prend une forme ontologique.
L'ontologie désigne le savoir représentant une vision particulière de l'univers.
Par contre, son usage pour décrire une culture se serait de plus en plus complexifié par l'apparition récente d'une multitude de nouvelles identités, par la politisation de plusieurs pans de population vivant la marginalisation et par une remise en question de la culture dominante~\citep[122]{Knopp2004}.
Le queer pourrait désigner l'ontologie propre au spectre de la diversité sexuelle apte à dépasser les normes séparant chacune des identités du spectre \lgbt{} et pouvant rendre compte d'une certaine cohésion qui rejoint historiquement les individus de ces communautés~\citep[122]{Knopp2004}.

Larry Knopp propose d'ailleurs la théorie de l'acteur-réseau pour parvenir à identifier et analyser toutes les composantes spatiales du queer --- comme le lieu et le mouvement --- et recréer une certaine cohésion de la théorie queer face aux divers mouvements qui la façonnent, comme le constructivisme ou le matérialisme.
Elle permettrait également d'échapper à la notion du territoire pour plutôt utiliser une vision plus relativiste de l'espace étant donné que les éléments d'un réseau peuvent croiser d'autres réseaux et porter plusieurs sens.
Des approches où l'espace peut prendre diverses formes et qualités~\citep{DiMeo1998} seraient beaucoup plus compatibles avec la théorie de l'acteur-réseau.

La géographie culturelle se penche donc depuis peu à ces nouveaux groupes sociaux, surtout avec l'avènement des approches postmoderne.
Celles-ci, dont la théorie queer fait partie, s'intéressent de nouveau à l'individu en prenant tout de même assise sur les travaux ultérieurs, notamment le structuralisme et la théorie critique.
Une des premières œuvres à avoir marqué le commencement de la géographie queer est~\citetitle{Bell1994} de~\citet{Bell1994} qui amena une sélection de textes, montrant la pertinence et les possibilités multiples de cette nouvelle branche de la géographie.
Néanmoins, dans plusieurs études, l'homme gai a été mis en priorité comme objet où on s'attarda à l'usage de certaines méthodes de géographie, comme le recours au territoire~\citep{Podmore2001,Oswin2008}.
On travaillait donc à comprendre la progression des villages gais, souvent menée par des hommes.
Les femmes lesbiennes et les individus caractérisés par d'autres formes de sexualités moins étudiées ont été mis de côté.
L'argument derrière cette division est celles-ci sont globalement moins visibles et présentes dans l'espace public et que l'on devrait s'attarder au ménage et à la maison pour traiter de leurs situations~\citep[333-334]{Podmore2001}.
Il s'agit, selon Podmore, dans l'article \citetitle{Podmore2001} plutôt d'un problème méthodologique alors que les concepts utilisés actuellement en géographie ne suffisent plus pour étudier les nouvelles formes identitaires.
Le recours concept de territoire  pour s'intéresser aux communautés sexuelles apparait donc problématique: non seulement on exclut les femmes, mais également d'autres groupes encore moins visibles, comme les transgenres et les bisexuels.
Il convient alors, en reprenant la position de Natalie Oswin, de traiter d'espaces queers plutôt que d'espaces dits \lgbt.
Cette nuance vise à éviter une essentialisation de l'autre sans mettre en priorité certains genres, utiliser des dichotomies fautives entre l'homosexuel face au \anglais{straight} et considérer \latin{de facto} un milieu queer comme un espace de résistance de dissidence~\citep{Oswin2008}.
En effet, comme l'a montré Nathaniel M. \citet{Lewis2011} dans son article \citetitle{Lewis2011}, les hommes gais, même sans espaces précis, héritent des normes de leur environnement sur l'identité sexuelle.
Ces derniers peuvent également induire un effet sur ces normes, l'auteur prenant exemple sur les familles homoparentales dans les banlieues.
Ces dernières offrent au chercheur un regard différent sur la définition d'un ménage de classe moyenne, particulièrement au sein d'un espace comme un voisinage~\citep[304]{Lewis2011}.

Nous avons dans cette partie réussi à apporter certaines nuances aux concepts de territoire et d'espace que nous allons utiliser dans le reste de ce mémoire, plus particulièrement dans le chapitre sur l'analyse.
Par contre, pour revenir à la géographie culturelle, nous introduirons dans la prochaine section Joël Bonnemaison et son concept de géosymbole, ainsi que d'autres auteurs pour définir et tenter de raffiner l'usage de ce concept.
Comme nous allons le voir, le géosymbole s'inscrit dans une culture en particulier et participe à la persistance des identités.
Pour pouvoir les identifier et les décrire, d'autres concepts nous apparaissent essentiels, et donc, pour la suite, nous allons nous plonger en premier lieu dans le domaine de la sémiotique.

\subsection{La sémiotique}
\label{ssub:la_semiotique} La sémiotique prend ses racines dans les travaux de plusieurs auteurs, principalement Ferdinand de Saussure que nous avons nommé précédemment en nous penchant sur l'évolution du sujet chez Hall.
Sa contribution la plus remarquée est son travail sur le langage qui a éventuellement donné naissance au courant du structuralisme~\citep{Noth1995}.
%Dans ses travaux sur la religion, Clifford Geertz considère que le géosymbole est \{phrase non terminée}
On retrouve aujourd'hui toute une variété d'auteurs et de mouvements s'intéressant aux symboles dans la communication, autant en musique qu'en théologie, mais dans le cas présent, nous nous intéresserons principalement aux auteurs en géographie.
Au sein de ceux-ci, plus particulièrement dans le champ des études tropicalistes, on peut nommer d'abord Joël Bonnemaison.
Ce dernier proposa et démontra l'importance du concept de géosymbole comme moyen utilisé par les groupes ethniques pour s'ancrer à l'espace habité qui deviendra le territoire par des itinéraires, des formes et des tracés dans le paysage~\citep{Bonnemaison1981}.
Mario Bédard va poursuivre la réflexion et inclure tout comportement culturel particulier rattaché à l'espace, et créer une typologie complexe des différents géosymboles que l'on retrouve dans le territoire~\citep{Bedard2002}.
Jérôme Monnet quant à lui traite plutôt des processus conscients de création de symboles, et des relations particulières qui lient les symboles à l'espace, au pouvoir et à l'identité, surtout dans le contexte occidental et capitaliste~\citep{Monnet1998}.
L'idée de contexte se joue d'ailleurs à plusieurs échelles, et il apparait important selon l'auteur de ne pas limiter l'étude d'un symbole à lui-même.
Il faudrait en fait considérer l'environnement du symbole, sachant que la présence d'un symbole d'un type peut amener la propagation de symboles similaires, comme dans le cas d'un centre commercial qui provoque l'arrivée d'autres commerces, comme symboles marchands~\citep[7-8]{Monnet1998}.
Il apparait donc possible, à l'aide des travaux effectués en sémiotique et en géographie culturelle, d'arriver à décrire l'espace occupé par les groupes et individus appartenant au spectre de la diversité sexuelle
Nous devons par contre tenir compte de certaines difficultés, comme le recours au territoire (qui peut masquer espaces dispersés ou en réseaux) ou encore à la matérialité des symboles, qui peuvent ne pas être suffisants comme appuis pour décrire une réalité spatiale soit temporaire, soit mobile.

Plus proche de la méthodologie, le géosymbole pourrait faire l'objet d'une analyse basée sur les sens visuels.
À cette idée, Rose résume les différents moyens de faire de la sémiologie une méthode et une analyse de la société~\citep[69--73]{Rose2001}.
On retrouverait deux courants principaux dans le domaine de la recherche en sémiologie qui déborde les champs disciplinaires traditionnels comme l'anthropologie et la géographie.
Le premier courant serait représenté par les chercheurs pratiquant la sémiologie traditionnelle ou courante (\anglais{mainstream} dans le texte), dont l'analyse vise à déconstruire les médias pour y chercher les signes, unités de base de la sémiotique.
Le deuxième courant, nommé sémiologie sociale (\anglais{social semiotics} dans le texte), consisterait en une analyse des faits sociaux, et plus particulièrement, comment les individus agissent et sont censés agir selon les lieux et les contextes.

Dans le chapitre suivant, nous reviendrons sur la sémiotique, mais du point de vue de la méthode.
Nous décrirons alors comment nous pourrons, à partir d'une déconstruction des images collectées, obtenir plus d'information sur le sens des géosymboles évoqués et ce qu'ils traduisent comme relation territoriale entre les espaces des villes traitées et les sous-groupes du spectre \lgbt{}.
Pour l'instant, nous voulons nous pencher sur les outils concepts utilisés par la sémiotique, traditionnelle ou sociale, pour déconstruire les symboles.

Selon Rose, le signe est le principal concept que l'on retrouve en sémiotique~\citeyearpar[74]{Rose2001}.
Il consiste en différentes informations qui circuleraient dans un système de communication, que ce soit entre deux individus ou un individu et son environnement.
Ce concept, dans le modèle de Saussure, est divisé en deux parties pour rendre compte de la relation existant entre un individu dans un système de communication et les objets ou idées auxquels il réfère.
Ce premier élément correspond au signifié qui consiste en l'objet ou l'idée auquel on octroie un signe.
Dans un contexte de communication ou non, le signifié en soi existe toujours.
Bien que certains signifiés dépendent de l'être humain pour exister, comme la philosophie ou les mathématiques ou des objets manufacturés, en général ces signifiés existent sans dépendance à un système de communication auquel on ferait référence.
Autrement dit, si c'est l'être humain dans sa communication qui permet une telle caractérisation d'un objet par le signifié, il demeure que ce dernier peut exister en soi.
Le signifiant quant à lui ne peut se séparer de ce système de communication.
En effet, ce signifiant forme en quelque sorte l'élément qui permet d'évoquer chez la personne jouant le rôle de communicateur l'idée rattachée au signifié~\citeyearpar[74]{Rose2001}.
Le mot \emph{roche} dans un tel système constitue un signifiant lié à la substance minérale séparée d'un substrat rocheux plus important qui est ici le signifié dans sa forme la plus générale et dont la forme peut différer selon les occasions et les contextes.
Enfin, ce signifiant n'est pas qu'un mot prononcé ou lu, il peut également consister en un média au sein duquel l'idée d'un signifié est transmise, comme une image, une vidéo, un poème, etc.

%\subsubsection{Signe}
Dans l'ensemble, cette définition du signe correspond aux théories de Saussure.
Elle présente rapidement des limites: si un signifiant peut prendre plusieurs formes, comment peut-on comparer facilement ceux-ci?
Si les individus utilisent divers signifiants pour communiquer, certains demandent un bagage culturel particulier pour être compris alors que d'autres sont plus universels.
Les travaux de Pierces permettent conceptuellement de dépasser ces limites.
Nous allons enrichir cet usage du signe en concordance avec la théorie développée par Pierce et présentée par Rose~\citeyearpar[78]{Rose2001}.
Selon l'auteur, le signe peut prendre plusieurs formes selon le média dans lequel on le trouve et le degré d'abstraction de différenciation qui peut exister entre le signifié et le signifiant.
Ainsi, différents termes ont été développés pour différencier les différents signes.
Parmi ceux-ci, on retrouve d'abord l'icône, qui consiste en une quasi-concordance entre le signifié et le signifiant.
Dans un dessin d'enfant, selon les cultures, un arbre est souvent dessiné avec un tronc brun, quelques branches et des feuilles vertes.
Cette image, sans recouper l'immense diversité des arbres, correspond tout de même à plusieurs types d'arbres que l'on retrouve dans un climat tempéré.
Par contre, ce même arbre pourrait figurer sur un roman ou un livre traitant de la vie, de la longévité ou encore de la généalogie dans un contexte familial: dans ce cas-ci, l'arbre est lié de façon abstraite à ces concepts.
On a donc affaire à un symbole, un signe dont la relation entre le signifiant et le signifié est définie de façon arbitraire et relative à un contexte culturel particulier.
Le troisième type de signe est l'index dans lequel le signifiant et le signifié ne sont a priori pas liés, de façon comparable au symbole, mais qu'en plus, le signifiant n'a pas de sens en soi sans le signifié.
En effet, dans les cas précédents, surtout dans des cas plus visuels, on ne peut séparer le signifiant du signifié, ou du moins, on peut trouver d'autres types de relations.
La longévité peut être signifiée par des signifiants différents, comme une personne très âgée, alors que le signifiant d'arbre rappelle le signifié \enquote{arbre}, sans mise en contexte.
Dans l'index, la personne en communication doit avoir connaissance du lien entre le signifiant et le signifié.
On peut penser par exemple aux feux de signalisations : les cercles rouges, jaunes et verts ont été désignés par les signifiants des différents types de restrictions ou de non-restrictions au déplacement en automobile.
Par contre, sans connaissance préalable des codes de la route, un individu ne peut comprendre ce lien de façon instinctive.
Un autre exemple serait un mot dans une langue; si l'individu qui communique n'est pas locuteur de cette langue ou ne possède pas les connaissances pour saisir de quoi il s'agit, l'ensemble des caractères ne peut rien évoquer du signifié qu'il désigne.

On le constate rapidement, ce modèle ne porte \latin{a priori} rien de sens géographique en soi.
Il ne permet pas facilement d'articuler les nuances propres à la culture et à l'identité d'un locuteur au-delà de la connaissance ou de l'ignorance d'un signe prenant la forme d'un index.
C'est ici que nous allons nous arrêter pour traiter des notions d'espaces et de territoire.
Nous nous attarderons ensuite à la synthèse à l'aide du concept de géosymbole tel qu'articulé par \citet{Bonnemaison1981}.


\section{Espaces et territoires}
\label{sec:espaces_et_territoires} Nous poursuivrons avec ces deux concepts plus géographiques, traités ensemble.
En effet, notre démarche s'inscrit dans une certaine critique du concept de territoire comme nous l'avons effleuré à la section \ref{ssub:la_diversit_sexuelle_dans_les_sciences_sociales_et_la_g_ographie}.
À nos yeux, celui-ci demande une part d'approfondissements.
Ces deux concepts, quoique semblables à première vue, prennent en géographie culturelle des sens différents.
Le concept d'espace d'abord englobe tout un point de vue rationnel sur les questions de distances entre les objets de ce monde et leurs positionnements.
Utilisé en mathématiques, en physique, et dans l'usage courant d'un point de vue pratique ou technique, l'espace se définit comme une construction intellectuelle volontairement neutre dont les propriétés sont quantifiables \citep[99]{DiMeo1998}.
On peut penser aux distances en kilomètres pour un voyageur entre deux villes, le calcul de la taille pour la construction d'une maison, ou encore du volume pour la quantification d'un liquide.
Mais nous allons le voir, ces calculs d'apparence banale semblent participer aujourd'hui à un point de vue particulier sur le monde qu'on décrira comme désenchanté.

L'utilisation d'un tel concept dans le cadre de la géographie culturelle peut de prime abord paraître contradictoire; en effet, ce champ de la géographie travaille  sur les populations et les territoires, selon leurs pratiques et leurs spécificités identitaires ou ethniques.
L'espace prendrait plutôt sa pertinence au sein de cette discipline dans le contexte de la géographie sociale par exemple, lorsque l'on s'intéresse aux enjeux propres aux déplacements.
On peut penser aussi au champ de la géographie physique ou de la biogéographie qui traitent bien souvent les substrats rocheux ou le monde du vivant comme faisant partie d'ensembles de données quantifiables.
Par contre, durant le dernier siècle, une partie de la géographie culturelle a porté son regard sur des populations des pays colonisés ou en voie de décolonisation en utilisant plutôt le concept de territoire.
Particulièrement, nous pensons à la géographie tropicale et son point de vue basé sur l'altérité entre les régions nordiques, dites normales, et les régions du sud~\citep[493]{Power2009} et également les travaux plus anciens en géographie régionale française s'intéressant strictement aux régions~\citep[31]{Courville1991}.
Ceux-ci travaillaient bien souvent sur des populations plutôt restreintes pensées comme des \emph{ethnies}.
Ces dernières semblaient à première vue vivre dans des milieux suffisamment isolés pour que les interactions interculturelles doivent être considérées comme quasi inexistantes.
De cette façon, cette géographie pouvait restreindre les caractéristiques de la population comme des sujets uniques.
Cette position nous apparait faussée, alors que l'on devrait plutôt comprendre que ces groupes sont limités artificiellement, sont objet à des interactions avec d'autres populations et évoluent,  comme le souligne~\citet[79--80]{DiMeo2007}.

Dans ce contexte, le territoire est perçu comme un plan de la réalité distinct se superposant à l'espace: il s'agirait d'un ensemble d'éléments abstraits et matériels permettant à la population d'ancrer son identité, son histoire comme son futur.
On retrouve cette définition chez plusieurs auteurs, notamment Bonnemaison,
Pour lui, le territoire:
\blockquote[{\cite[253]{Bonnemaison1981}}][]{[\ldots] n'est pas forcément clos, il n'est pas toujours un tissu spatial uni, il n'induit pas non plus un comportement nécessairement stable} comme pourrait l'être le concept d'espace, dans sa forme comme dans ses effets.
\citeauthor{DiMeo2007} nous offre également une définition du territoire, pour qui:
\blockquote[{\cite[76]{DiMeo2007}}][.]{L’assise territoriale, campée sur un réseau de lieux et d’objets géographiques, constitué en éléments patrimoniaux visibles, renforce l’image identitaire de toute collectivité. Elle lui dresse une scène et la pourvoit d’un contexte discursif de justification particulièrement efficace en ville où des lieux très denses, soigneusement et anciennement dénommés, s’inscrivent dans une totalité territoriale représentée, à la fois symbolique et fonctionnelle}. \todo{à approfondir}

Le désenchantement du monde tel que nommé précédemment est décrit dans les travaux de Max Weber comme le phénomène par lequel les explications reliées au mystère, la religion ou la superstition perdent leur place face à celles qu'offre la rationalité scientifique\citep[62]{Weber1963}.
On peut donc comprendre que c'est l'avancement scientifique et surtout sa méthode et ses découvertes qui affectera la société en mettant en danger la place de la religion et du mythe dans la société.
Enlevant une part importante du divin dans l'explication des phénomènes terrestres, par la théorie de l'évolution par exemple, le phénomène va prendre de l'ampleur par l'arrivée du capitalisme au \siecle{19}.
Par sa capacité à produire des marchandises en masse sans l'apport individuel de l'ouvrier --- sa subjectivité --- ce nouveau système social gérant le travail et les échanges rendra l'économie plus anonyme et aliénante.
Combinés, ces effets --- rationalité et aliénation --- éloigneront progressivement l'individu de sa capacité à expliquer le monde et à s'y conforter, perdant son pouvoir sur le matériel comme sur l'abstrait.

Ce développement affectera grandement les milieux de vie dans lesquels l'industrie s'implantera, principalement les villes.
Toujours plus demandantes en main-d’œuvre et offrant une quantité importante de marchandises à consommer, ces dernières prendront une taille de plus en plus conséquente alors que les quartiers évolueront pour nourrir cette industrie.
Des milieux villageois aux villes orientées vers les pouvoirs politiques, les sociétés se développeront désormais sur des nœuds urbains que l'urbanisme tentera de rationaliser par la poussée des champs architecturaux nouveaux, notamment par les travaux de Le Corbusier~\citep[170]{Pacione2009}.
De la naissance du mouvement fonctionnaliste en architecture, les bâtiments posséderont maintenant des fonctions et les villes seront pensées comme des machines dont l'efficacité doit être maximisée\citep[158]{Pacione2009}.
On peut souligner ici que l'avènement du capitalisme mettra en place un point de vue rationaliste de l'espace des villes, alors qu'on peut de moins en moins considérer celui-ci comme un territoire où les individus trouveront un sens à leur existence, mais plutôt, nous croyions, un espace à reterritorialiser.

Nous ne croyons pas qu'il s'agit ici d'un effet ayant pris emprise strictement en occident, alors que le capitalisme et d'une certaine façon la rationalité scientifique comme discours se sont étendus à grande échelle.
Néanmoins, malgré des effets sociaux très larges, nous considérons tout de même que les lieux touchés ont des particularités qui leur sont propres du point de vue de l'organisation spatiale.
La réponse sociale à ces effets se conjugue à des effets politiques particuliers, que ce soit le colonialisme, les régimes politiques en vigueur, etc.
Ainsi, nous reconnaissons l'importance de l'Histoire dans les processus sociaux régionaux ailleurs dans le monde.
Les différences spatiales contribueraient donc aux développements de différences socioculturelles et rendraient chaque groupe identitaire différent, quoique, avec l'avènement des communications de masse, le phénomène soit maintenant incertain.

Nous voulons montrer que dans les villes, l'espace prend une place particulière après ce que nous considérons comme un recul important de la territorialisation dans un sens traditionnel.
Après un développement certain des villes autour des industries, l'avènement d'un commerce mondial de plus en plus flexible et des entreprises toujours plus mobiles ont rendu les vieux centres urbains --- où s'étaient d'abord développées ces dernières --- moins intéressants.
Cette perte d'intérêt se retrouve autant pour les industries elles-mêmes, que pour les classes plus aisées bénéficiant du capital ou encore la classe moyenne en création.
Après avoir perdu en valeur, ces centres sont réhabités par les classes dominantes et rénovées selon les moyens des investisseurs accaparant ces espaces par le phénomène décrit par de nombreux chercheurs sous le terme de gentrification or d'embourgeoisement~\citep[voir][211--217]{Pacione2009}.

Néanmoins, nous n'assistons pas qu'à une simple \emph{désertification} des espaces dans lesquels les classes privilégiées prennent tous les espaces vacants et les individus appartenant aux classes inférieures sont laissés pour compte.
Plusieurs cas de reterritorialisation ont été identifiés dans la littérature, par une diversité d'acteurs, avec ou sans succès \citep[un exemple est le cas de Québec, où le quartier Saint-Roch est passé de quartier ouvrier  à un chantier de rénovation urbaine où se sont organisées plusieurs formes de résistances populaires. Voir][]{Hatvany2005}.
D'ailleurs, certains géographes ont relevé que cet espace présente tout de même des caractéristiques particulières.
Pour \citeauthor{Courville1991} notamment: \blockquote[{\cite[41]{Courville1991}}][.]{l'espace devient un médiateur du rapport entre individus, groupes et collectivités, un produit social à analyser comme tel, au milieu des pouvoirs et des rapports sociaux qui le structurent et l'organisent}.

C'est donc dire que cet espace urbain, selon son évolution, est maintenant le théâtre d'interactions multiples et enchevêtrées.
Des groupes comme les individus du spectre \lgbt{} ont vécu ces diverses interactions, en participant ou en subissant les phénomènes d’ embourgeoisement~\autocite{Podmore2001,Giraud2014,Hogan2005}.
D'autres ont plutôt participé à des luttes d'ordre économique ou politique sans qu'ils ne soient directement impliqués~\autocite{Kelliher2014}.

Leur présence dans ce jeu d'interactions offre à ces individus la possibilité de réaliser des rencontres qui mènent à une construction identitaire.
Comme souligné par~\citeauthor{DiMeo2007}: \blockquote[{\cite[81]{DiMeo2007}}][.]{\textelp{} la ville fournit un potentiel privilégié d’outils de recentrage pour toute identité individuelle. Par sa variété intrinsèque et par les innombrables repères sensibles et vécus qu’elle étale, par les \emph{affordances} (emphase de l'auteur) qu’elle sème dans le champ des perceptions individuelles, la ville file une trame dont ses habitants se servent sans restriction pour tisser et inventer leur propre identité}.

Ainsi, on arrive à relier d'une certaine façon les travaux de sociologie sur le sujet et l'identité tels que décrits à la Section~\ref{subsec:sujet_et_identité}.
Ces nouvelles identités, naissant en partie des luttes pour la reconnaissance et d'une certaine distance avec les identités nationales, trouvent également un lieu pour l'organisation dans les espaces urbains.

\section{Le géosymbole comme marqueur spatial}
\label{sec:le_symbole_comme_marqueur_spatial} Nous retiendrons, avec les quelques nuances soulevées dans la section précédente, le concept de territoire pour décrire l'espace investi par un groupe culturel particulier.
Pour établir un lien entre cette notion culturelle pour présenter l'espace et les façons dont les individus communiquent entre eux, en demeurant sensible aux questions identitaires, nous proposons l'usage du géosymbole.
Peu répandu dans la géographie anglo-saxonne, ce concept reste intéressant pour l'analyse des groupes culturels contemporains.

La culture n'est plus considérée en géographie culturelle comme un tout monolithique; il s'agit plutôt d'une multitude de visions du monde différentes ou divergentes.
Chez les premiers auteurs en géographie culturelle, la région, le paysage et le territoire ont été des concepts particulièrement importants  pour décrire les relations entre l'espace et la culture~\citep{Bonnemaison1981,Monnet1998,DiMeo1998,}.
Ceux-ci tendent par contre dans certains cas à proposer une vision incomplète de la culture, en mettant l'accent sur la vision de la culture dominante / hégémonique d'un espace~\citep[11-12]{Duncan1993}, ou en offrant une vision déformée des groupes culturels minoritaires ou marginalisés.

La culture occidentale par exemple n'est donc, en Amérique du Nord, qu'une des multiples cultures qui se partage l'espace et avec laquelle il y a négociation~\citep[11]{Duncan1993}.
Contrairement aux anciennes perspectives en géographie culturelle qui définissaient la culture comme une entité à part, homogène et où régnait une apparence de consensus~\citep{Duncan1980}, la réalité semble s'éloigner de cette image.
C'est surtout lorsque l'on s'intéresse aux faits politiques, socio-économiques et d'immigration que les sociétés apparaissent beaucoup plus diversifiées que ce qu'elles semblent l'être.
Il faut donc se distancer de cette perspective dite \enquote{superorganique}, selon \citet[198]{Duncan1980}, qui mène à la réification de la culture; on pourrait même dans ce cas-ci étendre cette précaution aux groupes culturels plus restreints, comme les membres de la diversité sexuelle.
En effet, depuis les dernières décennies, la diversité sexuelle engendre plusieurs nouvelles visions du monde par la résistance à l'hétérosexisme qu'on peut voir dans les différentes luttes, trans,  gaies et lesbiennes (et de plus en plus, bisexuelles et d'autres liées à d'autres identités).
Occupant une myriade d'espaces difficiles à situer précisément comparativement à la culture dominante, il apparait inapproprié d'utiliser les concepts spatiaux traditionnels propres à la géographie culturelle comme le territoire pour parvenir à reconnaître et étudier ces groupes.

%Il convient donc, pour l'étude géographique des minorités sexuelles, d'avoir recours à certains concepts moins rattachés à la matérialité de l'espace.
%\{Déjà formulé ainsi d'après Caroline}
En demeurant dans le champ de la géographie culturelle, nous proposons l'usage du géosymbole comme outil conceptuel pour situer les groupes culturels et comprendre leur relation avec l'espace.
En effet, les géosymboles consistent selon Bonnemaison en: \blockquote[{\cite[256]{Bonnemaison1981}}][.]{\textelp{} un lieu, un  itinéraire, une étendue qui, pour des raisons religieuses, politiques ou   culturelles prend aux yeux de certains peuples et groupes ethniques, une   dimension symbolique qui les conforte dans leur identité }.
En nous réappropriant cette définition dans le contexte des minorités sexuelles, nous pouvons affirmer que nous traitons ici d'un groupe culturel dont la formation s'apparente de certaines façons à un groupe ethnique~\citep{Sinfield1996}, mais sans nécessairement être attaché à territoire au sens traditionnel tel soulevé précédemment.
La sémiotique, soit l'étude plus large des symboles, apparait pertinente.
Nous considérons que les géosymboles sont une forme particulière de symboles à dimension spatiale et que selon les auteurs, ces derniers peuvent prendre plusieurs formes différentes, matérielles ou non tout en demeurant attachés à l'espace~\citep{Bonnemaison1981,Bedard2002}.
Nous soulevons donc ici une des limites conceptuelles du géosymbole décrit par Bonnemaison (1981): nous allons utiliser ce concept sans la facette territoriale, mais plutôt en se rattachant seulement à l'espace et à la temporalité de son occupation.

Ces géosymboles, bien que permettant de s'intéresser à la culture, n'effacent pas nécessairement les aspects sociaux d'un groupe.
Des concepts plus proches de la géographie sociale comme les classes sociales ou la racisation~\citep{Bonniol2005} demeurent essentiels à la compréhension des espaces gais et lesbiens~\citep[93]{Oswin2008}.
Ils pourraient permettre de comprendre la position de certains géosymboles et leur raison d'être, en montrant les clivages propres à un groupe culturel précis, comme le spectre \lgbt{}.

Par ailleurs, si dans l'imaginaire collectif, les \emph{gais} apparaissent comme un groupe homogène, la population \lgbt{} forme tout un spectre.
Sa compréhension demande la prise en compte de facteurs sociaux pour comprendre les variances et les divisions, le genre en premier lieu, mais également l'appartenance à une certaine classe sociale.
Cette classe peut être plus ou moins prompte à vouloir s'accorder ou rejeter les normes sociales hétérosexuelles et à générer des espaces particuliers ou à transformer certains espaces conçus comme normaux~\citep{Lewis2011}.
En effet, le contexte actuel dans la littérature pousse à reconnaître que l'espace en général facilite les relations hétérosexuelles au détriment des autres relations, et donc, des identités qui y sont rattachées~\citep{Brown2003}.
Il est ainsi important de s'intéresser aux différents espaces qui existent en marge ou en parallèle et de se questionner sur l'utilité de ces espaces pour les individus impliqués.

Alors que le territoire n'arrive pas à bien définir la spatialité des individus de la diversité sexuelle, le recours à la sémiotique et aux géosymboles pourrait permettre de voir émerger certaines formes spatiales.
On peut croire également que celle-ci offrirait la possibilité de montrer la diversité des lieux, leur position ne permettant pas une compréhension plus poussée de l'utilisation de l'espace par les communautés \lgbt{}.
À notre connaissance, aucun travail ne traite spécifiquement de la sémiotique en géographie queer et il s'agit d'une lacune par rapport au potentiel que contient cette approche dans l'étude de l'espace et au sein de la géographie culturelle.
L'étude des géosymboles permettrait d'atteindre un savoir difficile ou impossible à obtenir par une approche plus matérialiste s'arrêtant au territoire et qui ne prendrait pas en compte l'existence de réseaux dans l'espace.
D'emblée, nous pouvons déjà considérer que des événements  portent en eux des marques particulières, des logos ou des apparences qui entrent dans la définition du géosymbole.
À titre d'exemple, les Fiertés gaies, les manifestations politiques, les centres communautaires offrant des services aux individus porteurs du \vih, les bars lesbiens ou encore des lieux de dragues dans des espaces publics sont tous des d'espaces \lgbt{} pouvant utiliser des symboles pour marquer leur présence.

\blockquote[{\cite[108]{DiMeo1998}}][.]{\textelp{} le territoire multidimensionnel participe de trois ordres distincts. Il s'inscrit en premier lieu dans l'ordre de la matérialité, de la réalité concrète de cette terre d'où le terme tire son origine. Il relève en deuxième lieu de la psyché individuelle.
Sur ce plan, la territorialité s'identifie pour partie à un rapport a priori, émotionnel et présocial de l'homme à la terre. Il participe en troisième lieu de l'ordre des représentations collectives, sociales et culturelles. Elles lui confèrent tout son sens et se régénèrent, en retour, au contact de l'univers symbolique dont il fournit l'assise référentielle}.

\section{La diversité sexuelle en géographie}
\label{sec:la_diversit_sexuelle_en_g_ographie}
Les travaux arrimant l'ensemble de ces courants théoriques, l'étude des géosymboles avec un accent fort sur la sémiotique en relation avec des groupes dont l'identité se superpose à des ensembles culturels plus grands ne sont pas nombreux.
C'est encore plus vrai en ce qui concerne le cas plus spécifique des identités \lgbt{}.
Ceux-ci existent tout de même et il nous apparait important de nommer brièvement ceux qui existent dans le but précis de définir les angles morts où la recherche a lieu d'être.

L'ouvrage principal à rendre explicite l'usage des symboles par les communautés \lgbt{} sur lequel nous nous appuyons est \citetitle{Giraud2014} de \citet{Giraud2014} qui s'est plus particulièrement intéressé aux cas du Village gai de Montréal et du Marais de Paris selon une démarche comparative.
Plus près de notre champ d'intérêt, une section du livre s'est penchée sur les symboles utilisés par les communautés gaies des deux villes.
On y apprend comment cette visibilité a contribué à une certaine territorialisation des groupes gais.
Choisissant une perspective historique et centrée sur un seul groupe, les hommes homosexuels, la recherche reste muette sur les autres groupes \lgbt{} sachant que ceux-ci interagissaient avec ces hommes et partageaient certains espaces et une histoire interreliée~\citep{Remiggi2000,Demczuk1998,Podmore2001,Higgins1997,Higgins1999}.

Nous nous appuierons donc dans cette recherche sur une variété de travaux pour couvrir plus largement les communautés \lgbt{} québécoises.
Nous avons pensé notamment aux travaux de Julie Podmore, de Frank Remiggi et de Ross Higgins qui figurent parmi les auteurs principaux à traiter de la diversité sexuelle au Québec en géographie ou en anthropologie.
En général, ces travaux se sont limités aux identités et communautés gaies et lesbiennes; nous avons donc tenté, dans ce mémoire, d'aborder les géosymboles liés aux autres groupes.

Avec tout le corpus théorique et les travaux que nous avons nommés dans cette dernière sections, nous croyons avoir une base solide pour appuyer les résultats qui seront présentés au chapitre \ref{cha:de_la_clandestinite_a_l_espace_public} et à l'analyse de ceux-ci au chapitre \ref{cha:de_l_inclusion_la_diff_rence_par_le_symbole}.
Dans le prochain chapitre, nous nous pencherons sur la méthodologie.
Pour celui-ci, nous nous attarderons de nouveau à la sémiotique et aux géosymboles, et comment nous arriverons à faire la collecte et l'analyse de ceux-ci.
Nous reviendrons également sur l'aspect géographique de l'analyse par l'entremise de la cartographie à l'aide d'un \sig{} en plus de la littérature que nous abordée dans les dernières pages.

%\section{Vers une vision hétérogène de l'identité, de l'espace et de l'essence du symbole}
%\label{sec:vers_une_vision_h_t_rog_ne_de_l_identit_de_l_espace_et_de_l_essence_du_symbole}
%\{à revoir} \foreignblockquote{english}[{\cite[tel que cité
%dans][97]{Oswin2008}}][.]{As Elspeth Probyn has stated, sexual spaces \foreigntextquote{english}[{\citeyear[10]{Probyn1996}}][.]{are delineated through coincidence and not through exclusion}. Rather than clinging to the fiction that we can locate queer spaces that exist in coherent opposition to heterosexual spaces, we need to intensify examinations of what comes together in processes of sexualization}

%%% Local Variables:
%%% mode: latex
%%% TeX-master: "../../memoire-maitrise"
%%% End:
             % chapitre 1
%!TEX root = ../../memoire-maitrise.tex
\chapter{Méthodologie}
\label{cha:methodologie}

\chapterprecishere{\textquote{The growth of the pervert population of Brisbane, beautiful capital of Queensland, is astounding, and in the last year hundreds of these queer semi-feminine men have made the city their headquarters.
Now they have evolved into a cult, with two main sects, one on the north and the other on the south side of the town, with the river dividing them. 
And occasionally they meet at queer, indecent, degrading ceremonies when perverted lusts come into full play and shocking rituals are celebrated} \par\raggedleft--- \textup{The Arrow}, le 4 mars 1932}

Ce chapitre se penchera plus particulièrement sur la méthodologie et nous définirons le cadre de l'étude, soit le lieu et la durée dans laquelle a eu lieu la collecte de données. 

% La recherche se déroulera en plusieurs étapes : d'abord, faire
% un portrait de la création et de l'évolution des géosymboles \qus\ ou de la
% diversité sexuelle, dresser un bref portait historique des milieux de la
% diversité sexuelle au Québec et répertorier les géosymboles dans un
% échantillon varié de villes d'une taille minimale (selon les paramètres
% d'apparitions d'une communauté \lgbt).

% Le travail de recherche se terminera par une analyse de ces géosymboles selon
% leur usage et de la trame sémantique sous-jacente à leur déploiement dans
% l'espace, en tentant de comprendre la relation qui existe entre les
% communautés \lgbt{} et leur spatialité.

\section{Lieu d'étude}
\label{sec:lieu_d_tude}
L'espace couvert par cette étude comprend les milieux urbains fréquentés par et pour les individus ou groupes faisant partie du spectre de la diversité sexuelle, c'est-à-dire principalement les individus non-hétérosexuels ou dont l'identité de genre ne correspond pas à la norme sociale hétérosexuelle ou cisgenre\footnote{Terme inventé dans la foulée des luttes pour la reconnaissance des personnes trans et de leurs droits. 
Le terme cisgenre vise à remettre en  question la norme sociale sur l'assignation du genre à la naissance, pour  proposer plutôt que la concordance entre le genre et le sexe d'un individu est  une des possibilités d'identification de genre, comme le fait d'être trans (ou  non-binaire dans le genre, sans genre, etc., voir~\cite{Barker2015})~\citep[150]{McGeeney2015}.}. 
Cette définition large vise à rassembler en un groupe les personnes homosexuelles, lesbiennes, bisexuelles, trans, queer, ou encore en non-concordance avec le genre (bien que dans l'analyse, ces groupes seront analysés séparément ou conjointement selon le contexte). 
En considérant le temps alloué à la collecte de données, au contenu de certains matériaux d'archives et d'inégalités entres certaines identités dans l'ensemble de la société, nous n'avons pas couvert l'ensemble de ce spectre de façon égale. 
Nous reviendrons plus loin dans le chapitre sur les raisons de cette couverture inégale.

Plus spécifiquement, en concordance avec l'hypothèse émise dans le chapitre précédent, nous considérons que cet espace couvre de multiples lieux dispersés parmi les villes québécoises, à quelques expressions près, sachant que certains lieux en-dehors de la province sont publicisés dans les médias québecois. 
Ces espaces ont pris dans notre collecte de données la forme de bars, restaurants, commerces, rues, quartiers ou autres en correspondance avec les études déjà effectuées par d'autres chercheurs sur les pratiques d'organisation et de rencontre des minorités sexuelles~\citep{Higgins1999,Hinrichs2012}.

Selon la hiérarchie des villes québécoises, ce projet de recherche se penchera principalement sur deux groupes d'espaces en particulier, sois ceux des villes de Québec et Montréal. 
Celles-ci se démarquent d'une part par leur importance démographique, politique, sociale et culturelle et d'autre part par leur place dans le réseau urbain, ces deux villes occupants une position centrale. 
Nous nous attarderons de façon secondaire sur d'autres villes où il y a une communauté \lgbt{} organisant des activités publicisés, selon notre collecte de données. 
Nous nous pencherons plus particulièrement sur ces différentes villes plus loin dans ce chapitre à la Section~\ref{ssub:autres_villes}.

Ces diverses publicités référant aux activités des minorités sexuelles ont historiquement prises diverses formes, que ce soit sous la forme de feuillets ou de magazine culturistes~\citep{Higgins1999}~\footnote{On retrouve d'ailleurs ce   type de documents et plusieurs autres dans les \agq{}.}. 
Étant donné le cadre temporel de cette recherche, nous avons arrêté notre collecte de données sur les médias les plus importants en termes de distribution et de diversité de contenu, sois les revues \fugues{}, \sortie{} et les médias sociaux. 

Étant donné le cadre temporel dans lequel s’enchâsse ce mémoire, traité plus loin dans ce chapitre, nous avons basé l'analyse des lieux dans l'histoirique récent des communautés \lgbt{}, sachant tout de même qu'elle peut s'étendre à jusqu'aux débuts de la colonisation~\citep{Higgins1999}.
Nous tenterons tout de même de cerner brièvement le contexte récent des dernières décennies, selon les groupes analysés dans les chapitres subséquents et des informations disponibles dans la littérature. 
Ce cadre historique s'intéressera donc principalement au cas de Montréal étant donné le peu d'écrits existants dans le domaine scientifiques sur les autres villes québécoises. 
Un travail en archives plus poussé serait à faire pour celles-ci mais ceci est au-delà du champs couvert par ce mémoire.
\todo{Peut-être à revoir, pas clair selon Caroline}

\begin{figure}[ht]
	\begin{center}
		\includegraphics[width=18cm]{fig2.png}
	\end{center}
	\caption{Villes pour lesquelles des géosymboles ont été localisés à partir des
    magazines Sortie et Fugues.}\label{fig:carte_quebec}
\end{figure}

\subsubsection{La ville de Montréal}
\label{ssub:montreal}
Anciennement nommée Hochelaga puis Ville-Marie, Montréal est aujourd'hui la métropole de la province de Québec et la deuxième ville en taille au Canada après avoir été la première durant plusieurs décennies. 
\note{À quel point devrais-je décrire l'historique de la ville de Montréal? à compléter}


Historiquement, dans la métropole, nous pouvons croire qu'il exista pour les minorités sexuelles une forme d'organisation durant pratiquement toute l'histoire de la colonie canadienne-française jusqu’aux années 1950, organisation marquée avant tout par la clandestinité et l'invisibilité~\citep{Higgins1999}. 
Par son statut de métropole très tôt dans l'Histoire, Montréal rayonnait suffisamment pour qu'une partie des individus des minorités sexuelles connaissent ou côtoient d'autres individus de la ville, cette probabilité grandissant avec l'amélioration des moyens de communication et l'urbanisation grandissante des lieux centraux. 
Peu de données par contre permettent réellement de décrire avec précision les contours de ces rassemblements ou de ces rencontres; bien souvent, la connaissance que nous avons de la vie de ces personnes se résume aux faits divers qu'on retrouve dans les \emph{pages jaunes} de l'époque, des médias imprimés spécialisés dans les informations à sensation. 
\todo{Trouver la citation de Higgins sur les gens qui   trouvaient les lieux LGBT rapidement}\citep[]{Higgins1999}. 
On sait par exemple que certains cinémas, comme le Midway dans les années 1920, aurait été un lieu de rencontres entre hommes\citep[30]{Higgins1999} \todo{à compléter}.
\begin{figure}[ht]
	\centering
	\includegraphics[width=15cm]{carto/mtl.png}
	\caption{Arrondissements ciblés pour la collecte de données: ville de
    Montréal}\label{fig:espaces_montreal}
\end{figure}
Aujourd'hui, Montréal est une métropole bien établie à l'échelle du continent nord-américain. 
Possédant de nombreuses institutions post-secondaires, dont quatre université, elle attire en son sein une population immigrante importante et est réputée à certains égards pour l'ouverture qu'elle fait montre envers la communauté \lgbt{}. 
\todo{à terminer}Le Village gai, un des plus développés au monde, l'existence de festivals comme le \anglais{Black \& Blue}, les \anglais{Outgames}, la Fierté, etc.\@sont reconnus.

Les arrondissements du Plateau Mont-Royal et de Ville-Marie ont d'abord été ciblés pour la présence reconnue de plusieurs lieux \qus{}. 
D'abord, le lieu occupé actuellement par le quartier des spectacles a été il y a plusieurs décennies le \anglais{Red Light} du centre-ville et plusieurs cabarets ont été les premiers lieux de rassemblement d'individus \lgbt{}~\citep[198]{Podmore2015}.
Pas vraiment un lieu de communauté et de sécurité, le red light et ses cabarets servait plutôt de lieu de travail pour plusieurs travailleurs du sexes dont la sécurité était minée par la criminalité et une hétérogénéité de la clientèle qui ne protégeait pas des violences homophobes et transphobes~\parencite[91]{Higgins1999}. 
De cet espace, une partie de la \textquote{communauté} --- d'abord des investisseurs puis la clientèle --- s'est dirigée vers le quartier sud dans le même arrondissement pour créer ce qui allait devenir le Village gai. 
Une caractéristique intéressante est la proximité de cet espace d'institutions importantes comme l'\uqam{} et l'Université McGill; ces universités seront occupées durant ces mêmes années par divers groupes étudiants et politiques qui mèneront en partie plusieurs des luttes politiques pour la reconnaissance des droits et de la visibilité des gays et des lesbiennes.

Une autre partie de la clientèle \lgbt{} du \anglais{Red Light} s'est dirigée plus au nord, principalement les femmes. 
Nous savons aujourd'hui que plusieurs lesbiennes ont investi pendant près de deux décennies le Plateau Mont-Royal, principalement autour du boulevard Saint-Laurent~\citep[599]{Podmore2006} étant donné la présence de nombreux bars réservés aux femmes qui existèrent durant les années 80--90. 
Par contre, leur présence aujourd'hui s'est amoindrie; la gentrification et une certaine compétition avec le Village --- dont certains bars sont devenus mixtes et ont réussi à attirer une nouvelle clientèle lesbienne plus jeune --- en serait en partie la cause~\citep{Podmore2015}.

La figure~\ref{fig:espaces_montreal}\footnote{Afin de faciliter la lecture de la   vue d'ensemble de la ville de Montréal et pour ne pas encombrer la liste des   acronymes au début de ce mémoire, les noms complets des arrondissements de   Montréal se trouvent en annexes.\todo{à faire}} montre la position de ces deux arrondissements et nous verrons dans les chapitres suivants où ces espaces ont été situés en complément des données accumulées dans cette recherche \todo{Approfondir le contexte historique?}

\subsubsection{La ville de Québec}
Capitale de la province de Québec, la ville de Québec est également considérée comme la plus ancienne ville fondée par les Européens lors de la colonisation du continent américain. 
Anciennement, cet espace situé au nord de la rive du Saint-Laurent était connu sous le nom de Stadaconé et était un établissement iroquoiens~\citep[91]{Dickason1996}. 
La ville de Québec en soit, fondée par des colonisateurs français, devient une possession britannique durant le \siecle{18} tout en demeurant un lieu essentiellement habité par des canadiens-français, avec une minorité anglophone protestante dans la partie de la ville nommée la haute-ville de Québec\missref{}.

Si la ville a longtemps été limitée aux quartiers centraux, aujourd'hui, la ville de Québec possède une diversité importante de quartiers différents nés de la fusion d'anciennes municipalités et d'une croissance importante à partir du milieu du \siecle{20}. 
Un des legs de ces fusions de territoires a été le déplacement d'institutions importantes, comme l'Université Laval dans le quartier Sainte-Foy; Un des quartier les plus centraux et ancien, le quartier Saint-Jean-Baptiste, semble avoir été reconnu comme l'espace de vie des minorités sexuelles dans la capitale selon certaines entrevues avec des individus de la communauté gaie actuelle~\citep{CSJB2011}. 
Anciennement ouvrier, il s'agit aujourd'hui d'un quartier morcelé par la rénovation urbaine, la patrimonialisation et par un processus de gentrification~\citep{Hatvany2005,Mercier2014}.

Un espace important occupé par la communauté \lgbt{} a été le quartier Saint-Roch qui a également été habité et fréquenté par les minorités sexuelles.
Autre espace morcelé, on le considère aujourd'hui comme le centre-ville de la ville de Québec. 
Plus avancé dans son processus de gentrification, de nombreux espaces \lgbt{} comme des bars ont existés sur ses rues importantes, principalement la rue de la Couronne et aux environs. 
Aujourd'hui, ses fonctions ont changées et on y retrouve principalement des locaux loués par les organismes communautaires appartenant à la communauté \lgbt{} de Québec.

À notre connaissance, pratiquement aucune recherches ne se sont encore vraiment penchées à la population \lgbt{} de la capitale. 
Nous avons par contre écris précédemment un mémoire de baccalauréat qui s'est intéressé à cette question et qui a permis de dénombrer plusieurs lieux où la communauté serait active, mais nous ne possédons pas d'informations précise sur l'histoire de ces lieux, un exercice qui serait important à faire dans le futur. 
Si certains bars de la communauté semblent avoir disparus du paysage de la ville de Québec dans les dernières années dans les espaces décrits précédemment, de nombreux espaces fréquentés par les communautés \lgbt{} existent encore aujourd'hui. 
Certains sont maintenant présents en périphérie du centre-ville avec la dispersion importante des Université de Québec --- L'Université Laval et l'\uqar{} (qui est plutôt à Lévis) --- et des cégeps. 
Ils n'ont donc pas fait l'objet d'une collecte de données sur le terrain.

Pour l'instant, au su des informations précédentes, nous avons décidé de cibler nos efforts sur le centre-ville de Québec, dans l'arrondissement La-Cité-Limoilou, visible à la figure~\ref{fig:espaces_quebec}. 
Il s'agit de l'espace dans la ville de Québec où l'on retrouve le plus de lieux appartenant à la communauté \lgbt{} locale tout en étant l'espace choisi par l'organisation Alliance Arc-en-ciel de Québec pour les célébrations de la fête arc-en-ciel, le pendant local des fierté gaies.

\label{ssub:la_ville_de_quebec}
\begin{figure}[ht]
	\centering
	\includegraphics[width=15cm]{carto/qc.png}
	\caption{Arrondissement ciblé pour la collecte de données: ville de
    Québec}\label{fig:espaces_quebec}
\end{figure}


\subsubsection{Autres villes}
\label{ssub:autres_villes}
D'autres villes ont été envisagées pour ce travail de recherche. 
L'expérience terrain a par contre été limitée aux deux espaces décrits précédemment comme mentionné au départ de ce chapitre. 
Néanmoins, comme souligné précédemment, des données ont été trouvées dans plusieurs autres villes du Québec. 
Nous pensions déjà avant la collecte de données qu'il serait possible de traiter des villes de Rimouski et de Sherbrooke selon nos prores connaissances. 
Plus précisément, nous savions déjà à titre de membre de la communauté \lgbt{} que ces villes possèdent respectivement une communauté plus ou moins active. 
Nous envisagions donc de vérifier si ces villes sont des cas d'exception ou si celles-ci possèdent de par leur taille et leur place dans un réseau plus large de villes les caractéristiques nécessaires à l'apparition d'une communauté \lgbt.
L'usage de certains médias de la communauté que nous traiterons à la section \ref{sec:source_des_donn_es} nous ont effectivement permis de confirmer cette intuitions; nous avons dressé à la figure~\ref{fig:carte_quebec}. 
Selon les résultats de notre analyse de données présentée dans les prochains chapitre, nous arriverons en toute fin à dresser une hiérarchie des villes selon la complexité et la visibilité de leur communauté respective.

\section{Cadre temporel}
\label{sec:cadre_temporel}
Au niveau temporel, nous abordons l'époque contemporaine en couvrant le \siecle{20} au maximum, sachant que le sujet d'étude est particulièrement récent et que la majeure partie des données proviendront du dernier demi-siècle. 
En effet, la mise en contexte particulière du sujet nécessite une prise en considération de l'évolution historique des communautés formées par les minorités sexuelles. 
En effet, selon les circonstances historiques décrites dans la littérature~\citep{Spencer2005}, on peut estimer que les géosymboles de la diversité sexuelle actuelle dateraient au maximum des luttes ayant suivies les émeutes de Stonewall aux États-Unis mais que ces communautés ont existé plusieurs décennies avant de porter un discours politiques et engendré un mouvement civique de grande ampleur.

Par contre, en ce qui concerne les données collectées pour répondre à la question de recherche, la couverture temporelle est beaucoup plus courte et récente: nous couvrons les dix dernières années pour arriver à dresser un portrait actuel des géosymboles des communautés \lgbt{}. 
Les données étant particulièrement variées dans leur provenance, certaines ont été prises durant les mois précédents la rédaction du présent mémoire, alors que d'autres proviennent d'archives conservées et couvrant toute cette décennie. 
Nous décrirons plus en profondeur la couverture temporelle de ces données dans les Section~\ref{sec:source_des_donn_es} traitant des sources de données.

\section{Approche méthodologique}
\label{sec:approche_m_thodologique}
La découverte et l'analyse des géosymboles d'un groupe culturel donné ne s'appuie pas sur une méthodologie particulière; au contraire, ils apparaissent suite à une observation approfondie du groupe étudiée et de sa relation particulière avec le territoire. 
Nous croyons par contre que certaines approches méthodologiques sont plus appropriées selon les contextes de recherche et les groupes étudiés.

La question de la visibilité et de la présence en plus de l'acceptabilité sociale sont récurrentes dans l'histoire récente des minorités sexuelles en occident. 
Nous croyons donc que mettre de l'avant cette particularité culturelle devrait être un des motifs derrière le choix de l'approche méthodologique adoptée dans ce travail. 
La géographie visuelle semble ici la réponse à cette préoccupation. 
Mettant de l'avant les documents visuels comme matériel d'analyse, elle s'inscrit dans le champs plus large de l'analyse qualitative.

Contrairement à d'autres travaux en géographie culturelle, cette recherche va s'appuyer principalement sur l'observation et la recherche dans des archives plutôt que des entrevues avec des individus impliqués dans le sujet de recherche. 
Dans cette section, nous nous intéresserons à cette méthode alternatives qui devraient nous permettre de faire ressortir les différents géosymboles qu'on conçoit peupler les espaces urbains.

\subsection{Analyse visuelle}
\label{sub:analyse_visuelle}
L'analyse visuelle des territoires est profondément ancrée dans la discipline de la géographie. 
En effet, pour plusieurs penseurs\missref{}, la géographie, comparativement à d'autres disciplines en sciences humaines, demandent du chercheur qu'il se déplace sur son terrain d'étude pour pouvoir se l'approprier visuellement et arriver à en faire une analyse juste. 
Si elles ne sont pas toujours présentent dans les travaux des chercheurs, nombreuses sont les études de cas à intégrer des photographies des espaces étudiés, que ce soit en géographie physique où l'image peut servir à montrer au lecteur les différentes composantes du sous-sol ou pour la géographie humaine, à montrer un paysage ou une organisation spatiale humaine particulière. 
Également, un des outils de prédilection de la géographie est la carte pour la présentation de données, de plus en plus remplacée --- ou améliorée --- par les \sig{} qui remplissent cette fonction en intégrant des éléments d'analyses alimentés par des algorithmes. 
Par contre, si le visuel est aussi important, peu de travaux utilisent la photographie pour une raison autre que la démonstration~\citep[151]{Rose2008}.
Bien que la description peut servir des buts pertinents, comme la démonstration de l'évolution d'un espace dans le temps ou encore pour appuyer un argument, d'autres usages existent\parencite[158]{Rose2008}. 
Nous débuterons dont cette partie de l'approche méthodologique par l'utilisation de l'analyse visuelle dans notre recherche. 
La section suivante se penchera plus particulièrement sur l'usage des \sig{} dans la géographie culturelle.

Une des volontés derrière cette recherche est de poursuivre l'utilisation et l'expérimentation des méthodes visuelles entâmées par d'autres chercheurs durant la dernière décennie. 
Au-delà d'un simple renouement avec une pratique traditionnelle, nous considérons qu'il s'agit d'une méthodologie qui a le potentiel de faire le pont avec la théorie géographique, plus particulièrement en géographie culturelle vers une des composantes importantes des groupes culturels minoritaires ou marginalisées, la visibilité. 
De plus, les géosymboles que nous avons traité dans le dernier chapitres ont comme caractéristique d'être des symboles visuels, matériels ou immatériels, qui permettent d'articuler un territoire propre à groupe donné. 
Ainsi, les géosymboles jouent en quelque sorte le rôle des images de l'analyse visuelle. 
\citeauthor{Rose2012} caractérise d'ailleurs les images et les pratiques visuelles d'une façon similaire à la définition que nous avons des géosymboles, à ce savoir que : \foreignblockquote{english}[{\cite[32]{Rose2012}}][.]{\textelp{} the spaces and   practices of display \textelp{are} especially important to bear in mind given   the increasing mobility of images now; images appear and reappear in all sorts   of places, and those places, with their particular ways of spectating, mediate   the visual effects of those images}.
Repérer ces géosymboles pourrait se faire en travaillant directement avec les populations données, par l'observation et l'entrevue par exemple, méthode prisée dans la plupart des travaux déjà effectués \citep[][pour ne citer que ceux-ci]{Giraud2014, Podmore2015a, Higgins1999}. 
Mais nous pensons que le processus d'analyse du territoire d'un groupe peut se faire du point de vue inverse, en s'intéressant d'abord aux géosymboles que l'on retrouve préalablement dans un territoire et faire le pont entre ceux-ci et les travaux déjà effectués sur l'histoire, la politique ou la sociologie et surtout la géographie \lgbt. 
Autrement dit, nous envisageons aborder directement le territoire tel qu'il se présente matériellement et dans les médias \lgbt{} et utiliser les méthodes visuelles pour approcher les géosymboles, le tout en nous appuyant sur des technologies comme les applications cellulaires et les \sig.

% La géographie culturelle continua son existence durant ces mêmes décennies
% selon plusieurs sous-disciplines, comme le tropicalisme ou les études plus
% régionales comme au Québec.

Une auteur importante à avoir travaillé sur le domaine de l'analyse visuelle est Gillian~\citet{Rose2008} dont les travaux détaillent plusieurs façons dont les géographes utilisent les images dans leurs analyses. 
En plus de l'usage de description décrit précédemment, les images pourraient également être utilisées comme un outil de représentation, d'évocation ou encore servir de fragment de culture matérielle. 
Ce dernier point est également soutenu par \citet{Frosh2001} pour qui l'image, ou plus précisément la photographie est un fragment d'une performance sociale de représentation qui mérite analyse. 
Son pouvoir particulier de représenter des pans de la réalité donne au photographe un pouvoir particulier sur le sujet ainsi que sur l'observateur de la photographie.
Il apparait donc nécessaire dans notre recherche de prendre en compte cette réalité de deux façons : d'abord, reconnaître que les documents visuels analysés tirés des médias sociaux et des archives sont le fait d'individus possédant une certaine vision de la communauté /lgbt{} ou du moins, cherchent à la présenter d'une certaine façon, consciemment ou non. 
Ensuite, utilisant la photographie comme outil pour situer nous-mêmes certains géosymboles, nous avons nous-mêmes un biais envers l'objet étudié, malgré notre volonté de demeurer objectif, surtout dans un contexte où le sujet est social et culturel. 
Nous ne croyons pas par contre qu'il s'agit d'une faiblesse de notre perspective méthodologique mais plutôt une des particularités des méthodes qualitatives. 
Nous reconnaissons toutefois que dans des recherches subséquentes, il serait également intéressant d'utiliser des photographies produites consciemment dans le cadre d'une recherche mais par les individus appartenant au groupe culturel visé. 
Cette méthode a également fait ses preuves, notamment dans les travaux de \citet{Kwan2008},~\citet{Moore2008} et de~\citet{Markwell2000}.

En ce qui concerne la collecte de données sur le terrain, notre méthode se rapproche plutôt de l'article de~\citet{Leroy2010} dans lequel la photographie est utilisée pour montrer la diversité des représentations présentes dans la \anglais{Gay pride} parisienne. 
Si celui-ci ne décrit pas particulièrement comment l'échantillonage a été créé, nous croyons qu'il est nécessaire ici de faire cet exercice, tel que souligné par \citet[109]{Rose2012} qui constate que trop souvent, les travaux en sémiotique ne s'attardent pas suffisamment sur le contenu sélectionné, mettant l'emphase que sur le contenu intéressant à analyser.


% \todo{Parler de Suchar} La méthode de collecte de données s'inspire
% essentiellement de la technique

% \citep{Rose2012} \citep{Rose2008} \citep{Rose2003} \citep{Dorrian2003}
% \citep{Suchar1997} \citep{Frosh2006} \citep{Frosh2001}

L'analyse des données visuelles --- principalement celles qui ont été collectées dans les données d'archives --- va s'inspirer de différentes questions amenées par \citet[157]{Rose2008}, à savoir:
\begin{itemize}
	\item Qui utilise ces photographies, comment et pourquoi?
	\item Est-ce que l'usage de ces photographies à un effet particulier (sur les
    sujets, les auditeurs, etc.)?
	\item Où ces photographies ont-elles été prises?
	\item Est-ce que la localisation du sujet de la photographie est à prendre en
    compte? Comment?
	\item Et enfin, quel est l'impact des photographies sur les lieux où elles ont
    été prises mais également sur les lieux où elles sont utilisées/diffusées?
\end{itemize}

Ainsi, nous croyons que les différentes réponses apportées pourront ensemble dresser un portrait de la territorialité des groupes \lgbt{} du point de vue de la visibilité. 
Dans la section suivante, nous verrons comment ces imaginaires visuels pourront être situés dans l'espace, apportant ainsi des nuances ou des approfondissements aux différents portraits que nous dresserons de ces groupes.

\subsection{Géolocalisation du géosymbole}
\label{sub:g_olocalisation_du_sybole}
Ce n'est pas toutes les branches de la géographie qui s'attardent ou qui se sont attardées aux données visuelles, surtout depuis l'avènement de la géographie quantitative, constaté durant la deuxième moitié du \siecle{20} et du bond technologique apporté par l'informatisation. 
Le travail à partir de bases de données statistiques et géoréférencées permirent également aux chercheurs de prendre un certain recul vis-à-vis le sujet d'analyse et du même coup prendre une distance avec la partie terrain de la collecte de données. 
On doit tout de même souligner que si ces nouveaux outils informatiques permirent de couvrir des ensembles spatiaux beaucoup plus importants que précédemment, comme on peut le constater dans la branche de la géographie utilisant les méthodes d'analyses spatiales.\note{à garder ou impertinent?}

L'utilisation des \sig{} en géographie culturelle n'est par contre pas encore très répandue; on les retrouvent par contre fréquemment utilisés dans le cadre des analyses quantitatives en géographie urbaine, économique et dans les disciplines affiliées comme l'économie \missref{}. 
Une des volontés derrière cette recherche est d'arriver méthodologiquement à faire un pont entre ces techniques propres à la géographie quantitative et celles plutôt utilisées en géographie culturelles dans lesquelles la carte comme résultat d'un \sig{} joue plus souvent le rôle de description d'un espace à analyser. 
Nous croyons comme plusieurs autres auteurs \citep[4]{Elwood2009} que les \sig{} peuvent être un outil pratique à la compilation de données de sources multiples tout en permettant une localisation souvent très précise, surtout lorsque le chercheur travaillant à la collecte a accès à des adresses postales ou encore à un \gps.
Ce but est depuis peu partagé par d'autres chercheurs en géographie culturelle et en géosciences~\citep{Perkins2003,Elwood2011,Elwood2009,Kwan2008,Madden2009,Knigge2006,Jung2010}.


Comme le souligne~\cite{Kwan2008}: \foreignblockquote{english}[{\citeyear[444]{Kwan2008}}][.]{GIS-based data   analysis, mapping, and visualization are deployed to complement ortriangulate   (i.e., verify results using multiple data sources) the knowledge acquired   throughthe qualitative component of the research} 
Il devient ainsi possible d'ajouter certaines couches d'informations à d'autres dans un but d'enrichissement, en ajoutant une photographie à un lieu géolocalisé sur un \sig{} ou encore positionner une photographie dans l'espace et pouvoir comparer la position de chaque photographie. 
Dans un contexte multimédia, cette caractéristique peut s'appliquer à plusieurs types de médias, comme la photographie ou l'audio, malgré que ce type de données s'intègre mal à l'écriture dans le contexte de la rédaction d'un mémoire ou d'un article scientifique.

En plus de ces caractéristiques \citet{Elwood2011} s'est attardée à la définition de la géovisualisation comme méthode qualitative. 
Dans un but de comparaison dépassant le simple type de données traitées, Elwood souligne que contrairement à l'utilisation classique des \sig{} dans un contexte quantitatif: \foreignblockquote{english}[{\cite{Elwood2011}}][.]{\textelp{} what defines   these approaches as qualitative geovisualization is not absence of numeracy.
  Rather, it is their integration of multiple modes of representation –--  visual,textual, numerical --– and iterative interpretive analysis of these representations to tease out what they reveal about social and material situations. 
Most of these qualitative geovisualization methods emerge from qualitative GIS, but could clearly be applied to georeferenced multimedia drawn from the geoweb}. 
Notre recherche comprendra elle-même une multitude de médias visuels d'origine différentes, dont des données en ligne que l'on pourrait assimilé au \emph{géoweb} comme souligné dans cette citation d'\citeauthor{Elwood2011}. 
L'avantage ici des \sig{} sera de pouvoir incorporer ces images et en même temps de les situer pour mieux rendre compte de la dispersion de ces représentations et en même temps de simplifier la tâche de garder en mémoire la localisation des symboles trouvés par nous-même sur le terrain.

Suite à l'observation des espaces urbains ciblés et du travail en archives, l'ensemble des données on subit un traitement de géolocalisation pour la plupart et de codage pour la totalité. 
Nous avons décidé de ne pas effectuer de géoréférencement précis pour les symboles accumulés à partir des médias imprimés archivés par manque de qualité dans certaines données; dans certains cas, des adresses manquantes dans les symboles et des fermetures/dissolutions au fil des années nous ont empêché de retracer la localisation de chacun des espaces. 
De plus, certains symboles ne concernaient pas nécessairement des espaces spécifiques mais étaient des messages lancés à la communauté par des organisations hors communauté ou dont la localisation n'était pas, ou ne semblait pas \latin{a priori} pertinente. 
Nous pouvons penser par exemple aux messages publiés pas certains syndicats ou palliers de gouvernements nationaux.

Les données que nous avons donc géoréférencées consistent donc en les photographies prises durant les événements \lgbt{} auxquels nous avons participé et les données collectées sur les réseaux sociaux. 
Le géoréférencement s'est déroulé en plusieurs étapes et à l'aide de plusieurs outils et services différents. 
D'abord, à l'aide principalement du logiciel GIS Cloud et d'une façon secondaire Google Photo, nous avons accumulé des photographies des géosymboles dont le géoréférencement a été effectué à l'aide du \gps{} intégré au cellulaire utilisé pour la collecte de données. 
GIS Cloud était la solution retenue au départ et devait servir tout au long de la collecte, de façon exclusive. 
Par contre, l'usage de cette solution s'est avérée moins concluante qu'envisagé au départ. 
D'abord, il faut savoir que, tel que spécifié sur la page d'accueil de leur site internet, GIS Cloud consiste en un service de collecte sur le terrain et d’emmagasinage de données, en plus d'être utilisé pour la publication \citep{Cloud2014}. 
Plus spécifiquement, un des avantages de Cloud GIS pour la collecte de données sur le terrain dans le cadre d'une recherche utilisant des méthodes qualitatives est de permettre la construction d'un guide d'entretien similaire à celui utilisé par exemple dans le contexte d'entretien enregistrés. 
Avant la collecte, il est en effet possible pour le chercheur de construire des questions telles quelles seront utilisées pour la collecte.
Celles-ci peuvent être par exemple des questions fermées ou des questions ouvertes, ainsi que des questions qui sont en fait des objets capturés par le périphérique utilisé. 
Ainsi, dans le contexte de notre recherche, nous avons monté un questionnaire comprenant plusieurs questions sur le contexte de chaque point localisé, visible en annexes à la figure~\ref{ann:cloudgis}.

Au niveau pratique, lors du terrain, le logiciel s'est montré gourmand en ressources (avec pour conséquence une faible anatomie de l'appareil) et nécessitant une attention parfois difficilement conciliable avec le déroulement de l'activité en cours. 
Par exemple, dans les cas où nous avons eu à faire l'observation d'espaces comme le Village gai, nous avions tout le temps disponible pour nous arrêter devant un géosymbole potentiel, l'analyser, prendre des notes et faire un bon usage du questionnaire. 
Par contre, lors de manifestations par exemple, où plusieurs individus tenaient des pancartes, un discours était prononcé et que les individus en-dehors de l'événement réagissait, il devenait difficile de tout prendre en note à l'aide du seul logiciel. 
Par adaptation, il a été nécessaire de modifier notre usage de l'application pour poursuivre la collecte de données de façon efficace. 
D'abord nous avons décidé de nous servir de l'application Cloud GIS seulement à une reprise à chaque partie spécifique d'un événement pour marquer un point dans la base de données. 
Par la suite, il devenait possible de prendre des photos normalement en-dehors de l'application et de collecter des notes de terrain par nos propres moyens en tenant de l'heure et du contenu des notes et du géoréférencement des données présentes dans l'application cellulaire. 
Cette méthode s'est avérée plus efficace sachant qu'il était maintenant possible de faire un usage prolongé du cellulaire pour la collecte de données sans avoir à constamment synchroniser nos données collectées avec une base de données.

Lors du traitement des photographies, nous avons remarqué que l'ensemble des photographies prises sur cellulaire étaient déjà géoréférencées par défaut par le système d'exploitation de l'appareil, ce qui nous permit d'accélérer une partie du processus de localisation et d'augmenter la fiabilité du positionnement des données récoltées. 
Au final, grâce à cette fonction, nous avons obtenu un résultat similaire à ce qui avait été prévu au départ, c'est-à-dire le géoréférencement automatique des données. 
\todo{faire attention,   peut-être fusionner cette partie avec la section plus loin traitant des données, vu qu'une partie des informations se recoupe}

En ce qui concerne les données d'archives, le géoréférencement a été fait de façon manuelle à partir des adresses postales. 
Nous nous doutions que nous n'aurions pas toujours accès aux coordonnées des lieux ciblés dans la collecte de données mais nous croyions alors qu'il devrait être possible d'obtenir celles-ci à l'aide de certaines sources de données, comme les répertoires de Fugues ou encore Google Maps qui réussit normalement à garder en mémoire les lieux fermés mais possédant préalablement une entrée dans leur base de données à l'époque de leur fonctionnement. 
En effet, étant donné la couverture assez large de collecte de données sur dix années, certains établissements sont apparus et disparus, aujourd'hui remplacés par d'autres du même acabit ou d'espaces totalement différents. 
En fait, si certains géosymboles, comme les publicités, risquaient d'avoir des adresses intégrées, sachant qu'une publicité incite normalement le client potentiel à se déplacer sur les lieux du commerce ou du service, d'autres, comme des photos toutes simples, devaient être situées à l'aide du contenu qu'il supporte.  
Dans le cas de documents comme le Fugues, le contenu visuel est pratiquement toujours accompagné de textes comme des articles ou du moins des titres pouvant fournir de genre de données pour la localisation, mais en général nous disposions peu de données réellement utiles à la géolocalisation des données. 
À l'aide des répertoire qui sont compris dans ce média, nous avons tout de même pu situer géographiquement une certaine partie de ces documents.

Dans le même cadre d'idée, la participation à des activités de la communauté \lgbt{} lors des différents événements qui se sont déroulés durant le terrain ont permis, durant la planification, d'accéder à des données supplémentaires sur les réseaux sociaux. 
Nous savions préalablement à titre individuel que de nombreux événements sociaux sont organisés aujourd'hui à partir de certains sites web comprenant de telles fonctions. 
Le réseau retenu pour l'analyse est Facebook qui permet l'organisation d'événements et de leur publication auprès d'un grand nombre d'individus. 
Ces événements permettent de situer les différents événements, autant dans le lieu que dans l'espace tout en étant une plateforme pour publier des images porteuses de géosymboles.
\citep{Barkhuus2010} \citep{Boyd2010}

Au final, nous avons réussi à géoréférencer une majeure partie de ces données variées en un seul \sig{} et ainsi permettre d'offrir une lecture visuelle des multiples espaces \qus{} des espaces urbains.

\section{Source des données}
\label{sec:source_des_donn_es}
Tel qu'énoncé dans la section précédente, nous avons eu recours à diverses sources de données. 
Celles-ci ont été envisagées et retenues dans le but d'arriver à couvrir l'ensemble du spectre des minorités sexuelles, selon le niveau d'activité de chaque groupe ou sous-groupe tout en accumulant selon les nécessités et la découverte de ces sources de données d'autres données. 
Notre processus de collecte de données s'apparente d'une certaine façon aux méthodes de la théorie ancrée, dans lesquelles les chercheurs alimentent d'abord leur travail de théorisation a partir de premières données collectées pour ensuite continuer et améliorer la collecte selon les prémisses soulevées par une première analyse de ces données. 
C'est d'ailleurs cette méthode qui nous a poussé a inclure certaines données non-prévues au départ de cette recherche, comme les données trouvées sur les réseaux sociaux. 
Également, choisir dès le départ les sources de données demeurait un choix ardu selon les modalités de notre recherche. 
Couvrir l'ensemble du spectre \lgbt{} ne pouvait se résumer à notre avis à un seul média étant donné le risque de biais dans notre recherche.

Ceux-ci étant en général considérées comme marginalisées, certains de ces le sont plus que d'autres et des enjeux de pouvoir particulier existent entre elles, alors que certaines minorités ont obtenu une plus grande sympathie au sein de la société et disposent ainsi de moyens communicationnels bien différents que d'autres minorités. 
Ainsi, la diversité de média permet d'éviter certains biais que l'on croit exister au sein de certains médias ou certaines manifestations de la présence d'identités particulières. 
Il ne s'agit pas ici de faire une critique du public visé de certains médias, au contraire: certaines manifestations géosymboliques sont le fruits de certains groupes minoritaires pour des raisons particulières, par exemple une meilleure reconnaissance dans la loi qui n'affectent pas d'autres groupes marginalisées. 
Également, l'analyse des communautés \lgbt{} au Québec est amorcée depuis déjà quelques décennies. 
Nous nous servirons de ces nombreux travaux dans nos analyses.

\subsection{Données d'archives}
\label{sub:donn_es_d_archives}
Nous décrirons donc maintenant plus en profondeur ces diverses sources de données. 
Les sources secondaires sont composées, d'une part, de la littérature gaie et lesbienne présentement en circulation au Québec et, d'autre part, des données référencées des \agq{}. 
Ces archives, situées dans la ville de Montréal, ont le mandat de: \blockquote[{\cite{LAGQ2014}}][.]{\textelp{} de recevoir,   conserver et préserver tout document manuscrit, imprimé, visuel, sonore, et   tout objet témoignant de l'histoire des gais et lesbiennes du Québec   (\textsc{Canada})} et sont parmi les seules au Québec à disposer de telles données. 
Étant donné cet isolement institutionnel, les données qu'on peut y trouver sont nombreuses et débordant le cadre de ce travail de recherche. 
Dans ce cas où les données abondent et dont le traitement nécessiterait un temps et un effort qui n'apporterait pas d'informations supplémentaires dans le cadre du mémoire de recherche, nous avons décidé de limiter la collecte de données à un nombre restreint de médias. 
Avec l'aide du personnel et de la professeur Julie Podmore, des sources ont été sélectionnées pour leur pertinence et leur statut récent lors d'une première visite.

Le premier média sélection est le magazine Fugues. 
Celui-ci, selon son site internet, se décrit comme suit: \blockquote[{\cite{LesNitram2015}}][.]{
  Fondé à Montréal par les Éditions Nitram, Fugues est le plus important média gai au Québec. 
Depuis sa fondation en 1984, Fugues jouit d’une notoriété et d’une crédibilité qui n’a cessé de croître au fil des années. 
On y retrouve toute l’actualité gaie d’ici et d’ailleurs, ainsi qu'une foule de rubriques et chroniques. 
En livrant chaque mois un contenu éditorial fiable sur l’actualité et les enjeux de la communauté GLBT, Fugues permet aux gais et aux lesbiennes de la région de Montréal et du reste du Québec, de rester informé sur ce qui concerne spécifiquement leurs communautés. 
C’est pourquoi, plusieurs générations de gais et de lesbiennes du Québec apprécient beaucoup ce magazine et lui sont fidèles depuis trente ans} 
Comme on peut le noter dans cette description, le magazine privilégie un point de vue montréalais sans pour autant omettre l'activité des communauté \lgbt{} des autres villes québecoises, ce qui nous permet ainsi de trouver une grande diversité de données et de toucher à de nombreuses villes, plus qu'aucune autre source de données. 
L'ancienneté du média nous permet également de couvrir l'entièreté de l'époque désignée, sois les années entre 2005 et 2015. 
À raison de douze numéros par années, c'est presque 120 numéros qui seront analysés pour la collecte de données. 
Nous avons décidé d'arrêter la collecte au mois d'août 2015, malgré qu'à posteriori, la diversité de données recherchée dans cette recherche a été atteinte plus tôt.

% \begin{quote}
%   Fugues est offert sur différentes plateformes. La version imprimée est
% 	 publiée 12 fois par année et comprend, outre une synthèse de l’information
% 	 et nos suggestions de sorties pour le mois, des analyses, des débats
% 	 d’idées et de nombreuses chroniques.

%   Le site web Fugues.com, quant à lui, suit l’actualité de plus près, et
% 	 propose plusieurs galeries de photos et de vidéos. Des nouvelles viennent
% 	 alimenter quotidiennement le site et ces articles, régulièrement cités dans
% 	 d’autres médias, sont repris sur les réseaux sociaux.~\citep{LesNitram2015}
% \end{quote}

Le deuxième média sélectionné est le journal Sortie:
\blockquote[{\cite{AllianceArc2014}}][.]{ Communautaire et participatif, le journal Sortie est produit par l’Alliance Arc-en-ciel de Québec dans le but d’informer la population sur les réalités et les droits des personnes LGBT+.
  Il a pour mission de traiter des enjeux et des événements en lien avec la lutte contre l’homophobie et la transphobie. 
Cinq éditions paraissent chaque année, chacune imprimée à 10 000 exemplaires couleurs de format tabloïd. 
Elles sont distribuées gratuitement dans plus de 200 points stratégiques de Québec.
  De plus, son édition présente et ses archives se retrouvent en version intégrale sur le présent site web} 
Possédant moins de moyens que le magazine Fugues étant donné la vocation communautaire du journal, il s'agit tout de même d'une des sources les plus complètes que nous traiterons dans cette recherche.
En effet, le journal Sortie s'étend sur sept années, soit de mai 2007 à décembre 2014. 
Le nombre de numéro fluctue d'année en année; nous avons couverts dans cette recherche 32 journaux. 
Par contre, si le journal n'est pas officiellement publié selon le site Internet de l'Alliance arc-en-ciel aucun numéros n'ont été produits durant l'année 2015, dernière année couverte par cette collecte de données. 
Le choix de ce journal est sa position centrale à Québec, faisant contrepoids à la couverture plus montréalaise du magazine Fugues.

Pas un média à proprement parler, il a été convenu durant la sélection des sources de données d'inclure les archives du festival Pervers/Cité. 
Cette décision a été prise pour représenter également les milieux dits alternatifs existants au sein des minorités sexuelles québécoises. 
Ces archives sont les moins volumineuses et possèdent certaines lacunes: nous disposons de données que pour les données 2011, 2014 et 2015. 
Ces données sont également plus variées étant constituées de feuillets d'informations, de cartes et d'affiches qui semblent plutôt être tirées de manifestations politiques idéologiquement similaires mais sans lien au niveau de l'organisation à proprement parler.

Également, nous souhaitions inclure les archives des festival de Fierté Montréal et de Divers/Cité, son prédécesseur. 
Par contre, dans les deux cas, les archives possèdent trop peu de données: en ce qui concerne Fierté Montréal, l'événement semble encore trop récent pour que les \agq{} possèdent des documents sur celui-ci. 
Pour Divers/Cité, l'organisation a récemment déclaré faillite~\citep{Cormier2015} et il est attendu que les documents dont disposent les anciens propriétaires soient bientôt intégrées aux \agq{}. 
Pour l'instant, les \agq{} n'entreposent que quelques affiches et dépliants trop anciens pour être intégrées dans notre documentation. 
Néanmoins, on va le voir dans les prochains chapitres, le magazine Fugues est un des médias principaux où ces deux festivals ont affichés des publicités et programmes.

\subsection{Données collectées sur le terrain}
\label{sub:donnees_collectees_sur_le_terrain}
En parallèle à la consultation et à la collecte des données auprès des sources secondaires, des données primaires seront recueillies sur les espaces permanents de la diversité sexuelle comme le Village Gai de Montréal, mais aussi des espaces dits temporaires comme le vieux port de Montréal durant Divers/Cité ou la rue Saint-Jean-Baptiste durant la Fête Arc-en-ciel dans la ville de Québec. 
À l'aide d'Internet notamment, il sera possible de retracer une partie des évènements publics organisés par les communautés \lgbt{}, sachant que ceux-ci peuvent permettre une mobilisation hors-ligne, du moins en contexte politique et ainsi occuper l'espace~\citep[153-154]{Mercea2011}. 
Ces données doivent comprendre également des photographies prises sur le terrain dans des secteurs des villes reconnus pour abriter des espaces \qus{}. 
Les photographies serviront notamment à capturer la composante visuelle des géosymboles rencontrés pour en faire la recension et servir par la suite de matériel d'analyse. 
Ces photographies seront géoréférencées dans le but d'ajouter une couche d'information spatiale qui devrait faciliter l'analyse géographique.

Il n'est pas prévu dans le cadre de cette recherche de faire des entrevues; néanmoins, il est envisageable que certaines informations soient recueillies à l'occasion auprès de passants ou d'individus impliqués dans les organismes, évènements ou espaces identifiés si jamais il devait y avoir un échange fortuit avec ceux-ci. 
Ces informations seront recueillis dans ce que nous nommons un guide de relevé; les données qui y seront consignées un ajout complémentaire à la photographie et à l'identification des géosymboles. 
Il n'est donc pas envisagé de passer devant un comité d'éthique car il s'agira essentiellement d'observations personnelles sur l'environnement humain et bâti entourant les géosymboles rencontrés.

À l'aide des technologies à notre disposition, il a été possible de procéder à la géolocalisation en temps réel des données collectées. 
Préalablement à la collecte de données, plusieurs applications ont été recherchées et testées pour faciliter la collectes et s'assurer d'avoir facilement accès aux différentes localiations par \gps. 
Plusieurs options sont possibles, notamment Cloud GIS, logiciel propriétaire et dont les options dans la version gratuite sont limitées, et OpenDataKit, logiciel libre complet mais qui demanderait la mise en place d'un serveur~\citep{OpenDataKit2014}. 
Ces outils permettent l'enregistrement de données multimédias: dans le cas où des données écrites devraient être notées, notamment lors de discussions, on envisage d'utiliser des logiciels de codage réflexif~\footnote{Plus communément appelés en anglais   \cadqas}. 
Il sera possible de travailler soit avec Sonal ou avec \rqda{} selon le type de données collectées. 
Le guide d'entretien pourra également être analysé par la suite à l'aide des logiciels nommés précédemment.

\subsection{Sources secondaires}
\label{sub:sources_secondaires}
Ce travail de recherche s'appuie également sur des travaux déjà effectués sur les espaces \qus{} montréalais. 
En effet, malgré une certaine jeunesse des études sur les communautés \lgbt{} au Québec, plusieurs travaux ont déjà effectués dans les dernières décennies. 
On compte parmi ceux-ci le mémoire de \cite{Leznoff1954} qui consiste en une ethnographie des homosexuels durant les années 1950, les divers travaux de l'anthropologue et membre fondateur des \agq{}, Ross Higgins, ainsi que divers articles scientifiques; plusieurs de ces travaux ont été nommées dans le chapitre précédent à la section \ref{sec:la_diversit_sexuelle_en_g_ographie}. 
Nous tenons à souligner que nous nous appuierons grandement sur le travail de \citep{Giraud2014} dans une perspective de continuité et d'approfondissement du travail déjà effectué.

\section{Séjours}
\label{sec:s_jours}
L'ensemble des données sur le terrain ont été collectées durant l'été 2015. 
Une première collecte a d'abord été effectuée durant le mois de juin dans la ville de Québec à l'aide de données déjà connues d'une recherche précédente dans le cadre d'un mémoire de maîtrise sur les divers lieux des minorités sexuelles de la ville~\citep{Vachon2014}. 
Étant donné que nous possédions un lieu de résidence dans la ville de Québec, nous avons pu mettre à l'essai les outils de collecte de données sans avoir d'inquiétudes au niveau du temps nécessaire. 
De plus, durant cette période d'essai, aucun événement d'importance n'avait lieux en lien avec la communauté LGBT de Québec et l'occupation des lieux correspondait à l'achalandage régulier auquel on peut s'attendre durant l'année, si l'on fait fit d'une probable hausse d'activité lors d'événements d'importance ailleurs dans la ville. 
Durant cette première collecte de données, il est apparu nécessaire de détailler plus en profondeur le questionnaire construit à l'intérieur de l'interface de CloudGIS pour la suite de la collecte. 
Certaines limites quant aux outils utilisés, un cellulaire, nous a poussé à nous assurer d'avoir accès à une grande quantité de données cellulaires pour effectuer le transfert des données géolocalisées quotidiennement une fois débuté la collecte à Montréal.

Cette première partie effectuée, nous avons pu ainsi procéder au mois d'août à la plus grande part de la collecte de données dans la ville de Montréal.
Préalablement au départ, nous avons construit un calendrier en ligne dans lequel nous avons compilé l'ensemble des événements organisés par la communauté \lgbt{} prévus durant mon séjour. 
Une première partie de ces événements ont été trouvés à l'aide des sites internet officiels des organismes organisateurs d'événements comme Fierté Montréal ou Pervers/Cité. 
Étant donné la présence de plus en plus grande de Facebook dans la promotion des-dits événements, nous avons également utilisé ce réseau social pour obtenir plus d'informations sur certains événements ou encore découvrir d'autres qui n'ont pas été promus autrement sur  Internet. 
C'est le cas de la totalité des événements du Festival Qouleurs et des événements politiques. 
Les pages événementielles sur Facebook possédant un grand nombre d'informations tout en présentant des images promotionnelles au potentiel géosymbolique fort, elles ont été retenues dans nos données et feront l'objet dans les chapitres suivant d'une analyse plus poussée.

Notre collecte de données à Montréal a duré au total trois semaines. 
Les premiers jours ont été consacrés à une familiarisation avec les lieux durant lesquels nous avons visité le Village gai et observé les divers événements quotidiens et l'achalandage des lieux. 
À cause d'une hausse très forte d'activité vers la fin de la journée, étant donné la fonction commerciale des lieux, les matins ont été priorisés pour la photographie des géosymboles repérés. 
En collaboration avec les bénévoles des \agq{} pour l'établissement d'un horaire de recherche, nous avons pu faire le travail en archives et couvrir l'ensemble des événements LGBTQ qui se déroulaient durant le mois. 

Comme il était prévu préalablement à la collecte, durant le mois d'août sont planifiés une grande part des événements \lgbt{} durant l'année. 
Il s'agit d'ailleurs d'une particularité au Québec étant donné qu'ailleurs dans le monde ces événements ont lieu durant le début de l'été. 
Cette décision de produire la fierté annuelle à ce moment a été faite d'abord par Divers/Cité puis par Fierté Montréal et a été reprise également à Québec. 
D'autres événements d'envergure ont d'ailleurs été organisés en parallèle par d'autres groupes de la communauté et il est donc apparu pertinent de profiter de cette opportunité pour planifier la collecte de données à ce moment de l'année sachant quand même que le portrait dressé est partiellement contingent à ce moment de l'année. 

La collecte à Montréal terminée, le retour à Québec nous a permis d'être présent par la suite aux festivités de la fête arc-en-ciel. 
Cette dernière partie de la collecte de données a été facilité par notre participation comme bénévole à l'événement. 
Nous avons donc ainsi eu accès à l'ensemble de l'espace et du temps couvert par les festivités. 
D'autres espaces auraient été intéressants à couvrir, mais par conflit d'horaire il ne n'a pas été possible d'effectuer les déplacements nécessaires. 
C'est le cas des villes nommées à la Section~\ref{ssub:autres_villes}.


%%% Local Variables:
%%% mode: latex
%%% TeX-master: "../../memoire-maitrise"
%%% End:
             % chapitre 2, etc.
%!TEX root = ../../memoire-maitrise.tex

\chapter{De la clandestinité à l'espace public}
\label{cha:de_la_clandestinite_a_l_espace_public}
\todo{titres et sous-titre à revoir : ne concorde pas avec les données}
% \chapterprecishere{\textquote{Everyone needs a place. It shouldn't be inside of someone else.} \par\raggedleft--- \textup{Richard Siken}, Crush}

\section{Spatialité homosexuelle aux débuts de la pathologisation}
\label{sec:spatialit_homosexuelle_aux_d_buts_de_la_pathologisation}

\section{Politisation et place publique}
\label{sec:politisation_et_place_publique}



\subsection{Mouvement des droits civiques}
\label{sub:mouvement_des_droits_civiques}

\section{Institutionnalisation de la diversité sexuelle}
\label{sec:institutionnalisation_de_la_diversit_sexuelle}

% section institutionnalisation_de_la_diversit_sexuelle (end)
% chapter de_la_clandestinite_a_l_espace_public (end)

\blockquote[{\cite{Pervers/Cite2015}}][.]{Organisé en collaboration,
  Pervers/Cité est un festival d’été visant à faire des liens entre les groupes
  de justice sociale, les communautés queers et les visions radicales de ce
  qu'étaient et devraient être les Fiertés LGBT\@. Dans un climat où prévaut
  l’agenda corporatif gai et l’aseptisation homogénéisé des queers, Pervers/cité
  tâche de fournir des activités critiques et accessibles, destinées à redonner
  une colonne vertébrale au mouvement LGBT}

Le classement des codes en catégorie s'est fait en relation avec les images et textes affiliés avec les codes, car ces images en soi orientaient le sens du code, à défaut d'avoir un code dont le sens serait suffisemment précis pour ne pas porter à confusion. 
Par exemple, le code maquillage a été classé dans la catégorie \code{visuel} car le maquillage qu'on retrouvait dans certaines images servaient en fait de visuel au sein d'une publicité par exemple, plutôt que d'être un maquillage porté par un individu et qui dans ce cas, aurait été instinctivement placé dans une catégorie comme habillement ou dans public ciblé si celui-ci aurait été offert à des enfants.

\section{Québec: spectre large}
\label{sec:qu_bec_spectre_large}
Nous commencerons notre présentation des résultats des médias avec la ville de Québec. 
Comme mentionné dans le chapitre sur la méthodologie, il s'agit de la deuxième ville en importance à être étudiée. 
En plus de figurer dans le magazine Fugues, celle-ci possède son propre média local, le journal Sortie. 
Nous analyserons donc les symboles de la ville dans ces deux médias.

Dans le journal Sortie, c'est vingt et une images qui ont été relevées dont dix-neuf à propos de la ville de Québec. 
Les images utilisées sont particulièrement variées: si nous avons ciblé des publicités, en général elles ne mettent pas en promotion à Québec des produits ou des services d'ordre commercial, malgré que ces dernières existent. 
En effet, en terme de produits, les publicités sont pratiquement absentes, nous n'avons relevés qu'une publicité pour la boutique érotique Chez Priape qui n'a même pas d'enseigne à Québec, celle-ci se situant à Montréal. 
En terme de services, les symboles sont plutôt orientées vers les soirées organisées sois dans les bars de la capitale, le cabaret-bar le Drague et le bar le Saint-Matthew's, ou encore dans les saunas comme L'Hippocampe. 
Ces soirées visent sois un public général, sois un genre en particulier. 
Le Drague dans ses publicités offrent des soirées réccurentes dans le temps en visant des types de soirées basées sur le contenu plutôt que vers une clientèle précise. 
Alors que dans le journal Fugues, nous avons relevé dix-sept images traitant de la capitale. 
Cela se remarque dans les visuels utilisés: on utilise les couleurs de façon variées et les symboles ne semblent pas cibler un genre en particulier. 
Des photographies sont utilisées mais celles-ci mettent essentiellement en avant des personnificateurs féminins pour des spectacles de ce type. 
Un autre cas d'activité mixte est le cas d'une publicité pour une activité de danse, plus précisément du Queer tango, où l'image, si elle met de l'avant un couple masculin, ne semble pas non viser un genre en particulier. 
Il s'agit d'ailleurs de la seule image utilisée dans le contexte de la ville de Québec où le terme queer est utilisé. 
On peut douter ici qu'il s'agit d'un usage du terme dans un cadre politique, plutôt, le terme queer ici semble désigner le caractère négatif et alternatif mis de l'avant dans l'activité du tango, une danse habituellement pratiquée entre un homme et une femme.

Parmi les images d'activités s'adressant spécifiquement à un genre particulier, nous avons d'abord deux activités festives s'adressant aux femmes, visible à la figure~\ref{figs3132}. 
Il s'agit de soirées organisées par le magazine Saphomag, un magasine destinée à une clientèle lesbienne. 
Contrairement aux activités réservées aux hommes dont nous traiterons plus loin dans ce chapitre, celles-ci ne se déroulent pas dans des bars, mais dans des locaux loués à l'église St-Jean-Baptiste dans le quartier du même nom. 
Sachant qu'il n'y a plus de bars strictement lesbiens comme il a pu en avoir à Montréal~\citep{Podmore2006}, on peut comprendre en partie la nécessité de se tourner vers des lieux alternatifs, mais dans ce cas-ci, les organisateurs ne se sont pas tourné vers le bar mixte de Québec, le Drague. 
Il est important de souligner également que ces fêtes sont organisées dans le cadre de la fête arc-en-ciel de Québec; nous n'avons pas trouvé d'occurence de fêtes similaires durant le reste de l'année dans le journal Sortie. 
Ces activités mettent de l'avant des soirées de DJ mais également de peinture corporelle.

\begin{figure}
\centering
\subcaptionbox{Événement de 2007\label{fig31}}
{\includegraphics[width=9cm]{fig31.jpg}}
\subcaptionbox{Événement de 2008 (et événement mixte)\label{fig32}}
{\includegraphics[width=6cm]{fig32.jpg}}
\caption{Événements destinés aux femmes (lesbiennes) : deux éditions
  différentes}\label{figs3132}
\end{figure}

\section{Montréal: diversité des imaginaires}
\label{sec:montr_al_diversit_des_imaginaires}

\subsection{Fétichisme}
Nous retrouvons une bonne part d'imagerie liées au fétichisme sexuel dans les médias étudiés. 
Ce fétichisme est particulièrement apparent dans les espaces orientés vers les hommes gais. 
Nous retrouvons ces images presque essentiellement dans le magazine Fugues et dans une moindre mesure sur quelques événements Facebook, pour des événements touchant encore une fois cette communauté masculine. 
Parmi les bars mettant en avant un tel imaginaire, on peut nommer en ordre d'importance l'Aigle Noir, le Tool et le Drague, à Québec \todo{Vérifier l'ordre réel des bars utilisant ce type d'imagerie}. 
Le Drague fait figure d'exception car il s'agit du seul lieu hors Montréal de ce type, quoique cette impression n'est pas la norme pour cet espace. 
En effet, aujourd'hui le Drague n'utilise plus ce type d'imagerie: il s'agit maintenant d'un lieu essentiellement mixte et peu sexualisé, du moins dans les symboles invoqués. 
Auparavent, et dans les premières années traitées dans notre collecte de données, le troisième étage servait de lieu de drague utilisé par les hommes et strictement pour eux. 

Tel que souligné par \citet{Giraud2013a}, le bar l'Aigle Noir vise une clientèle masculine d'abord et plus particulièrement fétichiste. 
Ceci est d'autant plus visible par le choix du logo et des couleurs utilisées: le noir, couleur fréquemment utilisée pour les vêtements et articles de cuir, on y voit un homme habillé de ce qui semble être du cuir et un aigle. 
Cet aigle, utilisé dans de nombreuses cultures selon des sens variés, peut rappeler l'imagerie militaire, tel qu'utilisé par exemple par l'État américain. 
Cette interprétation cadrerait avec une part importante de la sous-culture fétichiste s'attardant plus particulièrement aux uniformes, dont les uniformes militaires. 
L'Aigle Noir ne fait pas montre de censure vis-à-vis le type d'activité à caractère fétichiste qui s'y produisent: << party bobettes >>, << soirées bulles >>, party latex, ventes << d'esclaves >>, etc. 
Ce type d'activités sont propres au fétichisme: en plus du cuir, d'autres matériaux sont utilisés selon les fétichismes, dont le latex et les sous-vêtements masculins sont également souvent priorisés. 
Les soirées bulles semblent être l'apanage de nombreux autres espaces: en effet, en plus d'être du genre d'activités proposées à l'Aigle Noir, on les retrouve également dans les saunas, autant de Montréal que de Québec. 
Les soirées d'esclaves s'inscrivent plus particulièrement dans le cadre du \bdsm{}\footnote{\citeauthor{Turley2015} définisent le \bdsm{} comme:   \foreignquote{english}{\textelp{} the umbrella term   used to describe a set of consensual sexual practices that usually involve an   eroticised exchange of power and the application or receipt of painful and/or   intense sensations (Barker et al., 2007). 
The range of \bdsm{}-related activities   is wide and complex. 
‘BDSM’ denotes the assorted consensual activities   involved in the experience of participating in \bdsm{}; bondage and discipline   (B\&D), dominance and submission (D/s), and sadism and masochism   (SM)~\citeyearpar[24]{Turley2015}.}}. 
On retrouve de moins en moins cet imaginaire dans les dernières années de notre collecte de données en ce qui concerne particulièrement les données d'archives et celà correspond à quelques discussions que nous avons eu sur le terrain avec certains bénévoles des \agq{}: le bar a évolué dans les dernières années pour satisfaire un plus grand éventail au niveau de la clientèle, tout en demeurant un bar essentiellement pour hommes gais. 

Certains lieux orientés vers les fétichisme continuent toutefois à exister: c'est le cas de soirées organisées au \emph{Bunker}, la section sous-sol du bar \emph{Les Katakombes}, un bar orienté vers la scène métal et rock en-dehors du Village gai. 
Ce sous-sol remplie d'une certaine façon le rôle qu'a joué par le passé le bar l'Aigle Noir, et offrant des activités orientées vers le fétichisme en plus d'être un \emph{Backroom}\footnote{À COMPLÉTER}. 
Encore une fois, l'imagerie fétichiste est mise de l'avant, en misant toujours sur une clientèle masculine. 
Au niveau de l'accessibilité, les lieux diffèrent : si les soirées semblaient gratuites dans l'Aigle Noir, \emph{Le Bunker} n'organise que des soirées épisodiques dont l'entrée est payante, au montant de 30\$. 

Dans l'événementiel, certaines soirées mettent de l'avant des codes similaires.
C'est le cas notemment de certaines soirées organisées dans le cadre de la Fierté 2015, comme la soirée  \emph{BlackNight}.
Dans celle-ci, le noir et le rouge sont utilisés avec comme point focal une photographie d'un homme portant la barbe et ce qui semble être un uniforme de cuir.
Un autre exemple est l'événement \emph{Beardrop édition Montréal} organisé par \emph{Scruff}\footnote{Application de rencontre et drague pour hommmes fonctionnant sur cellulaire à l'aide du \gps{} de l'utilisateur.}.
On retrouve dans l'imagerie utilisée les mêmes codes que ceux nommés précédemment, malgré que le nom de l'événement semble cibler plus particulièrement les hommes entrant dans la catégorie de \emph{bear}.
En fait, dans l'image, on peut voir deux hommes sveltes mais musclés dans une posture s'apparentant à une danse.
Les vêtements, plutôt qu'être des uniformes de style militaire ou policier comme pour la soirée \emph{BlackNight}, sont plutôt des tenues ressemblant à des tenues de travail pour l'un des hommes, et des sous-vêtements pour l'autre.
On retrouve donc qu'une partie des codes visuels permettant d'identifier les hommes typés \emph{bear}; on retrouve pas la carrure habituellement mise de l'avant, sois un surpoids important ou une carrure impressionante.

L'ensemble des exemples nommés précédemment se rapportent à la ville de Montréal.
Peu de lieux ou d'événements mettent vraiment de l'avant le fétichisme dans leur imagerie, sauf à quelques occasions à Québec.
En effet, on peut nommer déjà certaines soirées organisées dans le bar Saint-Matthew's qui mélangent fétichisme et masculinité dans la promotion de leur événements.
Également, comme dans la Fierté montréalaise, la marche pour la diversité sexuelle est un des moments où plusieurs individus décident de mettre de l'avant sur la place publique leur intérêt pour les sexualités alternatives.
On retrouve par exemple plusieurs personnes utilisant les drapeaux cuirs, latex et \bdsm{} ainsi que des costumes se rapportant à ces fétichismes ou à des pratiques affiliées, comme le \emph{Puppy play}\footnote{Jeu de rôle de domination et soumission dans laquelle le soumis joue le rôle d'un chien  et où le dominant joue le rôle de \emph{handler}, à savoir le \emph{propriétaire} du chien. Le tout est souvent appuyé par l'usage de certains accessoires comme des masques, queues en caoutchouc, des harnais, un collier, etc.}.

\subsection{Marches et manifestations}
\label{subsec:label}
\todo{Probablement que ça va changer de place, à voir} Montréal se distingue par l'activité politique de plusieurs groupes \lgbt{}.
Nous traiterons dans cette partie de deux cas en particulier, sois les marches organisées par les communautés lesbiennes et trans, sachant que celles-ci n'ont pas nécessairement de liens direct (on peut croire par contre que certains individus participent aux deux événements, nous reviendrons sur les raisons) mais qu'elles adoptent des stratégies similaires.

\subsubsection{Marche Trans}
\label{subsubsec:marchetrans}
La marche trans s'inscrit plus largement dans le cadre de la Fierté Trans, un événement organisé immédiatement avant la fierté de Fierté Montréal. 
Comme le laisse entendre le nom choisi, l'événement s'adresse plus particulièrement aux personnes trans, quoique ce dernier terme rassemble un grand nombre d'identités entourant le genre. 
Parmi ces identités, on compte les personnes trans\footnote{nous n'utiliserons pas les termes de transexuels ou de   transgenres, ceux-ci n'apparaissant pas dans aucun des documents que nous   avons traités entourant cet événement}, les personnes non-binaires dans le genre\footnote{Nous utiliserons une définition assez large du terme, en   considérant celui-ci comme représentant autant les personnes se disant   non-binaires que les personnes agenres, neutroïdes, demi-hommes, demi-femmes,   au genre fluctuant, etc.\citep[see][]{Barker2015}} et les personnes intersexes. 
Nous appuyons cette définition d'ensemble de trans par l'imagerie utilisée par les organisateurs, plus particulièrement l'affiche de la marche.
Celle-ci montre en effet un grand nombre de symboles de ces différentes identités, comme des drapeaux. 
Chacune des identités nommées précédemment est évoquée par un des drapeaux: \todo{faire la liste des drapeaux et des identités}.

Comme nous l'avons souligné dans le début de cette section, la marche Trans s'inscrit dans la fierté trans qui compte d'autres événements: plus particulièrement, on peut souligner les différentes fêtes et campagnes de financement qui ont tous eu lieu dans une même soirée au café Cléopâtre. 
Cet espace est particulier par sa proximité avec l'histoire des communautés \lgbt{} québecoises et de sa proximité relative avec la communauté trans.

La marche s'est fait un trajet assez linéaire et situé dans les espaces reconnus de la communauté \lgbt{}. 
Tel qu'on peut le voir à la figure \todo{insérer la   figure}, la manifestation commence à proximité du café Cléopâtre, traité précédemment, dans le coeur de l'ancien \anglais{Red Light} montréalais. 
Cette marche se dirige par la suite dans le Village Gai pour terminer dans le Parc Lafontaine, toujours à proximité du Village. 
Plusieurs arrêts ont marqués cette marche. 
D'abord, au tout départ, plusieurs intervenants ont procédé à des discours par rapport aux droits des personnes trans et dans un cas précis, le cas des femmes trans de couleur (tel que décrit par la banderole utilisée).
Quelques organismes étaient présents à la marche et visibles; nous avons remarqué la présence de l'\atq{} et du \rlq{}. 
On peut croire que d'autres organismes ou membres d'organismes étaient également présents, étant donné la présence de ceux-ci à la soirée précédent la marche, tel l'\astteq{}. 
Par la suite, la marche se dirigea vers le Village gai pour un arrêt au parc de l'espoir, un lieu commémoratif aux victimes du \sida{}~\citep{Lafontaine2012} connu pour avoir été le théâtre d'actions politiques par le passé.

La marche trans s'articule autour d'un discours particulier, étant donné le contexte politique dans lequel cette dernière s'inscrit et également de l'actualité québecoise au niveau législatif. 
En s'intéressant au texte d'invitation de la marche Trans tel que publié sur Facebook, on apprend que la marche s'appuie sur plusieurs revendications entourant le changement de statut légal de genre. 
Ces points tournent autour du statut de citoyennté, l'âge, le genre, les exigences médicales (au niveau chirurgical notamment) et du coût des démarches. 
D'autres revendications sont également supportées ayant moins à voir avec le statut légal, sois l'absence de ressources spécialisées en prévention du \vih{} pour les personnes trans et pour faciliter le changement de statut.

On va mieux le comprendre dans la section suivante, la marche trans utilise une stratégie similaire au groupe des personnes \dyke{}, sois l'utilisation de la marche pour s'affiche publiquement. 
Cette marche correspond à première vue à une manifestation politique dans laquelle la visibilité est extrêmement importante.

\subsubsection{Marche dyke}
\label{subsubsec:marchedyke}
La \dm{}, comparativement à la marche trans, ne s'inscrit pas dans un événement plus large. 
En fait, nous constatons que le choix de la date se fait plutôt en réponse à la fierté organisée par Fierté Montréal. 
Nous croyons effectivement que, selon les motifs politiques de l'événement, celle-ci vise à offrir une visibilité à la communauté \dyke{} que l'on ne retrouverait pas dans la Fierté plus traditionnelle, malgré qu'une marche est été aussi organisée pour les femmes, cette dernière étant soutenue par l'organisation de la Fierté. 
Par conflit d'horaire, nous n'avons été présent que pour la marche Dyke. 
Si cette section va principalement traiter de cette dernière, nous nous intéresserons également à la marche lesbienne selon les informations que nous avons pu accumuler sur Facebook et dans la documentation promotionnelle de Fierté Montréal.

La visibilité semble était le but principal derrière la \dm{}: ceci est particulièrement apparente par le choix esthétique de la bannière ornant l'événement Facebook, où on peut y voir des pictogrammes d'oeils ainsi que dans le slogan utilisé. 
On retrouve également cette visibilité portée par la volonté de manifester sa présence dans l'espace public où la visibilité est normalement mobilisée comparativement à l'espace privé qui s'appuie plutôt sur les notions d'intimité et par l'absence de d'observateur (la présence d'un tel observateur possible mais on parle plutôt ici d'intrusion et d'une certaine part de violence \todo{Trouver des références sur l'intimité dans l'espace privé}). 

Pour la suite de notre analyse de la \dm{}, nous nous pencherons sur les publics sollicités, à savoir quels individus sont invités à participer à la marche et vers qui le message de la marche est orientée. 
Pour répondre à ces questions, nous pouvons déjà solliciter le travail de~\cite{Podmore2015a} qui a travaillé sur l'édition 2012 de cette marche. 
S'appuyant sur les travaux déjà effectués sur des éditions d'autres villes d'Amérique du Nord, comme Chicago, on apprend que les marches \dykes{} prennent racines dans une certaine exclusion des femmes des marches qui désiraient s'organiser entre elles et faire valoir leur présence au sein du mouvement principal derrière les fiertés. 
En conséquent, celles-ci décidèrent de créer des leurs propres marches, celles-ci se déroulant quelques jours avant la marche officielle. 
C'est également la stratégie choisie par les organisatrices de la marche de Montréal des dernières années, qui ont toujours placé l'événement quelques jours avant les débuts de la semaine de la fierté, et non seulement avant la marche officielle qui se déroule habituellement dans la dernière fin de semaine de la fierté.

Ce positionnement particulier et cette division du mouvement semble conforter une certaine identité autour du mot \dyke{}, terme que~\citet{Podmore2015a} dans son article \citetitle{Podmore2015a} décrit comme politisé et radicalisé comparativement au terme plus général de lesbienne. 
Le terme \dyke{} d'ailleurs vise à englober une plus vase population que le terme lesbienne, alors que n'importe quelle femme non-hétérosexuelle peut s'identifier avec ce terme et se joindre au mouvement de la marche \dyke{}. 
Cette politisation s'exprime en même temps par une non-mixité qui vise à exclure les hommes de la marche --- exclusion basée sur le respect des principes de l'événement plutôt qu'une exclusion qu'on pourrait qualifier d'agressive. 
Les alliés intéressés par l'évènement mais n'entrant pas dans la catégorie \dyke{} étaient invitéEs à suivre la marche à l'extérieur, en marchant sur les trottoir. 
C'est ce que j'ai du faire, mais j'ai pu constater l'absence d'individus se réclamant ou agissant comme \emph{alliés} (tenant des pancartes ou scandant des slogans en marge de la marche). 
Un seul autre homme était présent. 
En discutant avec cette personne j'ai pu apprendre qu'il était là par curiosité et j'ai du moi-même dû lui expliquer que sa présence dans la marche n'était pas tolérée et quels étaient les principes soutenus par cette dernière.

Le public visé par la marche dans son ensemble est moins facilement définissable. 
En fait, nous pouvons considérer que l'ensemble de la société est visée par le message de la marche, étant donné que la visibilité comme telle s'exprime dans l'espace public, comme je l'ai souligné précédemment. 
Par contre, on ne peut ignorer le fait que la marche s'inscrit dans les pratiques d'autres événements politiques similaires. 
Nous ne pouvons pas clairement savoir s'il s'agit ici d'une tradition ou un message qui demeure encore porté à l'organisation principale de la fierté. 
Également, le choix du parcours peut nous renseigner sur le public ciblé. 
Contrairement à l'édition sur laquelle~\citet{Podmore2015a} a travaillée, la marche de 2015 a complètement évité le Village gai pour commencer plus au nord et terminer dans le quartier du \anglais{Mile End}.

Pour approfondir la question de la visibilité, l'article de \citet{Frosh2006} \citetitle{Frosh2006} offre une vue intéressante sur le partage d'un message --- \anglais{text} dans l'article --- à l'aide d'un média. 
Dans un contexte social où les interactions entre individus dans l'espace public sont réduites au minimum jusqu'à l'indifférence, se rendre visible auprès d'autrui permet d'agir à contre-courant et déranger cette indifférence. 

Pourtant, \citeauthor{Frosh2006} nous apprend que cette indifférence peut être une forme de respect ou d'intégration. 
Les pratiques sociales dans l'espace public qui se fondent sur l'indifférence témoignent chez les individus une forme de respect mutuel qui est bien différente d'une relation basée sur l'altérité et l'incompréhension. 
Si certains auteurs d'après \citeauthor{Frosh2006} considèrent la froideur des relations humaines --- l'inattention civile chez \citeauthor{Goffman1956} --- dans l'espace public comme une preuve d'un espace moralement vide et étranger, on apprend aussi que ce serait plutôt la réaction à la vision d'un autre qui sera le témoignage d'une forme de peur ou d'appréhension vis-à-vis l'autre~\citep[279--280]{Frosh2006}. 
L'action de réclamer cette visibilité dans la \dm{} pourrait être conçue comme une rupture volontaire de ce respect mutuel pour montrer que l'égalité sous-entendue n'est pas concrète et reste à faire. 
Cet acte montrerait dont une limite de l'inattention civile comme concept centré sur la communication et la perception, sois une certaine incapacité à tenir compte des minorités conçues comme invisibles ou, du moins, d'individus se regroupant autour d'une différence commune vis-à-vis la norme, ici hétérosexuelle et même masculine. 
L'appel à la non-mixité, si elle répond à une volonté de se retrouver entre individus partageant une oppression commune comme femme et lesbienne, permet également de d'avoir un contrôle sur l'image qu'elles véhiculent collectivement par rapport à l'observateur qui se trouve souvent être en position de pouvoir\todo{Traiter du   Male gaze avec les   références~\cite{Wood2004,Patterson2002,Skelton2002,Snow1989}}.

\section{Festivités s'étirant sur plusieurs jours}
\label{sec:festivitesplusieursjours}
Si les marches et manifestations sont symboliquement marquantes par l'usage de nombreux géosymboles et par une subversion partielle ou complète de l'espace public, il en demeure pas moins que leur présence est très circonscrite dans le temps.
Nous nous intéresserons dans cette section à des événements temporaires, mais s'étirant dans le temps.
La totalité de ceux-ci consistent en des célébrations de le diversité sexuelle, certaines bien connues du public, d'autres appartenant plutôt à une certaine contre-culture ou orientée vers les individus s'indentifiant spécifiquement à l'identité de genre ou l'orientation sexuelle ciblée.

\subsection{Fierté trans}
\label{subsec:fiertetrans}
Nous avons traités brièvement de la Fierté trans précédemment dans la section précédente à propos de la marche trans.

\subsection{Qouleur}
\label{subsec:qouleur}
Le terme de Qouleur désigne autant le nom de l'événement que le nom du collectif derrière celui-ci.
À première vue, on peut croire que le terme Qouleur peut désigner deux réalités, sois la diversité sexuelle par un lien avec le drapeau arc-en-ciel aux multiples couleurs et sois une désignation donnée aux personnes racisées, personnes de couleur.
Entendu comme un terme moins discriminant que le terme \emph{race}, le terme couleur prend ses origines dans \todo{Trouver l'origine du mot couleur}.
Bien qu'utilisé fréquemment pour désigner les personnes d'origine afro, le terme de couleur dans le nom de l'événement semble plutôt désigner la totalité des individus racisés, sachant que les événements se déroulant dans ce festival ciblent certains groupes précis.
Si certains de ceux-ci sont publics, plusieurs d'entre eux visent pas exemple les individus autochtones par l'appellation de \emph{two-spirits}

\subsection{Pervers/Cité}
\label{subsec:perverscite}
Pervers/Cité est un autre des nombreux événements se déroulant en parallèle à la Fierté Montréal du mois d'août.
L'existence de Pervers/Cité se démarque comme un événement créé en réaction à l'évolution des festivités plus traditionnelles de la communauté gaie montréalaise.
En effet, à la création de Pervers/Cité existait déjà un organisme organisant le fierté gaie montréalaise, Divers/Cité.
Ce dernier, nous y reviendrons plus tard, a d'abord été un événement à tendance communautaire et politique pour devenir une festival plutôt orienté vers la fête avec un gain de popularité des festivités.
Pervers/Cité est né d'une réponse à cette évolution qui a été dénoncée comme marchande et non-rassembleuse pour l'ensemble des minorités sexuelles montréalaises.
Organisée par une faction encore très politisée, Pervers/Cité souhaitaient alors offrir une alternative radicale aux festivités de Divers/Cité, l'événement se déroulant dans les mêmes journées mais à l'extérieur du périmètre occupé par Divers/Cité, sois la rue Sainte-Catherine et plus tard le vieux port de Montréal.
Cette tendance se poursuite aujourd'hui, alors que le festival s'étend sur une vaste territoire, visible à la figure \todo{Mettre la figure}.

\todo{À remplacer par la carte de Pervers/Cité}

\subsection{Fierté Montréal 2016}
\label{subsec:fiertemontreal2016}

\subsection{Fête Arc-en-ciel}
\label{subsec:fetearcenciel}




%%% Local Variables:
%%% mode: latex
%%% TeX-master: "../../memoire-maitrise"
%%% End:
             % chapitre 2, etc.
%!TEX root = ../../memoire-maitrise.tex

\chapter{De l'inclusion à la différence par le symbole}
\label{cha:de_l_inclusion_la_diff_rence_par_le_symbole}

% \chapterprecishere{\textquote{Everyone needs a place. It shouldn't be inside of someone else.” \par\raggedleft--- \textup{Richard Siken}, Crush}
Après avoir présenté les résultats de la collecte de données dans le chapitre précédent, nous nous procèderons dans celui-ci au portrait d'ensemble soulevé par les données.
Pour chaque géosymbole analysé précédemment, nous avons offert une brève analyse du sémiotique pour faire ressortir les messages que ceux-ci portent, les codes.
Également, l'approche géographique nous a poussé à localiser ces symboles et nous avons ainsi pu décrire du même coup l'étendu et la position des géosymboles rencontrés.
Ces informations, combinées ensemble, dressent plusieurs portraits simultanés vis-à-vis la panoplie des symboles rencontrés.
Par contre, nous croyons qu'il est nécessaire pour la suite, conformément aux points soulevés au chapitre 2, de faire les liens entre ces géosymboles pour arriver, par la suite, à faire une analyse d'ensemble.
Cette analyse sera donc une synthèse de notre démarche qui pourra par la suite servir à faire la comparaison avec d'autres territoires ailleurs, dans d'autres villes ailleurs qu'au Québec.



\todo{Fouiller \cite{Fyfe1988}, contenu pertinent selon \cite[11]{Rose2012}}

\subsection{Communauté imaginée}
\label{sub:communaut_imagin_e}
% Pourra être déplacé ailleurs
Devant ces nombreuses dissensions et absences d'échanges au sein ce qu'on nomme communauté~\lgbt{} ou gaie et lesbienne, on peut s'interroger sur ce qui en est de cette communauté, s'il en est vraiment une.
S'intéressant d'abord au nationalisme, le texte de Benedict Anderson offre une analyse intéressante de la question de nation, alors que celles-ci représentent bien souvent de grands ensembles liés par peu de choses, sinon un lieu de naissance commun et une culture commune.
\begin{quote}	
societies are sociological entities of such firm and stable reality that their members (A and D) can even be described as passing each other on the street, without ever becoming acquainted, and still be connected.37 \citep{Anderson1983}
\end{quote}

\begin{quote}
  That all these acts are performed at the same clocked, calendrical time, but   by actors who may be largely unaware of one another, shows the novelty of this imagined world conjured up by the author in his readers' minds.38
  \citep{Anderson1983}
\end{quote}

\begin{quote}
	
It should suffice to note that right from the start the image (wholly new to Filipino writing) of a dinner-party being discussed by hundreds of unnamed people, who do not know each other, in quite different parts of Manila, in a particular month of a particular decade, immediately conjures up the imagined community.
[\ldots]

Notice too the tone. 
While Rizal has not the faintest idea of his [28] readers' individual identities, he writes to them with an ironical intimacy, as though their relationships with each other are not in the smallest degree problematic.43
\end{quote}


\begin{figure}[ht]
	\centering
	\includegraphics[width=16cm]{fig4.jpg}
	\caption{Fête arc-en-ciel de Québec: marche de solidarité et journée
    communautaire\todo{corriger les étiquettes de noms sans accents}}
	\label{fig:figure4}
\end{figure}

\begin{figure}[ht]
	\centering
	\includegraphics[width=16cm]{fig3.jpg}
	\caption[]{Dyke March: trajet et moments importants de la manifestation}
	\label{fig:figure3}
\end{figure}

% subsection communaut_imagin_e (end)

\section{Premiers symboles liés à la diversité sexuelle}
\label{sec:premiers_symboles_li_s_la_diversit_sexuelle}

% section premiers_symboles_li_s_la_diversit_sexuelle (end)

\section{Symboles politiques et identitaire}
\label{sec:symboles_politiques_et_identitaire}

% section symboles_politiques_et_identitaire (end)

% chapter de_l_inclusion_la_diff_rence_par_le_symbole (end)
%%% Local Variables:
%%% mode: latex
%%% TeX-master: "../../memoire-maitrise"
%%% End:
             % chapitre 2, etc.
%!TEX root = ../../memoire-maitrise.tex

\chapter{Le paysage queer d'aujourd'hui}
\label{cha:le_paysage_queer_d_aujourd_hui}
Comme nous avons pu le voir dans notre analyse de données, le paysage géosymboles des communautés \lgbt{} varient énormément, que ce soit selon les axes du genre ou de l'orientation sexuelle, ou encore d'autres qu'on ne lit pas automatiment à la sexualité, comme la classe sociale ou l'ethnicité.
Ces variations débordant le cadre de la sexualité, elles confirment les propos de \todo{faire une citation adéquate de Jackson et Son}, comme quoi nous ne pouvons réduire notre analyse de l'hétérosexualité au simple cadre de la sexualité, celle-ci se manifestant dans l'ensemble des sphères de la société, au même titre que le capitalisme ou l'hégémonie blanche, par exemple.


Nous l'avons souligné déjà à quelques reprises dans ce mémoire, mais plusieurs limitent s'imposent à l'étendue des résultats et à l'analyse dont nous faisons du territoire des communautés \lgbt{} du Québec.

Y'a-t-il une place à la critique des différentes manifestations dont nous avons fait la revue?
Il ne s'agit pas du but premier de notre travail.
Par contre, nous pouvons constater certaines des tendances que nous avons traitées dans le premier chapitre, à savoir que les enjeux d'hétéronormativité semblent se manifester.

\section{Types de symboles}
\label{sec:types_de_symboles}


\section{Sens et utilité}
\label{sec:sens_et_utilit_}
\begin{quotation}
Ces relations se construisent comme une appropriation symbolique de l'espace, sous l'effet de forces qui tantôt unissent, tantôt opposent les acteurs sociaux. 
D'où l'idée qu'il existe, dans une société ou un milieu donné, plusieurs « types » et plusieurs « niveaux » de territorialités, celles-ci pouvant être symétriques ou non, selon la nature des échanges qui s'établissent dans le système (simples relations bilatérales ou coûts supérieurs à consentir qui mettent en danger la structure de ce système).\citep[41]{Courville1991}
\end{quotation}

\section{Types d'espaces rencontrés}
\label{sec:types_d_espaces_rencontr_s}


\section{Pistes de recherche futures}
\label{sec:pistes_de_recherches}

Suite aux différents résultats de ce mémoire ainsi que les conclusions soulevées par les différentes approches méthodologiques sur lesquelles s'assoient ce travail, il apparait maintenant nécessaire de poursuivre le travail auprès des différents groupes rencontrés sur le terrain.
En effet, le portrait dressés restent fortement influencées par ma perspective personnelle du chercheur, autant comme géographe que comme membre de la communauté LGBTQ, avec mes a priori et une volonté de demeurer objectif qui est nécessairement partielle.

Ce travail auprès de la population pourrait prendre diverses formes. 
D'abord, il est envisageable de maintenant prendre contact avec avec certains des organismes treprésent de ces groupes, que ce soit le GRIS-Montréal ou Qouleur par exemple et leur offrir, en plus des conclusions soulevées par ce mémoire, une possibilité de poursuivre la recherche en tentant d'appronfondir l'analyse de l'occupation de l'espace urbain par la population qu'ils représentent \todo{reformuler}.

Cet approdondissement pourrait prendre la forme du cartographie participative des espaces \lgbt{} par la population et pour celles-ci.
En effet, au-delà des cartographies qu'on retrouve à l'intérieur du Fugues, dans le domaine de la recherche \todo{trouver la citation de Podmore pour sa   cartographie} ou exceptionnelles dans le cadre de certains événements~\parencite{Pervers/Cite2015}, aucun outil ne centralise l'ensemble de ces connaissances. 
Comme nous le soulevons dans cette recherche, les espaces \lgbt{} dans les villes de Montréal ou de Québec sont multiples et méritent, en concordance avec la volonté de certains groupes comme ceux de la Dyke March, une meilleure visibilité.
Cette visibilité pourrait potentiellement offrir aux individus d'orientation ou de genre variés de retrouvés les gens qui leur ressemble et obtenir des ressources adaptées à ceux et celles-ci, que ce soit des lieux de socialisation comme les bars réputés sécuritaires ou des cliniques offrant des soins particuliers.

Également, un espace qui a très peu été touché par cette recherche est l'ensemble des villes et villages où s'organisent ou vivent des individus des minorités sexuelles. 


Un travail de plus grande ampleur au niveau géohistorique ouvrirait la possibilité à une étude plus approfondie des géosymboles, mais, en l'absence de ces données, il est difficile de décrire plus particulièrement les autres villes Québecoises possédant une communauté de minorités sexuelles. 
On peut toutefois nommer les villes de Rimouski, Gatineau, Saguenay et Trois-Rivières comme candidates à une analyse plus approfondie. 
Ces villes, par leur inscription au sein d'une structure régionale urbanisée et par leur proximité aux villes importantes de l'est du Canada. 
En effet, comme il le sera décrit dans les chapitres suivants, des données ont été recensés dans ces diverses villes, que ce soient des géosymboles ou du moins, des adresses et des contacts prouvant l'existence de telles communautés.

%%% Local Variables:
%%% mode: latex
%%% TeX-master: "../../memoire-maitrise"
%%% End:
             % chapitre 2, etc.
\chapter*{Conclusion}         % ne pas numéroter
\phantomsection\addcontentsline{toc}{chapter}{Conclusion} % dans TdM

% Une thèse ou un mémoire devrait normalement se terminer par une
% conclusion, placée avant les annexes, le cas échéant. Celle-ci est
% traitée comme un chapitre normal, sauf qu'elle n'est pas numérotée.

Comme nous avons pu le voir dans notre analyse de données, le paysage géosymbolique des communautés \lgbt{} varie énormément, que ce soit selon les axes du genre ou de l'orientation sexuelle, ou encore d'autres qu'on ne lie pas automatiquement à la sexualité, comme la classe sociale ou l'ethnicité.
Ces variations débordant la simple notion de sexualité, elles confirment les propos de X \todo{faire une citation adéquate de Jackson et Son}, comme quoi nous ne pouvons réduire notre analyse de l'hétérosexualité au simple champ social de la sexualité, celle-ci se manifestant dans l'ensemble des sphères de la société, au même titre que le capitalisme ou l'hégémonie blanche, par exemple.


Nous l'avons souligné déjà à quelques reprises dans ce mémoire, mais plusieurs limites s'imposent à l'étendue des résultats et à l'analyse que nous établissons du territoire des communautés \lgbt{} du Québec.

Y'a-t-il une place à la critique des différentes manifestations dont nous avons effectué la revue?
Il ne s'agit pas du but premier de notre travail.
Par contre, nous pouvons constater certaines des tendances que nous avons traitées dans le premier chapitre, à savoir que les enjeux d'hétéronormativité semblent se manifester.

Si nous revenons sur la conception geertzienne de la culture dans son analyse sémiotique, nous pouvons remarquer que notre travail s'est très peu attardé au \emph{modèles de} d'un/des modèles culturels \lgbt{}.
En effet, nous pouvons considérer \latin{a posteriori} que l'analyse des structures non-symboliques de la réalité n'a pas un sens particulier pour un sous-groupe comme les communautés \lgbt{}; cette conception du réel pourrait être celle de la société majoritaire, du modèle culturel occidental, américain, ou québecois selon la lentille.
Si ce travail ne s'y est pas attardé, ce pourrait être une avenue intéressante~\todo{à compléter}.

\section*{Types de symboles}
\label{sec:types_de_symboles}


\section*{Sens et utilité}
\label{sec:sens_et_utilit_}
\begin{quotation}
  Ces relations se construisent comme une appropriation symbolique de l'espace, sous l'effet de forces qui tantôt unissent, tantôt opposent les acteurs sociaux.
  D'où l'idée qu'il existe, dans une société ou un milieu donné, plusieurs « types » et plusieurs « niveaux » de territorialités, celles-ci pouvant être symétriques ou non, selon la nature des échanges qui s'établissent dans le système (simples relations bilatérales ou coûts supérieurs à consentir qui mettent en danger la structure de ce système).\citep[41]{Courville1991}
\end{quotation}

\section*{Types d'espaces rencontrés}
\label{sec:types_d_espaces_rencontr_s}


\section*{Pistes de recherche futures}
\label{sec:pistes_de_recherches}

À la suite des différents résultats de ce mémoire ainsi que les conclusions soulevées par les différentes approches méthodologiques sur lesquelles s'assoit ce travail, il apparait maintenant nécessaire de poursuivre notre démarche auprès des différents groupes rencontrés sur le terrain.
En effet, le portrait dressé reste fortement influencé par ma perspective personnelle de chercheur, autant comme géographe que comme membre de la communauté \lgbt{}, avec mes a priori et une volonté de demeurer objectif qui est nécessairement partielle.

Cet exercice auprès de la population pourrait prendre diverses formes.
D'abord, nous pensons qu'il est envisageable de contacter certains organismes représentant ces groupes, qu'il s'agisse du GRIS-Montréal ou Qouleur par exemple.
Un deuxième travail de recherche pourrait se donner comme objectif d'offrir, en plus des conclusions soulevées par ce mémoire, un approfondissement analytique de l'occupation de l'espace urbain par la population qu'ils représentent \todo{reformuler}.

Cet approfondissement pourrait prendre la forme de la cartographie participative des espaces \lgbt{} par la population et pour celles-ci.
En effet, au-delà des cartographies qu'on retrouve à l'intérieur du Fugues, dans le domaine de la recherche \todo{trouver la citation de Podmore pour sa   cartographie} ou exceptionnelle dans le cadre de certains événements~\parencite{Pervers/Cite2015}, aucun outil ne centralise l'ensemble de ces connaissances.
Comme nous le soulevons dans cette recherche, les espaces \lgbt{} dans les villes de Montréal ou de Québec sont multiples et méritent, en concordance avec la volonté de certains groupes comme ceux de la Marche Dyke, une meilleure visibilité.
Cette visibilité pourrait offrir aux individus d'orientation ou de genre variés de retrouver les gens qui leur ressemble et obtenir des ressources adaptées à ceux et celles-ci, que ce soit des lieux de socialisation comme les bars réputés sécuritaires ou des cliniques offrant des soins particuliers.

Également, nous avons peu été traité d'un ensemble d'espaces dans cette recherche, soit les villes régionales et villages où s'organisent ou vivent des individus des minorités sexuelles.


Un travail de plus grande ampleur dans le champ géohistorique ouvrirait la possibilité à une étude plus approfondie des géosymboles, mais, en l'absence de ces données, il est difficile de décrire plus particulièrement les autres villes Québécoises possédant une communauté de minorités sexuelles.
On peut toutefois nommer les villes de Rimouski, Gatineau, Saguenay et Trois-Rivières comme candidates à une analyse plus approfondie.
Ces villes, par leur inscription au sein d'une structure régionale urbanisée et par leur proximité à d'autres centres urbains importants de l'est du Canada, nous apparaissent comme candidates intéressantes pour un travail subséquent.
En effet, comme nous l'avons décrit dans les chapitres précédents, nous possédons des données recensées dans ces diverses villes, soit des géosymboles ou des adresses et des contacts prouvant l'existence de telles communautés.

%%% Local Variables:
%%% mode: latex
%%% TeX-master: "../../memoire-maitrise"
%%% End:

            % conclusion


%\bibliography{}                 % production de la bibliographie
%\printbibliography[nottype=manual,nottype=online]
%\printbibliography[type=manual,type=online,sorting=nyt]
\printbibliography%[sorting=nyt]



\appendix                       % annexes le cas échéant

%!TEX root = memoire-maitrise.tex
\chapter{Matériel de collecte de données}     % numérotée

% Please add the following required packages to your document preamble:
% \usepackage{booktabs}
% \begin{table}[]
% \centering
\begin{longtable}{p{.20\textwidth}  p{.80\textwidth}}

% \begin{tabular}{@{}ll@{}}
\toprule
Question                   & Réponse                                                                                                                       \\ \midrule
Ville                      & \begin{tabular}[c]{@{}l@{}}\text{\Circle} Montréal\\ \text{\Circle} Québec\end{tabular}                                                                  \\ \midrule
Nom                        & \emph{texte}                                                                                                                       \\ \midrule
Médium                     & \begin{tabular}[c]{@{}l@{}}\text{\Circle} Événement\\ \text{\Circle} Mobilier urbain\\ \text{\Circle} Bâtiment\end{tabular}                                          \\ \midrule
Mixité                     & \begin{tabular}[c]{@{}l@{}}\text{\Circle} Mixte\\ \text{\Circle} Non-mixte\end{tabular}                                                                  \\ \midrule
Type de non-mixité         & \begin{tabular}[c]{@{}l@{}}\text{\Square} Trans\\ \text{\Square} Lesbienne\\ \text{\Square} Personnes de couleur\\ \text{\Square} Autochtone\\ \text{\Square} Autre\end{tabular} \\ \midrule
Événement connu            & \begin{tabular}[c]{@{}l@{}}\text{\Circle} Fierté 2015\\ \text{\Circle} Fierté trans\\ \text{\Circle} Pervers/Cité\\ \text{\Circle} Qouleur\\ \text{\Circle} Non/Autre\end{tabular}           \\ \midrule
Persistance                & \begin{tabular}[c]{@{}l@{}}\text{\Circle}  Permanent\\ \text{\Circle}  Temporaire\end{tabular}                                                             \\ \midrule
Notes écrites              & \emph{texte}                                                                                                                       \\ \midrule
Notes audios               & \emph{enregistrement au dictaphone}                                                                                                \\ \midrule
Environnement sonore       & \emph{enregistrement au dictaphone}                                                                                                \\ \midrule
Photographie du géosymbole & \emph{fichier matriciel}                                                                                                           \\ \midrule
Temps                      & \emph{généré automatiquement}                                                                                                      \\ \midrule
Longitude                  & \emph{généré automatiquement} \\ \midrule
Latitude                   & \emph{généré automatiquement}                                                                                                      \\ \bottomrule
\caption[Questionnaire pour \anglais{Cloud GIS}]{Questionnaire construit sur l'application Cloud GIS pour la collecte de données sur terrain}
\label{ann:cloudgis}
\end{longtable}
% \end{tabular}
% \end{table}

%%% Local Variables:
%%% mode: latex
%%% TeX-master: "memoire-maitrise"
%%% End:
                % annexe A


\end{document}
