%!TEX program = xelatex

\documentclass[MScGeogr,nobabel,nonatbib,12pt]{ulthese}
  %% Encodage utilisé pour les caractères accentués dans les fichiers
  %% source du document. Les gabarits sont encodés en UTF-8. Inutile avec
  %% XeLaTeX, qui gère Unicode nativement.
  \ifxetex\else \usepackage[utf8]{inputenc} \fi

\usepackage{xunicode} % Pour l'encodage UTF-8
\usepackage{polyglossia} % Équivalent de Babel
\usepackage[autostyle]{csquotes}
\setdefaultlanguage{french} % Set default language for the Polyglossia package
\setotherlanguage{english}
%\usepackage{natbib}
% Start of 'ignore natbib' hack
%\let\bibhang\relax
%\let\citename\relax
%\let\bibfont\relax
%\let\citeauthor\relax
%\let\Citeauthor\relax
%\let\citeyear\relax
%\expandafter\let\csname ver@natbib.sty\endcsname\relax
% End of 'ignore natbib' hack
\usepackage[style=authoryear,sortcites,backend=biber,maxbibnames=99,natbib=true]{biblatex}

% Siècles
\def\siecle#1{\textsc{\romannumeral #1}\textsuperscript{e}~siècle}
\def\anglais#1{\textit{#1}}
\def\latin#1{\textit{#1}}
\def\code#1{\texttt{#1}}

% Gestion des acronymes
%\usepackage{acronym}
\usepackage{silence}
\WarningFilter{glossaries}{Overriding \printglossary}
\WarningFilter{glossaries}{overriding `theglossary'}
\usepackage[acronym,xindy,sort=def,toc,smallcaps,nomain,nonumberlist]{glossaries}
\usepackage{enumitem}
\usepackage{xspace}
%\renewcommand{\theglossary}{\begin{description}[font=\normalfont]}
\makeglossaries
%!TEX root = memoire-maitrise.tex
% \newacronym{<label>}{\textsc{<abbrv>}}{<full>}
\newacronym{accm}{\textsc{Accm}}{\emph{AIDS Community Care Montreal}}
\newacronym{agq}{\textsc{Agq}}{Archives gaies du Québec}
\newacronym{alccva}{\textsc{Alccva}}{Centre communautaire pour les modes de vie alternatifs}
\newacronym{ascgcn}{\textsc{Ascgcn}}{Association Socio-Culturelle Gaie de la Capitale Nationale}
\newacronym{astteq}{\textsc{Astt}(e)\textsc{q}}{Action Santé Travesti(e)s \& Transsexuel(le)s du Québec}
\newacronym{atq}{\textsc{Atq}}{Aide aux trans du Québec}
\newacronym{bdsm}{\textsc{Bdsm}}{<< Bondage, discipline, domination, soumission, sadisme et masochisme >>}
\newacronym{cadqas}{\textsc{Cadqas}}{\emph{Computer Assisted Qualitative Data AnalysiS}}
\newacronym{ccglm}{\textsc{Ccglm}}{Centre communautaire des gais et lesbiennes de Montréal}
\newacronym{cocqsida}{\textsc{Cocq-Sida}}{Coalition des organismes communautaires québécois de lutte contre le sida}
\newacronym{cpavih}{\textsc{Cpavih}}{Comité des personnes atteintes du VIH du Québec}
\newacronym{csq}{\textsc{Csq}}{Centrale des syndicats du Québec}
%\newacronym{dj}{\textsc{Dj}}{Centrale des syndicats du Québec}
\newacronym{gps}{\textsc{Gps}}{Global Positioning System}
\newacronym{ggul}{\textsc{Ggul}}{Groupe gai de l'Université Laval}
\newacronym{grip}{\textsc{Grip}}{Groupe de recherche à intérêt public}
\newacronym{gris}{\textsc{Gris}}{Groupe régional d'intervention sociale}
\newacronym{itss}{\textsc{Itss}}{Infections transmissibles sexuellement et par le sang}
\newacronym{irc}{\textsc{Irc}}{\emph{Internet Relay Chat}}
\newacronym{lgbtqia2}{\textsc{Lgbtqia2}}{Lesbien(ne)s, gays, bisexuel-le-s, trans--, queer, intersexes, asexuel-le-s, bispirituel-le-s}
\newacronym{lstw}{\textsc{Lstw}}{\emph{Lez spread the word}}
\newacronym{mai}{\textsc{Mai}}{Musée d'arts interculturels}
\newacronym{mcm}{\textsc{Mcm}}{Monsieur Cuir Montréal}
\newacronym{mbr}{\textsc{Mbr}}{\emph{Montreal Bear Rendez-vous}}
\newacronym{miels}{\textsc{Miels}-Québec}{Mouvement D'informations \& D'entraide Dans La Lutte Contre Le Sida}
\newacronym{npd}{\textsc{Npd}}{Nouveau Parti Démocratique du Canada}
\newacronym{pflag}{\textsc{Pflag}}{\emph{Parents and Friends of Lesbians and Gays of Canada}}
\newacronym{rlq}{\textsc{Rlq/Qln}}{Réseau des lesbiennes du Québec}
\newacronym{rqda}{\textsc{Rqda}}{\emph{R's Qualitative Data Analysis}}
\newacronym{sida}{Sida}{Syndrome d'immuno-déficience humaine}
\newacronym{sig}{\textsc{Sig}}{Système d'informations géographiques}
\newacronym{vih}{\textsc{Vih}/Sida}{Virus d'immuno-déficience humaine / Syndrome d'immuno-déficience humaine}
\newacronym{uqam}{\textsc{Uqam}}{Université du Québec à Montréal}
\newacronym{uqar}{\textsc{Uqar}}{Université du Québec à Rimouski}
\newacronym{om}{OM}{Outremont}
\newacronym{ls}{LS}{LaSalle}
\newacronym{mr}{MR}{Mont-Royal}
\newacronym{vm}{VM}{Ville-Marie}
\newacronym{pm}{PM}{Le Plateau-Mont-Royal}
\newacronym{hs}{HS}{Hampstead}
\newacronym{so}{SO}{Le Sud-Ouest}
\newacronym{rp}{RP}{Rivière-des-Prairies-Pointe-aux-Trembles}
\newacronym{lc}{LC}{Lachine}
\newacronym{dv}{DV}{Dorval}
\newacronym{mn}{MN}{Montréal-Nord}
\newacronym{is}{IS}{L'Île-Bizard-Sainte-Geneviève}
\newacronym{kl}{KL}{Kirkland}
\newacronym{do}{DO}{Dollard-des-Ormeaux}
\newacronym{sv}{SV}{Senneville}
\newacronym{ac}{AC}{Ahuntsic-Cartierville}
\newacronym{cl}{CL}{Côte-Saint-Luc}
\newacronym{ln}{LN}{Saint-Léonard}
\newacronym{mo}{MO}{Montréal-Ouest}
\newacronym{pc}{PC}{Pointe-Claire}
\newacronym{id}{ID}{L'Île-Dorval}
\newacronym{mh}{MH}{Mercier-Hochelaga-Maisonneuve}
\newacronym{cn}{CN}{Côte-des-Neiges-Notre-Dame-de-Grâce}
\newacronym{ro}{RO}{Rosemont-La Petite-Patrie}
\newacronym{lr}{LR}{Saint-Laurent}
\newacronym{bf}{BF}{Beaconsfield}
\newacronym{vs}{VS}{Villeray-Saint-Michel-Parc-Extension}
\newacronym{wm}{WM}{Westmount}
\newacronym{me}{ME}{Montréal-Est}
\newacronym{aj}{AJ}{Anjou}
\newacronym{pr}{PR}{Pierrefonds-Roxboro}
\newacronym{bv}{BV}{Sainte-Anne-de-Bellevue}
\newacronym{vd}{VD}{Verdun}
\newacronym{bu}{BU}{Baie-d'Urfé}


% Macros (surtout des acronymes)
\newcommand\agq{\gls{agq}\xspace}
\newcommand\alccva{\gls{alccva}\xspace}
\newcommand\ascgcn{\gls{ascgcn}\xspace}
\newcommand\bdsm{\gls{bdsm}\xspace}
\newcommand\atq{\gls{atq}\xspace}
\newcommand\astteq{\gls{astteq}\xspace}
\newcommand\cadqas{\gls{cadqas}\xspace}
\newcommand\ccglm{\gls{ccglm}\xspace}
\newcommand\ggul{\gls{ggul}\xspace}
\newcommand\gps{\gls{gps}\xspace}
\newcommand\grip{\gls{grip}\xspace}
\newcommand\irc{\gls{irc}\xspace}
\newcommand\lgbt{\gls{lgbtq}\xspace}
\newcommand\miels{\gls{miels}\xspace}
\newcommand\mtl{Montréal\xspace}
\newcommand\qc{Québec\xspace}
\newcommand\dyke{\emph{dyke}\xspace}
\newcommand\dykes{\emph{dykes}\xspace}
\newcommand\Dyke{\emph{Dyke}\xspace}
\newcommand\Dykes{\emph{Dykes}\xspace}
\newcommand\dm{\emph{Dyke March} de Montréal\xspace}
\newcommand\mai{\gls{mai}\xspace}
\newcommand\qu{\emph{queer}\xspace}
\newcommand\qus{\emph{queers}\xspace}
\newcommand\rlq{\gls{rlq}\xspace}
\newcommand\rqda{\gls{rqda}\xspace}
\newcommand\sida{\gls{sida}\xspace}
\newcommand\sig{\gls{sig}\xspace}
\newcommand\vih{\gls{vih}\xspace}
\newcommand\uqam{\gls{uqam}\xspace}
\newcommand\uqar{\gls{uqar}\xspace}
\newcommand\fugues{\emph{Fugues}\xspace}
\newcommand\sortie{\emph{Sortie}\xspace}
\renewcommand{\arraystretch}{1.5}

% Source pour les figures
\newcommand{\source}[1]{\caption*{\hfill Source: {#1}} }

% Annotations
\newcommand{\note}[1]{\color{red}(#1!)\color{black}} % Note importante, en rouge
\newcommand{\missref}{\note{[REF]}} % Besoin d'une référence
\newcommand{\todo}[1]{\textcolor{blue}{[Todo: #1]}} % À faire

% Pour l'interligne
\usepackage{setspace}

\graphicspath{{images/}} % Définir un dossier par défaut pour les images
%\usepackage{float}
%\usepackage{subfig}
\usepackage{subcaption} % Plus de legendes possibles
%\usepackage{caption}
\usepackage{wasysym} % Cases à cocher
\usepackage{longtable} % Tableaux longs

  %% Utilisation d'une autre police de caractères pour le document.
  \setmainfont{TeX Gyre Pagella}      % texte en Pagella (Palatino)
  %\setmathfont{TeX Gyre Pagella Math} % mathématiques en Pagella (Palatino)
  %\setmainfont{TeX Gyre Termes}       % texte en Termes (Times)
  %\setmathfont{TeX Gyre Termes Math}  % mathématiques en Termes (Times)

  %% Gestion des hyperliens dans le document. S'assurer que hyperref
  %% est le dernier paquetage chargé.
  \usepackage{hyperref}
  \hypersetup{colorlinks,allcolors=ULlinkcolor}

  %% Options de mise en forme du mode français de babel. Consulter la
  %% documentation du paquetage babel pour les options disponibles.
  %% Désactiver (effacer ou mettre en commentaire) si l'option
  %% 'nobabel' est spécifiée au chargement de la classe.
  %\frenchbsetup{%
  %  CompactItemize=false,         % ne pas compacter les listes
  %  ThinSpaceInFrenchNumbers=true % espace fine dans les nombres
  %}

  %% Style de la bibliographie.
  %\bibliographystyle{}
\setcounter{biburlnumpenalty}{100}
\addbibresource{~/Dropbox/bibliographie/literature_repository/library.bib} % Name of the file (without the .bib extension) to use as source for bibliography references
%\DeclareFieldFormat{url}{\space\url{#1}}
%\DeclareFieldFormat{urldate}{\addcomma\space\bibstring{urlseen}\space#1}
%\setcounter{biburlnumpenalty}{100} % Permet une brisure dans les DOI
\DeclareSourcemap{%
  \maps[datatype=bibtex]{%
    \map{%
      \pertype{article}
       \step[fieldset=issn, null]
       \step[fieldset=url, null]
    }
    \map{%
      \pertype{collection}
       \step[fieldset=url, null]
    }
    \map{\pertype{book}
       \step[fieldset=url, null]
       \step[fieldset=doi, null]
    }
    \map{\pertype{online}
%       \step[fieldset=url, null]
       \step[fieldset=doi, null]
    }
    \map{%
      \step[fieldset=language, null]
    }
  }
}
\DefineBibliographyStrings{french}{%
  urlseen = {visité le},
  in = {},
  mathesis = {Mém.\ de maît.}
}
%\defbibfilter{ressources}{ % Créer une catégorie bibliographique pour "outils" (permet d'exclure)
  %type=manual or
  %type=online
%}

  %% Déclarations de la page titre. Remplacer les éléments entre < >.
  %% Supprimer les caractères < >. Couper un long titre ou un long
  %% sous-titre manuellement avec \\.
  \titre{Au-delà du territoire}
  % \titre{Ceci est un exemple de long titre \\
  %   avec saut de ligne manuel}
  \soustitre{Espaces et géosymboles de la diversité sexuelle}
  % \soustitre{Ceci est un exemple de long sous-titre \\
  %   avec saut de ligne manuel}
  \auteur{Frédéric Vachon}
  \programme{Maîtrise en Sciences géographiques}
  \annee{2016}

\begin{document}
\frontmatter                    % pages liminaires
\pagetitre                      % production de la page titre
\chapter*{Résumé}                      % ne pas numéroter
\phantomsection\addcontentsline{toc}{chapter}{Résumé} % inclure dans TdM

\begin{otherlanguage*}{français}
  Ce travail de recherche vise à recenser les géosymboles utilisés par les communautés gaies, lesbiennes, bisexuelles, transgenres, queers et d'autres identités liées à l'orientation sexuelle ou au genre pour marquer leur territoire.
  Cette démarche s'appuie sur les travaux théoriques de nombreux auteurs en géographie culturelle et tente d'établir un lien avec les avancées de la théorie queer et de la géographie sexuelle.
  À l'aide de données collectées dans les villes de Québec et de Montréal durant l'été 2015, des médias Fugues et Sortie, de documents d'archives et des réseaux sociaux, il a été possible d'établir un portait diversifié des différents territoires de ces communautés, surtout en milieu urbain.
  Il ressort que Montréal possède de nombreux territoires, le Village gai n'étant qu'un de ceux-ci parmi d'autres, certains étant éphémères et d'autres plus permanents selon les fonctions.
  D'autres espaces existeraient également en dehors de la métropole, et si l'on retrouve des traces de ceux-ci dans les médias étudiés, cette recherche ouvre la porte à une plus grande exploration des espaces ruraux et des milieux urbains plus modestes.
\end{otherlanguage*}
                % résumé français
\chapter*{Abstract}                      % ne pas numéroter
\phantomsection\addcontentsline{toc}{chapter}{Abstract} % inclure dans TdM

\begin{otherlanguage*}{english}
  Text of English abstract.
\end{otherlanguage*}
              % résumé anglais
\cleardoublepage
\tableofcontents                % production de la TdM
\cleardoublepage
\listoftables                   % production de la liste des tableaux
\cleardoublepage
\listoffigures                  % production de la liste des figures
\cleardoublepage
\printglossary[type=\acronymtype,title=Liste des acronymes,toctitle=Liste des acronymes] % production de la liste des acronymes
\cleardoublepage
\dedicace{Dédicace si désiré}
\cleardoublepage
\epigraphe{Texte de l'épigraphe}{Source ou auteur}
\cleardoublepage
\chapter*{Remerciements}         % ne pas numéroter
\phantomsection\addcontentsline{toc}{chapter}{Remerciements} % inclure dans TdM

% Caroline Desbiens
% Marie-Hélène Vandersmissen
% Julie Podmore
% Olivier
% Charles
% Joëlle
% Les bénévoles de la fête arc-en-ciel
% Laurence Simard-Gagnon
% Benoît Lalonde
         % remerciements
%!TEX root = memoire-maitrise.tex
\chapter*{Avant-propos}         % ne pas numéroter
\phantomsection\addcontentsline{toc}{chapter}{Avant-propos} % inclure dans TdM

\section{Problématique}
\label{sec:problematique}
\todo{Section entière à retravailler}


\subsection{Énoncé du problème}
\label{sub:enonce_du_probleme}
\todo{à refaire, étant donné que celle du plan de recherche se retrouve dans
  tout ce chapitre. Peut-être faire un résumé? Prendre les objectifs de
  l'introduction et les rapatrier ici?}

Ce mémoire de maîtrise s'inscrit dans la suite d'un cheminement qui m'a amené à me pencher scientifiquement sur un ensemble plus large, un groupe d'individu auquel je considère moi-même appartenir, la communauté \lgbt{}, souvent appellée gaie et lesbienne, à laquelle j'ajoute les individus queers et les asexuels/alliés.
Nous voulons ici présenter les différentes façons qu'ont ces communautés d'utiliser l'espace au Québec et comment elles marquent les territoires qu'elles investissent pour montrer leur présence et surtout, pour construire un espace à leur image.
Ainsi, nous essayerons également de nous intéresser aux raisons pour lesquelles ces orientations sexuelles et ces identités de genre ont modelées les espaces qu'elles fréquentent et pourquoi on ne peut parler d'un simple groupe homogène comme l'acronyme \lgbt{} ou ses nombreuses variantes peuvent le laisser penser.
Plutôt, comme nous le verrons, si les façons de marquer l'espaces sont particulièrement variées, les groupes qui sont représentés le sont tout autant et ces derniers ne cohabitent pas nécessairement.

Ce mémoire fait suite à un essai qui a été produit afin de recenser l'ensemble des lieux \lgbt{} dans la ville de Québec.
Nous voulions d'abord débuter notre travail de recherche en nous intéressant à une ville importante mais ne présentant pas les caractéristiques des métropoles souvent abordées dans les études \lgbt{}.

\section*{Objectifs et hypothèse}
Ce mémoire vise rendre compte de notre recherche visant à dresser un portrait des géosymboles des espaces \emph{queers} en milieux urbains, plus particulièrement ceux des villes de Montréal et de Québec, mais aussi des villes de plus petites envergure.
Pour arriver à atteindre ce but général, nous nous sommes également fixé des objectifs plus spécifiques.
D'abord, cette recherche a permis de développer une méthodologie permettant l'identification et le catalogage des géosymboles marquants les espaces \emph{queers}, en tenant compte du caractère éphémère ou permanent, matériel ou immatériel de ceux-ci.
Ensuite, nous avons pris à tâche de rechercher et répertorier les géosymboles \emph{queers} en tenant en compte de l'environnement et de la forme de ceux-ci dans une variété relative de milieux urbains (métropolitains et non métropolitains) pour par après les géolocaliser.
Nous avons aussi voulu dresser les différences et ressemblances entre les différentes expressions spatiales et symboliques de la diversité sexuelle, notamment selon le genre et l'orientation sexuelle.
Enfin, nous avons cherché à faire progresser la connaissance portant sur les espaces \lgbt{} urbains de la province du Québec en prenant en premier comme terrain d'étude les villes de Québec et de Montréal puis les villes de taille inférieures selon la présence d'une communauté \lgbt{}.
La figure~\ref{fig:arrondissementsmtl} montre l'emplacement de ces deux arrondissements et nous verrons dans les chapitres suivants où ces espaces ont été situés en complément des données accumulées dans cette recherche \todo{approfondir le contexte historique?}

\begin{figure}[ht]
 \centering
 \includegraphics[width=1\textwidth]{arrondissementsmtl}
 \caption[Arrondissements ciblés: ville de Montréal]{Arrondissements ciblés pour la collecte de données: ville de Montréal}\label{fig:arrondissementsmtl}
\end{figure}


\subsection*{Question de recherche}
\label{sub:hypothese}
\todo{À simplifier}
%L'ensemble des objectifs précédents ont servi à répondre à la question de recherche que nous exposerons dans les prochaines lignes.
Pour répondre à ces différents objectifs, nous avons articulé la question de recherche suivante: comment les communautés \lgbt{} articulent-elles une relation particulière avec l'espace, et comment celles-ci s'affichent-elles dans celui-ci?
Nous supposons que les groupes et individus \lgbt{} occupent l'espace d'une façon particulière et que cette manière d'occuper l'espace sert à se regrouper autour d'une identité partagée et de l'afficher.
Cette occupation s'exprime par une variété de géosymboles signifiant la différence ou l'inclusion des individus selon cette identité.
L'existence de ces espaces, et des géosymboles les identifiant est permanente ou temporaire, selon les modalités particulières de l'espace.
Ces géosymboles marquants les espaces des communautés \lgbt{} sont récents; leur apparition serait survenue à partir de la \emph{révolution sexuelle}, une période durant laquelle les normes sexuelles se seraient en apparence assouplies.
Plus précisément, pour les communautés \lgbt{}, ce changement débuterait à partir des émeutes de Stonewall aux États-Unis et se serait répandu ailleurs en occident et donc au Québec.
%AJOUT-----------------------------
Ces géosymboles se seraient propagés d'abord dans les milieux urbains selon des paramètres propres à l'histoire des géosymboles, du milieu urbain, de la taille de ce milieu et de sa place dans la hiérarchie urbaine.

Pour conclure, le prochain chapitre servira à confirmer ces choix d'objectifs de recherche par un approfondissement de la problématique de recherche.
           % avant-propos

\mainmatter                     % corps du document
%!TEX root = memoire-maitrise.tex
\chapter*{Introduction}         % ne pas numéroter
\phantomsection\addcontentsline{toc}{chapter}{Introduction} % inclure dans TdM

%\chapterprecishere{<< Everyone needs a place. It shouldn't be inside of someone else.” \par\raggedleft--- \textup{Richard Siken}, Crush}
Avant d'entamer mes études universitaires à Québec, j'ai vécu loin des grandes villes, d'abord dans la région de Bellechasse, puis durant une bonne partie de mon enfance et de mon adolescence dans la ville de Rimouski, dans l'est du Québec.
Rimouski n'est pas la ville régionale typique organisée autour d'une industrie particulière; au contraire, il s'agit d'une ville réputée pour ses services et sa proximité avec le fleuve, possédant de nombreuses institutions gouvernementales et d'éducation.
Sans vouloir porter de jugement sur ce type de ville, les valeurs étaient plus traditionnelles que dans une ville comme Montréal, mais restaient tout de même assez progressistes comparativement à l'image répandue des villages et villes industrielles.
Par contre, pour celles et ceux dont les préférences, les valeurs, ou les comportements débordaient un tant soit peu de la norme, les espaces où se retrouver étaient peu nombreux, à moins que cette déviance soit fondée sur l'attachement à un style musical comme la culture punk ou metal.
Toutes deux étaient bien représentées durant mon adolescence; j'y ai d'ailleurs participé.
Par contre, si la déviance était basée sur le fait d'être homosexuel\ldots{}

Adolescent, discuter du sujet de la diversité sexuelle, c'était une chose possible que durant les cours d'éducation sexuelle, quelques fois par années, par un ou une enseignante plus ou moins intéressée.
Même s'il s'agissait d'un début, c'était tout de même mieux à certains égards qu'aujourd'hui; ces cours ont été annulés pour être remplacés par des interludes par-ci par-là dans les cours considérés comme plus sérieux, comme ceux de français, de biologie, etc.
Lorsque venait la puberté et qu'on commençait à se poser des questions sur ses préférences sexuelles, sur nos amours, il y avait peu de place vers où se tourner.
Il restait l'Internet, avec ses premiers sites de rencontres rudimentaires (mais explicites) pour hommes de 18 ans et plus et les canaux \irc{} (beaucoup plus austères) où il était possible de publier sa petite annonce, en espérant trouver l'amour ou, la plupart du temps, une relation sexuelle sans lendemain.
Une fois la décision prise de se rencontrer, plusieurs options s'offraient à nous; on allait chez l'autre s'il habitait un appartement seul, on se retrouvait dans le Tim Hortons du centre-ville pour faire connaissance, ou l'on se retrouvait en dehors de la ville, en forêt sur le bord de la rivière Rimouski où d'autres se rencontraient aussi, chacun dans sa voiture, les fenêtres embuées.
J'ai appris tardivement qu'il s'organisait des soirées dans une taverne près du cégep, puis dans une autre, après que la précédente ferma ses portes pour être rouverte par le même propriétaire sur une autre rue.
À moins de s'informer par Internet, rien n'indiquait clairement que, cette soirée-là, il était possible de trouver des personnes à l'orientation sexuelle similaire.
Il n'y avait pas vraiment de femmes lesbiennes ou d'individus s'identifiant socialement comme trans.
On y voyait par contre tel chauffeur de taxi, ou tel ami du secondaire, et l'on rencontrait un nombre non négligeable de nouvelles personnes qu'on ne croyait pas avoir jamais croisée, malgré la taille modeste de la ville.
Enfin, j'ai aussi entendu parler de la discothèque rétro du centre-ville où plusieurs personnes non-straights allait danser sans que lieu soit considéré comme le bar gai de ville.
Il faut dire également que c'est là que les mineurs allaient essayer d'entrer avant d'avoir atteint la majorité et où certaines personnes plus âgées, femmes ou hommes, s'intéressaient aux jeunes qui y dansaient.

Tous ces lieux, je les ai connus ou fréquentés il y a plusieurs années, en les découvrant par le bouche-à-oreille (numérique comme matériel).
Peu de choses les identifiaient, et le reste de la population rimouskoise pouvait passer à proximité de ces espaces, le jour comme la nuit, sans nécessairement remarquer qu'il s'agissait d'espaces queers.
Mais, quand je suis arrivé à la ville de Québec, quand j'ai été à Montréal dans le Village gai, ou, plus récemment, quand j'ai été à New York ou à Philadelphie dans les centres-villes, j'ai pu apercevoir des drapeaux multicolores, ceux qu'on voyait parfois à la télévision et un peu partout sur Internet (en allant sur les bons sites).
Le symbole de l'arc-en-ciel, bien que représentant l'ensemble de la communauté \lgbt{}, m'est toujours apparu comme symbole identifiant la communauté gaie masculine, et parfois aussi les femmes lesbiennes, plus que les autres.
Bien que sachant l'existence des personnes trans, intersexes et bisexuelles, celles-ci ne semblaient pas présentes où je posais mon regard.
S'agissait-il d'un phénomène de rareté, ou n’étais-je tout simplement pas apte à regarder aux endroits, à m'intéresser aux bonnes personnes?
Ou peut-être que je considérais mon prochain comme étant hétérosexuel ou homosexuel, sans autres issues possibles?

L'idée d'écrire ce mémoire m'est venue durant mes études au baccalauréat en géographie.
Domaine réputé analyser l'espace, souvent selon l'un ou l'autre des axes physiques ou humains, peu laissait penser qu'il était possible de s'intéresser à des groupes, à des personnes plus marginalisées.
J'ai eu la chance de rencontrer certaines personnes, certains groupes, qui m'ont permis de constater que cette possibilité existait bel et bien.
Il m'est donc apparu envisageable d'utiliser cette discipline pour répondre à ces interrogations que j'ai cultivé durant bien des années.

Je m'inscris, dans cette recherche, dans le domaine de la géographie culturelle, en usant des travaux déjà effectués auprès des communautés \lgbt{}.
Les intérêts associés à cette recherche sont multiples.
Sans prétendre innover sur une base conceptuelle, cette étude pourrait permettre d'abord de conjuguer un champ de la géographie culturelle, l'étude des géosymboles, à la géographie \emph{queer} ou sexuelle.
Il deviendrait possible de développer une méthodologie pour l'étude des géosymboles à la jonction de ces deux champs, et enfin, de mettre à l'étude des espaces, les villes Québécoises, rarement traitées comme sujet dans les études gaies et lesbiennes jusqu'à maintenant, sauf exception \parencite{Chamberland1993a,Podmore2006,Podmore2001,Hebert2012,Hunt2008,Laprade2014}.

Il sera possible de développer une méthodologie pour la reconnaissance, la description et l'usage des géosymboles des espaces \emph{queers} en milieu urbain.
Il apparait important de dépasser le concept de territoire au sein de la
géographie.
Si les géosymboles demeurent pertinents comme il sera démontré dans la partie~\ref{sec:problematique}, il convient de complexifier l'usage de ceux-ci en géographie.
Ceci passe par une prise en compte des éléments immatériels ou temporaires dans l'espace, sachant que certains groupes culturels ne possèdent pas d'assises territoriales stabilisées et tangibles, mais font plutôt usage de l'espace de manière ponctuelle ou subversive \parencite{Talburt2012}.

Le domaine de la géographie sexuelle n'est pas un domaine d'étude courant dans
la recherche francophone en géographie: la majeure partie des travaux sont effectués en anglais dans des villes Américaines ou en Europe, en Belgique ou en France \parencite{Blidon2010,Blidon2006,Cattan2010,Deligne2006}.
Moins répandus aussi, on retrouve de plus en plus de travaux de recherches dans d'autres pays développés ou postcoloniaux dans la littérature scientifique anglophone, en Chine ou à Singapour par exemple \parencite{Oswin2014a,Kong2012}.
Cela pose un problème sachant que des différences culturelles majeures existent entre États et régions.
Il apparait donc nécessaire d'enrichir le domaine de la géographie sexuelle au Québec, un territoire peu étudié en français notamment en dehors de la ville de Montréal.


\section*{Objectifs et hypothèse}
Ce mémoire vise rendre compte de notre recherche visant à dresser un portrait des géosymboles des espaces \emph{queers} en milieux urbains, plus particulièrement ceux des villes de Montréal et de Québec, mais aussi des villes de plus petites envergure.
Pour arriver à atteindre ce but général, nous nous sommes également fixé des objectifs plus spécifiques.
D'abord, cette recherche a permis de développer une méthodologie permettant l'identification et le catalogage des géosymboles marquants les espaces \emph{queers}, en tenant compte du caractère éphémère ou permanent, matériel ou immatériel de ceux-ci.
Ensuite, nous avons pris à tâche de rechercher et répertorier les géosymboles \emph{queers} en tenant en compte de l'environnement et de la forme de ceux-ci dans une variété relative de milieux urbains (métropolitains et non métropolitains) pour par après les géolocaliser.
Nous avons aussi voulu dresser les différences et ressemblances entre les différentes expressions spatiales et symboliques de la diversité sexuelle, notamment selon le genre et l'orientation sexuelle.
Enfin, nous avons cherché à faire progresser la connaissance portant sur les espaces \lgbt{} urbains de la province du Québec en prenant en premier comme terrain d'étude les villes de Québec et de Montréal puis les villes de taille inférieures selon la présence d'une communauté \lgbt{}.
% subsection objectifs_specifiques (end)

\subsection*{Question de recherche} % (fold)
\label{sub:hypothese}
\todo{À simplifier}
L'ensemble des objectifs précédents ont servi à répondre à la question de recherche que nous exposerons dans les prochaines lignes.
Nous supposons que les groupes et individus \lgbt{} occupent l'espace d'une façon particulière selon leur propre identité.
Cette occupation s'exprime par une variété de géosymboles signifiant la différence ou l'inclusion des individus selon, encore une fois, cette identité.
L'existence de ces espaces, et des géosymboles les identifiant est permanente ou temporaire, selon les modalités particulières de l'espace.
Ces géosymboles marquants les espaces des communautés \lgbt{} sont récents; leur apparition serait survenue à partir de la \emph{révolution sexuelle}, une période durant laquelle les normes sexuelles se seraient en apparence assouplies.
Plus précisément, pour les communautés \lgbt{}, ce changement débuterait à partir des émeutes de Stonewall aux États-Unis et se serait répandu ailleurs en occident et donc au Québec.
%AJOUT-----------------------------
Ces géosymboles se seraient propagés d'abord dans les milieux urbains selon des paramètres propres à l'histoire des géosymboles, du milieu urbain, de la taille de ce milieu et de sa place dans la hiérarchie urbaine.

Pour conclure, le prochain chapitre servira à confirmer ces choix d'objectifs de recherche par un approfondissement de la problématique de recherche.

%%% Local Variables:
%%% mode: latex
%%% TeX-master: "memoire-maitrise"
%%% End:

%!TEX root = ../../memoire-maitrise.tex
\chapter{Éléments conceptuels et problématique}
\label{cha:elements_conceptuels_et_problematique}

\chapterprecishere{\textquote{J’ai abandonné depuis longtemps l’idée qu’une
    vérité immanente se trouve dans la sexualité, qu’elle soit marquée par le
    péché ou l’émancipation. J’ai aussi abandonné l’idée qu’il existe quelque
    chose qu’on appelle \enquote{la sexualité}. Il existe plutôt des sexualités
    multiples, des sexualités dominantes et des sexualités
    marginalisées.} \par\raggedleft--- \textup{Anne Archet}, Sexe et liberté}

Ce chapitre servira à approfondir d'une part les concepts que nous avons utilisés dans cette recherche et à faire un tour d'horizon de l'état de la littérature dans le domaine de la géographie sexuelle pour développer la problématique.

\section{Revue de la littérature et principaux concepts}
\label{sec:revue_de_la_litterature_et_principaux_concepts} 

\todo{À utiliser ailleurs peut-être, Caroline recommande de commencer au paragraphe suivant. Également, il faudrait simplifier}
Étant donné la largeur relative de la recherche, nous avons cherché des définitions et des approfondissements aux concepts abordés d'une façon multidisciplinaire, sachant que certains de ces concepts ont été utilisés autant en géographie qu'en anthropologie et en sociologie, principalement ceux tournant autour de l'identité et de la symbolique. 
Afin de garder le focus sur notre sujet de recherche, sois la géosymbolique des espaces et territoires des minorités sexuelles, nous tenterons de contextualiser tout au long de notre texte les liens qu'ont ces concepts avec le sujet et quels sont les limites épistémologiques à lesquelles nous nous frotterons. 

Ce travail de recherche s’appuiera sur différents domaines de la littérature scientifique: d'abord, sur les étude \qus\ et plus particulièrement la géographie \qu\ et sur la sémiotique, en mettant l'accent encore une fois sur les liens avec la géographie. 

\todo{À déplacer?}Plus particulièrement, les études \qus\ sont un champ pluridisciplinaire des sciences sociales. 
Trouvant son origine d'abord dans le corpus de la \anglais{French Theory}, la théorie \qu\ est développée par un ensemble d'auteurs attachés au post-structuralisme dans des disciplines aussi diverses que la littérature, la philosophie ou les sciences sociales. 
Le terme \qu, d'abord utilisé par les mouvements sociaux \lgbt{} lors de la crise du \sida, est une réappropriation d'une insulte dirigée vers les homosexuels~\citep{Laprade2014}.
La théorie \qu\ traite particulièrement des normes sociales comme objet plutôt que sur des identités sexuelles particulières; le concept de négativité, de performativité et l'intersectionnalité sont particulièrement utilisés pour traiter des phénomènes de marginalisation.

\subsection{Le regard anthropologique sur la culture}
\label{subsec:le_regard_anthropologique_sur_la_culture} 
Le premier concept que nous aborderons est celui de la culture. 
Celle-ci étant polysémique selon le contexte et l'usage, nous voulons décrire rapidement ces différents usages pour arriver à comprendre comment cell-ci s'articule dans le domaine de la géographie sexuelle mais également dans un contexte de regard sur soi au sein des minorités sexuelles. 
En effet, nous le verrons plus loin, plusieurs positions sont débattues dans la communauté \lgbt{}, une s'inscrivant dans une identité forte et une autre dans une position négative et anti-normative.

% Dans le cadre de ce travail de recherche, nous nous inscrivons dans une 
% définition sémiotique de la culture inspirée des travaux structuralistes et
% post-structuralistes en anthropologie. Pour d'abord arriver à bien comprendre
% cette définition de la culture, nous allons nous pencher plus particulièrement
% sur le texte \textquote{La religion comme système culturel} de \citet{Geertz1972}.
% Malgré que le sujet principal du texte de Geertz est la religion et consiste en
% son analyse, nous pouvons y voir ici une description avancée d'un système
% culturel particulier dans lequel nous retrouvons les bases pouvant servir à la
% description d'autres systèmes culturels, comme celui de la communauté \lgbt{}.

Dans \citetitle{Geertz1972}, Geertz s'inscrit dans une critique des travaux précédents en anthropologie religieuse, laquelle comme sous-discipline de l'anthropologie serait stagnante au niveau théorique en citant continuellement les mêmes auteurs dont on remarque autant l'inspiration sociologique que anthropologique (en se basant autant sur les travaux de Durkheim que sur ceux de Malinowski par exemple~\citep[20]{Geertz1972}). 
Ce texte vise donc à proposer de nouvelles bases théoriques sur lesquelles l'analyse anthropologique pourrait s'approfondir et continuer à évoluer. 
Geertz précède donc son analyse des systèmes religieux\todo{À étoffert, que sont-ils?} par une définition renouvelée de ce qu'est la culture ou plutôt le système culturel. 
Pour lui: \blockquote[{\cite[21]{Geertz1972}}][.]{\textelp{} il désigne
  un modèle de significations incarnées dans des symboles qui sont transmis à
  travers l'histoire, un système de conceptions héritées qui s'expriment
  symboliquement, et au moyen desquelles les hommes [\latin{sic}] communiquent,
  perpétuent et développent leur connaissance de la vie et leurs attitudes
  devant elle}.

Si la culture apparaît bel et bien comme un système abstrait persistant dans le temps au-delà de la vie des individus composant la société, la définition offerte par Geertz ne laisse pas sous-entendre que nous avons affaire à un entité réifiée ou superorganique pour reprendre les termes de \citet{Duncan1980}. 
Les individus composant la société héritent donc des connaissances offertes par la culture pour arriver à comprendre le monde où les symboles qui s'y trouvent portent des significations particulières.

Les symboles plus particulièrement sont pour Geertz dans son modèle\todo{à étoffer}, en reprenant les termes de~\cite{Langer1962}: \textquote{\textelp{} tout objet, acte,
  événement, propriété ou relation qui sert de véhicule à un concept --- le
  concept est la \textquote{signification du symbole }~\citep[
  23--24]{Geertz1972}}. 
Il est important de souligner que les objets en soi que l'on pourrait assimiler à des symboles demeurent ce qu'ils sont matériellement; Geertz prend l'exemple d'une maison qui, si celle-ci peut être un objet concret sans significations autres que sa matérialité, peut également jouer ou non le rôle d'un symbole particulier selon le regard qu'on lui pose en tant qu'être humain appartenant à une culture particulière. 
Autrement dit, au-delà de sa matérialité, sa forme, sa position ou sa composition la maison peut être le témoignage d'un fait culturel particulier. 
On pourrait extrapoler en considérant que cette maison informe le public sur le statut social de la personne. 
\todo{à   reformuler:} Elle hériterait donc dans sa forme d'une forme architecturale propre à la culture dans laquelle elle s'inscrit.

Cette manière de donner forme aux choses matérielles ou abstraites, de leur donner une signification sous la forme de symbole est selon Geertz le fait des programmes fournis par les modèles culturels~\citep[25]{Geertz1972}. 
Ces modèles agissent en deux temps: d'abord, ils créent les symboles en se basant sur le réel, en prenant assise sur les structures non symboliques déjà existantes, ce que Geertz nomment des \emph{modèles de}. 
L'autre forme de modèle, les \emph{modèles pour} agissent plutôt en orientant les structures non symboliques et en créant des liens entre elles qui n'existent pas nécessairement au préalable~\citep[26--27]{Geertz1972}. 
Ces deux phénomènes sont en fait les deux manières qu'ont les modèles de donner sens aux symboles qu'ils contiennent; ils\todo{à revérifier} orientent la compréhension des symboles en calquant ceux-ci sur le réel non-symbolique et en liant ensemble les éléments composant ce réel.

Ces modèles culturels \todo{à préciser, réitérer on ne sait pas de quoi il s'agit}tels que décrits par Geertz s'inscrivent dans une contexte culturel réputé homogène. 
En effet, les différentes analyses faites par l'auteur traitent des religions todo{expliquer d'avantage} (l'objet d'analyse) de façon singulière ou encore sur la place de la religion dans la société de façon générale, sans traiter d'un contexte en particulier. 
Il s'agit d'une des limites à prendre en compte dans la suite du présent texte; en effet, Geertz n'entreprend pas dans cette recherche de traiter des effets propres aux mélanges culturels qui surviennent par exemple avec l'immigration ou la genèse de nouveaux phénomènes culturels, comme dans le cas qui nous intéresse les minorités \lgbt{}.

On trouve tout de même dans son texte des éléments orientant la relation particulière d'un individu et de la religion qui peut servir d'introduction à la suite de ce travail sur l'identité. 
En effet, Geertz, plus loin dans son texte, traite des dispositions propres à l'individu s'insérant dans un contexte culturel particulier. 
La culture transmets ces dispositions à effectuer certaines activités qui permettent à l'individu de s'identifier à la culture dont il fait partie au-delà des fonctions premières et des motivations derrières la pratique en question. 
Le principe d'identité est simple à comprendre dans le contexte religieux décrit dans le texte: l'individu religieux pratique dans sa vie la prière et d'autres activités religieuses selon une probabilité plus élevé avec l'intensité de son sentiment religieux. 
Il est important de souligner ici qu'on parle principalement de la probabilité d'un acte de survenir: l'acte en soi n'est pas une nécessité. 
En société, par exemple, on s'attend à ce que l'individu réputé religieux agissent selon certaines dispositions propre au mode de vie religieux~\citep[28--30]{Geertz1972}. 
Dans le contexte des identités \lgbt{}, on ne peut s'arrêter seulement au pratiques pour traiter d'une potentielle identité.

\todo{À lier avec Geertz}
Les individus \lgbt{}, à leur naissance, s'insèrent de facto dans un contexte culturel dont les symboles n'ont que peu a voir avec les identités sexuelles non-hétérosexuelles. 
En effet, dans le contexte occidental, les individus sont considérés comme hétérosexuels par défaut, à un point tel qu'il ne s'agit pas, en général, d'une identité particulière mais d'une norme, cette dernière étant escamotée\todo{revoir le terme?} par des identités nationales, régionales, \emph{ethniques}, etc.
Nous ne répondrons par immédiatement à ce problème; nous allons plutôt maintenant nous pencher sur l'identité comme concept et comment celui-ci a été manié et pensé par la théorie queer, pour ensuite revenir sur la place de l'identité chez les groupes \lgbt{}.
\todo{vérifier l'enchaînement des parties}

\subsection{Le sujet et identité}
\todo{Revoir le titre de la section}
\label{subsec:sujet_et_identité} Durant les dernières décennies en occident et ailleurs dans le monde, de nombreux groupes identitaires semblent avoir fait surface, surtout dans les milieux urbains. 
On peut penser notamment aux groupes ethniques nés de l'immigration, aux groupes d'intérêts envers des objets culturels particulier (genres musicaux, dessins animés, cinéma) et également aux groupes nés de l'identification à des pratiques sexuelles différentes où une non-coïncidence du genre de la personne avec celui fixé à la naissance. 
Le but de ce mémoire est de traiter des concepts de culture et d'identité en lien avec les individus appartenant au spectre \lgbt{} et d'aborder les enjeux particulier que pose l'étude géographique de cette partie de la population, en considérant le contexte historique récent de rassemblement politique et d'organisation communautaire auquel plusieurs des individus du spectre \lgbt{} ont participé au cours des dernières décennies, des émeutes de Stonewall à la crise du \sida\ des années 80 jusqu'à aujourd'hui\todo{Si on ne traite pas de Stonewall plus loin dans le texte, décrire l'événement ici}. 
En effet, nous tenterons de comprendre si la communauté \lgbt{} forme une culture en soi au sein d'un groupe culturel plus large et si l'orientation sexuelle peut être considérée comme une forme particulière d'identité au sein de ce groupe culturel particulier. 
% Étant donné
% le contexte dans lequel s'inscrit ce travail, nous tenterons, au su des
% conclusions des analyses précédentes, de tenter de traiter des enjeux
% géographiques entourant l'analyse de ce groupe culturel, ou des individus
% rattachés à ces ou cette identité(s).

Nous débuterons notre démarche par l'étude du concept de culture d'un point de vue symbolique en se basant sur les travaux récents de Clifford Geertz pour ensuite introduire le concept d'identité par les travaux de Stuart Hall\todo{revoir la formulation, vu qu'on traite de Geertz précédemment}. 
Par les deux définitions qui se dégageront du survol de ces textes, nous pourrons d'un côté comprendre l'évolution du concept d'identité d'un point de vue généalogique tout en ayant une définition de la culture et voir comment le premier concept s'articule avec le second. 
Nous nous pencherons ensuite sur les enjeux entourant les populations \lgbt{} du point de vue de la définition: avons-nous affaire à un groupe culturel particulier ou plutôt une série d'individus ne possédant en commun que des pratiques sexuelles similaires et une volonté relative d'intégration sociale commune? 
Enfin, nous tenterons de nous pencher sur les enjeux géographiques particuliers à prendre en compte pour l'analyse de la population \lgbt{}, autant d'un point de vue de la pratique sexuelle que sous la forme d'un groupe culturel et comment ces différents points de vue se sont manifestés au sein de ce champs disciplinaire.

% Mon projet de recherche dans le cadre de mon mémoire porte plus
% particulièrement sur une analyse des symboles \lgbt{} ou queers qu'on retrouve
% dans les espaces urbains. Ces symboles devraient permettre d'identifier les
% différents espaces queers dans les territoires connus pour abriter une
% population non-hétérosexuelle importante – comme le Village gai à Montréal –
% mais également ceux qu'on retrouve ailleurs ou temporairement et dont
% l'existence est peu connue en-dehors des groupes \lgbt{}.

\subsection{Le regard socio-historique sur l'identité}
\label{sec:le_regard_sociohistoirique_sur_l_identit_} Maintenant que nous avons une définition du concept de culture, nous allons traiter du concept d'identité tel que défini par~\citet{Hall1996a} dans le texte \citetitle{Hall1996a}. 
Hall propose une généalogie simplifiée du concept d'identité en lien avec les différentes étapes socio-économiques des derniers siècles et donc des courants de pensées y étant liés\todo{trop vague}. 
Cette généalogie permet de rendre compte de l'évolution de l'identité individuelle, en montrant que celle-ci est passée selon trois stades différents.

Le premier type identitaire selon \citeauthor{Hall1996a} est le sujet des lumières qui apparait à partir du \siecle{16}. 
Sans trop entrer dans détails de sa production, Hall souligne qu'il s'agit du sujet né des idées des lumières. 
On a affaire à un sujet dont la capacité principale est d'être et de penser en reprenant les idées de Rousseau dans ce cas-ci. 
Rattaché à la nation, son identité est peu développée et les conflits de classes ne sont pas encore clairement présents: on a affaire à un individu dont les caractéristiques, comme le statut social, sont conçues comme immuables et banales comparativement aux autres membres de la société~\citeyearpar[596]{Hall1996a}.

Le sujet sociologique, deuxième évolution du concept d'identité, nait en même temps que la sociologie devient une discipline autonome séparée du domaine de l'économie, des sciences politiques mais surtout de la psychologie. 
En effet, selon Hall, celui-ci apparait lorsque la psychologie et la psychanalyse prennent de l'importance et concentrent leur analyse sur l'individu dans son plus fort intérieur et ses relations particulières avec son environnement immédiat lors de la constitution de sa personnalité, à savoir les membres de sa famille et surtout ses parents. 
Le domaine du social devient l'apanage de la sociologie et l'identité agit ici comme l'intermédiaire entre cet individu réputé particulier par sa psychologie personnelle en relation avec le monde social, considéré non pas comme une foule d'individu mais plutôt un ensemble de structures économiques et sociales ayant une influence cruciale sur la place de l'individu dans la société. 
Ce sujet sociologique est donc en relation avec les autre individus ainsi qu'avec \textquote[{\citeyear[597]{Hall1996a}}][]{les valeurs, les
  significations et les symboles --- la culture --- du monde qu'il ou elle
  habite}. 
Ce sujet apparait également avec la constitution de l'individualisme. 
Il possède également des caractéristiques particulières très peu développées, alors que son essence est similaire à celle de ses concitoyens au sein de la nation et des individus appartenant à la même classe sociale. 
En somme, les systèmes symboliques prennent une importance qui dépasse les individus malgré qu'on commence à voir apparaître une complexification des identités possibles suite au bouleversement des institutions réputées immuables par le renversement bourgeois des traditions et des systèmes de royauté. \todo{Dater!}

Le sujet postmoderne serait le sujet le plus récent et celui cadrerait à notre époque, à partir de la fin du \siecle{20}. 
Selon \citeauthor{Hall1996a}, ce sujet aurait pris la place du sujet sociologique suite à une déstabilisation de l'identité portée par le sujet moderne et donc sa fragmentation au fil des générations. 
Hall recense cinq causes à cette déstabilisation qui se retrouvent dans divers travaux sur la société et l'individu. 
La première de ces causes provient des travaux de la théorie marxiste. 
En effet, dans celle-ci, la faculté d'action (\anglais{agency}) est remise en question par la reconnaissance de structures sociales et économiques ayant un impact particulièrement important sur le devenir de l'individu. 
Cet individu seul n'a d'ailleurs plus d'essence particulière: plutôt, il acquiert de l'extérieur nombreuse de ses caractéristiques particulières, bien souvent par la classe sociale de laquelle il est issu~\citeyearpar[606]{Hall1996a}.

Le deuxième décentrement proviendrait de la décomposition psychologique de l'individu, notamment dans les travaux de Freud et plus tard de Lacan. 
Cette décomposition remet radicalement son l'auteur la position de Rousseau sur la Raison individuelle: l'individu n'est maintenant plus maître de sa vie, mais possède en lui des pulsions aussi variées qu'incomprises et une partie de lui-même, héritée socialement, le dépasse et a une influence très importante sur sa propre vie. 
Les autres autour de lui, la famille d'abord, ont des rôles particulier dans sa vie qui l'amènent maintenant à ne plus se construire lui-même de façon autonome mais plutôt en relation avec les autres, selon des versions symboliques que ceux-ci représentent pour l'adulte à venir~\citeyearpar[ 607--608]{Hall1996a}.

La troisième cassure selon Hall est celle opérée par les travaux de Ferdinand de Saussure sur le langage. 
Selon lui, la langue et les mots ne sont en aucun cas la possession des individus, au contraire. 
Plutôt, les individus s'insèrent dans un système complexe de symboles et utilisent les mots pour transmettre des significations, des messages aux autres sans toutefois être certains de la réception de ceux-ci étant donné l'évolution rapide du sens des mots. 
Le langage dépasse donc chaque personne et est un fait de société qui les précède. 
La langue prend d'ailleurs place dans la psyché de chaque individu et structure bon nombre de ses pensées~\citeyearpar[608--609]{Hall1996a}

Les travaux de \citeauthor{Foucault2004a} sur le pouvoir disciplinaire consistent en la quatrième cassure. 
Principalement dans \citetitle{Foucault2004a}~\citeyearpar{Foucault2004a}, \citeauthor{Foucault2004a} dresse une généalogie des moyens disciplinaires utilisés par les sociétés occidentales pour faire justice en société. 
On y apprend que continuellement, les techniques disciplinaires vont devenir de plus en plus douces et passer d'un pouvoir strictement extérieur à l'individu, incarné dans la royauté d'abord, à un système judiciaire de plus en plus distant dans la société pour prendre en même temps place à l'intérieur de l'individu. 
Par des dispositifs de surveillance toujours plus élaborés, non seulement les criminels seront surveillés, mais également le reste de la société, dans les écoles et les hôpitaux par exemple grâce aux technologies développées dans les centres pénitenciers, les prisons, les cachots. 
Ces institutions, par leur pouvoir grandissant et leur capacité à surveiller, en viennent à acquérir suffisamment de connaissances sur les individus pour devenir des agent normatifs puissants~\citep[608--609]{Hall1996a}.

La cinquième et dernière cassure vient de la place grandissante qu'a joué le féminisme dans le monde occidental. 
Hall considère que ce sont les mouvements politiques et intellectuels qui ont permis cette cassure, en remettant en question les rapports de genres entre les individus d'abord et en ouvrant la porte à la contestation de nombreux groupes marginalisés ou dont les idées politiques remettaient en question le système social et politique au-delà de la lutte des classes. 
Ce mouvement permit la mise en place des identités politiques pour chacun de ces nouveaux mouvements de contestation et permit de nouvelles formes de contestation sociales et politiques~\citeyearpar[610]{Hall1996a}.


Comme on peut le voir, le sujet sociologique et le sujet post-moderne sont
particulièrement similaires sur plusieurs points: plusieurs des facteurs ayant
mis en place le sujet sociologique en sont venu à le déstabiliser avec les
décennies, quoique Hall ne fait pas une généalogie claire et datée de ces
déstabilisations; il faut plutôt s'appuyer sur les généalogies internes à chacun
des travaux ou évènements et leur moment d'apparition dans l'histoire moderne.
On doit plutôt concevoir ceux-ci d'une part comme des facteurs de
déstabilisation et des œuvres les expliquant, qui ont amené la mise en place du
sujet post-moderne. Celui-ci, dépassé par le langage, l'économie ou les rapports
de pouvoirs est de plus en plus individualisé par la société tout en y étant
enchevêtré. En ceci, comparativement au sujet des lumières, il n'est pas
nécessairement raisonnable et son expérience individuelle est beaucoup plus
importante dans sa constitution identitaire. Au-delà de la complexification des
rapports sociaux, cette individualisation a ouvert de nouvelles voies pour
l'individu lui permettant d'arriver à coordonner sa place en société. Pour
\citeauthor{Hall1996a}, ce sujet:
\foreigntextquote{english}[{\citeyear[598]{Hall1996a}}][]{\textelp{} assumes
  different identities at different times, identities which are not unified
  around a coherent \emph{self}. Within us are contradictory identities pulling
  in different directions, so that our identifications are continuously being
  shifted about. If we feel we have a unified identity from birth to death, it
  is only because we construct a comforting story or ``narrative of the self''
  about ourselves}. 
Nous pouvons croire que cette \emph{narration du soi} s'incarne non pas dans une rationalité objective et indépendante comme pour le sujet des lumières, mais plutôt dans une rationalité subjective. 
Cette rationalité subjective répondrait donc à deux défis: d'abord s'adapter à son environnement immédiat, spatial ou temporel, puis assembler en soi un récit qui arrive à surmonter certaines contradictions inhérentes aux multiples identités individuelles. 

Ce que nous appellons la rationalité subjective n'est pas une contruction stable et encore moins indépendante de l'individu.
Au-delà des contradictions internes, les autres individus peuvent également contredire ce récit et identités.
La politique pour Hall jouerait ce rôle:
\foreigntextquote{english}[{\citeyear[610]{Hall1996a}}][]{Since
  identity shifts according to how the subject is addressed or represented,
  identification is not automatic, but can be won or lost. It has become
  politicized. This is sometimes described as a shift from a politics of (class)
  identity to a politics \emph{difference}}. 
Ces changements importants chez l'individu ouvrent la porte à d'autres formes d'identités jusque là improbables.

Nous pouvons conclure cette partie en soulignant que ces chamboulements et les évènements historiques du dernier siècle ont ainsi permis l'éclosion des identités \lgbt{}, notamment par le traitement qu'a subi l'orientation sexuelle par le système de santé, par la psychanalyse et une distance du pouvoir qui permis l'apparition de groupes d'affinités autour de la question sexuelle. 
La reconnaissance de ces nombreuses identités dans la société et de l'intersection de celles-ci dans les individus doit être prise en compte dans une analyse géoculturelle de la société et des groupes culturels.
En effet, l'individu geerztien que nous avons traité plus tôt dans le chapitre ne serait plus dépendant à un seul système de symboles mais bien à plusieurs dont la géographie et la temporalité serait changeante.
L'incompréhension envers des contradictions ne créerait plus \textquote[]{une angoisse très forte dès qu'il sent que ces symboles peuvent ne pas pouvoir répondre à tel ou tel aspect de l'expérience}~\citep[33]{Geertz1972}, mais une adaptation et un changement d'identité.
Ainsi, on arrive d'une certaine façon à dépasser certaines des limites inhérentes à la vision \emph{geertzienne} de la culture dans laquelle l'attachement à une identité comme la religion pour un individu passait principalement par la pratique.
L'identité se construirait plutôt dans l'interaction avec autrui et la médiation de ces identités avec les forces politiques les régissant ou les contredisants.
Néanmoins, l'usage du terme d'identité n'est pas nécessairement aujourd'hui l'apanage des chercheurs et certains contextes, principalement dans le domaine du politique, permettent une auto-réflexivité sur ce statut d'identité.

\subsection{Diversité sexuelle et identité}
\label{sec:diversit_sexuelle_et_identit_} Comme nous l'avons vu dans la section précédente, l'identité de minorité sexuelle peut être considérée comme une création récente propre au contexte post-moderne des dernières décennies. 
Ce constat s'explique notamment par le contexte historique récent dans lequel s'inscrit cette identité ainsi que par la correspondance à certaines caractéristiques proposées par Hall, à savoir qu'il s'agit d'une identité dans laquelle les individus s'inscrivent en parallèle à d'autres formes identitaires, comme la communauté ethnique, nationale ou encore de genre, par exemple.
Nous introduiront pour la suire le travail de~\citet{Sinfield1996}, qui, dans \citetitle{Sinfield1996}, traite des difficultés et contradictions propres à l'usage de l'identité chez les communautés \lgbt{}.
Ce dernier met en rapport l'identité et les phénomènes historiques récents de libération sexuelle en occident.
Ce rapport complexe viendrait du développement d'un regard critique vis-à-vis l'essentialisation des communautés sexuelles au sein de certains des premiers travaux liés à la théorie queer qui tentèrent de dépasser la \emph{pathologisation} de la sexualité comme on la retrouvait dans les travaux en psychanalyse. 
En effet, le texte de Sinfield s'intéresse plus particulièrement aux considération stratégiques et historiques entourant la mise en place d'une identité homosexuelle pour la communauté elle-même et du point de vue des penseurs de l'identité sexuelle, notamment Foucault.

\subsubsection{Minorités et universel}
\label{sub:minorit_s_et_universel}
Le texte de Sinfield débute par la présentation de deux manières de voir l'homosexualité; d'abord, un point de vue de \emph{minoritarisation} ou de marginalisation qui voit l'homosexuel, gay ou lesbienne, comme un groupe d'individus ayant un style de vie particulier, tel un groupe ethnique.
L'universalisation le second point de vue possible lors de l'analyse de l'homosexualité voit celle-ci comme un comportement potentiel chez tout et un chacun: tous peuvent être à un moment ou à un autre commettre un acte homosexuel et il n'y a pas lieu de parler d'identité ou de culture comme on traiterait d'une groupe ethnique~\citep[271]{Sinfield1996}.

Le point de vue \emph{minoritarisant} va à l'encontre du point de vue constructiviste répandu dans les études sur le queer et l'identité sexuelle, inspirée des travaux \citet{Foucault2011}, de \citet{Rubin2010} et de \citet{Butler2007} en considérant les individus à la sexualité déviante, les homosexuel-e-s dans ce cas-ci, comme des groupes identitaires particuliers. 
Au contraire, le point de vue d'universalisation est en congruence avec le champs de pensée constructiviste; en effet, on voit la sexualité comme une donnée variable chez les individus dont la position sociale, l'éducation et l'environnement auront un effet prépondérant et dont le sens prendra une valeur différente selon la culture traitée. 
Contrairement au point de vue \emph{minoritarisant}, le point de vue universalisant voit l'homosexualité comme une attitude, une pratique sexuelle possible pour chaque individu, peu importe la culture. 
Cette dernière déterminera si la pratique homosexuelle est tolérée, encouragée ou discriminée et marginalisée~\citep[271]{Sinfield1996}.

Selon Sinfield, les gais et les lesbiennes ont historiquement pris une position stratégique les rapprochant du point de vue \emph{minoritarisant} en empruntant une dynamique de revendication et de lutte sociale similaire à celles des groupes ethniques, notamment des mouvements pour les droits civiques afro-américains~\citep[271]{Sinfield1996}. 
Sinfield nomme cette stratégie le cadre de l'ethnicité-et-des-droits (\anglais{ethnicity-and-rights}). 
Le développement de ce cadre, au-delà de la simple imitation des groupes ethniques, s'est fait dans le cadre de l'État de droit où, pour améliorer sa position sociale et réduire la marginalisation, les individus doivent, pour reprendre les termes de \citet{Sinfield1996}, \foreigntextquote{english}[{\citeyear[272]{Sinfield1996}}][.]{\textelp{} to   compartmentalize their complex subjectivities in order to \emph{make a claim}} (envers le pouvoir). 
Cette compartimentation de la subjectivité individuelle amène les individus touchés à mettre de l'avant une caractéristique d'eux-mêmes et donc de vivre une rapprochement avec les autres individus touchés et qui se reconnaissent dans cette identité potentielle. 
Ce cadre stratégique ne laisse pas entendre qu'il n'existait pas de groupes d'individus gays et lesbiens avant que ceux-ci revendiquent des droits selon Sinfield, mais bien que ces revendications ont amené ces groupes à se voir: \textquote{\textelp{} as gay in   the terms of a discourse of ethnicity-and-rights} ~\citep[272]{Sinfield1996} et que ces regroupement par affinités se sont mutés en groupe identitaires avec un poids politique. 
Sinfield souligne plusieurs problèmes dans la poursuite de ce cadre; d'abord cette nouvelle identité et cette genèse culturelle peut entrer en contradiction avec les autres identités assumées par les individus y prenant part. 
Elle désengage également le reste de la société de poser une réflexion profonde sur la sexualité comme le propose la pensée \emph{universalisante} qui fut assumée en partie par certains groupes plus radicaux (dont le mouvement queer, qui par définition vise une refonte radicale des normes sur la sexualité et le genre plutôt que l'acquisition de droits)~\citep[273]{Sinfield1996}.

Plus loin dans son texte, Sinfield explique les différences culturelles dans lesquelles ont eu lieu le développement de mouvements de contestations pour l'amélioration des conditions de vie des gays et lesbiennes, plus particulièrement entre les États-Unis et la Grande-Bretagne. 
Pour l'auteur, le cadre de l'ethnicité-et-des-droits se traduit de différentes manières selon la région étudiée: en Grande-Bretagne, la concession de droit s'inscrit dans la suite de l'État-providence, où l'état anglais concède des acquis supplémentaires dans la perspective d'assurer à tous les citoyens un mode de vie décent. 
Aux États-Unis, on s'inspire plutôt des valeurs traditionnelles américaines qui s'orientent surtout vers la liberté aux individus~\citep[274]{Sinfield1996}. 
En cherchant à obtenir cette liberté offerte par la société américaine, les groupes ethniques s'appuient du même coup sur ce que Sinfield nomme le mythe de la pluralité américaine, qui laisse entendre que chaque groupe culturel est égal et peut revendiquer un accès égal aux mêmes ressources que les autres dans un cadre compétitif. 
C'est ce mode stratégique qui se serait par la suite répandu dans les autres mouvements de contestations et qui aurait stratégiquement mis à l'avant-plan le modèle de l'ethnicité-et-des-droits.

\subsubsection{Diaspora et hybridité}
\label{sub:diaspora_et_hybridit_} Pour faire face au problème de la multiplicité des origines des individus appartenant à l'identité homosexuelle ou lesbienne, Sinfield propose de concevoir cette identité comme nécessairement hybride. 
Cette hybridité est conceptuellement analogue à l'hybridité qui s'impose aux individus appartenant à une diaspora, donc qui se retrouvent géographiquement à distance du lieux d'origine de leur culture d'appartenance. 
À titre d'exemple, on peut notamment penser à la diaspora juive ou encore la population afro-américaine.
Cette hybridité est pour l'auteur un moyen qu'ont ces individus de résister d'une part à l'assujettissement de leur identité avec leur milieu d'accueil et d'autre part conserver une partie de leur culture: \foreignblockquote{english}[{\cite[278]{Sinfield1996}}][.]{`Diaspora' \textelp{}   usually invokes a true point of origin, and an authentic line --- hereditary   and/or historical --- back to that. However, diasporic Black culture, Hall   says, is defined `not by essence or purity, but by the recognition of a   necessary heterogeneity and diversity; by a conception of ``identity'' which   lives with and through, not despite, difference; by hybridity'}

Cette hybridité peut donc être conçue comme participant à une forme d'ethnogenèse tout en possédant un potentiel politique: au lieu de répondre à certains archétypes que la société d'accueil impose sur l'identité des groupes culturels provenant d'une diaspora, ceux-ci peuvent participer à la conception de leur identité en reprenant certains traits culturels:
\foreignblockquote{english}[{\cite[277]{Sinfield1996}}][.]{Stuart Hall traces
  two phases in self-awareness among British Black people. In Do the first,
  `Black' is the organizing principle: instead of colluding with hegemonic
  versions of themselves, Blacks seek to make their own images, to represent
  themselves. In the second phase (which Hall says does not displace the first)
  it is recognized that representation is formative --- active, constitutive ---
  rather than mimetic}.
Néanmoins, dans le cas de la culture afro-américaine, nous nous retrouvons dans un contexte où ce concept de culture est en compétition avec celui de la race selon l'auteur, où une certaine \emph{essentialisation} par le racisme maintient cette version hégémonique d'eux-mêmes.

Pour comprendre qu'il existerait une culture née par l'hybridité chez les individus gais et lesbiennes, l'auteur considère que l'on doit se baser sur l'histoire des individus plutôt que s'attarder seulement à l'Histoire au sens large des sociétés. 
En effet:
\foreignblockquote{english}[{\cite[280]{Sinfield1996}}][.]{\textelp{} for lesbians
  and gay men the diasporic sense of separation and loss, so far from affording
  a principle of coherence for our subcultures, may actually attach to aspects
  of the (heterosexual) culture of our childhood, where we are no longer `at
  home'. Instead of dispersing, we assemble.

  The hybridity of our subcultures derives not from the loss of even a mythical
  unity, but from the difficulty we experience in envisioning ourselves beyond
  the framework of normative heterosexism --- the \emph{straightgeist} \textelp{}}
Dans ce contexte, on peut dénoter que l'auteur souligne une des particularités de la culture dominante: son caractère essentiellement hétérosexuel au niveau des normes, ou hétéronormatif (voir partie~\ref{sec:enjeux_g_ographiques_du_recours_l_identit_}). 
Le départ de la culture hétérosexuelle ou \anglais{straightgeist} à laquelle tous les individus de la culture homosexuelle doivent répondre est partielle; à tout moment, les gays et lesbiennes pour ne nommer que ceux-ci doivent composer avec le reste de la culture hétérosexuel dans les autres sphères de leur vie, que ce sois à l'école, au travail ou dans l'espace public. 
C'est de cette négociation inévitable avec la culture dominante que pourrait se justifier le caractère hybride de la culture homosexuelle. 
Les objets culturels, les pratiques culturelles et sociales s'inscrivent dans cette culture hybride et peuvent donc ou non être comprises par la culture dominante.

Pour conclure cette partie, notons que le trait commun partagé par les gays et lesbiennes dans le cadre de l'analyse par l'ethnicité-et-les-droits est l'altérité vécue par les individus non-hétérosexuels:
\foreignblockquote{english}[{\cite[289]{Sinfield1996}}][.]{Our apparent unity is
  founded in the shared condition of being not-heterosexual --- compare `people
  of colour', whose collocation derives from being not-white}. 
Étant donné cette emphase sur la non-correspondance à une norme, la communauté gaie et lesbienne, et on pourrait rajouter bisexuelle et trans-, est nécessairement très large et diverse. 
Sinfield hésite donc à parler ici d'une culture en soi; on propose plutôt l'usage du concept de sous-culture qui rendrait mieux ce caractère de diversité et reconnaître le côté construit et récente de celle-ci:
\foreignblockquote{english}[{\cite[289]{Sinfield1996}}][.]{It is to protect my
  argument from the disadvantages of the ethnicity model that I have been
  insisting on `subculture', as opposed to `identity' or `community': I envisage
  it as retaining a strong sense of diversity, of provisionality, of
  constructedness}.

% , Hall \& Gay introduisent le concept d'identité pour traiter des groupes
% sociaux et culturels qui s'opposerait à l'ancien sujet moderne. À partir de ce
% concept, il devient a priori possible de traiter de groupes ou communautés comme
% les homosexuels, bisexuels, trans- et queers dans le contexte culturel précisé
% précédemment.


\subsubsection{Enjeux géographiques du recours à l'identité}
\label{sec:enjeux_g_ographiques_du_recours_l_identit_} 
%Les textes suivant
% permettraient de situer l'usage de la culture dans un contexte précis. Par
% exemple, \textquote{The Location of Culture: The Urban Culturalist Perspective}
% propose l'étude culturelle des phénomènes urbains, alors que les études urbaines
% utilisent normalement des méthodes quantitatives pour traiter des mêmes
% questions (Borer, 2006). Les parties précédentes se sont principalement
% intéressées à l'analyse générale des concepts d'identité, de culture et
% d'identité sexuelle en demeurant essentiellement dans un contexte sociologique.
% Pour la poursuite de ce texte, nous nous intéresserons plus particulièrement au
% domaine spatial de ces concepts en tentant d'apposer une regard géographique sur
% le culture. %, plus particulièrement par un regard sur la ville comme espace
% culturel. Celle-ci, en plus d'être un des milieux les plus populeux qu'on
% retrouve dans plusieurs des sociétés humaines, sinon la totalité de nos jours,
% permet de comprendre les enjeux entourant la mixité sociale et culturelle

%Pour cette partie, nous pencherons sur deux textes de Michael Borer, à savoir \textquote{The Location of Culture: The Urban Culturalist Perspective} (2006) et \textquote{From Collective Memory to Collective Imagination} (2010).~\citep{Borer2006}
%Le deuxième texte de Borer, \textquote{From Collective Memory to Collective
%Imagination >> propose l'analyse spatiotemporelle des phénomènes culturels en
%milieu urbain, un point de vue méthodologique qui rejoint celui de Larry Knopp.~\citep{Borer2010}

En géographie Queer on retrouve les deux paradigmes soulevés à la partie~\ref{sub:minorit_s_et_universel}, à savoir un partage entre une analyse autour de la diversité comme construction sociale et une autre centrée sur l'identité gaie, lesbiennes, bisexuelle ou trans-. 
Plus particulièrement, la pensée géographique peut se pencher sur les espaces occupés par les gays et lesbiennes, ou plutôt s'intéresser au caractère normatif des espaces. 
Les premiers travaux en géographie sexuelle se sont principalement attardés au premier point de vue alors que récemment, durant les vingts dernières années, on remet en question le point de vue \emph{minoritarisant} qu'on retrouve toujours dans certains textes comme celui de~\citet{Sinfield1996} pour plutôt se pencher sur les normes sociales et leurs rapports avec l'espace. 
C'est ce dernier point de vue que défend et explique Natalie~\citet{Oswin2008} dans l'article \citetitle{Oswin2008} que nous traiterons dans la suite de ce texte.

Dans ce texte, Oswin s'oppose à l'idée que les espaces queers puissent être des lieux en opposition totale avec les normes de la société dominante. 
Plutôt, la recherche récente en géographie sexuelle a permis de rendre compte que c'est par la présence d'individus dont les actes ou la manière de performer le genre ou l'identité sexuelle ne correspond pas aux normes sexuelles dominantes que les espaces de la société apparaissent comme hétéronormatifs par le jeu de pouvoir qui s'y établit. 
Ce jeu de pouvoir s'établirait de plusieurs manières, par exemple par la marginalisation (violence homophobe, exclusion), par une présence accrue du pouvoir policier à proximité des espaces queers ou pas le refus des instances gouvernementales de répondre aux demandes des populations \lgbt{} (durant la crise du \sida, par exemple).

Ce jeu de pouvoir sur les normes sociales se manifestent particulièrement dans les espaces réputés occupés par des individus appartenant au spectre \lgbt{} où plusieurs visions de l'homosexualité luttent elles-mêmes entre elles. 
En effet, comparativement à l'idée d'une culture homosexuelle uniforme et partagée par les membres de la communautés gaie et lesbienne, les auteurs en géographie queer ont plutôt montrer que plusieurs groupes luttent sois par un élargissement des normes pour l'inclusion sociale, le point de vue \emph{assimilationniste} alors que d'autres, les \emph{libérationnistes}, souhaitent plutôt remettre en question certaines normes qu'ils considèrent empruntées à la culture hétérosexuel et répliquées à l'intérieur même des espaces queers, comme les villages gais. 
Oswin définie cette homonormativité à partir d'une citation de Lisa Duggan: 
\foreignblockquote{english}[{\cite[tel que cité
  dans][92]{Oswin2008}}][]{\foreigntextquote{english}[{\cite[50]{Duggan2003}}][]{a
    politics that does not contest dominant heteronormative assump- tions and
    institutions, but upholds and sus- tains them, while promising the
    possibility of a demobilized gay constituency and a privatized,
    depoliticized gay culture anchored in domesticity and consumption}}.

Un autre point important du texte d'Oswin est la réinterprétation du sens des multiples identités que peut posséder un individu du spectre \lgbt{}. 
Au lieu de se baser sur l'hybridité, ces identités sont plutôt conçues comme des sources d'oppressions, au niveau de la racisation, de la classe sociale ou du genre par exemple. 
Dans de nombreux espaces queers, il a été remarqué que souvent le pouvoir était détenu par des individus dits privilégiés sur d'autres bases identitaires que la simple orientation sexuelle. 
Souvent également, au sein même des communautés gaies, d'autres forme d'identité sexuelle ou de genre sont mises de côté, comme la bisexualité ou les individus trans-~\citep[93]{Oswin2008}.
Il est donc pour l'auteur important de reconnaître le potentiel qu'ont les chercheurs de réifier les communautés des milieux qu'ils étudient, notamment en recréant certains hiérarchie en omettant de reconnaître les inégalités sociales.

Pour la suite, nous nous intéresserons plus particulièrement aux méthodes offertes par la géographie pour traiter efficacement des questions de diversité sexuelle et d'espace. 
Larry~\citet{Knopp2004}, un des premiers chercheurs à lier les études queers à la géographie culturelle, traite dans \citetitle{Knopp2004} de la théorie de l'Acteur-Réseau qui pourrait permettre méthodologiquement de dépasser certaines limites de l'identité comme concept pour traiter des populations \lgbt{} tout en s’efforçant de fuir certains déterminismes en géographie culturelle~\citep{Knopp2004}. 
En effet, plutôt que s'appuyer sur un point de vue \emph{minoritarisant} des groupes et communautés \lgbt{}, Knopp reconnait d'emblée le potentiel qu'ont ces groupes d'avoir un effet sur les structures sociales de pouvoir. 
Ce texte s'inscrit donc ainsi moins dans l'étude d'un groupe culturel particulier que sur les conflits vis-à-vis les normes sociales d'un ensemble culturel. 
Knopp utilise dans son texte des termes similaires à ceux de \citet{Sinfield1996}, à savoir que les groupes queers seraient entre autres hybrides au niveau identitaire et dont la présence sociale prendrait la forme d'une diaspora. 
En effet, si les espaces queers sont les espaces vers lesquels se dirigent les membres de ces communautés dans le texte de \citet{Sinfield1996}, Knopp considère plutôt que ce sont les déplacements dans le temps et l'espace qui sont formateurs des identités queers et que la géographie a le potentiel de rendre compte de ces déplacements par les théories \emph{non-représentationnelles}, dont nous ne traiterons pas ici.

Ces déplacements ont en effet un sens particulier:
\foreignblockquote{english}[{\cite[123]{Knopp2004}}][.]{for gays, lesbians,
  bisexuals, transgenders, and other queers, as for other oppressed groups, this
  means seeking people, places, relationships, and ways of being that provide
  the physical and emotional security, the wholeness as individuals and as
  collectivities, and the solidarity that are denied us in a heterosexist world}
Plutôt que d'être les héritiers directs d'une culture queer, on doit plutôt concevoir que les individus queers possède bel et bien la culture dominante, mais que leur intégration sociale passe par d'autres trajets que ceux proposé normalement par la culture dominante.

Knopp avance même que ces déplacements ont une importance assez forte pour être génératrice d'une ontologie particulière:
\foreignblockquote{english}[{\cite[123]{Knopp2004}}][.]{It is also about
  testing, exploring, and experimenting with alternative ways of \emph{being},
  in contexts that are unencumbered by the expectations of tight-knit family,
  kinship, or community relationships—no matter how accepting these might be
  perceived to be} 
En même temps que les individus \lgbt{} quittent le contexte familial hétérosexuel comme début de parcours et principal lieu d'acquisition de culture, ils transportent avec eux des éléments de cette culture et la transforment au gré de leurs expériences. 
Pour Knopp, l'expérience queer en soi provoque la constitution de nouvelles données culturelles, spatiales et participe donc à cette hybridité de l'identité queer.
\foreignblockquote{english}[{\cite[130]{Knopp2004}}][.]{As queer bodies and
  subjectivities circulate through (and constitute) time and space, they leave
  legacies, absorb others, and mutate. They spread information, values, and
  culture, and constitute barriers to such spreads at the same time. This is
  diffusion par \emph{excellence}}

Si ces caractéristiques sont particulièrement importantes, Knopp souligne tout de même que ces processus formateurs au niveau identitaire ne sont pas l'apanage des individus \lgbt{} mais peuvent être considérés comme des expérience probable pour tous les individus; elles ne sont pas essentielles à l'expérience \lgbt{}.
Néanmoins, le contexte normatif entourant la sexualité pousse tout de même les populations \lgbt{} à ces parcours de façons plus particulières: les chemins de vie empruntés ressembleraient à des migrations vers l'acceptation sociale de l'identité sexuelle, qu'elle passe par l'anonymat ou pas l'inclusion au sein d'un espace \lgbt{}. 
Knopp appuie cette caractéristique par plusieurs travaux en géographie queer qui reconnaissent l'importance du parcours et du déplacement spatio-temporel chez les individus \lgbt{}~\citep[123]{Knopp2004}.


% j'envisage dans mon travail d'utiliser le texte \textquote{Critical geographies and the
% uses of sexuality: deconstructing queer space}~\citep{Oswin2008} qui apporte
% certaines critiques sur l'usage de concepts géographiques que l'on retrouve dans
% les travaux portant sur le queer. Celui-ci devrait permettre d'éviter certains
% raccourcis qui pourraient subvenir en tentant d'utiliser les textes précédents
% dans un contexte ethnographique en gardant certaines avancées propres au point
% de vue queer, comme la fluidité des identités, leurs superpositions, etc. 

\subsubsection{Synthèse}
\label{sec:synth_se} Cette dernière partie tentera de proposer une synthèse des parties précédentes en lien avec la culture ou les identités de la diversité sexuelle. 
Nous avons d'abord traité de la question de la culture d'un point de vue anthropologique, à partir de Clifford Geertz. 
Si une culture est bel et bien un système de pratiques et de conceptions héritées, on pourrait croire qu'il pourrait exister une culture \lgbt{} étant donné les pratiques particulières des communautés gaies et lesbiennes, notamment par l'existence de milieux de vie particuliers, de pratiques sociales et sexuelles différentes des hétérosexuels et d'une histoire particulière à cette communauté (qui demeure tout de même imbriquée dans celle de l'Occident). 
Le contexte social, politique et intellectuel du dernier siècle semble avoir été propice à l'éruption des communautés gaies et lesbiennes, comme nous l'apprend la partie sur le texte de Hall. 
En effet, si, des lumières jusqu'à la modernité, il était difficile d'imaginer l'existence d'identités basées sur la sexualité, différents décentrements et déstabilisations du sujet moderne ont amené cette possibilité par une remise en question de l'uniformité relative des individus au sein d'une même culture. 
Est-ce toutefois suffisant pour dire qu'il existerait aujourd'hui des cultures basées sur des identités? 
En ce qui concerne la question de l'orientation sexuelle du moins, la réponse n'est claire et l'on remet même en question l'idée de s'attarder à l'identité dans un contexte où ces individus, les individus de la diversité sexuelle, sont souvent victime de marginalisation dans le contexte culturel occidental. 
Leur présence témoigne des instabilités propres au nouveau sujet post-moderne qui vit dans un contexte culturel aux normes témoignant de jeux de pouvoirs bien présents, autant pour les individus de la diversité sexuelle que pour le reste de la société. 
Au-delà des enjeux stratégique liés à l'étude des individus \lgbt{}, dans une perspective de changement social, on pourrait croire que ces déstabilisations continuent d'avoir lieu et que la critique des normes sociales et des enjeux de pouvoir demeure l'avenue à privilégier, dans un contexte de l'étude de la culture occidentale.

Nous pouvons conclure en soulignant que les liens entre identité et culture ne sont pas encore clairement définis; l'identité comme concept laisse croire qu'il est nécessaire d'appartenir à une culture précise pour se lier à une identité précise. 
Pourtant, malgré l'ambiguïté entourant l'idée de culture ou de sous-culture \lgbt{}, on peut difficilement remettre en question l'idée d'identité, sachant l'existence de groupes sociaux, culturels et d'espaces \lgbt{} dans la société occidentale.

La question du pouvoir reste à creuser plus en profondeur, tel que souligné pas Oswin dans son texte. 
Celle-ci propose également l'étude de l'hétérosexualité, une avenue intéressante pour voir si cette identité sexuelle existe comme celle des autres formes d'orientation sexuelles et peut-être ainsi arriver à voir les effets probables de la normalisation de la sexualité à une plus grande échelle au sein de la culture occidentale~\citep[100]{Oswin2008}. 
Enfin, il serait également pertinent d’entamer une étude auprès des groupes \lgbt{} pour savoir qu'elle est la place de la communauté dans leur identité et s'il existe, selon eux, les éléments nécessaires pour parler d'une culture queer, que ce soit au niveau historique ou au niveau des significations particulières des éléments et pratiques composant les espaces et la temporalité queer.


\subsubsection{La diversité sexuelle dans les sciences sociales et la
géographie}
\label{ssub:la_diversit_sexuelle_dans_les_sciences_sociales_et_la_g_ographie}
Nous travaillerons dans cette étude à partir de l'identité sexuelle et du genre.
\citet{Foucault2011} décrit dans son œuvre \citetitle{Foucault2011} le processus par lequel les relations entre individus de même sexe sont devenues par la médecine notamment, une forme de trouble de la sexualité jusqu'à être aujourd'hui, après bien des luttes sociales, une identité particulière~\citep{Foucault2011}. 
Plusieurs autres auteurs, en études féministes notamment avec Gayle Rubin et Judith Butler~\citep[98]{Marcus2005}, ont permis de montrer que le genre est aussi un construit social ou plutôt une performance qui n'est pas l'essence de l'individu~\citep{Butler2007}. 
On remet ainsi en question l'idée de nature en sexualité ainsi que la possibilité que les comportements dits anormaux, comme l'homosexualité, la bisexualité ou un genre non-binaire, soient en fait des comportements sociaux qui ne concordent tout simplement pas avec les normes sociales.

Ce travail théorique fut plus tardivement repris par la géographie avec une certaine justification: à la même époque que les études sur le genre avaient lieu et sortaient du seul cadre de la psychanalyse~\citep{Rubin2011a,Rubin2011}, les luttes sociales entreprises par les gais et lesbiennes prenaient place dans l'espace public~\citep[422-427]{Spencer2005}. 
Dans les grandes villes des États-Unis d'abord, puis progressivement d'Europe et du Canada, on a assisté à l'apparition des villages gais, lieux d'abord utilisés pour la protection par l'anonymat, puis, avec ces luttes, le partage d'un mode de vie et de lutte politique. 
Relativement invisibles, les espaces urbains habités par les gais et lesbiennes acquièrent une visibilité supplémentaire en même temps que l'on commença à s'intéresser aux liens entre l'espace et la sexualité dans un contexte ou l'identité sexuelle semblait détenir une ontologie particulière.

Pour certains auteurs, ce développement identitaire prend une forme ontologique.
L'ontologie, terme utilisé pour désigner le savoir représentant une vision particulière de l'univers, est un concept dont l'usage serait de plus en plus complexifié par l'apparition récente d'une multitude de nouvelles identités, par la politisation de plusieurs pans de population vivant la marginalisation et par une remise en question de la culture dominante~\citep[122]{Knopp2004}. 
Le \qu\ pourrait désigner l'ontologie propre au spectre de la diversité sexuelle étant apte à dépasser les normes séparant chacune des identités du spectre \lgbt{} et pouvant rendre compte d'une certaine cohésion qui rejoint historiquement les individus de ces communautés~\citep[122]{Knopp2004}.

Larry Knopp propose d'ailleurs la théorie de l'Acteur-Réseau pour parvenir à identifier et analyser toutes les composantes spatiales du \qu\ comme le lieu et le mouvement et recréer une certaine cohésion de la théorie \qu\ face aux divers mouvements qui la façonnent, comme le constructivisme, le matérialisme. 
Elle permettrait également d'échapper à la notion du territoire pour plutôt utiliser une vision plus relativiste de l'espace étant donné que les éléments d'un réseau peuvent croiser d'autres réseaux et être polysémiques. 
Des approches où l'espace peut prendre diverses formes et qualités~\citep{DiMeo1998} seraient beaucoup plus compatibles avec la théorie de l'Acteur-Réseau.

La géographie culturelle s'intéresse donc depuis peu à ces nouveaux groupes sociaux, surtout avec l'avènement des approches postmodernes dont la théorie \qu\ fait partie qui s'intéressent de nouveau à l'individu, en prenant tout de même assise sur les travaux ultérieurs, notamment le structuralisme et la théorie critique. 
Une des premières œuvres à avoir marqué le commencement de la géographie \qu\ est~\citetitle{Bell1994} de~\citet{Bell1994} qui amena une sélection de textes, montrant la pertinence et les possibilités multiples de cette nouvelle branche de la géographie. 
Néanmoins, dans plusieurs études, l'homme gai fut priorisé comme objet où on s'attarda à l'usage de certaines méthodes de géographie, comme le recours au territoire~\citep{Podmore2001,Oswin2008}. 
On travaillait donc à comprendre la progression des villages gais, souvent menée par des hommes, alors que les femmes lesbiennes et autres formes de sexualités moins étudiées furent mises de côté par l'argument que celles-ci sont souvent moins visibles et présentes dans l'espace public, et que l'on devrait s'attarder au ménage et à la maison pour traiter de leurs situations~\citep[333-334]{Podmore2001}. 
Il s'agit, selon Podmore dans \citetitle{Podmore2001} plutôt d'un problème méthodologique alors que les concepts utilisés actuellement en géographie ne suffisent plus pour étudier les nouvelles formes identitaires. 
Le recours au territoire comme concept géographique pour s'intéresser aux communautés sexuelles apparait donc problématique: non seulement on exclut les femmes, mais également d'autres groupes encore moins visibles, comme les transgenres et les bisexuels. 
Il convient donc, en reprenant la position de Natalie Oswin, de traiter d'espaces \qus\ pour éviter une essentialisation de l'autre, de ne pas prioriser certains genres, d'utiliser des dichotomies fautives entre l'homosexuel face au \anglais{straight} et considérer \latin{de facto} un milieu \qu\ comme un espace de résistance de dissidence~\citep{Oswin2008}. 
En effet, comme l'a montré Nathaniel M. \citet{Lewis2011} dans \citetitle{Lewis2011}, les hommes gais, même sans espaces précis, héritent des normes de leur environnement sur l'identité sexuelle, mais peuvent également avoir un effet sur ces normes, l'auteur prenant exemple sur les familles homoparentales dans les banlieues qui offrent un regard différent sur la définition d'un ménage de classe moyenne au sein du voisinage~\citep[304]{Lewis2011}.

\subsubsection{La sémiotique}
\label{ssub:la_semiotique} \todo{à revoir} La sémiotique prend ses racines dans les travaux de plusieurs auteurs, principalement Ferdinand de Saussure qui a également donné naissance au courant du structuralisme~\citep{Noth1995}. 
Dans ses travaux sur la religion, Clifford Geertz considère que le géosymbole est On retrouve aujourd'hui toute une variété d'auteurs et de mouvements, autant en musique qu'en théologie, mais dans le cas présent, nous nous intéresserons principalement aux auteurs en géographie. 
Au sein de ceux-ci, on peut nommer d'abord Joël de Bonnemaison qui démontra l'importance des géosymboles comme moyen utilisé par les groupes ethniques pour s'ancrer à l'espace habité qui deviendra le territoire par des itinéraires, des formes et des tracés dans le paysage~\citep{Bonnemaison1981}. 
Mario Bédard va poursuivre la réflexion et inclure tout comportement culturel particulier rattaché à l'espace, et créer une typologie complexe des différents géosymboles que l'on retrouve dans le territoire~\citep{Bedard2002}. 
Jérôme Monnet quant à lui traite plutôt des processus conscients de création de symboles, et des relations particulières qui lient les symboles à l'espace, au pouvoir et à l'identité, particulièrement dans le contexte occidental et capitaliste~\citep{Monnet1998}. 
L'idée de contexte se joue d'ailleurs à plusieurs échelles, et il apparait important selon l'auteur de ne pas limiter l'étude d'un symbole seul, mais considérer son environnement, sachant que la présence d'un symbole d'un type peut amener la propagation de symboles similaires, comme dans le cas d'un centre d'achat qui provoque l'arrivée d'autres commerces, comme symboles marchands~\citep[7-8]{Monnet1998}.
Il apparait donc possible, à l'aide des travaux effectués en sémiotique et en géographie culturelle, d'arriver à décrire l'espace occupé par les groupes et individus appartenant au spectre de la diversité sexuelle en tenant compte de certaines difficultés, comme le recours au territoire (qui peut masquer espaces dispersés ou en réseaux) ou encore à la matérialité des symboles, qui peuvent ne pas être suffisants comme appuis pour décrire une réalité spatiale soit temporaire, soit mobile.

\citet[105--109]{Rose2012} fait une synthèse des différents moyens de faire de la sémiologie une méthode et une analyse de la société. 
On retrouverait deux courants principaux dans le domaine de la recherche en sémiologie qui déborde les champs disciplinaires traditionnels comme l'anthropologie et la géographie.
Le premier courant serait les chercheurs pratiquant la sémiologie dite  
Le deuxième courant, nommé sémiologie sociale (\anglais{social semiotics}), consisterait en une analyse des faits sociaux

Dans le chapitre suivant, nous reviendrons sur la sémiotique mais du point de vue de la méthode. 
Nous décrirons alors comme nous pourrons, à partir d'une déconstruction des images collectées, obtenir plus d'information sur le sens des géosymboles évoqués et ce qu'ils traduisent comme relation territoriales entre les espaces des villes traitées et les sous-groupes du spectre \lgbt{}.

%\paragraph{Signifiant, Signifié et Référent}
Selon \citet[113][]{Rose2012}, le signe est le principal concept que l'on retrouve en sémiotique. 
Il consiste en les différentes informations qui circuleraient dans un système de communication, que ce soit entre deux individus ou un individu et sont environnement. 
Ce concept, dans le modèle de Saussure, est divisé et deux parties pour rendre compte de la relation existant entre un individu dans un système de communication et les objets ou idées à lesquelles il réfère. 
Ce premier élément est le signifié qui consiste en l'objet ou l'idée auquel on octroie un signe. 
Dans un contexte de communication ou non, le signifié en soi existe toujours. 
Bien que certains signifiés dépendent de l'être humain pour exister, comme philosophie ou les mathématiques ou des objets manufacturés, en général ces signifiés ont une existence propre non-dépendante d'un système de communication dans lequel on ferait référence. 
Autrement dit, si c'est l'être humain dans sa communication qui permet une telle caractérisation d'un objet par le signifié, il demeure que ce dernier peut exister en soi. 
Le signifiant quant à lui ne peut exister qu'au sein de ce système de communication. 
En effet, ce signifiant est en quelque sorte l'élément qui permet d'évoquer chez la personne jouant le rôle de communicateur l'idée liée au signifié. 
Le mot \emph{roche} dans un tel système est un signifiant lié à la substance minérale séparée d'une substrat rocheux plus important qui est ici le signifié dans sa forme la plus générale et dont la forme peut différer selon les occasions et les contextes. 
Enfin, ce signifiant n'est pas qu'un mot prononcé ou lu, il peut également consister en tout média dans lequel l'idée d'un signifié est transmise, comme une image, un vidéo, un poème, etc.

%\paragraph{Signe}
Dans l'ensemble cette définition du signe correspond aux théories de Saussure.
Elle présente rapidement des limites: si un signifiant peut prendre plusieurs formes, comment peut-on comparer facilement ceux-ci entre eux? 
Si les individus utilisent divers signifiants pour communiquer, certains demandent un bagage culturel particulier pour être compris alors que d'autres sont plus universels.
Les travaux de Pearsons permettent conceptuellement de dépasser ces limites.
Nous allons enrichir cet usage du signe en concordance avec la théorie développée par Pearson. 
Selon lui, le signe peut prendre plusieurs formes selon média dans lequel on le trouve et le degré d'abstraction de différenciation qui peut exister entre le signifié et le signifiant. 
Ainsi, différents termes ont été développés pour différencier les différentes signes.
Parmi ceux-ci, on retrouve d'abord l'icône, qui consiste en une quasi-concordance entre le signifié et le signifiant. 
Dans un dessin d'enfant, selon les cultures, un arbre est souvent dessiné avec un tronc brun, quelques branches et des feuilles vertes. 
Cette image, sans correspondre à l'immense diversité des arbres, correspond tout de même à plusieurs types d'arbres que l'on retrouve dans un climat tempéré. 
Par contre, ce même arbre pourrait figurer sur un roman ou un livre traitant de la vie, de la longévité ou encore de la généalogie dans un contexte familial: dans ce cas-ci, l'arbre est lié de façon abstraite à ces concepts. 
On a donc affaire à un symbole, un signe dont la relation entre le signifiant et le signifié est définie de façon arbitraire et relative à un contexte culturel particulier. 
Le troisième type de signe est l'index dans lequel le signifiant et le signifié n'ont a priori pas de liens, de façon similaire au symbole, mais qu'en plus, le signifiant n'a pas de sens en soi sans le signifié. 
En effet, dans les cas précédents, surtout dans des cas plus visuels, on ne peut séparer le signifiant du signifié, ou du moins, on peut trouver d'autres types de relations. 
La longévité peut est signifiée par d'autres signifiants comme une personne très âgée, et le signifiant d'arbre rappelle très rapidement le signifié d'arbre sans mise en contexte. 
Dans l'index, la personne en communication doit avoir connaissance du lien entre le signifiant et le signifié. 
On peut penser par exemple aux feux de signalisations : les cercles rouges, jaunes et verts ont été désignés par être les signifiants des différents types de restrictions ou de non restrictions au déplacement en automobile, mais sans connaissance préalable des codes de la route, un individu ne peut comprendre ce lien de façon instinctive. 
Un autre exemple: un mot dans une langue. 
Si l'individu qui communique n'est pas locuteur de cette langue ou ne possède pas les connaissances pour comprendre quoi il s'agit, l'ensemble des caractères ne peut rien évoquer du signifié qu'il désigne.

On le constate rapidement, ce modèle n'a \latin{a priori} rien de très géographique en soi et ne permet pas facilement d'articuler les nuances propres à la culture et à l'identité d'un locuteur au-delà de la connaissance ou de l'ignorance d'un signe prenant la forme d'un index. 
C'est ici que nous allons nous arrêter pour traiter des notions d'espaces et de territoire, avant de faire la synthèse grâce au concept de géosymbole tel qu'articulé par \citet{Bonnemaison1981}.


\subsection{Espaces et territoires}
\label{sec:espaces_et_territoires} Nous poursuivrons avec ces deux concepts les plus géographiques, traités ensemble car notre démarche s'inscrit dans une certaine critique du concept de territoire qui demande une part d'approfondissements, \todo{repasser sur cette phrase}sans pour autant entrer dans une épistémologie trop approfondie car il ne s'agit pas ici du but de cette recherche. 
Ces deux concepts, quoique semblable à première vue, prennent en géographie culturelle des sens différents. 
Le concept d'espace d'abord se veut être un point de vue rationnel sur les questions de distances entre les objets de ce monde et leurs positionnement \todo{à reformuler}. 
Utilisé en mathématiques, en physique, et dans l'usage courant d'un point de vue pratique ou technique, l'espace est par définition une construction intellectuelle volontairement neutre dont les propriétés sont quantifiables \missref{Di Méo?}.
On peut penser aux distances en kilomètres pour un voyageur entre deux villes, le calcul de la taille pour la construction d'une maison, ou encore du volume pour la quantification d'un liquide. 
Mais nous allons le voir, ces calculs d'apparence banales semblent participer aujourd'hui à un point de vue particulier sur le monde qu'on décrira comme désenchanté.

L'utilisation d'un tel concept dans le cadre de la géographie culturelle peut à prime abord paraître contradictoire; en effet, ce champs de la géographie travaille plutôt sur les populations et les territoires, selon leurs pratiques culturelles et leurs spécificités identitaires ou ethniques. 
L'espace prendrait plutôt sa pertinence au sein de cette discipline dans le contexte de la géographie sociale par exemple, lorsque l'on s'intéresse aux enjeux propres aux déplacements, ou encore en géographie physique ou biologique qui traitent bien souvent les substrats physiques ou le monde du vivant dans des ensembles comme les \todo{trouver le mot manquant} comme des données quantifiables. 
La géographie culturelle par contre, durant le dernier siècle, telle que pratiquée par des géographes du nouveau/missref et de l'ancien monde/missref, travaillaient sur des populations des pays colonisés ou en voie de décolonisation en utilisant plutôt le concept de territoire. 
Nous pensons plus particulièrement à la géographie tropicale et son point de vue basé sur l'altérité entre les régions nordiques dites normales et les régions du sud~\citep[493]{Power2009} et également les travaux plus anciens en géographie régionale française s'intéressant strictement aux régions~\citep[31]{Courville1991}. 
Ceux-ci travaillaient bien souvent sur des populations plutôt restreintes pensées comme des ethnies \todo{à retravailler}.
Ces dernières semblaient à première vue vivre dans des milieux suffisamment isolées pour que les interactions inter culturelles doivent être considérées comme quasi-inexistantes et de cette façon, restreindre les caractéristiques de la population comme des faits uniques plutôt que l'objet d'interactions avec d'autres populations destinées à évoluer, avec ou sans la présence des schémas de colonisation~\citep[79--80]{DiMeo2007}.

Dans ce contexte, le territoire est perçu comme un autre plan de la réalité se superposant à l'espace: il s'agirait d'un ensemble d'éléments abstraits et matériels permettant à la population d'ancrer son identité, son histoire comme son futur. 
On retrouve cette définition chez plusieurs auteurs, notamment chez~\citeauthor{Bonnemaison1981} chez qui le territoire:
\blockquote[{\cite[253]{Bonnemaison1981}}][.]{Les sociétés humaines ont une
  conception différente du territoire. Il n'est pas forcément clos, il n'est pas
  toujours un tissu spatial uni, il n'induit pas non plus un comportement
  nécessairement stable}. 
\citeauthor{DiMeo2007} nous offre également une définition du territoire, pour qui:
\blockquote[{\cite[76]{DiMeo2007}}][.]{L’assise territoriale, campée sur un
  réseau de lieux et d’objets géographiques, constitués en éléments patrimoniaux
  visibles, renforce l’image identitaire de toute collectivité. Elle lui dresse
  une scène et la pourvoit d’un contexte discursif de justification
  particulièrement efficace en ville où des lieux très denses, soigneusement et
  anciennement dénommés, s’inscrivent dans une totalité territoriale
  représentée, à la fois symbolique et fonctionnelle}. \todo{à approfondir}

Le désenchantement du monde tel que nommé précédemment est décrit dans les travaux de Max Weber comme le phénomène par lequel les explications reliés au mystère, la religion ou la superstition perdent leur place face à celles qu'offrent la rationalité scientifique. 
On peut donc comprendre que c'est l'avancement scientifique et surtout sa méthode et ses découvertes qui aura un effet social qui mettra en danger la place de la religion et du mythe dans la société. 
Enlevant une part importante du divin dans l'explication des phénomènes terrestres, par la théorie de l'évolution par exemple, le phénomène va prendre de l'ampleur par l'arrivée du capitalisme au \siecle{19} qui, par sa capacité à produire des marchandises en masse sans l'apport individuel de l'ouvrier --- sa subjectivité --- rendra l'économie de plus en plus anonyme et aliénante. 
Combiné, ces des effets --- rationalité et aliénation --- éloigneront de plus en plus l'individu de sa capacité à expliquer le monde et à s'y conforter, perdant son pouvoir sur la matériel comme sur l'abstrait.

Ce développement aura un effet certain sur les milieux de vie dans lesquels l'industrie s'implantera, principalement les villes: de plus en plus demandantes en main-d’œuvre et offrant une quantité importantes de marchandises à consommer, ces dernières prendront une taille de plus en plus conséquente alors que les quartiers se développeront pour nourrir cette industrie. 
Des milieux villageois aux villes orientées vers les pouvoirs politiques, les sociétés se développeront désormais sur des nœuds urbains que l'urbanisme tentera de rationaliser par la poussée des champs architecturaux nouveaux, notamment par les travaux du Corbusier. 
De la naissance du mouvement fonctionnaliste en architecture, les bâtiment auront maintenant des fonctions et les villes seront pensées comme des machines dont l'efficacité doit être maximisée\missref{}. 
On peut noter ici que l'avènement du capitalisme mettra en place un point de vue rationaliste du l'espace des villes, alors qu'on peut de moins en moins considérer celui-ci comme un territoire où les individus trouveront un sens à leur existence\todo{Reformuler la conclusion. 
Également, voir à ce que la fin du paragraphe précédent ne répète pas inutilement celle-ci}.

Nous ne croyons pas qu'il s'agit ici d'un effet ayant pris emprise strictement en occident, alors que le capitalisme et d'une façon la rationalité scientifique s'est étendue à grande échelle. 
Néanmoins, malgré des effets sociaux très larges, nous considérons tout de même que les lieux touchés par ces effets ont des particularités qui leurs sont propres au niveau de l'organisation spatiale mais surtout, de la réponse sociale à ces effets qui se conjuguent à des effets politiques particulier, que ce soit le colonialisme, les régimes politiques en vigueur, etc. 
Ainsi, nous reconnaissons l'importance de l'Histoire dans les processus sociaux régionaux de part le monde. 
Les différences spatiales contriburaient donc aux développements de différences socio-culturelles quant à l'emprise \todo{à compléter}


Notre point ici est de montrer que dans les villes, l'espace prend une place particulière après ce que nous voyons comme un recul important de la territorialisation dans un sens traditionnel pouvant se reporter à une vision étroite du concept d'ethnie. 
De plus, nous le verrons dans la section \todo{à   compléter} Ces effets sont encore visible aujourd'hui à l'échelle même des villes, alors qu'après un développement certains des villes autour des industries, l'avènement d'un commerce mondial de plus en plus flexible et des entreprises toujours plus mobiles ont rendu les vieux centres urbains où s'étaient d'abord développées ces dernières moins intéressants, que ce soit pour elles-mêmes que pour les classes plus aisées. 
Après avoir perdu en valeur, ces centres sont de nouveaux réhabités par les classes aisées et reconditionnés selon les moyens des investisseurs s'accaparant ces espaces par le phénomènes décrits par de nombreux chercheurs sous le terme de gentrification or d'embourgeoisement.

Néanmoins, nous n'assistons pas qu'à une simple \emph{désertification} des espaces dans lesquels les classes prévilégiées prennent tous les espaces vacants et les individus appartenant aux classes inférieures sont laissées pour compte.
Plusieurs cas de reterritorialisation ont été identifiés dans la littérature, par une diversité d'acteurs, avec ou sans succès \citet{Hatvany2005}\todo{trouver d'autres références}. 
D'ailleurs certains géographes ont relevé que cet espace présente tout de même des caractéristiques particulières. 
Pour \citeauthor{Courville1991} notamment:
\blockquote[{\cite[41]{Courville1991}}][.]{l'espace devient un médiateur du
  rapport entre individus, groupes et collectivités, un produit social à
  analyser comme tel, au milieu des pouvoirs et des rapports sociaux qui le
  structurent et l'organisent}.

C'est donc dire que cet espace urbain, selon son évolution, est maintenant le théâtre d'interactions multiples et enchevêtrées. 
Des groupes comme les individus du spectre \lgbt{} ont d'ailleurs fait l'expérience de ces multiples interactions, en participant ou en subissant les phénomènes de gentrification~\autocite{Podmore2001,Giraud2014,Hogan2005} ou en se mêlant à des luttes d'ordre économiques et politiques auxquelles ils ne font pas partie a priori~\autocite{Kelliher2014} \todo{trouver des références liées aux luttes   anti-capitalistes, angleterre et autres}

Leur présence dans ce jeu d'interactions offrent à ces individus la possibilité de réaliser des rencontres qui mènent à une construction identitaire. 
Comme le souligne~\citeauthor{DiMeo2007}: \blockquote[{\cite[81]{DiMeo2007}}][.]{\textelp{}
  la ville fournit un potentiel privilégié d’outils de recentrage pour toute
  identité individuelle. Par sa variété intrinsèque et par les innombrables
  repères sensibles et vécus qu’elle étale, par les \emph{affordances} (emphase
  de l'auteur) qu’elle sème dans le champ des perceptions individuelles, la
  ville file une trame dont ses habitants se servent sans restriction pour
  tisser et inventer leur propre identité}.

Ainsi, on arrive à faire le lien d'une certaine façon avec les travaux de sociologie sur le sujet et l'identité tel que décrit à la Section~\ref{subsec:sujet_et_identité}. 
Ces nouvelles identités, naissants en partie des luttes pour la reconnaissance et d'une certaine distance avec les identités nationales, trouvent également un lieu pour l'organisation dans les espaces urbains.

% \todo{passer la hache?} \note{Le terme territoire n'est pas utilisé car il
% apparait complexe de déterminer un territoire selon la définition typiquement
% utilisée en géographie culturelle; dans la partie~\ref{sub:enonce_du_probleme},
% la question de territorialité sera approfondie. Parmi les espaces urbains
% envisagés, on peut compter prioritairement ceux des villes de Montréal et Québec
% (voir figure~\ref{fig:carte_quebec}), et hypothétiquement ceux de villes de plus
% petite envergure selon la première partie de la collecte de données (voir à ce
% propos la Section~\ref{sec:source_des_donn_es}).}

% \note{concept on l'a vu peut présenter des lacunes importantes si elle tend à
% une réification des groupes culturels} \todo{insérer à quelque part?}



\subsection{Le géosymbole comme marqueur spatial}
\label{sec:le_symbole_comme_marqueur_spatial} Nous retiendrons, avec les quelques nuances soulevées dans la section précédente, le concept de territoire pour décrire l'espace investi par un groupe culturel particulier. 
Pour faire le pont entre cette notion culturelle de décrire l'espace et les façons dont les individus communiquent entre eux, en demeurant sensible aux questions identitaires, nous proposons l'usage du géosymbole. 
Peu répandu dans la géographie anglo-saxonne, ce concept demeure intéressant pour l'analyse des groupes culturels contemporain.

La culture n'est plus considérée en géographie culturelle comme un tout monolithique; il s'agit plutôt d'une multitude de visions du monde différentes ou divergentes. 
La région, le paysage et le territoire furent des concepts particulièrement importants chez les premiers auteurs en géographie culturelle pour décrire les relations entre l'espace et la culture~\citep{Bonnemaison1981,Monnet1998,DiMeo1998,}, mais ceux-ci tendent dans certains cas à offrir une vision incomplète de la culture, en mettant l'emphase sur la vision de la culture dominante / hégémonique d'un espace~\citep[11-12]{Duncan1993}, ou en offrant une vision déformée des groupes culturels minoritaires ou marginalisés.

La culture occidentale par exemple n'est donc, en Amérique du Nord, qu'une des multiples cultures qui se partagent l'espace et avec laquelle il y a négociation~\citep[11]{Duncan1993}. 
Contrairement aux anciennes perspectives en géographie culturelle qui faisaient de la culture une entité à part, homogène et où régnait une apparence de consensus~\citep{Duncan1980}, la réalité semble s'éloigner de cette image, surtout lorsque l'on s'intéresse aux faits politiques, socio-économiques et d'immigration qui montrent que les sociétés sont beaucoup plus diversifiées que ce qu'elles semblent être. 
Il faut donc s'éloigner de cette perspective dite superorganique selon Duncan qui mène à la réification de la culture; on pourrait même dans ce cas ci étendre cette précaution aux groupes culturels plus restreints, comme les membres de la diversité sexuelle. 
En effet, depuis les dernières décennies, la diversité sexuelle est féconde de plusieurs nouvelles visions du monde par la résistance à l'hétérosexisme qu'on peut voir dans les différentes luttes gaies et lesbiennes.
Occupant une myriade d'espaces difficiles à situer précisément comparativement à la culture dominante, il apparaît inapproprié d'utiliser les concepts spatiaux traditionnels propres à la géographie culturelle comme le territoire pour parvenir à reconnaître et étudier ces groupes.

Il convient donc, pour l'étude géographique des minorités sexuelles, d'utiliser certains concepts moins rattachés à la matérialité de l'espace. 
En demeurant dans le champs de la géographie culturelle, nous proposons l'usage du géosymbole comme outil conceptuel pour situer les groupes culturels et comprendre leur relation avec l'espace. 
En effet, les géosymboles consistent selon Bonnemaison en: \blockquote[{\cite[256]{Bonnemaison1981}}][.]{\textelp{} un lieu, un   itinéraire, une étendue qui, pour des raisons religieuses, politiques ou   culturelles prend aux yeux de certains peuples et groupes ethniques, une   dimension symbolique qui les conforte dans leur identité }. 
En reprenant cette définition dans le contexte des minorités sexuelles, nous pouvons affirmer que nous traitons ici d'un groupe culturel dont la formation s'apparente de certaines façons à un groupe ethnique~\citep{Sinfield1996} mais sans nécessairement être attaché à territoire au sens traditionnel comme soulevé précédemment. 
La sémiotique, sois l'étude plus large des symboles, apparaît donc pertinente car nous considérons que les géosymboles sont une forme particulière de symboles à dimension spatiale et que selon les auteurs, ces derniers peuvent prendre plusieurs formes différentes, matérielles ou non tout en demeurant attaché à l'espace~\citep{Bonnemaison1981,Bedard2002}. 
Nous soulevons donc ici une des limites conceptuelles du géosymbole décrit par Bonnemaison (1981): nous allons utiliser ce concept sans la facette territoriale mais plutôt en se rattachant seulement à l'espace et à la temporalité de son occupation.

Ces géosymboles, bien que permettant de s'intéresser à la culture, ne font pas nécessairement une ombre sur les aspects sociaux d'un groupe; des concepts plus proches de la géographie sociale comme les classes sociales ou la racisation~\citep{Bonniol2005} demeurent nécessaires à la compréhension des espaces gais et lesbiens~\citep[93]{Oswin2008} et pourraient permettre de comprendre la position de certains géosymboles et leur raison d'être, en montrant les clivages propres à un groupe culturel précis, comme le spectre \lgbt{}.

Par ailleurs, ce qui pourrait apparaître comme un groupe homogène, les gais, est en fait tout un spectre qui demande la prise en compte de facteurs sociaux pour comprendre les variances et les divisions, le genre en premier lieu mais également l'appartenance à une certaine classe plus ou moins prompte à vouloir s'accorder ou rejeter les normes sociales hétérosexuelles et à être générateurs d'espaces particuliers ou à transformer certains espaces conçus comme normaux~\citep{Lewis2011}. 
En effet, dans un contexte où l'on reconnaît que l'espace en général facilite les relations hétérosexuelles au détriment des autres relations et donc des identités qui y sont rattachées~\citep{Brown2003}, il est important de s'intéresser aux autres espaces qui existent en marge ou en parallèle et de se questionner sur l'utilité de ces espaces pour les individus impliqués.

Alors que le territoire n'arrive pas à bien définir la spatialité des individus de la diversité sexuelle, le recours à la sémiotique et aux géosymboles pourraient permettre de voir émerger certaines formes spatiales ou du moins montrer la diversité des lieux et leur position et permettre une compréhension plus poussée de l'utilisation de l'espace des communautés \lgbt{}. 
À notre connaissance, aucun travail ne traite spécifiquement de la sémiotique en géographie \qu\ et il s'agit d'une lacune vis-à-vis le potentiel que contient cette approche dans l'étude de l'espace et au sein de la géographie culturelle.
L'étude des géosymboles permettrait d'atteindre un savoir difficile ou impossible à obtenir par une approche plus matérialiste s'arrêtant au territoire sans prendre en compte l'existence de réseaux. 
D'emblée, nous pouvons déjà considérer que des évènements comme les \anglais{gay pride} (ou Fierté gaie), les manifestations politiques pour les droits gays et lesbiens, les centres communautaires offrant des services aux individus porteur du \vih, les bars lesbiens ou encore des lieux de dragues dans des espaces publics sont tous des exemples d'espaces \lgbt{} portant en eux des marques particulières, des logos ou des apparences qui entrent dans la définition du géosymbole.

\blockquote[{\cite[108]{DiMeo1998}}][.]{\textelp{} le territoire multidimensionnel
  participe de trois ordres distincts. Il s'inscrit en premier lieu dans l'ordre
  de la matérialité, de la réalité concrète de cette terre d'où le terme tire
  son origine. Il relève en deuxième lieu de la psyché individuelle. Sur ce
  plan, la territorialité s'identifie pour partie à un rapport a priori,
  émotionnel et présocial de l'homme à la terre. Il participe en troisième lieu
  de l'ordre des représentations collectives, sociales et culturelles. Elles lui
  confèrent tout son sens et se régénèrent, en retour, au contact de l'univers
  symbolique dont il fournit l'assise référentielle}.



\subsection{La diversité sexuelle en géographie}
\label{sec:la_diversit_sexuelle_en_g_ographie} \todo{à garder ou pas?} 
Les travaux arrimant l'ensemble de ces courants théoriques, l'étude des géosymboles avec un accent fort sur la sémiotique en relation avec des groupes dont l'identité se superpose à des ensemble culturels plus grand ne sont pas légions, surtout en ce qui concerne le cas plus spécifique des identités \lgbt{}. 
Ceux-ci sont tout de même existants et il nous apparaît important de nommer brièvement ceux qui existent dans le but précis de définir les angles morts où la recherche a lieu d'être.

L'ouvrage principal à faire la lumière sur l'usage des symboles par les communauté \lgbt{} sur lequel nous nous appuyons est le livre \citetitle{Giraud2014} de \citet{Giraud2014} qui s'est plus particulièrement penché sur les cas du Village gai de Montréal et du Marais de Paris selon une démarche comparative. 
Plus près de nos intérêts, une section du livre s'est plus particulièrement penchée sur les symboles utilisés par les communautés gaies des deux villes. 
On y apprend plus particulièrement comment cette visibilité a contribué à une certaine territorialisation des groupes gais. 
Choisissant une perspective historique et centrée sur un seul groupe, les hommes gais, la recherche reste muette sur les autres groupes \lgbt{}, sachant que ceux-ci interagissaient avec ces hommes et partageaient certaines espaces et une histoire inter-reliée~\citep{Remiggi2000,Demczuk1998,Podmore2001,Higgins1997,Higgins1999}.

Nous nous appuyerons donc dans cette recherche sur d'autres travaux pour couvrir plus largement les communautés \lgbt{} québecoises. 
Nous avons penser notamment aux de Julie Podmore, de Frank Remiggi et de Ross Higgins qui figurent parmi les auteurs principaux à traiter de la diversité sexuelle au Québec en géographie ou en anthropologie.
\todo{à compléter}

\subsection{Vers une vision hétérogène de l'identité, de l'espace et de
l'essence du symbole}
\label{sec:vers_une_vision_h_t_rog_ne_de_l_identit_de_l_espace_et_de_l_essence_du_symbole}
\todo{à revoir} \foreignblockquote{english}[{\cite[tel que cité
  dans][97]{Oswin2008}}][.]{As Elspeth Probyn has stated, sexual spaces
  \foreigntextquote{english}[{\citeyear[10]{Probyn1996}}][.]{are delineated
    through coincidence and not through exclusion}. Rather than clinging to the
  fiction that we can locate queer spaces that exist in coherent opposition to
  heterosexual spaces, we need to intensify examinations of what comes together
  in processes of sexualization}

%%% Local Variables:
%%% mode: latex
%%% TeX-master: "../../memoire-maitrise"
%%% End:

%!TEX root = ../../memoire-maitrise.tex
\chapter{Méthodologie}
\label{cha:methodologie}

\chapterprecishere{\textquote{The growth of the pervert population of Brisbane, beautiful capital of Queensland, is astounding, and in the last year hundreds of these queer semi-feminine men have made the city their headquarters.
Now they have evolved into a cult, with two main sects, one on the north and the other on the south side of the town, with the river dividing them. 
And occasionally they meet at queer, indecent, degrading ceremonies when perverted lusts come into full play and shocking rituals are celebrated} \par\raggedleft--- \textup{The Arrow}, le 4 mars 1932}

Ce chapitre se penchera plus particulièrement sur la méthodologie et nous définirons le cadre de l'étude, soit le lieu et la durée dans laquelle a eu lieu la collecte de données. 

% La recherche se déroulera en plusieurs étapes : d'abord, faire
% un portrait de la création et de l'évolution des géosymboles \qus\ ou de la
% diversité sexuelle, dresser un bref portait historique des milieux de la
% diversité sexuelle au Québec et répertorier les géosymboles dans un
% échantillon varié de villes d'une taille minimale (selon les paramètres
% d'apparitions d'une communauté \lgbt).

% Le travail de recherche se terminera par une analyse de ces géosymboles selon
% leur usage et de la trame sémantique sous-jacente à leur déploiement dans
% l'espace, en tentant de comprendre la relation qui existe entre les
% communautés \lgbt{} et leur spatialité.

\section{Lieu d'étude}
\label{sec:lieu_d_tude}
L'espace couvert par cette étude comprend les milieux urbains fréquentés par et pour les individus ou groupes faisant partie du spectre de la diversité sexuelle, c'est-à-dire principalement les individus non hétérosexuels ou dont l'identité de genre ne correspond pas à la norme sociale hétérosexuelle ou cisgenre\footnote{Terme inventé dans la foulée des luttes pour la reconnaissance des personnes trans et de leurs droits. 
Le terme cisgenre vise à remettre en  question la norme sociale sur l'assignation du genre à la naissance, pour  proposer plutôt que la concordance entre le genre et le sexe d'un individu est  une des possibilités d'identification de genre, comme le fait d'être trans (ou  non binaire dans le genre, sans genre, etc., voir~\cite{Barker2015})~\citep[150]{McGeeney2015}.}. 
Cette définition large vise à rassembler en un groupe les personnes homosexuelles, lesbiennes, bisexuelles, trans, queers, ou encore en non-concordance avec le genre (bien que dans l'analyse, ces groupes seront analysés séparément ou conjointement selon le contexte). 
En considérant le temps alloué à la collecte de données, au contenu de certains matériaux d'archives et d'inégalités entre certaines identités dans l'ensemble de la société, nous n'avons pas couvert l'ensemble de ce spectre de façon égale. 
Nous reviendrons plus loin dans le chapitre sur les raisons de cette couverture inégale.

Plus spécifiquement, en concordance avec l'hypothèse émise dans le chapitre précédent, nous considérons que cet espace couvre de multiples lieux dispersés parmi les villes québécoises, à quelques expressions près, sachant que certains lieux en dehors de la province sont publicisés dans les médias québécois. 
Ces espaces ont pris dans notre collecte de données la forme de bars, restaurants, commerces, rues, quartiers ou autres en correspondance avec les études déjà effectuées par d'autres chercheurs sur les pratiques d'organisation et de rencontre des minorités sexuelles~\citep{Higgins1999,Hinrichs2012}.

Selon la hiérarchie des villes québécoises, ce projet de recherche se penchera principalement sur deux groupes d'espaces en particulier, sois ceux des villes de Québec et Montréal. 
Celles-ci se démarquent d'une part par leur importance démographique, politique, sociale et culturelle et d'autre part par leur place dans le réseau urbain, ces deux villes occupant une position centrale. 
Nous nous attarderons de façon secondaire sur d'autres villes où il y a une communauté \lgbt{} organisant des activités publicisées, selon notre collecte de données. 
Nous nous pencherons plus particulièrement sur ces différentes villes plus loin dans ce chapitre à la Section~\ref{ssub:autres_villes}.

Ces diverses publicités référant aux activités des minorités sexuelles ont historiquement prises diverses formes, que ce soit sous la forme de feuillets ou de magazine culturistes~\citep{Higgins1999}~\footnote{On retrouve d'ailleurs ce   type de documents et plusieurs autres dans les \agq{}.}. 
Étant donné le cadre temporel de cette recherche, nous avons arrêté notre collecte de données sur les médias les plus importants en termes de distribution et de diversité de contenu, soit les revues \fugues{}, \sortie{} et les médias sociaux. 

Étant donné le cadre temporel dans lequel s’enchâsse ce mémoire, traité plus loin dans ce chapitre, nous avons basé l'analyse des lieux dans l'historique récent des communautés \lgbt{}, sachant tout de même qu'elle peut s'étendre à jusqu'aux débuts de la colonisation~\citep{Higgins1999}.
Nous tenterons tout de même de cerner brièvement le contexte récent des dernières décennies, selon les groupes analysés dans les chapitres subséquents et des informations disponibles dans la littérature. 
Ce cadre historique s'intéressera donc principalement au cas de Montréal étant donné le peu d'écrits existants dans le domaine scientifique sur les autres villes québécoises. 
Un travail en archives plus poussé serait à faire pour celles-ci, mais ceci est au-delà du champ couvert par ce mémoire.
\todo{Peut-être à revoir, pas clair selon Caroline}

\begin{figure}[ht]
	\begin{center}
		\includegraphics[width=18cm]{fig2.png}
	\end{center}
	\caption{Villes pour lesquelles des géosymboles ont été localisés à partir des
    magazines Sortie et Fugues.}\label{fig:carte_quebec}
\end{figure}

\subsubsection{La ville de Montréal}
\label{ssub:montreal}
Anciennement nommée Hochelaga puis Ville-Marie, Montréal est aujourd'hui la métropole de la province de Québec et la deuxième ville en taille au Canada après avoir été la première durant plusieurs décennies. 
\note{À quel point devrais-je décrire l'historique de la ville de Montréal? à compléter}.


Historiquement, dans la métropole, nous pouvons croire qu'il exista pour les minorités sexuelles une forme d'organisation durant pratiquement toute l'histoire de la colonie canadienne-française jusqu’aux années 1950, organisation marquée avant tout par la clandestinité et l'invisibilité~\citep{Higgins1999}. 
Par son statut de métropole très tôt dans l'Histoire, Montréal rayonnait suffisamment pour qu'une partie des individus des minorités sexuelles connaissent ou côtoient d'autres individus de la ville, cette probabilité grandissant avec l'amélioration des moyens de communication et l'urbanisation grandissante des lieux centraux. 
Peu de données par contre permettent réellement de décrire avec précision les contours de ces rassemblements ou de ces rencontres; bien souvent, la connaissance que nous avons de la vie de ces personnes se résume aux faits divers qu'on retrouve dans les \emph{pages jaunes} de l'époque, des médias imprimés spécialisés dans les informations à sensation. 
\todo{Trouver la citation de Higgins sur les gens qui   trouvaient les lieux LGBT rapidement}\citep[]{Higgins1999}. 
On sait par exemple que certains cinémas, comme le Midway dans les années 1920, aurait été un lieu de rencontres entre hommes\citep[30]{Higgins1999} \todo{à compléter}.
\begin{figure}[ht]
	\centering
	\includegraphics[width=15cm]{carto/mtl.png}
	\caption{Arrondissements ciblés pour la collecte de données: ville de
    Montréal}\label{fig:espaces_montreal}
\end{figure}
Aujourd'hui, Montréal est une métropole bien établie à l'échelle du continent nord-américain. 
Possédant de nombreuses institutions postsecondaires, dont quatre universités, elle attire en son sein une population immigrante importante et est réputée à certains égards pour l'ouverture dont elle fait montre envers la communauté \lgbt{}. 
\todo{à terminer}Le Village gai, un des plus développés au monde, l'existence de festivals comme le \anglais{Black \& Blue}, les \anglais{Outgames}, la Fierté, etc.\@sont reconnus.

Les arrondissements du Plateau-Mont-Royal et de Ville-Marie ont d'abord été ciblés pour la présence reconnue de plusieurs lieux \qus{}. 
D'abord, le lieu occupé actuellement par le quartier des spectacles a été il y a plusieurs décennies le \anglais{Red Light} du centre-ville et plusieurs cabarets ont été les premiers lieux de rassemblement d'individus \lgbt{}~\citep[198]{Podmore2015}.
Pas vraiment un lieu de communauté et de sécurité, le Red Light et ses cabarets servait plutôt de lieu de travail pour plusieurs travailleurs du sexe dont la sécurité était minée par la criminalité et une hétérogénéité de la clientèle qui ne protégeait pas des violences homophobes et transphobes~\parencite[91]{Higgins1999}. 
De cet espace, une partie de la \textquote{communauté} --- d'abord des investisseurs puis la clientèle --- s'est dirigée vers le quartier sud dans le même arrondissement pour créer ce qui allait devenir le Village gai. 
Une caractéristique intéressante est la proximité de cet espace d'institutions importantes comme l'\uqam{} et l'Université McGill; ces universités seront occupées durant ces mêmes années par divers groupes étudiants et politiques qui mèneront en partie plusieurs des luttes politiques pour la reconnaissance des droits et de la visibilité des gays et des lesbiennes.

Une autre partie de la clientèle \lgbt{} du \anglais{Red Light} s'est dirigée plus au nord, principalement les femmes. 
Nous savons aujourd'hui que plusieurs lesbiennes ont investi pendant près de deux décennies le Plateau Mont-Royal, principalement autour du boulevard Saint-Laurent~\citep[599]{Podmore2006} étant donné la présence de nombreux bars réservés aux femmes qui existèrent durant les années 80--90. 
Par contre, leur présence aujourd'hui s'est amoindrie; l’ embourgeoisement et une certaine compétition avec le Village --- dont certains bars sont devenus mixtes et ont réussi à attirer une nouvelle clientèle lesbienne plus jeune --- en seraient en partie la cause~\citep{Podmore2015}.

La figure~\ref{fig:espaces_montreal}\footnote{Afin de faciliter la lecture de la   vue d'ensemble de la ville de Montréal et pour ne pas encombrer la liste des   acronymes au début de ce mémoire, les noms complets des arrondissements de   Montréal se trouvent en annexes.\todo{à faire, et à présenter au début du mémoire, selon Caroline}} montre la position de ces deux arrondissements et nous verrons dans les chapitres suivants où ces espaces ont été situés en complément des données accumulées dans cette recherche \todo{approfondir le contexte historique?}

\subsubsection{La ville de Québec}
Capitale de la province de Québec, la ville de Québec est également considérée comme la plus ancienne ville fondée par les Européens lors de la colonisation du continent américain. 
Anciennement, cet espace situé au nord de la rive du Saint-Laurent était connu sous le nom de Stadaconé et était un établissement iroquoien~\citep[91]{Dickason1996}. 
La ville de Québec en soi, fondée par des colonisateurs français, devient une possession britannique durant le \siecle{18} tout en demeurant un lieu essentiellement habité par des Canadiens français, avec une minorité anglophone protestante dans la partie de la ville nommée la haute-ville de Québec\missref{}.

Si la ville a longtemps été limitée aux quartiers centraux, aujourd'hui, la ville de Québec possède une diversité importante de quartiers différents nés de la fusion d'anciennes municipalités et d'une croissance importante à partir du milieu du \siecle{20}. 
Un des legs de ces fusions de territoires a été le déplacement d'institutions importantes, comme l'Université Laval dans le quartier Sainte-Foy; un des quartiers parmi les plus centraux et anciens, le quartier Saint-Jean-Baptiste, semble avoir été reconnu comme l'espace de vie des minorités sexuelles dans la capitale selon certaines entrevues avec des individus de la communauté gaie actuelle~\citep{CSJB2011}. 
Anciennement ouvrier, il s'agit aujourd'hui d'un quartier morcelé par la rénovation urbaine, la patrimonialisation et par un processus d’ embourgeoisement~\citep{Hatvany2005,Mercier2014}.

Un espace important occupé par la communauté \lgbt{} a été le quartier Saint-Roch qui a également été habité et fréquenté par les minorités sexuelles.
Autre espace morcelé, on le considère aujourd'hui comme le centre-ville de la ville de Québec. 
Plus avancé dans son processus d’ embourgeoisement, de nombreux espaces \lgbt{} comme des bars ont existés sur ses rues importantes, principalement la rue de la Couronne et aux environs. 
Aujourd'hui, ses fonctions ont changé et l'on y retrouve principalement des locaux loués par les organismes communautaires appartenant à la communauté \lgbt{} de Québec.

À notre connaissance, pratiquement aucune recherche ne s’ est encore vraiment penchée sur la population \lgbt{} de la capitale. 
Nous avons par contre écrit précédemment un mémoire de baccalauréat qui s'est intéressé à cette question et qui a permis de dénombrer plusieurs lieux où la communauté serait active, mais nous ne possédons pas d'informations précises sur l'histoire de ces lieux, un exercice qui serait important à faire dans le futur. 
Si certains bars de la communauté semblent avoir disparu du paysage de la ville de Québec ces dernières années dans les espaces décrits précédemment, de nombreux espaces fréquentés par les communautés \lgbt{} existent encore aujourd'hui. 
Certains sont maintenant présents en périphérie du centre-ville avec la dispersion importante des universités de Québec --- L'Université Laval et l'\uqar{} (qui est située à Lévis) --- et des cégeps. 
N'étant pas visibles dans les espaces publics, ils n'ont donc pas fait l'objet d'une collecte de données sur le terrain ciblé comme l'ont pu être les bars ou certains organismes au centre-ville.
Par contre, à quelques occasions on a pu voir les groupes en charge de ces espaces présents durant les événements \lgbt{} de la ville de Québec ou dans les médias: dans ces cas particuliers, ils ont été identifiés et participèrent aux données.

Pour ce mémoire, au su des informations précédentes, nous avons décidé de cibler nos efforts sur le centre-ville de Québec, dans l'arrondissement La-Cité-Limoilou, visible à la figure~\ref{fig:espaces_quebec}. 
Il s'agit de l'espace dans la ville de Québec où l'on retrouve le plus de lieux appartenant à la communauté \lgbt{} locale tout en étant l'espace choisi par l'organisation Alliance Arc-en-ciel de Québec pour les célébrations de la Fête Arc-en-ciel, le pendant local des fiertés gaies.

\label{ssub:la_ville_de_quebec}
\begin{figure}[ht]
	\centering
	\includegraphics[width=15cm]{carto/qc.png}
	\caption{Arrondissement ciblé pour la collecte de données: ville de Québec}\label{fig:espaces_quebec}
\end{figure}


\subsubsection{Autres villes}
\label{ssub:autres_villes}
D'autres villes ont été envisagées pour ce travail de recherche. 
L'expérience terrain a par contre été limitée aux deux espaces décrits précédemment comme il a été mentionné au départ de ce chapitre. 
Néanmoins, comme souligné précédemment, des données ont été trouvées dans plusieurs autres villes du Québec. 
Nous pensions déjà avant la collecte de données qu'il serait possible de traiter des villes de Rimouski et de Sherbrooke selon nos propres connaissances. 
Plus précisément, nous savions déjà à titre de membre de la communauté \lgbt{} que ces villes possèdent respectivement une communauté plus ou moins active. 
Nous envisagions donc de vérifier si ces villes sont des cas d'exception ou si celles-ci possèdent, en raison de leur taille et leur place dans un réseau plus large de villes, les caractéristiques nécessaires à l'apparition d'une communauté \lgbt.
L'usage de certains médias de la communauté que nous traiterons à la section \ref{sec:source_des_donn_es} nous a effectivement permis de confirmer cette intuition; nous avons dressé à la figure~\ref{fig:carte_quebec} une carte dressant l'ensemble des lieux où ont été localisés des géosymboles durant la collecte de données à l'échelle du Québec. 
Selon les résultats de notre analyse de données présentée dans les prochains chapitres, nous arriverons en toute fin à dresser une hiérarchie des villes selon la complexité et la visibilité de leur communauté respective.

\section{Cadre temporel}
\label{sec:cadre_temporel}
Pour le cadre temporel, nous abordons l'époque contemporaine en couvrant le \siecle{20} au maximum, sachant que le sujet d'étude est particulièrement récent et que la majeure partie des données proviendront du dernier demi-siècle. 
En effet, la mise en contexte du sujet nécessite une prise en considération de l'évolution historique des communautés formées par les minorités sexuelles. 
En effet, selon les circonstances historiques décrites dans la littérature~\citep{Spencer2005}, on peut estimer que les géosymboles de la diversité sexuelle actuelle dateraient au maximum des luttes ayant suivi les émeutes de Stonewall aux États-Unis, mais que ces communautés ont existé plusieurs décennies avant de porter un discours politique et engendrer un mouvement civique de grande ampleur.

Par contre, en ce qui concerne les données collectées pour répondre à la question de recherche, la couverture temporelle est beaucoup plus courte et récente: nous couvrons les dix dernières années pour arriver à dresser un portrait actuel des géosymboles des communautés \lgbt{}. 
Les données étant particulièrement variées dans leur provenance, certaines ont été prises durant les mois précédents la rédaction du présent mémoire, alors que d'autres proviennent d'archives conservées et couvrant toute cette décennie. 
Nous décrirons plus en profondeur la couverture temporelle de ces données dans les Sections~\ref{sec:source_des_donn_es} traitant des sources de données.

\section{Approche méthodologique}
\label{sec:approche_m_thodologique}
La découverte et l'analyse des géosymboles d'un groupe culturel donné ne s'appuient pas sur une méthodologie particulière; au contraire, ils apparaissent à la suite d'une observation approfondie du groupe étudié et de sa relation particulière avec le territoire. 
Nous croyons par contre que certaines approches méthodologiques sont plus appropriées selon les contextes de recherche et les groupes étudiés.

Les questions de la visibilité, de la présence et de l'acceptabilité sociale sont récurrentes dans l'histoire récente des minorités sexuelles en occident. 
Nous croyons donc que mettre cette particularité culturelle de l'avant devrait être un des motifs derrière le choix de l'approche méthodologique adoptée dans ce travail. 
La géographie visuelle semble ici la réponse à cette préoccupation. 
Utilisant des documents visuels comme matériau d'analyse, elle s'inscrit dans le champ plus large de l'analyse qualitative.

Contrairement à d'autres travaux en géographie culturelle, cette recherche s'est appuyée principalement sur l'observation et la recherche dans des archives plutôt que des entrevues avec des individus \lgbt{} ou témoins des géosymboles mobilisés par les communautés traitées. 
Cette décision est motivée par une volonté de couvrir dans le cadre restreint d'un mémoire l'ensemble du spectre \lgbt{}.
Nous gardons donc en tête que ce travail possède donc des limites importantes quant à sa représentativité, élément que nous traiterons en discussion.
Dans cette section, nous nous intéresserons donc à ces méthodes alternatives en analyse qualitative qui devraient nous permettre de faire ressortir les différents géosymboles qui devraient marquer les espaces urbains d'une présence des communautés \lgbt{}.

\subsection{Analyse visuelle}
\label{sub:analyse_visuelle}
L'analyse visuelle des territoires est profondément ancrée dans la discipline de la géographie. 
En effet, pour plusieurs penseurs\missref{}, la géographie, comparativement à d'autres disciplines en sciences humaines, demande du chercheur qu'il se déplace sur son terrain d'étude pour pouvoir se l'approprier visuellement et arriver à en faire une analyse juste. 
Si le caractère visuel n'est pas toujours explicitement présent dans les travaux des géographes, nombreuses sont les études de cas à intégrer des photographies des espaces étudiés.
On peut penser aux études en géographie physique par exemple où l'image sous forme de photographie peut servir à montrer au lecteur les différentes composantes d'un sous-sol.
En géographie humaine, les photographies peuvent plutôt servir à montrer un paysage ou une organisation spatiale humaine particulière.

Également, un des outils de prédilection de la géographie est la carte pour la présentation de données, progressivement remplacée --- ou améliorée --- par les \sig{} qui remplissent cette fonction en intégrant des éléments d'analyse alimentés par des algorithmes. 
Par contre, si le visuel est aussi important, peu de travaux utilisent la photographie pour une raison autre que la démonstration~\citep[151]{Rose2008}.
Bien que la description peut servir des buts pertinents, comme la démonstration de l'évolution d'un espace dans le temps ou encore pour appuyer un argument, d'autres usages existent~\parencite[158]{Rose2008}. 
Nous commencerons donc cette partie de l'approche méthodologique par une description de l'utilisation de l'analyse visuelle dans notre recherche. 
La section suivante se penchera plus particulièrement sur l'usage des \sig{} dans la géographie culturelle.

Une des volontés derrière cette recherche est de poursuivre l'utilisation et l'expérimentation des méthodes visuelles entamées par d'autres chercheurs durant la dernière décennie. 
Au-delà d'un simple renouement avec une pratique traditionnelle, nous considérons qu'il s'agit d'une méthodologie qui a le potentiel de faire le pont avec la théorie géographique, plus particulièrement en géographie culturelle vers une des composantes importantes des groupes culturels minoritaires ou marginalisées, la visibilité. 
De plus, les géosymboles que nous avons traités dans le dernier chapitre ont comme caractéristique d'être des symboles visuels, matériels ou immatériels, qui permettent d'articuler un territoire propre à un groupe donné. 
Ainsi, les géosymboles jouent en quelque sorte le rôle des images de l'analyse visuelle. 
\citeauthor{Rose2012} caractérise d'ailleurs les images et les pratiques visuelles d'une façon analogue à la définition que nous avons des géosymboles, à ce savoir que : \foreignblockquote{english}[{\cite[32]{Rose2012}}][.]{\textelp{} the spaces and   practices of display \textelp{are} especially important to bear in mind given   the increasing mobility of images now; images appear and reappear in all sorts   of places, and those places, with their particular ways of spectating, mediate   the visual effects of those images}.
Repérer ces géosymboles pourrait se faire en travaillant directement avec les populations données, par l'observation et l'entrevue par exemple, méthode prisée dans la plupart des travaux déjà effectués \citep[][pour ne citer que ceux-ci]{Giraud2014, Podmore2015a, Higgins1999}. 
Mais nous pensons que le processus d'analyse du territoire d'un groupe peut se faire du point de vue inverse, en nous intéressant d'abord aux géosymboles que l'on retrouve préalablement dans un territoire et en faisant le pont entre ceux-ci et les travaux déjà effectués sur l'histoire, la politique ou la sociologie et surtout la géographie \lgbt. 
Autrement dit, nous envisageons d'aborder directement le territoire tel qu'il se présente matériellement et dans les médias \lgbt{} et d'utiliser les méthodes visuelles pour approcher les géosymboles, le tout en nous appuyant sur des technologies comme les applications cellulaires et les \sig.

% La géographie culturelle continua son existence durant ces mêmes décennies
% selon plusieurs sous-disciplines, comme le tropicalisme ou les études plus
% régionales comme au Québec.

Une auteure importante à avoir travaillé sur le domaine de l'analyse visuelle est Gillian~\citet{Rose2008} dont les travaux détaillent plusieurs façons dont les géographes utilisent les images dans leurs analyses. 
En plus de l'usage de descriptions décrit précédemment, les images pourraient également être utilisées comme un outil de représentation, d'évocation ou encore servir de fragment de culture matérielle. 
Ce dernier point est également soutenu par \citet{Frosh2001} pour qui l'image, ou plus précisément la photographie, est un fragment d'une performance sociale de représentation qui mérite analyse. 
La capacité de représenter des pans de la réalité donne au photographe un pouvoir particulier sur le sujet ainsi que sur l'observateur de la photographie.
Il apparait donc nécessaire dans notre recherche de prendre en compte cette réalité de deux façons: d'abord, reconnaître que les documents visuels analysés tirés des médias sociaux et des archives sont le fait d'individus possédant une certaine vision de la communauté \lgbt{} ou du moins, cherchent à la présenter d'une certaine façon, consciemment ou non. 
Ensuite, utilisant la photographie comme outil pour situer certains géosymboles, nous avons nous-même un biais envers l'objet étudié, malgré notre volonté de demeurer objectif, surtout dans un contexte où le sujet est social et culturel. 
Nous ne croyons pas par contre qu'il s'agit d'une faiblesse de notre perspective méthodologique, mais plutôt une des particularités des méthodes qualitatives. 
Nous reconnaissons toutefois que dans des recherches subséquentes, il serait également intéressant d'utiliser des photographies produites consciemment dans le cadre d'une recherche, mais par les individus appartenant au groupe culturel visé. 
Cette méthode a également fait ses preuves, notamment dans les travaux de \citet{Kwan2008},~\citet{Moore2008} et de~\citet{Markwell2000}.

En ce qui concerne la collecte de données sur le terrain, notre méthode se rapproche plutôt de l'article de~\citet{Leroy2010} dans lequel la photographie est utilisée pour montrer la diversité des représentations présentes dans la Fierté parisienne. 
Si celui-ci ne décrit pas particulièrement comment l'échantillonnage a été créé, nous croyons qu'il est nécessaire ici de faire cet exercice, tel que souligné par \citet[109]{Rose2012} qui constate que, trop souvent, les travaux en sémiotique ne s'attardent pas suffisamment sur le contenu sélectionné, ne mettant l'accent que sur le contenu intéressant à analyser.


% \todo{Parler de Suchar} La méthode de collecte de données s'inspire
% essentiellement de la technique

% \citep{Rose2012} \citep{Rose2008} \citep{Rose2003} \citep{Dorrian2003}
% \citep{Suchar1997} \citep{Frosh2006} \citep{Frosh2001}

L'analyse des données visuelles --- principalement celles qui ont été collectées dans les données d'archives --- s'est inspirée de différentes questions amenées par \citet[157]{Rose2008}, à savoir:
\begin{itemize}
	\item Qui utilise ces photographies, comment et pourquoi?
	\item Est-ce que l'usage de ces photographies a un effet particulier (sur les sujets, les auditeurs, etc.)?
	\item Où ces photographies ont-elles été prises?
	\item Est-ce que la localisation du sujet de la photographie est à prendre en compte? Comment?
	\item Et enfin, quel est l'impact des photographies sur les lieux où elles ont été prises, mais également sur les lieux où elles sont utilisées/diffusées?
\end{itemize}

Ainsi, nous croyons que les différentes réponses apportées pourront ensemble dresser un portrait de la territorialité des groupes \lgbt{} du point de vue de la visibilité. 
Dans la section suivante, nous verrons comment ces imaginaires visuels pourront être situés dans l'espace, apportant ainsi des nuances ou des approfondissements aux différents portraits que nous dresserons de ces groupes.

\subsection{Géolocalisation du géosymbole}
\label{sub:g_olocalisation_du_sybole}
Ce n'est pas toutes les branches de la géographie qui s'attardent ou qui se sont attardées aux données visuelles, surtout depuis l'avènement de la géographie quantitative, constaté durant la deuxième moitié du \siecle{20} et du bond technologique apporté par l'informatisation. 
Le travail à partir de bases de données statistiques et géoréférencées permit également aux chercheurs de prendre un certain recul du sujet d'analyse et du même coup prendre une distance avec la partie terrain de la collecte de données. 
On doit tout de même souligner que si ces nouveaux outils informatiques permirent de couvrir des ensembles spatiaux beaucoup plus importants que précédemment, comme on peut le constater dans la branche de la géographie utilisant les méthodes d'analyses spatiales.\note{à garder ou impertinent?}

L'utilisation des \sig{} en géographie culturelle n'est par contre pas encore très répandue; on les retrouve par contre fréquemment utilisés dans le cadre des analyses quantitatives en géographie urbaine, économique et dans les disciplines affiliées comme l'économie \missref{}. 
Une des volontés derrière cette recherche est d'arriver méthodologiquement à faire un pont entre ces techniques propres à la géographie quantitative et celles plutôt utilisées en géographie culturelles dans lesquelles la carte comme résultat d'un \sig{} joue plus souvent le rôle de description d'un espace à analyser. 
Nous croyons comme plusieurs autres auteurs \citep[4]{Elwood2009} que les \sig{} peuvent être un outil pratique à la compilation de données de sources multiples tout en permettant une localisation souvent très précise, surtout lorsque le chercheur travaillant à la collecte a accès à des adresses postales ou encore à un \gps.
Ce but est depuis peu partagé par d'autres chercheurs en géographie culturelle et en géosciences~\citep{Perkins2003,Elwood2011,Elwood2009,Kwan2008,Madden2009,Knigge2006,Jung2010}.


Comme le souligne~\cite{Kwan2008}: \foreignblockquote{english}[{\citeyear[444]{Kwan2008}}][.]{GIS-based data   analysis, mapping, and visualization are deployed to complement or triangulate   (i.e., verify results using multiple data sources) the knowledge acquired through the qualitative component of the research} 
Il devient ainsi possible d'enrichir certaines couches d'informations en y liant d'autres, en ajoutant une photographie à un lieu géolocalisé sur un \sig{} ou encore positionner une photographie dans l'espace et pouvoir comparer la position de chaque photographie. 
Dans un contexte multimédia, cette caractéristique peut s'appliquer à plusieurs types de médias, comme la photographie ou l'audio, bien que ce type de données s'intègre mal à l'écriture dans le contexte de la rédaction d'un mémoire ou d'un article scientifique.

En plus de ces caractéristiques \citet{Elwood2011} s'est attardée à la définition de la géovisualisation comme méthode qualitative. 
Dans un but de comparaison dépassant le simple type de données traitées, Elwood souligne que contrairement à l'utilisation classique des \sig{} dans un contexte quantitatif: \foreignblockquote{english}[{\cite{Elwood2011}}][.]{\textelp{} what defines   these approaches as qualitative geovisualization is not absence of numeracy.
  Rather, it is their integration of multiple modes of representation –--  visual,textual, numerical --– and iterative interpretive analysis of these representations to tease out what they reveal about social and material situations. 
Most of these qualitative geovisualization methods emerge from qualitative GIS, but could clearly be applied to georeferenced multimedia drawn from the geoweb}. 
Notre recherche comprendra elle-même une multitude de médias visuels d'origine différente, dont des données en ligne que l'on pourrait assimilé au \emph{géoweb} comme souligné dans cette citation d'\citeauthor{Elwood2011}. 
L'avantage ici des \sig{} sera de pouvoir incorporer ces images et en même temps de les situer pour mieux rendre compte de la dispersion de ces représentations et en même temps de simplifier la tâche de garder en mémoire la localisation des symboles trouvés par nous-même sur le terrain.

À la suite de l'observation des espaces urbains ciblés et du travail en archives, l'ensemble des données on subit un traitement de géolocalisation pour la plupart et de codage pour la totalité. 
Nous avons décidé de ne pas effectuer de géoréférencement précis pour les symboles accumulés à partir des médias imprimés archivés par manque de qualité dans certaines données; dans certains cas, des adresses manquantes dans les symboles et des fermetures/dissolutions au fil des années nous ont empêché de retrouver la localisation de chacun des espaces. 
De plus, certains symboles ne concernaient pas nécessairement des espaces précis, mais consistaient en des messages lancés à la communauté par des organisations hors communauté ou localisés  dans des lieux qui ne semblait pas \latin{a priori} pertinents. 
Nous pouvons penser par exemple aux messages publiés pas certains syndicats ou paliers de gouvernements nationaux.

Les données que nous avons géoréférencées consistent donc en des photographies prises durant les événements \lgbt{} auxquels nous avons participé et les données collectées sur les réseaux sociaux. 
Le géoréférencement s'est déroulé en plusieurs étapes et à l'aide de différents outils. 
D'abord, à l'aide principalement du logiciel GIS Cloud et d'une façon secondaire Google Photo, nous avons accumulé des photographies des géosymboles dont le géoréférencement a été effectué en utilisant le \gps{} intégré au cellulaire utilisé pour la collecte de données. 
Nous avons retenu GIS Cloud comme au départ, celui-ci devant servir tout au long de la collecte, de façon exclusive. 
Par contre, l'usage de cette solution s'est avéré moins concluant qu'envisagé au départ. 
D'abord, il faut savoir que, comme énoncé sur la page d'accueil de leur site internet, GIS Cloud consiste en un service de collecte sur le terrain et d’emmagasinage de données, en plus d'être utilisé pour la publication \citep{Cloud2014}. 
Plus spécifiquement, Cloud GIS présente plusieurs avantages quant à la collecte de données sur le terrain dans le cadre d'une recherche utilisant des méthodes qualitatives. 
Ce service permet la construction d'un guide d'entretien similaire à celui utilisé par exemple dans le contexte d'entretiens enregistrés. 
Avant la collecte, il est en effet possible pour le chercheur de construire des questions telles quelles seront utilisées pour la collecte.
Celles-ci peuvent être des questions fermées ou des questions ouvertes, ainsi que des questions qui sont en fait des objets capturés par le périphérique utilisé. 
Ainsi, dans le contexte de notre recherche, nous avons monté un questionnaire comprenant plusieurs questions sur le contexte de chaque point localisé, visible en annexes à la figure~\ref{ann:cloudgis}.

En pratique, lors du terrain, le logiciel s'est montré gourmand en ressources (avec pour conséquence une faible anatomie de l'appareil) et nécessitant une attention parfois difficilement conciliable avec le déroulement de l'activité en cours. 
Par exemple, dans les cas où nous avons eu à observer les espaces comme le Village gai, nous avions tout le temps disponible pour nous arrêter devant un géosymbole potentiel, l'analyser, prendre des notes et bien utiliser le questionnaire. 
Par contre, lors de manifestations par exemple, où plusieurs individus tenaient des pancartes, un discours était prononcé et que les individus en-dehors de l'événement réagissait, il devenait difficile de tout prendre en note à l'aide du logiciel. 
Par adaptation, il a été nécessaire de modifier notre usage de l'application pour poursuivre la collecte de données de façon efficace. 
D'abord, nous avons décidé de nous servir de l'application Cloud GIS seulement à une reprise à chaque partie explicite d'un événement pour marquer un point dans la base de données. 
Par la suite, il devenait possible de prendre des photos normalement en dehors de l'application. 
Nous pouvions du même coup collecter des notes de terrain par nos propres moyens.
Les données étant datées, nous pouvions prendre compte de l'heure, du contenu des notes et du géoréférencement des données présentes dans l'application cellulaire. 
Cette méthode s'est avérée plus efficace sachant qu'il était maintenant possible d'utiliser de facon prolongé du cellulaire pour la collecte de données sans avoir à constamment synchroniser nos données collectées avec une base de données.

Lors du traitement des photographies, nous avons remarqué que l'ensemble des photographies prises sur cellulaire étaient déjà géoréférencées par défaut par le système d'exploitation de l'appareil. 
Cette opération nous permit d'accélérer une partie du processus de localisation et d'augmenter la fiabilité du positionnement des données récoltées. 
Par cette fonction, nous avons obtenu un résultat similaire à ce qui avait été prévu au départ, c'est-à-dire le géoréférencement automatique des données. 
\todo{faire attention,   peut-être fusionner cette partie avec la section plus loin traitant des données, vu qu'une partie des informations se recoupe}

En ce qui concerne les données d'archives, le géoréférencement a été fait de façon manuelle à partir des adresses postales. 
Nous avons envisagé préalablement à la collecte que nous n'aurions pas toujours accès aux coordonnées des lieux ciblés. 
Nous croyions alors qu'il devrait être possible d'obtenir celles-ci à l'aide de certaines sources de données, comme les répertoires de Fugues ou encore Google Maps.
L'outil de Google possède un historique permettant normalement de garder en mémoire les lieux fermés, mais possédant préalablement une entrée dans leur base de données à l'époque de leur fonctionnement. 
Étant donné la couverture assez large de collecte de données sur dix années, certains établissements sont apparus et ont disparu, aujourd'hui remplacés par d'autres du même acabit ou d'espaces totalement différents. 
Il est donc possible de croire que ceux-ci ont été référencés à un moment ou à un autre sur Google Maps, mais nous n'avons pas obtenu plus d'informations sur la méthodologie de cette compagnie.
Nous ignorons également à quel moment les lieux ont commencé à être répertoriés de façon systématique.
Nous pensons par contre que les espaces plus urbanisés, comme Montréal et Québec, ont accès à un référencement plus systématique que les lieux moins centraux.
En ce qui concerne les adresses de certains géosymboles,  les publicités risquaient d'avoir des adresses intégrées, sachant qu'une publicité incite normalement le client potentiel à se déplacer sur les lieux du commerce ou du service.
D'autres par contre, comme des photos toutes simples, ont dû être situées à l'aide du contenu qu'il accompagne, s'il était présent.  
Dans le cas de documents comme le Fugues, le contenu visuel est pratiquement toujours accompagné de textes comme des articles ou du moins des titres pouvant fournir de genre de données pour la localisation. 
Par contre, en général nous disposions de peu de données réellement utiles à la géolocalisation des données. 
À l'aide des répertoires qui sont compris dans ce média, nous avons tout de même pu situer géographiquement une certaine partie de ces documents.

Dans le même cadre d'idée, la participation à des activités de la communauté \lgbt{} lors des différents événements qui se sont déroulés durant le terrain a permis, pour la planification, d'accéder à des données supplémentaires sur les réseaux sociaux. 
Nous savions préalablement à titre individuel que de nombreux événements sont organisés aujourd'hui à partir de certains sites web comprenant de telles fonctions. 
Le réseau retenu pour l'analyse est Facebook qui permet l'organisation d'événements et de leur publication auprès d'un grand nombre d'individus. 
Ces événements permettent de situer les différents événements, autant dans le lieu que dans l'espace tout en étant une plateforme pour publier des images porteuses de géosymboles.
\citep{Barkhuus2010} \citep{Boyd2010}

Finalement, nous avons réussi à géoréférencer une majeure partie de ces données variées en un seul \sig{} et ainsi permettre d'offrir une lecture visuelle des multiples espaces \qus{} des espaces urbains.

\section{Source des données}
\label{sec:source_des_donn_es}
Comme énoncé dans la section précédente, nous avons eu recours à diverses sources de données. 
Celles-ci ont été envisagées et retenues dans le but d'arriver à couvrir l'ensemble du spectre des minorités sexuelles, conformément au niveau d'activité de chaque groupe ou sous-groupe. 
Par la découverte de ces sources de données, nous avons pu relever d'autres données, sur Internet principalement ou dans des fonds alternatifs en archives. 
Notre processus de collecte de données s'apparente d'une certaine façon aux méthodes de la théorie ancrée.
Pour celle-ci, les chercheurs alimentent d'abord leur travail de théorisation à partir de premières données collectées pour ensuite continuer et améliorer la collecte selon les prémisses soulevées par une première analyse de ces données. 
C'est d'ailleurs cette méthode qui nous a poussé à inclure certaines données non prévues au départ de cette recherche, comme les données trouvées sur les réseaux sociaux. 
Également, choisir dès le départ les sources de données demeurait un choix ardu selon les modalités de notre recherche. 
Couvrir l'ensemble du spectre \lgbt{} ne pouvait se résumer à notre avis à un seul média étant donné le risque de biais dans notre recherche.

Ceux-ci étant en général considérées comme marginalisées, certains de ces groupes le sont plus que d'autres. 
Des enjeux de pouvoir particulier existent entre elles, alors que certaines minorités ont obtenu une plus grande sympathie au sein de la société et disposent donc de moyens communicationnels bien différents que d'autres minorités. 
Ainsi, la diversité de média permet d'éviter certains biais que l'on croie exister au sein de certains médias ou certaines manifestations de la présence d'identités particulières. 
Il ne s'agit pas ici de critiquer le public visé de certains médias, au contraire. 
Certaines manifestations géosymboliques sont le fruit de certains groupes minoritaires pour des raisons particulières, par exemple une meilleure reconnaissance dans la loi qui n'affecte pas d'autres groupes marginalisés. 
Également, l'analyse des communautés \lgbt{} au Québec est amorcée depuis déjà quelques décennies. 
Nous nous servirons de ces nombreux travaux dans nos analyses.

\subsection{Données d'archives}
\label{sub:donn_es_d_archives}
Nous décrirons donc maintenant plus en profondeur ces diverses sources de données. 
Les sources secondaires sont composées, d'une part, des médias imprimés gais et lesbiens en circulation en 2015 au Québec et, d'autre part, des données référencées des \agq{}. 
Ces archives, uniques au Québec, sont situées dans la ville de Montréal, précédemment dans le Quartier latin pour maintenant être dans le Village gai.
Elles ont le mandat de: \blockquote[{\cite{LAGQ2014}}][.]{\textelp{} de recevoir,   conserver et préserver tout document manuscrit, imprimé, visuel, sonore, et   tout objet témoignant de l'histoire des gais et lesbiennes du Québec   (\textsc{Canada})} et sont parmi les seules au Québec à disposer de telles données. 
Étant donné cet isolement institutionnel, les données qu'on peut y trouver sont nombreuses et débordent le cadre de ce travail de recherche. 
Dans certains cas, les données abondaient et le traitement aurait nécessité un temps et un effort qui n'aurait pas apporter d'informations supplémentaires dans le cadre du mémoire de recherche. 
Nous avons donc décidé de limiter la collecte de données à un nombre restreint de médias. 
Avec l'aide du personnel et de la professeur Julie Podmore, des sources ont été sélectionnées pour leur pertinence et leur statut récent lors d'une première visite.

Le premier média sélection est le magazine Fugues. 
Celui-ci, selon son site internet, se décrit comme suit: \blockquote[{\cite{LesNitram2015}}][.]{
  Fondé à Montréal par les Éditions Nitram, Fugues constitue le plus important média gai au Québec. 
Depuis sa fondation en 1984, Fugues jouit d’une notoriété et d’une crédibilité qui n’a cessé de croître au fil des années. 
On y retrouve toute l’actualité gaie d’ici et d’ailleurs, ainsi qu'une foule de rubriques et chroniques. 
En livrant chaque mois un contenu éditorial fiable sur l’actualité et les enjeux de la communauté GLBT, Fugues permet aux gais et aux lesbiennes de la région de Montréal et du reste du Québec, de rester informé sur ce qui concerne spécifiquement leurs communautés. 
C’est pourquoi plusieurs générations de gais et de lesbiennes du Québec apprécient beaucoup ce magazine et lui sont fidèles depuis trente ans} 
Comme on peut le noter dans cette description, le magazine privilégie un point de vue montréalais sans pour autant omettre l'activité des communautés \lgbt{} des autres  villes québécoises. 
Cette couverture nous a permis ainsi de trouver une grande diversité de données et de toucher à de nombreuses villes, plus qu'aucune autre source de données. 
L'ancienneté du média nous permet également de couvrir l'entièreté de l'époque désignée, plus précisément l'intervalle entre les années entre 2005 et 2015. 
À raison de douze numéros par années, c'est presque 120 numéros qui seront analysés pour la collecte de données. 
Nous avons décidé d'arrêter la collecte au mois d'août 2015, bien qu'a posteriori, la diversité de données recherchée dans cette recherche aille été atteinte plus tôt.

% \begin{quote}
%   Fugues est offert sur différentes plateformes. La version imprimée est
% 	 publiée 12 fois par année et comprend, outre une synthèse de l’information
% 	 et nos suggestions de sorties pour le mois, des analyses, des débats
% 	 d’idées et de nombreuses chroniques.

%   Le site web Fugues.com, quant à lui, suit l’actualité de plus près, et
% 	 propose plusieurs galeries de photos et de vidéos. Des nouvelles viennent
% 	 alimenter quotidiennement le site et ces articles, régulièrement cités dans
% 	 d’autres médias, sont repris sur les réseaux sociaux.~\citep{LesNitram2015}
% \end{quote}

Nous avons sélectionné comme deuxième média le journal Sortie:
\blockquote[{\cite{AllianceArc2014}}][.]{ Communautaire et participatif, le journal Sortie est produit par l’Alliance Arc-en-ciel de Québec dans le but d’informer la population sur les réalités et les droits des personnes LGBT+.
  Il a pour mission de traiter des enjeux et des événements en lien avec la lutte contre l’homophobie et la transphobie. 
Cinq éditions paraissent chaque année, chacune imprimée à 10 000 exemplaires couleurs de format tabloïd. 
Elles sont distribuées gratuitement dans plus de 200 points stratégiques de Québec.
  De plus, son édition présente et ses archives se retrouvent en version intégrale sur le présent site web} 
Possédant moins de moyens que le magazine Fugues étant donné la vocation communautaire du journal, il s'agit tout de même d'une des sources les plus complètes que nous traiterons dans cette recherche.
En effet, le journal Sortie s'étend sur sept années, soit de mai 2007 à décembre 2014. 
Le nombre de numéros fluctue d'année en année; nous avons couvert dans cette recherche 32 journaux. 
Par contre, si le journal n'est pas officiellement publié selon le site internet de l'Alliance Arc-en-ciel aucun numéro n'a été produit durant l'année 2015, dernière année couverte par cette collecte de données. 
Le choix de ce journal se justifie par sa position centrale à Québec, faisant contrepoids à la couverture plus montréalaise du magazine Fugues.

Pas un média à proprement parler, il a été convenu durant la sélection des sources de données d'inclure les archives du festival Pervers/Cité. 
Cette décision a été prise pour représenter également les milieux dits alternatifs existants au sein des minorités sexuelles québécoises. 
Ces archives, les moins volumineuses, possèdent certaines lacunes: nous disposons de données que pour les données 2011, 2014 et 2015. 
Ces données sont aussi plus variées étant constituées de feuillets d'informations, de cartes et d'affiches qui semblent plutôt être tirées de manifestations politiques idéologiquement similaires, mais sans liens en ce qui concerne l'organisation à proprement parler.

Également, nous souhaitions inclure les archives des festivals de Fierté Montréal et de Divers/Cité, son prédécesseur. 
Par contre, dans les deux cas, les archives possèdent trop peu de données: en ce qui concerne Fierté Montréal, l'événement et l'organisme semblent encore trop jeunes pour que les \agq{} possèdent des documents sur celui-ci. 
Pour Divers/Cité, l'organisation a récemment déclaré faillite~\citep{Cormier2015} ; nous pouvons croire que les documents dont disposent les derniers propriétaires seront éventuellement intégrés aux \agq{}. 
Pour l'instant, les \agq{} n'entreposent que quelques affiches et dépliants trop anciens pour être intégrées dans notre documentation. 
Néanmoins, on va le voir dans les prochains chapitres, ces deux festivals ont affiché des publicités et leurs programmes respectifs surtout dans le magazine Fugues .

\subsection{Données collectées sur le terrain}
\label{sub:donnees_collectees_sur_le_terrain}
En parallèle à la consultation et à la collecte des données auprès des sources secondaires, des données primaires ont été recueillies sur les espaces permanents de la diversité sexuelle comme le Village gai de Montréal. 
Aussi, des espaces dits temporaires ont été sondés, comme le vieux port de Montréal durant Divers/Cité ou la rue Saint-Jean-Baptiste pendant la Fête Arc-en-ciel dans la ville de Québec. 
À l'aide d'internet notamment, il sera possible de retracer une partie des événements publics organisés par les communautés \lgbt{}, sachant que ceux-ci peuvent permettre une mobilisation hors-ligne, du moins en contexte politique et ainsi occuper l'espace~\citep[153-154]{Mercea2011}. 
Ces données doivent comprendre également des photographies prises sur le terrain dans des secteurs des villes reconnus pour abriter des espaces \qus{}. 
Les photographies serviront notamment à capturer la composante visuelle des géosymboles rencontrés pour la recenser et servir par la suite de matériau d'analyse. 
Ces photographies seront géoréférencées dans le but d'ajouter une couche d'information spatiale qui devrait faciliter l'analyse géographique.

Il n'est pas prévu dans le cadre de cette recherche d'accorder des entrevues; néanmoins, il est envisageable que certaines informations soient recueillies à l'occasion auprès de passants ou d'individus impliqués dans les organismes, événements ou espaces identifiés si jamais il devait y avoir un échange fortuit avec ceux-ci. 
Ces informations seront recueillies dans ce que nous nommons un guide de relevé; les données qui y seront consignées un ajout complémentaire à la photographie et à l'identification des géosymboles. 
Il n'est donc pas envisagé de passer devant un comité d'éthique, car il s'agira essentiellement d'observations personnelles sur l'environnement humain et bâti entourant les géosymboles rencontrés.

À l'aide des technologies à notre disposition, nous avons pu procéder à la géolocalisation en temps réel des données collectées. 
Préalablement à la collecte de données, plusieurs applications ont été recherchées et testées pour faciliter la collecte et s'assurer d'avoir facilement accès aux différentes localisations par \gps. 
Plusieurs options sont possibles, notamment Cloud GIS, logiciel propriétaire et dont les options dans la version gratuite sont limitées, et OpenDataKit, logiciel libre complet, mais qui demanderait la mise en place d'un serveur~\citep{OpenDataKit2014}. 
Ces outils permettent l'enregistrement de données multimédias: dans le cas où des données écrites devraient être notées, notamment lors de discussions, on envisage d'utiliser des logiciels de codage réflexif~\footnote{Plus communément appelés en anglais   \cadqas}. 
Il sera possible de travailler soit avec Sonal ou avec \rqda{} selon le type de données collectées. 
Le guide d'entretien pourra également être analysé par la suite à l'aide des logiciels nommés précédemment.

\subsection{Sources secondaires}
\label{sub:sources_secondaires}
Ce travail de recherche s'appuie également sur des travaux déjà effectués sur les espaces \qus{} montréalais. 
En effet, malgré une certaine jeunesse des études sur les communautés \lgbt{} au Québec, plusieurs travaux s'y sont intéressés dans les dernières décennies. 
On compte parmi ceux-ci le mémoire de \cite{Leznoff1954} qui consiste en une ethnographie des homosexuels durant les années 1950, les divers travaux de l'anthropologue et membre fondateur des \agq{}, Ross Higgins, ainsi que divers articles scientifiques; plusieurs de ces travaux ont été nommés dans le chapitre précédent à la section \ref{sec:la_diversit_sexuelle_en_g_ographie}. 
Nous tenons à souligner que nous nous appuierons grandement sur le travail de \citep{Giraud2014} dans une perspective de continuité et d'approfondissement du travail déjà effectué.

\section{Séjours}
\label{sec:s_jours}
L'ensemble des données sur le terrain ont été collectées durant l'été 2015. 
Une première collecte a d'abord été effectuée durant le mois de juin dans la ville de Québec. 
Nous avons effectué celle-ci à l'aide de données déjà connues d'une recherche précédente dans le cadre d'un mémoire de maîtrise sur les divers lieux des minorités sexuelles de la ville~\citep{Vachon2014}. 
Étant donné que nous possédions un lieu de résidence dans la ville de Québec, nous avons pu mettre à l'essai les outils de collecte de données sans s'inquiéter concernant le temps nécessaire. 
De plus, au cours de cette période d'essai, aucun événement d'importance n'avait lieu en lien avec la communauté \lgbt{} de Québec. 
L'occupation des lieux correspondait à l'achalandage typique auquel on peut s'attendre durant l'année, si l'on fait fit d'une probable hausse d'activité lors d'événements d'importance ailleurs dans la ville. 
Durant cette première collecte de données, il est apparu nécessaire de détailler plus en profondeur le questionnaire construit à l'intérieur de l'interface de CloudGIS pour la suite de la collecte. 
Certaines limites quant aux outils utilisés, un cellulaire, nous a poussé à nous assurer d'avoir accès à une importante quantité de données réseau pour effectuer le transfert des données géolocalisées quotidiennement une fois débutée la collecte à Montréal.

Cette première partie effectuée, nous avons pu ainsi procéder au mois d'août à la plus grande part de la collecte de données dans la ville de Montréal.
Préalablement au départ, nous avons construit un calendrier en ligne dans lequel nous avons compilé l'ensemble des événements organisés par la communauté \lgbt{} prévus durant mon séjour. 
Une première partie de ces événements ont été trouvés à l'aide des sites internet officiels des organismes organisateurs d'événements comme Fierté Montréal ou Pervers/Cité. 
Étant donné la présence de plus en plus grande de Facebook dans la promotion desdits événements, nous avons également utilisé ce réseau social pour obtenir plus d'informations sur certains événements. 
Facebook permis aussi de découvrir d'autres moments qui n'ont pas été promus autrement sur  internet. 
C'est le cas de la totalité des événements du festival Qouleurs et des événements politiques. 
Les pages événementielles sur Facebook possédant un grand nombre d'informations tout en présentant des images promotionnelles au potentiel géosymbolique fort, elles ont été retenues dans nos données et feront l'objet dans les chapitres suivants d'une analyse plus poussée.

Notre collecte de données à Montréal a duré au total trois semaines. 
Les premiers jours ont été consacrés à une familiarisation avec les lieux durant lesquels nous avons visité le Village gai et observé les divers événements quotidiens et l'achalandage des lieux. 
À cause d'une hausse très forte d'activité vers la fin de la journée, étant donné la fonction commerciale des lieux, les matins ont été priorisés pour la photographie des géosymboles repérés. 
En collaboration avec les bénévoles des \agq{} pour l'établissement d'un horaire de recherche, nous avons pu travailler en archives et couvrir l'ensemble des événements \lgbt{} qui se déroulaient durant le mois. 

Comme il était prévu à la collecte, durant le mois d'août sont planifiés une grande part des événements \lgbt{} de l'année. 
Il s'agit d'ailleurs d'une particularité au Québec étant donné qu'ailleurs dans le monde ces événements ont lieu durant le début de l'été. 
Cette décision de produire la fierté annuelle à ce moment a été prise d'abord par Divers/Cité puis par Fierté Montréal et a été reprise également à Québec. 
D'autres événements d'envergure ont d'ailleurs été organisés en parallèle par différents groupes de la communauté. 
Il est donc apparu pertinent de profiter de cette occasion pour planifier la collecte de données à ce moment de l'année sachant quand même que le portrait dressé y est partiellement contingent. 

La collecte à Montréal terminée, le retour à Québec nous a permis de participer par la suite aux festivités de la Fête Arc-en-ciel. 
Cette dernière partie de la collecte de données a été facilitée par notre participation comme bénévole à l'événement. 
Nous avons donc ainsi eu accès à l'ensemble de l'espace et du temps couvert par les festivités. 
D'autres espaces nous auraient intéressantés, mais par conflit d'horaire il ne n'a pas été possible d'effectuer les déplacements nécessaires. 
C'est le cas des villes nommées à la Section~\ref{ssub:autres_villes}.


%%% Local Variables:
%%% mode: latex
%%% TeX-master: "../../memoire-maitrise"
%%% End:

%!TEX root = ../../memoire-maitrise.tex

\chapter{De la clandestinité à l'espace public}
\label{cha:de_la_clandestinite_a_l_espace_public}
\todo{titres et sous-titre à revoir : ne concorde pas avec les données}
% \chapterprecishere{\textquote{Everyone needs a place. It shouldn't be inside of someone else.} \par\raggedleft--- \textup{Richard Siken}, Crush}

\section{Spatialité homosexuelle aux débuts de la pathologisation}
\label{sec:spatialit_homosexuelle_aux_d_buts_de_la_pathologisation}

\section{Politisation et place publique}
\label{sec:politisation_et_place_publique}



\subsection{Mouvement des droits civiques}
\label{sub:mouvement_des_droits_civiques}

\section{Institutionnalisation de la diversité sexuelle}
\label{sec:institutionnalisation_de_la_diversit_sexuelle}

% section institutionnalisation_de_la_diversit_sexuelle (end)
% chapter de_la_clandestinite_a_l_espace_public (end)

\blockquote[{\cite{Pervers/Cite2015}}][.]{Organisé en collaboration,  Pervers/Cité est un festival d’été visant à faire des liens entre les groupes de justice sociale, les communautés queers et les visions radicales de ce qu'étaient et devraient être les Fiertés LGBT\@.
Dans un climat où prévaut l’agenda corporatif gai et l’aseptisation homogénéisée des queers, Pervers/Cité tâche de fournir des activités critiques et accessibles, destinées à redonner une colonne vertébrale au mouvement LGBT}.

Les codes ont été classés en catégories relativement aux images et textes qui sont liés aux codes.
Ceci s'explique par le fait que les images en soi orientaient le sens du code, à défaut d'avoir un code dont le sens serait suffisamment précis pour ne pas porter à confusion.
Par exemple, nous avons classé le code \code{maquillage} dans la catégorie \code{visuel}, car celui qu'on retrouvait dans certaines images servait en fait de visuel au sein d'une publicité.
Plutôt qu'être un maquillage porté par un individu qui dans ce cas, aurait été instinctivement placé dans une catégorie comme \code{habillement} ou \code{public ciblé} si celui-ci avait été offert à des enfants.

\section{Québec: spectre large}
\label{sec:qu_bec_spectre_large}
Nous commencerons notre présentation des résultats par les géosymboles de la ville de Québec.
Comme mentionné dans le chapitre sur la méthodologie, il s'agit de la deuxième ville en importance que nous analyserons.
En plus de figurer dans le magazine Fugues, celle-ci possède son propre média local, le journal Sortie.
Nous analyserons donc les symboles de la ville dans ces deux médias.

Dans le journal Sortie, nous avons relevé vingt et une images, dont dix-neuf à propos de la ville de Québec.
Les images utilisées sont particulièrement variées: si nous avons ciblé des publicités, en général elles ne mettent pas en promotion à Québec des produits ou des services d'ordre commercial, bien que ces dernières existent.
En effet, en matière de produits, les publicités sont pratiquement absentes; nous n'avons relevé qu'une seule pour la boutique érotique Chez Priape qui n'a pourtant pas de succursales à Québec, celle-ci se situant à Montréal.
Sur le plan des services, les symboles sont plutôt orientés vers les soirées organisées dans les bars de la capitale, le cabaret-bar Le Drague et le Bar St-Matthew’s, ou encore dans les saunas comme L'Hippocampe.
Ces soirées ciblent un public général ou un genre en particulier.
Le Drague dans ses publicités offre des soirées récurrentes dans le temps en visant des types d'événements basés sur le contenu au lieu d'une clientèle précise.
Dans le journal Fugues, nous avons plutôt relevé dix-sept images traitant de la capitale.
Cela se remarque dans les visuels: on se sert des couleurs de façon variées et les symboles ne semblent pas cibler un genre en particulier.
Des photographies sont utilisées, mais celles-ci mettent essentiellement en avant des personnificateurs féminins pour des spectacles de ce type.
Un autre cas d'activité mixte est celui d'une publicité pour une activité de danse, plus précisément du tango queer, où l'image, si elle montre un couple masculin, ne semble pas non viser un genre en particulier.
Elle est d'ailleurs la seule image utilisée dans le contexte de la ville de Québec où figure le terme queer.
On peut douter ici qu'il s'agisse d'un usage du terme dans un cadre politique.
Plutôt, le mot queer semble désigner le caractère négatif et alternatif mis de l'avant dans l'activité du tango, une danse habituellement pratiquée entre un homme et une femme.

Parmi les images d'activités s'adressant spécifiquement à un genre particulier, nous avons d'abord deux activités festives pour femmes, visibles à la figure~\ref{figs3132}.
Il s'agit de soirées organisées par le magazine Saphomag, un magazine destiné à une clientèle lesbienne.
Contrairement aux activités réservées aux hommes dont nous traiterons plus loin dans ce chapitre, celles-ci ne se déroulent pas dans des bars, mais dans des locaux loués à l'église Saint-Jean-Baptiste dans le quartier du même nom.
Sachant qu'il n'y a plus de bars strictement lesbiens comme il a pu en avoir à Montréal~\citep{Podmore2006}, on peut comprendre en partie la nécessité d'investir des lieux alternatifs.
Dans ce cas-ci, les organisateurs ne se sont pas tournés vers le bar mixte de Québec, Le Drague.
Il est important de souligner également que ces fêtes sont organisées dans le cadre de la Fête Arc-en-ciel de Québec; nous n'avons pas trouvé d'occurrences d'événements similaires durant le reste de l'année dans le journal Sortie.
Ces activités consistent fréquemment en des soirées animées par des disk-jockey et quelques fois sont des ateliers de peinture corporelle.

\begin{figure}
\centering
\subcaptionbox{Événement de 2007\label{fig31}}
{\includegraphics[width=9cm]{fig31.jpg}}
\subcaptionbox{Événement de 2008 (et événement mixte)\label{fig32}}
{\includegraphics[width=6cm]{fig32.jpg}}
\caption{Événements destinés aux femmes (lesbiennes) : deux éditions
  différentes}\label{figs3132}
\end{figure}

\section{Montréal: diversité des imaginaires}
\label{sec:montr_al_diversit_des_imaginaires}

\subsection{Fétichisme}
Nous avons remarqué une grande part d'imagerie liée au fétichisme sexuel dans les médias étudiés.
Ce fétichisme est particulièrement apparent dans les espaces orientés vers les hommes gais.
Nous retrouvons ces images presque essentiellement dans le magazine Fugues et dans une moindre mesure sur quelques événements Facebook, pour des activités touchant encore une fois cette communauté masculine.
Parmi les bars mettant en avant un tel imaginaire, on peut nommer en ordre d'importance l'Aigle Noir, le Tool et Le Drague, à Québec \todo{Vérifier l'ordre réel des bars utilisant ce type d'imagerie}.
Le Drague fait figure d'exception, car il s'agit du seul lieu hors Montréal de ce genre, quoique cette impression ne soit pas la norme pour cet espace.
En effet, aujourd'hui Le Drague s'est tourné vers d'autres types d'imageries: il s'agit maintenant d'un lieu essentiellement mixte et peu sexualisé, du moins dans les symboles évoqués.
Auparavant, et dans les premières années traitées dans notre collecte de données, le troisième étage servait d'espace de drague utilisé par les hommes et strictement pour eux.

Comme souligné par \citet{Giraud2013a}, le bar l'Aigle Noir vise une clientèle masculine d'abord et plus particulièrement fétichiste.
Ceci est d'autant plus visible par le choix du logo et des couleurs: le noir --- couleur fréquente pour les vêtements et les articles de cuir, un homme habillé de ce qui semble être du cuir et un aigle.
Cet aigle, figurant dans de nombreuses cultures selon des sens variés, peut rappeler l'imagerie militaire, tel qu'utilisé par exemple par l'État américain.
Cette interprétation cadrerait avec une part importante de la sous-culture fétichiste s'attardant plus particulièrement aux uniformes.
L'Aigle Noir ne fait pas montre de censure quant au type d'activité à caractère fétichiste qui s'y produisent: << party bobettes >>, << soirées bulles >>, party latex, ventes << d'esclaves >>, etc.
Ce type d'activités est propre au fétichisme de la sous-culture cuir.
En plus de ce matériau, d'autres sont utilisés selon les individus pratiquant le kink, comme le latex, ou certains vêtements non acceptés dans l'espace public, comme les sous-vêtements masculins, plus particulièrement les \anglais{jock-straps}\footnote{Suspensoir en français. Nous utilisons de préférence le terme \anglais{jock-strap} en concordance avec le terme utilisé dans les publicités. }.
Les soirées bulles semblent être une activité répandue dans de nombreux espaces: en de plus d'être un des genres d'activités proposées à l'Aigle Noir, on les retrouve également dans les saunas, autant de Montréal que de Québec.
Les soirées d'esclaves s'inscrivent plus particulièrement au sein du cadre du \bdsm{}\footnote{\citeauthor{ Turley2015} définisent le \bdsm{} comme: \foreignquote{ english}{\textelp{} the umbrella term used to describe a set of consensual sexual practices that usually involve an eroticised exchange of power and the application or receipt of painful and/or intense sensations (Barker et al., 2007).
The range of \bdsm{}- related activities is wide and complex.
“BDSM” denotes the assorted consensual activities involved in the experience of participating in \bdsm{}; bondage and discipline (B\&D), dominance and submission (D/s), and sadism and masochism (SM)~\citeyearpar[24]{Turley2015}.}}.
On retrouve de moins en moins cet imaginaire dans les dernières années couvertes par notre collecte de données en ce qui concerne celles extraites d'archives.
Cela correspond à quelques discussions que nous ayons eu sur le terrain avec certains bénévoles des \agq{}: le bar a évolué dans les dernières années pour satisfaire un plus grand éventail de clientèle, tout en demeurant un lieu de consommation essentiellement pour hommes gais.

Certains lieux orientés vers le fétichisme continuent toutefois à exister: c'est le cas de soirées organisées au \emph{Bunker}, la section sous-sol de \emph{Les Katakombes}, un bar lié aux scènes musicales métal et rock en dehors du Village gai.
Ce sous-sol rempli d'une certaine façon le rôle qu'a joué par le passé le bar l'Aigle Noir, et offrent des activités orientées vers le fétichisme en plus d'être un \emph{backroom}\footnote{À COMPLÉTER}.
Dans ce cas-ci aussi, l'imagerie fétichiste est utilisée, en misant toujours sur une clientèle masculine.
En ce qui touche l'accessibilité, les lieux diffèrent : si les activités semblaient gratuites dans l'Aigle Noir, \emph{Le Bunker} n'organise que des soirées épisodiques dont l'entrée est payante, à un tarif de 30\$.

Dans l'événementiel, certaines soirées utilisent des codes similaires.
C'est le cas notamment de certaines se déroulant dans le cadre de la Fierté 2015, comme l'événement  \emph{BlackNight}.
Dans celle-ci, le noir et le rouge sont utilisés avec comme point focal une photographie d'un homme portant la barbe et ce qui semble être un uniforme de cuir.
L'événement \emph{Beardrop, édition Montréal}, organisé par \emph{Scruff}\footnote{Application cellulaire de rencontre et de drague pour hommes fonctionnant à l'aide d'un \gps{} ,} est un autre exemple.
En effet, on retrouve dans l'imagerie utilisée les mêmes codes que ceux nommés précédemment, bien que le nom de l'événement semble cibler plus particulièrement les hommes entrant dans la catégorie de \emph{bear}.
En fait, dans l'image, on peut voir deux hommes sveltes, mais musclés dans une posture qui s'apparente à une danse.
Les vêtements, plutôt qu'être des uniformes de style militaire ou policier comme pour la soirée \emph{BlackNight}, sont des tenues ressemblant à des habits de travail pour l'un, et des sous-vêtements pour l'autre.
On utilise donc qu'une partie des codes visuels permettant d'identifier les hommes rentrant dans la catégorie de \emph{bear}; on ne retrouve pas la carrure habituellement mise de l'avant, soit un surpoids important ou une musculature prononcée.

L'ensemble des exemples décrits précédemment se rapportent à la ville de Montréal.
Peu de lieux ou d'événements emploient vraiment le fétichisme dans leur imagerie, sauf à quelques occasions à Québec.
En effet, on peut nommer déjà certaines soirées organisées dans le bar St-Matthew's qui mélangent fétichisme et masculinité dans la promotion de leurs événements.
Également, comme dans la Fierté montréalaise, la marche pour la diversité sexuelle est un des moments où plusieurs individus décident de montrer sur la place publique leur intérêt pour les sexualités alternatives.
On retrouve par exemple plusieurs personnes utilisant les drapeaux cuirs, latex et \bdsm{} ainsi que des costumes se rapportant à ces fétichismes ou à des pratiques affiliées, comme le \emph{puppy play}\footnote{Jeu de rôle de domination et soumission dans lequel le soumis incarne un chien  et le dominant un \anglais{handler}, à savoir le \emph{propriétaire} de l'animal.
Le tout est souvent appuyé par l'usage de certains accessoires comme des masques, queues en caoutchouc, des harnais, un collier, etc.}.

\subsection{Marches et manifestations}
\label{subsec:label}
\todo{Probablement que ça va changer de place à voir}
Montréal se distingue par l'activité politique de plusieurs groupes \lgbt{}.
Nous traiterons dans cette partie de deux cas en particulier: les marches organisées par les communautés lesbiennes et celle de la communauté trans.
Si celles-ci n'ont pas nécessairement de liens directs (on peut croire par contre que certains individus participent aux deux événements, nous reviendrons sur les raisons), elles adoptent des stratégies similaires.

\subsubsection{Marche Trans}
\label{subsubsec:marchetrans}
La Marche Trans s'inscrit plus largement dans le cadre de la Fierté Trans, un événement organisé immédiatement avant la fierté de Fierté Montréal.
Comme le laisse entendre le nom choisi, la marche s'adresse plus particulièrement aux personnes trans, quoique ce dernier terme rassemble un grand nombre d'identités entourant le genre.
Parmi ces identités, on compte les personnes trans\footnote{nous n'utiliserons pas les termes transsexuels ou transgenres, car ceux-ci n'apparaissent dans aucun document traité entourant cet événement}, les personnes non binaires dans le genre\footnote{Nous utiliserons une définition assez large du terme; nous considérerons que celui-ci représente autant les personnes se disant non binaires que celles sans genre, neutrois, demi-hommes, demi-femmes, au genre fluctuant, etc.\citep[see][]{Barker2015}}et les personnes intersexes.
Nous appuyons cette définition d'ensemble de trans par l'imagerie utilisée par les organisateurs, plus particulièrement l'affiche de la marche.
Celle-ci montre en effet un grand nombre de symboles de ces différentes identités, comme des drapeaux.
Chacune de celles nommées précédemment est évoquée par un des drapeaux: \todo{faire la liste des drapeaux et des identités}.

Comme nous l'avons souligné dans le début de cette section, la Marche Trans s'inscrit dans la Fierté Trans, cette dernière comportant d'autres événements.
Ceux-ci consistent en différentes fêtes et campagnes de financement qui ont toutes eu lieu dans une même soirée au café Cléopâtre.
Cet espace est particulier par sa proximité avec l'histoire des communautés \lgbt{} québécoises et de sa proximité relative avec la communauté trans.

La marche s'est fait un trajet assez linéaire et situé dans les espaces reconnus de la communauté \lgbt{}.
Tel qu'on peut le voir à la figure \todo{insérer la   figure}, la manifestation commence à proximité du café Cléopâtre, traité précédemment, dans le cœur de l'ancien \anglais{Red Light} montréalais.
Cette marche se dirige par la suite dans le Village gai pour terminer dans le parc La Fontaine, toujours à proximité du Village.
Plusieurs arrêts ont marqué cette marche.
Au départ, plusieurs intervenants ont procédé à des discours par rapport aux droits des personnes trans et dans un cas précis, le cas des femmes trans de couleur (tel que décrit par la banderole utilisée).
Quelques organismes étaient à la marche tout en étant visibles; nous avons noté la présence de l'\atq{} et du \rlq{}.
On peut croire que d'autres organismes ou membres d'organismes participaient aussi étant donné la présence de ceux-ci durant la soirée qui précéda la marche, comme l'\astteq{}.
Par la suite, la marche se dirigea vers le Village gai pour un arrêt au parc de l'espoir, un lieu commémoratif aux victimes du \sida{}~\citep{Lafontaine2012} connu pour avoir été le théâtre d'actions politiques par le passé.

La Marche Trans s'articule autour d'un discours particulier, étant donné le contexte politique dans lequel cette dernière s'inscrit et également de l'actualité québécoise du point de vue législatif.
En s'intéressant au texte d'invitation de la Marche Trans tel que publié sur Facebook, on apprend que la marche s'appuie sur plusieurs revendications entourant le changement de statut légal de genre.
Ces points tournent autour du statut de citoyenneté, l'âge, le genre, les exigences médicales (au niveau chirurgical notamment) et du coût des démarches.
D'autres revendications étaient également soutenues, ayant moins à voir avec le statut légal comme l'absence de ressources spécialisées, notamment en prévention du \vih{} chez les personnes trans et de soutien pour le changement de statut.

On va mieux le comprendre dans la section suivante, mais la Marche Trans utilise une stratégie comparable au groupe des personnes \dyke{}, consistant en l'utilisation de la marche pour s'afficher publiquement.
Cette marche correspond à première vue à une manifestation politique dans laquelle la visibilité est extrêmement importante.
En effet, l'ensemble du parcours a été choisi en correspondance avec les zones de grandes affluences des quartiers centraux, dans le cas qui nous intéresse, la rue Sainte-Catherine.
De plus, comme le défilé de Québec et la marche Dyke, les participants prenaient part activement à la marche en prenant la rue.

Nous avons remarqué aussi une présence policière importante pour la gestion du traffic.
Certains participants ont décidé de leur propre intiative d'inscrire des slogans en lien avec le militantisme trans, de façon analogue à la marche Dyke.
Par contre, dans ce cas-ci, cet activisme n'était pas encadré.
Visible à la figure \todo{Mettre la figure avec la craie}, le \emph{graffiti} a été fait rapidement et en ciblant un sauna du village gai.
Selon le message du dessin, le sauna aurait un historique transphobe.


\subsubsection{La Marche Dyke}
\label{subsubsec:marchedyke}
La Marche Dyke, comparativement à la Marche Trans, ne s'inscrit pas dans un événement plus large.
En fait, nous constatons que le choix de la date se fait plutôt en réponse à la fierté organisée par Fierté Montréal.
Nous croyons effectivement que, selon les motifs politiques de l'événement, celle-ci vise à offrir une visibilité à la communauté \dyke{} que l'on ne retrouverait pas dans la Fierté plus traditionnelle.
Ça l'a été le cas d'ailleurs, une marche a été organisée pour les lesbiennes en général, cette dernière étant soutenue par l'organisation de la Fierté.
Par conflit d'horaire, nous n'avons été présent que pour la Marche Dyke.
Si cette section va principalement traiter de cette dernière, nous nous intéresserons également à la marche lesbienne selon les informations que nous avons pu accumuler sur Facebook et dans la documentation promotionnelle de Fierté Montréal.

La visibilité semble être le but premier derrière la Marche Dyke: ceci est particulièrement apparent par le choix esthétique de la bannière ornant l'événement Facebook, où l'on peut y voir des pictogrammes d'œil ainsi que dans le slogan utilisé.
On retrouve également cette visibilité portée par la volonté de manifester sa présence dans l'espace public où la visibilité est normalement mobilisée.
Ceci fait contraste avec l'espace privé qui s'appuie plutôt sur les notions d'intimité et par l'absence d'observateur.
La présence d'un tel observateur demeure possible, mais on parle  ici d'intrusion et d'une certaine part de violence (\todo{Trouver des références sur l'intimité dans l'espace privé}).

Pour la suite de notre analyse de la Marche Dyke, nous nous pencherons sur les publics sollicités, à savoir quels individus sont invités à participer à la marche et vers qui le message de la marche est orienté.
Pour répondre à ces questions, nous nous appuierons sur le travail de~\cite{Podmore2015a} sur l'édition 2012 de cette marche.
Se basant sur les recherches effectuées sur des éditions d'autres villes américaines, comme Chicago, on apprend que les marches \dykes{} prennent racine dans une certaine exclusion des femmes des fiertés mixtes.
Elles sont donc apparues par un désir d'auto-organisation et pour faire valoir leur présence au sein du mouvement principal derrière les fiertés.
Par conséquent, celles-ci décidèrent de créer leurs propres marches, celles-ci se déroulant quelques jours avant la marche officielle.
C'est également la stratégie choisie par les organisatrices de la marche de Montréal des dernières années, qui ont toujours placé l'événement quelques jours avant les débuts de la semaine de la fierté.
La marche des femmes quant à elle se situe durant la semaine de travail, donc, avant la marche officielle qui se déroule habituellement dans la dernière fin de semaine de la fierté.

Ce positionnement particulier et cette division du mouvement semblent conforter une certaine identité autour du mot \dyke{}, terme que~\citet{Podmore2015a} dans son article \citetitle{Podmore2015a} décrit comme politisé et radicalisé comparativement au terme plus général de lesbienne.
Le terme \dyke{} vise d'ailleurs à englober une plus vaste population que le terme lesbienne alors que n'importe quelle femme se considérant comme non-hétérosexuelle peut s'identifier avec ce terme et se joindre au mouvement de la marche \dyke{}.
Cette politisation s'exprime en même temps par une non-mixité qui vise à exclure les hommes de la marche --- exclusion basée sur le respect des principes de l'événement plutôt qu'une exclusion qu'on pourrait qualifier d'agressive.
Les alliés intéressés par l'événement, mais n'entrant pas dans la catégorie \dyke{} étaient invités à suivre la marche à l'extérieur, en marchant sur les trottoirs.
C'est ce que j'ai dû faire, mais j'ai pu constater l'absence d'individus se réclamant ou agissant comme \emph{alliés} (tenant des pancartes ou scandant des slogans en marge de la marche).
Selon notre observation, un seul autre homme était présent et se tenait à l'écart de la marche après avoir été informé de la non-mixité en vigueur.
Le public visé par la marche dans son ensemble est moins facilement définissable.
En fait, nous pouvons considérer que le message de la marche est orienté vers l'ensemble de la société, étant donné que la visibilité comme telle s'exprime dans l'espace public, comme je l'ai souligné précédemment.
Par contre, on ne peut ignorer le fait que la marche s'inscrit dans les pratiques d'autres événements politiques similaires.
Nous ne pouvons pas avec assurance savoir s'il s'agit ici d'une tradition ou un message qui demeure encore porté à l'organisation principale de la fierté.
Également, le choix du parcours peut nous renseigner sur le public ciblé.
Contrairement à l'édition sur laquelle~\citet{Podmore2015a} a travaillé, la marche de 2015 a complètement évité le Village gai pour commencer plus au nord et terminer dans le quartier du \anglais{Mile-End}.

Pour approfondir la question de la visibilité, l'article de \citet{Frosh2006} \citetitle{Frosh2006} offre une vue intéressante sur le partage d'un message --- \anglais{text} dans l'article --- à l'aide d'un média.
Dans un contexte social où les interactions entre individus dans l'espace public sont réduites au minimum jusqu'à l'indifférence, se rendre visible auprès d'autrui permet d'agir à contre-courant et déranger cette indifférence.

Pourtant, \citeauthor{Frosh2006} nous apprend que cette indifférence peut être une forme de respect ou d'intégration.
Les pratiques sociales dans l'espace public qui se fondent sur l'indifférence témoignent chez les individus une forme de respect mutuel qui est bien différente d'une relation basée sur l'altérité et l'incompréhension.
Si certains auteurs d'après \citeauthor{Frosh2006} considèrent la froideur des relations humaines --- l'inattention civile chez \citeauthor{Goffman1956} --- dans l'espace public comme une preuve d'un espace moralement vide et étranger, on apprend aussi que ce serait plutôt la réaction à la vision d'un autre qui sera le témoignage d'une forme de peur ou d'appréhension devant l'autre~\citep[279--280]{Frosh2006}.
L'action de réclamer cette visibilité dans la Marche Dyke pourrait être conçue comme une rupture volontaire de ce respect mutuel pour montrer que l'égalité sous-entendue n'est pas concrète et reste à faire.
Cet acte exposerait donc une limite de l'inattention civile comme concept centré sur la communication et la perception.
Ce dernier aurait en fait une certaine incapacité à tenir compte des minorités conçues comme invisibles ou d'individus se regroupant autour d'une différence marginalisante et commune envers les normes hétérosexuelles et masculines.
L'appel à la non-mixité, s'il répond à une volonté de se retrouver entre dykes, permet également d'avoir un contrôle sur l'image qu'elles véhiculent collectivement par rapport à l'observateur potentiel.
Cet observateur, dans l'espace public, serait l'ensemble de la société civile.
Cette société serait constituée d'individus en position de pouvoir en ce qui concerne les axes de la sexualité et du genre, mais également d'autres individus marginalisés\todo{Traiter du   Male gaze avec les   références~\cite{Wood2004,Patterson2002,Skelton2002,Snow1989}}.

\section{Festivités s'étirant sur plusieurs jours}
\label{sec:festivitesplusieursjours}
Si les marches et manifestations sont symboliquement marquantes par l'usage de nombreux géosymboles et par une subversion partielle ou complète de l'espace public, il n’en demeure pas moins que leur présence est très circonscrite dans le temps.
Nous nous intéresserons, dans cette section, à des événements temporaires, mais s'étirant sur une longue période, de l'ordre de plusieurs journées à plusieurs semaines.
La totalité de ceux-ci consiste en des célébrations de la diversité sexuelle, certaines bien connues du public, d'autres appartenant plutôt à une certaine contre-culture du spectre \lgbt.
Certaines en particulier sont orientées vers les individus s'identifiant spécifiquement à une identité de genre ou orientation sexuelle ciblée.

\subsection{Fierté Trans}
\label{subsec:fiertetrans}
Nous avons traité brièvement de la Fierté Trans dans la section précédente à propos de la Marche Trans.
En fait, la marche s'inscrit dans l'ensemble de la Fierté Trans, celle-ci étant l'activité principale autour de laquelle tourne le reste des événements.
La Fierté Trans, comme la Marche Dyke, se déroule dans les jours précédents la fierté traditionnelle.
Elle débute avec une soirée organisée au cabaret Cléo où des spectacles variés ont lieu, ces derniers consistant surtout en des performances de dragues par des artistes trans de la communauté.
Si ces performances sont connues habituellement pour être pratiquées par des hommes homosexuels, dans ce cas-ci quelques femmes ont également joué un rôle de drag-king, soit une personnification masculine.
Pour assurer l'accessibilité de l'événement, les organisateurs ont fixé un prix libre à l'entrée; les recettes du spectacle servent au financement du groupe l'\astteq{}\footnote{Selon leur site Internet: \textquote{aSTT(e)Q a pour mission de favoriser la santé et le bien-être des personnes trans par l’intermédiaire du soutien par les pairs et de la militance, de l’éducation et de la sensibilisation, de l’empowerment et de la mobilisation.}}

\subsection{Qouleur}
\label{subsec:qouleur}
Le terme de Qouleur désigne autant le nom de l'événement que le nom du collectif derrière celui-ci.
À première vue, on peut croire que le terme Qouleur peut symboliser deux réalités, sois la diversité sexuelle par un lien avec le drapeau arc-en-ciel aux multiples couleurs et sois une désignation donnée aux personnes racisées, personnes de couleur.
Entendu comme un terme moins discriminant que le terme \emph{race}, le terme \emph{couleur} prend ses origines dans \todo{Trouver l'origine du mot couleur}.
Bien qu'utilisé fréquemment pour désigner les personnes d'origine afro, le terme de couleur dans le nom de l'événement semble plutôt signifier la totalité des individus racisés, sachant que les événements se déroulant dans ce festival ciblent certains groupes précis.
Si certains de ceux-ci sont publics, plusieurs d'entre eux visent par exemple les individus autochtones par l'appellation  \emph{bispirituelle}.

Une des activités accessibles à tous était la visite du \mai{}.
Celle-ci consistait d'abord en une exposition d'œuvres de personnes racisées, trans ou hors cette communauté, mais dont les sujets traités pouvaient être liés au thème de cette édition du festival Qouleur de 2015, l'amour et la rage.
Par contre, si cette exposition était publique, il n'était pas possible d'interagir avec d'autres individus de la communauté; lors de notre visite, en semaine et sur l'heure du midi, personne n'était présent, sinon les chargés de l'accueil du musée.

Parmi les œuvres, on en retrouvait certaines mettant en scène des membres de la communauté.
La plus importante, en taille et en centralité dans le local de l'exposition, était une vidéo projetée sur une toile géante dans laquelle des personnes interviewées décrivaient leurs expériences de rage et d'amour en tant que trans et individus racisés.
On pouvait reconnaître là certaines personnes qui participaient également à la Fierté Trans.

Comme d'autres événements, certaines activités ont tourné autour de la sexualité.
C'est le cas d'une soirée \bdsm{} pour les individus trans et queers.
Étant donné le degré d'intimité impliquée dans ce genre d'activité, le lieu n'était pas communiqué sur Facebook et il était nécessaire de prendre contact avec les organisateurs pour être convié à l'événement.
Par manque d'informations, il est difficile de dire si cet événement était ouvert à toute personne s'identifiant comme trans, ou queer, ou s’il était nécessaire de connaître personnellement une des personnes organisatrices de l'événement.
Également, aucun visuel n'était utilisé.

\subsection{Pervers/Cité}
\label{subsec:perverscite}
Pervers/Cité est un des nombreux événements parallèles à la Fierté Montréal.
L'existence de Pervers/Cité se démarque comme un événement créé en réaction à l'évolution des festivités plus traditionnelles de la communauté gaie montréalaise.
En effet, à la création de Pervers/Cité existait déjà un organisme organisant la fierté, Divers/Cité.
Ce dernier a d'abord été un événement à tendance communautaire et politique pour devenir un festival plutôt orienté vers la fête avec un gain de popularité des festivités.
Pervers/Cité est né d'une réponse à cette évolution qui a été dénoncée comme marchande et non-rassembleuse pour l'ensemble des minorités sexuelles montréalaises.
En fait, progressivement, Divers/Cité s'est mis à délaisser le côté militant des festivités de la fierté pour  devenir un événement s'apparentant à un festival musical.
Divers/Cité a été à sa fin un événement essentiellement touristique ou ludique par la place prépondérante des spectacles de \emph{DJ} et autres musiciens liés à la musique électronique.
Organisée par une faction encore très politisée, Pervers/Cité souhaitait alors offrir une alternative radicale aux festivités de Divers/Cité.
L'événement a donc été planifié de façon à se dérouler durant les mêmes journées, mais à l'extérieur du périmètre occupé par Divers/Cité, sois la rue Sainte-Catherine et plus tard le vieux port de Montréal.
Cette tendance s'est poursuivie aujourd'hui; le festival s'étend encore sur un vaste territoire qui est visible à la figure \todo{mettre la figure}.
En général, les espaces utilisés par les organisateurs de Pervers/Cité ne sont pas des lieux nécessairement occupés quotidiennement par des individus queers.
Ceci s'explique notamment par les moyens limités d'un tel type d'organisation, fonctionnant sans financement autre que le bénévolat et les contributions volontaires, et aussi par une volonté de demeure accessible.
Les centres communautaires, les locaux d'universités ou les espaces publics ont donc été choisis pour limiter les frais encourus tout en n’obligeant pas les participants à acheter des consommations ou à payer des prix d'entrée élevés comme ceci peut être le cas dans les bars.
Le Salon du livre Queer entre les couvertures fait figure d'exception en se déroulant dans le village.
Une partie des ateliers ont eu lieu dans le local de l'Astérisk, un espace communautaire visant la communauté \lgbt{}, situé sur la rue Amherst.

\begin{figure}[ht]
	\centering
	\includegraphics[width=15cm]{images/fig33.jpg}
	\caption{Carte produite par Pervers/Cité présentant les différents lieux des activités du festival}\label{fig:carte_perverscite}
\end{figure}

Les visuels utilisés dans les publicités et événements Facebook n'avaient pas un thème ou une esthétique commune; on peut penser que les activités étaient organisées en soi par une multitude de personnes proposant leur propre matériel publicitaire, mais nous manquons d'informations à ce sujet.
En général par contre, les visuels utilisaient des termes à connotation vulgaire, sinon sexuelle, et souvent en anglais.
Par exemple, dans le cadre d'une projection publique et d'un spectacle, le nom choisi est \emph{Whorelock/up}, pouvant être traduit par \emph{Enfermé par les putes}.
Dans un autre cas, l'événement nommé était \anglais{Sick, Sad Summer}, traduisible par \emph{Triste été malade} et dessiné de façon apparentée à du sang, sur laquelle est utilisée l'image d'une femme corpulente et partiellement nue sous une averse.
Le nom de Pervers/Cité apparait dans un coin, dont le point sur le \enquote{i} de cité (et tous les autres de l'affiche) est remplacé par un cœur brisé.
Si la première image ne présentait pas de code particulier en dehors du titre, on trouve ici une image plus complexe et codifiée.
On peut faire le lien entre cette affiche et d'autres ayant été utilisées par des événements non queers dans lesquels la chaleur de l'été figure, avec toute une série de signifiants positifs faisant l'expression du bonheur et du plaisir.

\begin{figure}
\centering
\subcaptionbox{Affiche de l'activité \anglais{Sick, Sad Summer}\label{fig:affiche_sicksadsummer}}
{\includegraphics[width=9cm]{fig34.jpg}}
\subcaptionbox{Événement de 2008 (et événement mixte)\label{fig32}}
{\includegraphics[width=6cm]{fig32.jpg}}
\caption{Exemple d'affiches variées d'activités organisées dans le cadre de Pervers/Cité}\label{figs3132}
\end{figure}

\subsubsection{Activités}
\label{subsec:activitesperverscite}
Les activités offertes dans le cadre de Pervers/Cité sont très variées.
En effet, les sujets couverts semblent parfois ne pas avoir de liens directs: certains traitent plus particulièrement du militantisme, d'autres de la sexualité et certains s'apparentent plus à des jeux ou des activités sociales.
En général du moins, l'ensemble de celles-ci touchent des thématiques subversives, soit par l'utilisation particulière de l'espace, soit par les sujets traités.

Parmi les activités tournant autour de la thématique du militantisme, on peut nommer la projection du film \emph{Pride} et de la conférence subséquente sur les liens entre syndicalisme et droits des communautés \lgbt{}.
Ne se déroulant pas nécessairement dans l'espace public, ces deux activités ont eu lieu dans des lieux semi-publics, sois un centre communautaire et un local de l'Université Concordia.
Pour l'occasion était invité un des acteurs  des premières liaisons entre syndicalistes et groupe \lgbt{} en Grande-Bretagne.

Le Salon du livre \emph{Queers entre les couvertures} est une autre des activités importantes du festival Pervers/Cité.
Il s'agit également d'une des rares activités à se dérouler dans le village, dans le cas présent sur la rue Amherst dans le Centre communautaire de loisirs Sainte-Catherine d'Alexandrie.
Prenant l'ensemble de l'espace loué, le lieu
\subsection{Fierté Montréal 2016}
\label{subsec:fiertemontreal2016}
La Fierté Montréal propose à Montréal l'un des plus importants rassemblements \lgbt en quantité d'individus impliqués et d'événements se déroulant dans la ville de Montréal.
La grande majorité de ces activités sont organisées dans le Village gai ou dans ses environs.
Une exception d'importance est le lieu choisi pour la marche de la fierté, celle-ci se déroulant sur le boulevard René-Lévesque situé au sud de la rue Sainte-Catherine, l'axe principal du Village gai.
Ce choix est récent en ce qui concerne l'histoire de la communauté \lgbt montréalaise.
Avant, la marche de Divers/Cité était dans le Village, mais par une augmentation croissante de l'affluence de l'événement, il est devenu nécessaire de modifier le trajet de cette dernière.
Fierté Montréal organisant aujourd'hui les festivités de la fierté, les organisateurs et organisatrices ont choisi le boulevard pour des raisons qui semblent être d'ordre de capacité.
En effet, ce boulevard à quatre voies est beaucoup plus large que la rue Sainte-Catherine, surtout durant la saison estivale.
Cette dernière devient piétonne durant les festivités et une partie de la surface est occupée par les terrasses des restaurants et des bars.
La présence d'une foule très importante, confirmée par notre présence à l'édition 2016, avec un très grand nombre de chars allégoriques, semble justifier ce choix de localisation.

La journée communautaire est l'une des activités principales de la semaine de la fierté, quoiqu’étant l'une des moins festives.
En effet, durant celle-ci, l'espace de la rue Sainte-Catherine, demeurant piétonnier, est occupé cette fois par un grand nombre de kiosques et d'installations d'abord réservées aux groupes communautaires gravitant au sein du spectre \lgbt.
On retrouve tout de même une présence commerciale importante, avec plusieurs banques et entreprises offrants des produits destinés, de près ou de loin, à la communauté \lgbt.
Nous pensons surtout aux entreprises Viagra et Trojan qui offrent des produits pour aux hommes principalement, ces derniers étant historiquement investis dans le Village.
D'autres acteurs privés étaient présents, comme \anglais{General Mills}, qui faisait concorder l'esthétique d'un de leurs produits, les céréales \anglais{Lucky Charms}, et l'arc-en-ciel utilisé par les communautés \lgbt.
Comme durant l'ensemble des activités de la semaine organisée par Fierté Montréal, un grand nombre d'institutions bancaires était présent.
On peut noter principalement le cas de la Banque TD qui a été commanditaire des festivités, mais les autres banques majeures de la province et les caisses Desjardins participaient également.

La marche en soi est un événement mettant en scène un grand nombre de symboles très variés, représentant en effet la disparité des organisations s'arrimant de près ou de loin à la culture ou aux buts politiques des communautés \lgbt.
Prenant ses assises dans... \todo{écrire l'histoire des marches de la fierté}.
Les premiers instants de la marche marquent en soi la volonté des organisateurs de montrer le plus grand nombre d'identités possibles.
Un des premiers groupes à ouvrir la marche a été un contingent de drapeaux des différentes identités affiliées au spectre \lgbt{}.
Cette affiliation variait d'une identité à l'autre, certains drapeaux ne représentant pas nécessairement des groupes inscrits dans l'acronyme \lgbt.
Les drapeaux identifiés dans ce contingent correspondent aux identités gaies, lesbiennes, bisexuelles, trans, \emph{genderqueers}, asexuelles, amoureuses du latex, du cuir, alliées.
Il était possible durant le reste de la marche de constater la présence de différents contingents liés à ces identités.
En ce qui concerne les identités plus connues publiquement, comme l'identité homosexuelle masculine, celle-ci faisait apparition dans un grand nombre de contingents, souvent sportifs.
Les autres identités étaient moins représentées, comme les personnes trans ou les femmes lesbiennes, ou encore totalement absentes, en ce qui concerne les identités genderqueers ou asexuelles, quoiqu'il soit envisageable que des individus s'identifiant à ces identités participent à des contingents sans cette visibilité.
En effet, étant les différences entre le genre et l'orientation sexuelle, nous pouvons croire que des personnes trans prenaient part à des groupes plutôt affiliés à leur orientation sexuelle.
Il demeure par contre qu'il existe une très grande inégalité dans la visibilité de ces différentes identités, pour des raisons que nous traiterons au chapitre suivant.

La présence de groupes politiques est un autre fait important quant à la mixité de la marche.
Presque tous les partis fédéraux canadiens et provinciaux québécois étaient présents, à l'exception notable du parti conservateur du Canada.
Par contre, cette présence était inégale quant à la visibilité de l'ensemble de ces derniers.
Nous avons noté la forte visibilité du parti libéral du Canada qui a utilisé un autobus et un grand nombre d'affiches représentant le logo du parti aux couleurs de l'arc-en-ciel.
Québec Solidaire avait aussi une présence importante, mais comme les autres partis, n'adaptait pas son visuel aux couleurs mises de l'avant dans la marche.
Pour plusieurs de ces partis, les chefs étaient présents pour saluer la foule.

En plus des partis, des groupes, certains communautaires, ont aussi participé en portant un message politique.
C'était le cas notamment du Séro Syndicat / Blood Union, qui, au lieu de seulement faire acte de présence comme certains organismes communautaires traitant de ces dynamiques, portaient des slogans et des visuels similaires ou identiques à ceux de \anglais{Act-Up}.
Par exemple, dans ce contingent, il était possible de voir le triangle rose et le slogan \anglais{Silence = Death} ou Silence = Mort.

Tout comme la journée communautaire, les banques étaient présentes et avaient souvent leurs propres contingents aux couleurs de leur logos et de l'arc-en-ciel.
\subsubsection{Activités}
\label{subsec:activitesfiertemontreal}



\subsection{Fête Arc-en-ciel}
\label{subsec:fetearcenciel}
Dernier événement temporaire traité dans cette section, la Fête Arc-en-ciel se déroule dans la ville de Québec durant le congé de la fête du Travail, sois à la première fin de semaine du mois de septembre.
Comme pour la Fierté Montréal, dans ce cas-ci aussi nous avons un exemple d'un événement d'envergure ne se déroulant pas durant le début de l'été contrairement aux autres événements similaires ailleurs dans le monde.
Contrairement à la Fierté de Montréal, le moment choisi est particulièrement tard dans la saison estivale.
Normalement, les événements festifs cherchant à accueillir de vastes publics visent surtout les mois durant lesquels les personnes ayant des emplois traditionnels ou les étudiants sont en vacances.
Par contre, cette décision pourrait s'expliquer par une proximité géographique de Montréal.
La métropole pourrait en effet faire une trop grande compétition aux festivités de la capitale dans l'éventualité où les deux événements auraient lieu en même temps.
Cette hypothèse se confirme également par la présence de plusieurs groupes de Québec dans les événements publics de la fierté montréalaise, durant les journées communautaires et le défilé principalement.
Aussi, à l'échelle temporelle, la Fête Arc-en-ciel s'étire dans une plus courte durée que la fierté montréalaise.
En effet, celle-ci ne dure que 4 jours, sois du jeudi au dimanche, avec une quantité d'activités prévues beaucoup plus importante dans les deux dernières journées, sachant que pour plusieurs il s'agit de journées de congé.

Tout comme la fierté montréalaise, la Fête Arc-en-ciel de Québec occupe symboliquement l'espace public de façon marquée.
La partie ouest de la rue Saint-Jean-Baptiste et le carré d'Youville ont été décorés de nombreux fanions, affiches et ornements aux couleurs de l'arc-en-ciel.
Également, durant le samedi et le dimanche, la rue Saint-Jean-Baptiste sur une partie de sa longueur est fermée à la circulation automobile.
D'autres décorations arc-en-ciel sont alors ajoutées sur les rues et sur les trottoirs, ceux-ci n'étant plus utilisés obligatoirement par les piétons.
La côte Sainte-Geneviève est également décorée tout en ne permettant pas la circulation piétonnière.
En effet, si celle-ci est toujours fermée à la circulation automobile, durant les deux derniers jours de la Fête Arc-en-ciel elle devient un espace où sont organisées des activités pour la Fête Arc-en-ciel. Parmi ces activités, on dénombre par exemple des \anglais{stands} de photographie, des jeux pour enfants, et également des lieux de restauration.

\subsubsection{Activités}
\label{subsec:activitesfetearcenciel}
Étant donné le cadre temporel plus restreint, moins d'activités se déroulent dans le cadre de la Fête Arc-en-ciel que dans plusieurs des événements traités précédemment.
Néanmoins, une certaine diversité existe analogue à la fierté montréalaise, sois un mélange de soirées dans les discothèques et bars, des conférences et des spectacles sur la place publique.

Comme dans la fierté montréalaise, on peut constater une volonté de l'organisation de cibler un spectre plus large que la simple communauté gaie masculine.
Une conférence par exemple a eu lieu sur les enjeux vécus par les personnes trans la veille du début de la Fête Arc-en-ciel et à l'extérieur de celle-ci, dans un bar de la basse-ville.
Si la visibilité de cet événement est moindre que plusieurs autres sur le plan physique, on retrouvait quand même celui-ci dans la publicité du festival ainsi que sur \emph{Facebook}.

Certains événements ont ciblé plus particulièrement la communauté lesbienne, comme \todo{trouver exemples}.

En ce qui concerne les événements pour hommes gais, on peut nommer les soirées \emph{Boys Gone Wild}\todo{Vérifier si c'est le bon nom} et \todo{trouver l'autre événement} se déroulant tous les deux au Bar \emph{Saint-Matthew's}, un bar pour hommes comme nous l'avons souligné à la section \todo{trouver le numéro de section si déjà écrit}.
Ces activités, bien que s'inscrivant dans l'édition 2015 de la Fête Arc-en-ciel, animent également la communauté gaie de la capitale durant le reste de l'année.
Ces événements se produisent à raison de trois à quatre fois par année et sont organisés par l'Alliance Arc-en-ciel en collaboration avec le Bar \emph{Saint-Matthew's}.
Souvent thématiques, ces soirées permettent aux hommes de se rencontrer dans un espace \emph{de facto} non mixte, sans être affiché comme tel, comparativement aux activités organisées par les communautés montréalaises \emph{dykes} ou racisées.
La non-mixité sert donc ici surtout à rendre possibles des rencontres entre hommes plutôt qu'être une revendication politique.
Les thématiques choisies témoignent de cet apolitisme, par un accent mis sur des fêtes traditionnelles, comme Noël, ou certains fétichismes communs à la communauté gaie, comme l'équipement de sport, la masculinité, le cuir, etc.
D'après l'offre de bars et d'autres lieux de sociabilité dans la ville de Québec, ces activités semblent répondre à un manque de bars spécialisés permettant de satisfaire les différents goûts et intérêts des hommes gais.
Ceci est d'autant plus plus visible en comparaison avec la ville de Montréal qui possède en son sein un nombre varié de lieux de rencontre.

Ciblant une partie de la communauté plus politisée, le dernier événement a été une soirée \qu{}, celle-ci étant organisée aussi en dehors du secteur principal de la Fête Arc-en-ciel.
Il s'agissait de l'activité la plus excentrée; celle-ci s'est déroulée dans le bar L'Anti situé dans la partie nord du quartier Saint-Roch, en basse-ville de Québec.
Cet événement mixte s'inscrit dans la suite d'autres à thématiques similaires organisées dans les éditions précédentes de la Fête Arc-en-ciel ou de manière indépendante ailleurs dans l'année.
La soirée queer et ses prédécesseurs se sont surtout déroulées dans des bars alternatifs, mais non affiliés de façon évidente à la communauté \lgbt.
Malgré la charge symbolique portée par le thème \qu, il ne s'agissait pas en soi d'un événement politique, mais d'une soirée mélangeant musique, chant et danse.
Le premier spectacle était l'œuvre d'un collectif \qu rimouskois dans lequel le jeu autour du genre était clairement affiché par l'usage du drag.
Il s'agissait également pour cette occasion d'une activité de financement pour le \ggul.
Si plusieurs autres soirées visaient un genre en particulier, celle-ci se démarquait surtout par la jeunesse de l'auditoire et la mixité des personnes présentes.
Nous pouvons croire que le terme \qu, utilisé récemment au Québec \todo{voir si je pourrais citer Laprade ici}, semble être connu surtout par la jeunesse \lgbt.
Certains participants qui semblaient plus âgés que la moyenne du public sont venus au début de la soirée pour participer, mais ils ne sont pas restés tout au long, en général.

Une autre particularité de la Fête Arc-en-ciel était d'inviter des personnalités connues ou représentatives des minorités sexuelles.
Dans l'édition de 2016, la personne choisie a été Michèle Richard, désignée comme icône sois pour sa popularité chez une partie du spectre \lgbt{}, sois pas son interprétation de la chanson disco \emph{I will survive} de \todo{Trouver le nom de la chanteuse}.
Si cette chanson traite normalement du sentiment de rupture et de la volonté d'indépendance de la chanteuse, dans ce cas précis, l'expression \emph{I will survive}\footnote{\emph{Je vais survivre} dans l'interprétation de Michèle Richard}, fait écho à la résilience des communautés \lgbt depuis sa sortie de la clandestinité et plus particulièrement de la crise du \vih{}.

Enfin, on peut souligner le virage familial mis de l'avant par l'organisation de la Fête Arc-en-ciel dans la volonté d'organiser un pique-nique.
En effet, ce type d'événement semble plutôt être une occasion pour les familles, homoparentales ou non, de se rencontrer et de partager un moment ensemble.
Cette absence de contenu politique ou festif peut témoigner d'une ouverture perçue plus grande de la société au sujet des enjeux \lgbt, ici la famille, au sein des marcheurs.
Un contexte sans revendications dans lequel les participants pratiquent une activité pouvant être qualifiée de normale --- concordant avec la norme de la famille hétérosexuelle --- montre plutôt que ces familles peuvent être semblables aux autres et utiliser l'espace public comme n'importe quel ménage.

La marche en soi est un événement mettant en scène une quantité et une variété importante de symboles, représentant la disparité des organisations s'arrimant de près ou de loin à la culture ou aux buts politiques des communautés \lgbt.
En effet, tout comme l'équivalent montréalais, on peut constater un grand nombre de drapeau et de logos disséminés entre les participants.
Également, plusieurs personnalités publiques s'affichent comme adhérant à la marche.
Étant donné le cadre plus réduit de la marche de Québec en comparaison avec celle de Montréal, l'ensemble des participants étaient réunis en une seule congrégation marchant dans les rues de la ville.
En soi, le tout était comparable à une manifestation politique plutôt qu'à un défilé de la fierté plus traditionnel.
En effet, dans ceux-ci, les groupes sont séparés par thématiques, ils utilisent des véhicules divers pour occuper l'espace de façon plus évocatrice et structurée et le public est à l'extérieur dans une position d'observation, ce qui n'a pas été le cas ici.
Plutôt, les personnes connues, souvent des politiciens, se placèrent à l'avant de la marche et ont tenu une bannière fournie par l'organisation de l'événement permettant d'identifier facilement le contexte de cette marche.
D'autres individus plus visibles étaient aussi présents, sois des personnes appartenant au spectre \bdsm{}.
Ceux-ci étaient localisés directement derrière les personnalités publiques et sont apparus à plusieurs reprises dans les photographies que nous avons prises de l'événement.
Nous avons noté la présence de participants vêtus de latex, de cuir et également de deux pratiquants du \anglais{puppy play}, par les symboles marquant leurs vêtements, mais aussi l'usage de masque à l'effigie de chien.
Ces personnes détonnaient du reste du défilé sachant que les autres n'étaient pas vêtus de façon particulière, à l'exception de quelques accessoires à l'effigie du drapeau arc-en-ciel.

Ce format de marche permit de faire participer le public.
En effet, dans un défilé traditionnel, la foule est invitée à occuper les marges, sois les trottoirs autour des boulevards ou des rues utilisées.
Elle est alors bien souvent immobile, mais les participants ont ainsi la possibilité de voir l'ensemble de la marche.

Dans celle de Québec, la marche était composée d'individus n'ayant pas de rôles particuliers à jouer.
Le public extérieur était donc surtout constitué de passants, de touristes ou encore de consommateurs dans les restaurants et les magasins longeant les rues où passait la marche.
Ce public était plutôt passif, mais à de nombreuses reprises, les personnes à l'avant de la marche lançaient des appels à crier et à applaudir les participants et, surtout, à montrer leur ouverture à la diversité sexuelle.
En général, les gens répondaient à la demande et nous n'avons pas remarqué de réactions négatives ou d'homophobie en réaction à ces appels.
Certains des marcheurs avaient amené avec eux des pancartes affichant des slogans, mais en comparaison moins de symboles étaient convoyés par le défilé de Québec que celui de Montréal, en proportion par rapport à la quantité de participants.

Tout comme le défilé de Montréal, la marche de Québec s'est déroulée en écart du centre géographique des célébrations.
Commençant au Carré d'Youville où étaient organisés plusieurs spectacles extérieurs, la marche s'est poursuivie sur la rue Saint-Jean-Baptiste dans la section opposée au secteur occupé par les kiosques des journées communautaires.
Après être passée devant un lot de restaurants et de commerces, elle a continué dans une partie moins fréquentée du quartier pour se terminer face au parlement provincial.
Ce choix de parcours contraste particulièrement avec celui de Montréal où l'axe choisi, le boulevard René-Lévesque, est beaucoup plus large et offre une plus grande visibilité.
Celui de la marche de Québec, s'il a débuté dans un endroit très passant, s'est poursuivi dans des espaces peu visités et visibles du quartier, au détriment d'une certaine visibilité.
Par contre, ce choix de passage semble avoir permis une plus grande densité de personnes participantes et ainsi, donner l'impression d'une grande participation, un effet plus difficile à atteindre sur un boulevard plus large.

\begin{figure}[ht]
\centering
\includegraphics[width=12cm]{dense.png}\caption{\label{fig:label} }
\end{figure}


\subsubsection{Activités}
\label{subsec:activitesfiertemontreal}



% \subsection{Fête Arc-en-ciel}
% \label{subsec:fetearcenciel}
% Dernier événement temporaire traité dans cette section, la Fête Arc-en-ciel se déroule dans la ville de Québec durant le congé de la fête du Travail, sois durant la première fin de semaine du mois de septembre.
% Comme pour la Fierté Montréal, dans ce cas-ci aussi nous avons un exemple d'un événement d'envergure ne se déroulant pas durant le début de l'été contrairement aux autres événements ailleurs dans le monde.
% Le moment choisi est d'ailleurs particulièrement tard dans la saison estivale, sachant qu'à ce moment de l'année plusieurs personnes ne sont plus en vacances comme il est coutume de le faire durant les mois de juin, juillet et d'août, ou sont de retour à l'école pour les étudiants.
% Par contre, on peut croire que cette raison s'explique par une proximité géographique de Montréal qui ferait une trop grande compétition aux festivités de la capitale si jamais les deux événements auraient lieu en même temps.
% Cette hypothèse se confirme également par la présence de plusieurs groupes de Québec dans les événements publics de la fierté montréalaise, durant les journées communautaires et le défilé principalement.

% La Fête Arc-en-ciel s'étire dans une plus courte durée que la fierté montréalaise.
% En effet, celle-ci ne dure que 4 jours, sois du jeudi au dimanche, avec une quantité d'activités prévues beaucoup plus importante dans les deux dernières journées, sachant que pour plusieurs il s'agit de journées de congé.
% Tout comme la fierté montréalaise, la Fête Arc-en-ciel de Québec occupe symboliquement l'espace public de façon marquée: une partie importante de la rue Saint-Jean-Baptiste et le carré d'Youville sont décorés de nombreux fanions, affiches et ornements aux couleurs de l'arc-en-ciel.
% Également, durant le samedi et le dimanche, la rue Saint-Jean-Baptiste sur une partie de sa longueur est fermée à la circulation automobile.
% D'autres décorations arc-en-ciel sont alors ajoutées sur les rues et sur les trottoirs, ceux-ci n'étant plus utilisés obligatoirement par les piétons.
% La côte Sainte-Geneviève est également décorée tout en étant fermée d'une certaine façon à la circulation piétonnière.
% En effet, si celle-ci est toujours fermée à la circulation automobile, durant les deux derniers jours de la Fête Arc-en-ciel elle devient un espace où sont organisées des activités pour la Fête Arc-en-ciel et devient également un lieu de restauration.

% \subsubsection{Activités}
% \label{subsec:activitesfetearcenciel}
% Étant donné le cadre temporel plus restreint, moins d'activités se déroulent dans le cadre de la Fête Arc-en-ciel que dans plusieurs des événements traités précédemment.
% Néanmoins, une certaine diversité existe similaire à la fierté montréalaise, sois un mélange de soirées dans les discothèques et bars, des conférences et des spectacles sur la place publique.

% Comme dans la fierté montréalaise, on peut noter une certaine volonté de l'organisation de cibler un spectre plus large que la simple communauté gaie masculine.
% Une conférence par exemple a eu lieu sur les enjeux vécus par les personnes trans la veille du début de la Fête Arc-en-ciel et à l'extérieur de celle-ci.
% Si la visibilité de cet événement est moindre que plusieurs autres sur le plan physique, on retrouvait quand même l'événement dans la publicité du festival ainsi que sur \emph{Facebook}.

% Certains événements ont ciblé plus particulièrement la communauté lesbienne, comme \todo{trouver exemples}.

% En ce qui concerne les événements pour hommes gais, on peut nommer les soirées \emph{Boys Gone Wild}\todo{Vérifier si c'est le bon nom} et \todo{trouver l'autre événement} se déroulant tous les deux au bar \emph{Saint-Matthew's}, un bar pour hommes comme nommé à la section \todo{trouver le numéro de section si déjà écrit}.

% Une autre particularité de la Fête Arc-en-ciel et d'inviter des personnalités connues ou représentatives des minorités sexuelles.
% Dans l'édition de 2016, la personne invitée a été Michèle Richard, désignée comme icône sois pour sa popularité chez une partie du spectre \lgbt{}, sois pas son interprétation de la chanson Disco \emph{I will survive} de \todo{Trouver le nom de la chanteuse}.
% Si cette chanson traite normalement du sentiment de rupture et de la volonté d'indépendance de la chanteuse, dans ce cas précis, l'expression \emph{I will survive} ou \emph{Je vais survivre} dans l'interprétation de Michèle Richard, fait écho à la résilience des communautés \lgbt depuis sa sortie de la clandestinité et plus particulièrement de la crise du \vih{}.

% Enfin, on peut souligner le virage familial mis de l'avant par l'organisation de la Fête Arc-en-ciel dans la volonté d'organiser un pique-nique.

%%% Local Variables:
%%% mode: latex
%%% TeX-master: "../../memoire-maitrise"
%%% End:

%!TEX root = ../../memoire-maitrise.tex

\chapter{De l'inclusion à la différence par le symbole}
\label{cha:de_l_inclusion_la_diff_rence_par_le_symbole}

% \chapterprecishere{\textquote{Everyone needs a place. It shouldn't be inside of someone else.” \par\raggedleft--- \textup{Richard Siken}, Crush}
Après avoir présenté les résultats de la collecte de données dans le chapitre précédent, nous procèderons dans celui-ci au portrait d'ensemble soulevé par ces dernières.
Pour chaque géosymbole analysé précédemment, nous avons offert une brève analyse de la sémiotique pour faire ressortir les messages que ceux-ci portent, les codes.
Également, l'approche géographique nous a poussé à localiser ces symboles et nous avons ainsi pu décrire du même coup l'étendue et l'emplacement des géosymboles rencontrés.
Ces informations, combinées ensemble, dressent plusieurs portraits simultanés de la panoplie des symboles rencontrés.
Par contre, nous croyons qu'il est nécessaire pour la suite, conformément aux points soulevés au chapitre 2, de faire les liens entre ces géosymboles pour arriver, par la suite, à faire une analyse d'ensemble.
Cette analyse sera donc une synthèse de notre démarche précédente.
Elle pourra subséquemment servir à faire la comparaison avec d'autres territoires, dans d'autres villes ailleurs qu'au Québec dans une perspective de facilitation d'autres analyses et comprendre si les populations \lgbt{} hors Québec sont similaires a celles d'ici.

\todo{Fouiller \cite{Fyfe1988}, contenu pertinent selon \cite[11]{Rose2012}}

\subsection{Communauté imaginée}
\label{sub:communaut_imagin_e}
% Pourra être déplacé ailleurs
Devant ces nombreuses dissensions et l'absence apparente d'échanges au sein de ce qu'on nomme communauté~\lgbt{} ou gaie et lesbienne, on peut s'interroger sur ce qui en est de cette communauté, s'il en est vraiment une.
S'intéressant d'abord au nationalisme, le texte de Benedict Anderson offre une analyse intéressante de la question de nation, alors que celles-ci représentent bien souvent de grands ensembles liés par peu de choses, sinon un lieu de naissance commun et une culture commune.
\begin{quote}	
[..] Societies are sociological entities of such firm and stable reality that their members […] can even be described as passing each other on the street, without ever becoming acquainted, and still be connected.~\citep[25]{Anderson1983}
\end{quote}

\begin{quote}
  That all these acts are performed at the same clocked, calendrical time, but by actors who may be largely unaware of one another, shows the novelty of this imagined world conjured up by the author in his readers’ minds.
\citep[26]{Anderson 1983}
\end{quote}

\begin{quote}
	
It should suffice to note that right from the start the image (wholly new to Filipino writing) of a dinner-party being discussed by hundreds of unnamed people, who do not know each other, in quite different parts of Manila, in a particular month of a particular decade, immediately conjures up the imagined community.
[\ldots]

Notice too the tone. 
While Rizal has not the faintest idea of his [28] readers’ individual identities, he writes to them with an ironical intimacy, as though their relationships with each other are not in the smallest degree problematic.43
\end{quote}

\begin{figure}[ht]
	\centering
	\includegraphics[width=16cm]{fig4.jpg}
	\caption{Fête Arc-en-ciel de Québec: marche de solidarité et journée
    communautaire\todo{corriger les étiquettes de noms sans accents}}
	\label{fig:figure4}
\end{figure}

\begin{figure}[ht]
	\centering
	\includegraphics[width=16cm]{carte_fierte.png}
	\caption[]{Fierté Montréal: Trajet du défilé, emplacement de la journée communautaire et des événements de la semaine des festivités}
	\label{fig:figure3}
\end{figure}

\begin{figure}[ht]
	\centering
	\includegraphics[width=16cm]{fig3.jpg}
	\caption[]{Marche Dyke : trajet et moments importants de la manifestation}
	\label{fig:figure3}
\end{figure}

\begin{figure}[ht]
	\centering
	\includegraphics[width=16cm]{marchetrans.png}
	\caption[]{Fierté trans: trajet de la marche et localisation des événements}
	\label{fig:figure3}
\end{figure}

\section{Premiers symboles liés à la diversité sexuelle}
\label{sec:premiers_symboles_li_s_la_diversit_sexuelle}


\section{Symboles politiques et identitaires}
\label{sec:symboles_politiques_et_identitaire}
Certains événements historiques récents semblent laisser croire à une moins grande nécessité d'un mouvement politique au sein des communautés \lgbt{}.
En effet, le \sida{} n'est plus une urgence à aborder pour les individus \lgbt{} comme dans la population générale avec l'institutionnalisation de la lutte a l'épidémie, plusieurs lois ont été votées visant un arrêt de l'homophobie dans les écoles ou la discrimination basée sur l'orientation sexuelle dans différentes sphères de la société.

Nos données montrent par contre que des symboles politiques existent encore aujourd'hui.
Différentes causes sont soutenues selon les groupes étudiés, alors que ces messages politiques s'opposent à l'occasion.
C'est le cas par exemple de Pervers/Cité qui, dans son nom, récupère celui d'une institution importante de la communauté \lgbt{} pour la critiquer.
Du même coup, l'organisation queer semble critiquer les normes sociales entourant la sexualité par une réappropriation du terme \emph{pervers}. 

Les événements entourant la Fierté Trans nous apparaissent être ceux mettant en scène les géosymboles les plus clairement revendicatifs.
Partageant avec la Marche Dyke certains médias comme les banderoles, pancartes et occupations de la rue sous la forme d'une manifestation, la communauté trans au moment de la collecte de données attendait l'instauration d'une loi concernant le statut de personne trans.
Ainsi, divers géosymboles rencontrés visaient ces changements légaux en cours, mais pas nécessairement tous.
Plusieurs d'entre eux semblaient plutôt envoyer un message sur la place publique vis-à-vis de la légitimité des identités trans ou genderqueer.
L'utilisation de chaînes ou le fait de cacher les mamelons d'un homme trans par exemple montre l'usage de la politique dans un but de changement social en dehors du cadre légal vers, plutôt, un assouplissement des normes sociales visant la société en général.

\section{Une variété de groupes et de symboles}
\label{sec:une_variete_de_groupes_et_de_symboles}
Nous pouvons constater avec les résultats qu'avec une plus grande agglomération apparait une plus grande diversité de lieux et de groupes.
Si la ville de Montréal est reconnue pour son Village gai, une observation plus minutieuse des médias \lgbt montre qu'un grand nombre d'individus se sont organisés entre eux dans ce territoire ou en périphérie.
Nous avons déjà fait démonstration de cette diversité dans le chapitre précédent en séparant les géosymboles de la ville de Montréal en différentes catégories selon les gens et événements rencontrés.

Pour la communauté trans, il existe un groupe affinitaire organisant la Fierté Trans, événement comportant une manifestation et une soirée spectacle, et un autre visant une portion de cette communauté, les personnes racisées, par le festival Qouleur.
Les personnes queers s'investissent notamment dans Pervers/Cité, le Salon du livre Queer entre les couvertures et dans certains contingents de la marche de la Fierté. 
Après notre collecte, nous avons également pu apprendre l'existence d'autres événements prenant place dans les lieux nommés dans la section précédente; par contre, débordant le cadre que nous nous sommes fixé, nous ne nous attarderons pas sur ceux-ci.

Les communautés lesbiennes montréalaises semblent se séparer selon l'identification comme dyke, bien qu'il soit possible que, en dehors des faits marquants que nous avons abordés, ces communautés soient moins divisées qu'il n'y parait par cette distinction.
En effet, nous avons remarqué le faible nombre de lieux spécifiquement pour lesbiennes.
Les événements étant plus nombreux, souvent apolitiques si nous ne tenons pas compte des marches organisées en août, ces personnes s'identifiant comme femmes non-straights doivent volontairement se rejoindre en de telles situations lorsqu'elles se présentent.
On peut également supposer, à la suite de la division historique amenée par \citet{Giraud2014}, que la séparation soit plutôt de l'ordre de l'âge, les différentes générations de lesbiennes ne partagent pas nécessairement les mêmes intérêts.
Nos données étant limitées en ce sens, plus de travail serait à faire avec les médias lesbiens.

Les communautés gaies masculines semblent quant à elles se séparer dans une très grande variété d'événements souvent très publicisés, mais dont les moyens variant également si on regarde l'ensemble ou l'affiliation a d'autres identités, comme le queer.
Certains espaces arrivent difficilement à s'afficher avec évidence, par manque d'expérience ou de capacité financière des personnes qui les maintiennent, alors que d'autres utilisent une variété d'outils importante. 
Parmi ces outils, nous pensons à la production de publicités professionnelles, à l'utilisation d'un grand nombre de médias, ou encore en occupant des espaces difficilement abordables par des groupes possédant moins de ressources, comme des locaux commerciaux sur des artères urbaines importantes.

\section{Des communautés à la mixité variable}
La mixité des événements et des lieux \lgbt{} n'est pas une caractéristique commune, comme nous avons pu le constater dans la présentation de nos résultats.
En effet, diverses raisons poussent certains organisateurs d'événements à limiter l'accès, ou non, à certains groupes.
Cette division s'effectue souvent sur la base du genre, mais dans certains cas, par le vécu de marginalisation.
En fait, nous pourrions formuler qu'une forme de \emph{ségrégation raciale} aurait lieu, mais cette notion ne représente par précisément la réalité de cette non-mixité.

\section{De la nature pour hommes}
\label{sec:de_la_nature_pour_hommes}
Avec une population dépassant toute autre agglomération au Québec, le grand Montréal doit posséder par défaut un nombre d'individus \lgbt{} plus important qu'ailleurs.
Avec le grand degré d'organisation que nous avons constaté à la suite d'autres chercheurs, on remarque que la population masculine, après avoir construit un territoire important qu'est le Village gai, a également investi d'autres lieux secondaires.
La ville offrant anonymat et possibilités pour des rencontres, elle demeure limitante quant à celles qui sont possibles dans les espaces publics.

Comme nous l'avons soulevé au chapitre précédent, plusieurs endroits ont été investis dans la couronne montréalaise.
Les campings par exemple semblent compléter en partie les avantages offerts par la Village gai pour les hommes.
Dans de tels territoires, les espaces ouverts (sans emmurement physique), comparables aux espaces publics urbains, sont beaucoup moins surveillés, de tels lieux de villégiature étant \latin{de facto} privés.
Selon les symboles utilisés, sois des corps masculins nus, la sexualité est mise au premier plan.
La pratique du nudisme est également encouragée selon les campings.
Dans de tels cas, on s'attend à ce que ce soit essentiellement des hommes qui fréquentent ces lieux.

La proximité de Montréal semble témoigner de la nécessité d'une population importante à proximité pour assurer la pérennité de ceux-ci.
Cette pérennité semble surtout être liée au nombre de personnes potentielles intéressées à investir ces territoires; les régions moins urbanisées ont souvent accès beaucoup plus facilement à des espaces non urbanisés et à l'apparence naturelle.
Nous utilisons le terme naturel en référence au code soulevé au chapitre précédent; en effet, ce caractère naturel est opposé ici à la notion d'urbain plutôt qu'à la notion courante de culture.
Les espaces ruraux sont tout aussi \emph{non naturels} que ceux en ville; par contre, la présence d'un couvert végétal important porte en soi une signification d'une moins grande trace anthropique.

Certains auteurs ont abordé le rapprochement de certains groupes \lgbt{} avec les milieux ruraux.
C'est le cas dans \citetitle{Bell1995a} de \citeauthor{Bell1995a}, cet article abordant les imaginaires de la campagne queer.
Selon le genre et l'identité, certains groupes rechercheraient, dans certains romans ou en regardant des photographies d'époques, différents éléments dans une territorialisation de la campagne.
Les paysages ruraux, avec leur aspect naturel, peuvent évoquer une sensualité par l'usage de métaphore sur la vie sauvage ou le secret dans l'immensité des paysages~\citep[114]{Bell1995a}.
Pour les femmes, les espaces ruraux possèdent la qualité d'être des lieux où il serait possible d'échapper à la domination hétérosexuelle et masculine des espaces urbains.
La nature serait donc un territoire vierge où il serait possible de reconstruire des microsociétés égalitaires, des utopies.

Ce sont en général des éléments que nous avons retrouvés dans les publicités des campings en nature, mais l'absence de données pour les femmes ne peut nous permettre d'affirmer l'existence de ce genre d'utopies au Québec.
En même temps, le sens politique d'un tel projet se marie mal avec le média qu'est la publicité; la visibilité d'un tel lieu ne serait pas un but recherché, au contraire.

\section{Des régions peu organisées}
\label{sec:des_regions_peu_organisees}
Le niveau d'organisation spatiale des territoires en région semble moins développé que dans les agglomérations de Québec ou de Montréal.
En effet, la durée de vie des lieux est souvent basse pour les espaces commerciaux alors que ceux qui sont communautaires, s'ils persistent dans le temps, sont également peu visibles.
Malgré tout, le magazine Fugues nous a permis de constater l'existence de tels micro territoires, ceux-ci ayant été dénombrés à la fin du chapitre précédent.
This is not working. Will this? Ça va être drôle. Mais je pense que ça va
gosser un peu au final.
Et merde, va falloir que j'arrange ça.

Historiquement, les espaces ruraux n'ont pas eu la même utilité que ceux en ville pour les individus \lgbt{}, principalement au niveau politique.
Avec la dispersion spatiale, des valeurs réputées plus conservatrices au sein de la population générale et le faible nombre de ressources ont rendu l'organisation de communauté \lgbt{} plus complexe.
Comme nous l'avons souligné dans la section précédente, les espaces utopiques où des regroupements \lgbt{} auraient pu subsister n'auraient pas nécessairement besoin d'une visibilité, rendant difficile la possibilité de les situer par le seul usage de médias visuels.

%%% Local Variables:
%%% mode: latex
%%% TeX-master: "../../memoire-maitrise"
%%% End:

% %!TEX root = ../../memoire-maitrise.tex

\chapter{Le paysage queer d'aujourd'hui}
\label{cha:le_paysage_queer_d_aujourd_hui}
Comme nous avons pu le voir dans notre analyse de données, le paysage géosymbolique des communautés \lgbt{} varie énormément, que ce soit selon les axes du genre ou de l'orientation sexuelle, ou encore d'autres qu'on ne lie pas automatiquement à la sexualité, comme la classe sociale ou l'ethnicité.
Ces variations débordant la simple notion de sexualité, elles confirment les propos de X \todo{faire une citation adéquate de Jackson et Son}, comme quoi nous ne pouvons réduire notre analyse de l'hétérosexualité au simple champ social de la sexualité, celle-ci se manifestant dans l'ensemble des sphères de la société, au même titre que le capitalisme ou l'hégémonie blanche, par exemple.


Nous l'avons souligné déjà à quelques reprises dans ce mémoire, mais plusieurs limites s'imposent à l'étendue des résultats et à l'analyse que nous établissons du territoire des communautés \lgbt{} du Québec.

Y'a-t-il une place à la critique des différentes manifestations dont nous avons effectué la revue?
Il ne s'agit pas du but premier de notre travail.
Par contre, nous pouvons constater certaines des tendances que nous avons traitées dans le premier chapitre, à savoir que les enjeux d'hétéronormativité semblent se manifester.

\section{Types de symboles}
\label{sec:types_de_symboles}


\section{Sens et utilité}
\label{sec:sens_et_utilit_}
\begin{quotation}
Ces relations se construisent comme une appropriation symbolique de l'espace, sous l'effet de forces qui tantôt unissent, tantôt opposent les acteurs sociaux.
D'où l'idée qu'il existe, dans une société ou un milieu donné, plusieurs « types » et plusieurs « niveaux » de territorialités, celles-ci pouvant être symétriques ou non, selon la nature des échanges qui s'établissent dans le système (simples relations bilatérales ou coûts supérieurs à consentir qui mettent en danger la structure de ce système).\citep[41]{Courville1991}
\end{quotation}

\section{Types d'espaces rencontrés}
\label{sec:types_d_espaces_rencontr_s}


\section{Pistes de recherche futures}
\label{sec:pistes_de_recherches}

À la suite des différents résultats de ce mémoire ainsi que les conclusions soulevées par les différentes approches méthodologiques sur lesquelles s'assoit ce travail, il apparait maintenant nécessaire de poursuivre notre démarche auprès des différents groupes rencontrés sur le terrain.
En effet, le portrait dressé reste fortement influencé par ma perspective personnelle de chercheur, autant comme géographe que comme membre de la communauté \lgbt{}, avec mes a priori et une volonté de demeurer objectif qui est nécessairement partielle.

Cet exercice auprès de la population pourrait prendre diverses formes.
D'abord, nous pensons qu'il est envisageable de contacter certains organismes représentant ces groupes, qu'il s'agisse du GRIS-Montréal ou Qouleur par exemple.
Un deuxième travail de recherche pourrait se donner comme objectif d'offrir, en plus des conclusions soulevées par ce mémoire, un approfondissement analytique de l'occupation de l'espace urbain par la population qu'ils représentent \todo{reformuler}.

Cet approfondissement pourrait prendre la forme de la cartographie participative des espaces \lgbt{} par la population et pour celles-ci.
En effet, au-delà des cartographies qu'on retrouve à l'intérieur du Fugues, dans le domaine de la recherche \todo{trouver la citation de Podmore pour sa   cartographie} ou exceptionnelle dans le cadre de certains événements~\parencite{Pervers/Cite2015}, aucun outil ne centralise l'ensemble de ces connaissances.
Comme nous le soulevons dans cette recherche, les espaces \lgbt{} dans les villes de Montréal ou de Québec sont multiples et méritent, en concordance avec la volonté de certains groupes comme ceux de la Marche Dyke, une meilleure visibilité.
Cette visibilité pourrait offrir aux individus d'orientation ou de genre variés de retrouver les gens qui leur ressemble et obtenir des ressources adaptées à ceux et celles-ci, que ce soit des lieux de socialisation comme les bars réputés sécuritaires ou des cliniques offrant des soins particuliers.

Également, nous avons peu été traité d'un ensemble d'espaces dans cette recherche, soit les villes régionales et villages où s'organisent ou vivent des individus des minorités sexuelles.


Un travail de plus grande ampleur dans le champ géohistorique ouvrirait la possibilité à une étude plus approfondie des géosymboles, mais, en l'absence de ces données, il est difficile de décrire plus particulièrement les autres villes Québécoises possédant une communauté de minorités sexuelles.
On peut toutefois nommer les villes de Rimouski, Gatineau, Saguenay et Trois-Rivières comme candidates à une analyse plus approfondie.
Ces villes, par leur inscription au sein d'une structure régionale urbanisée et par leur proximité à d'autres centres urbains importants de l'est du Canada, nous apparaissent comme candidates intéressantes pour un travail subséquent.
En effet, comme nous l'avons décrit dans les chapitres précédents, nous possédons des données recensées dans ces diverses villes, soit des géosymboles ou des adresses et des contacts prouvant l'existence de telles communautés.

%%% Local Variables:
%%% mode: latex
%%% TeX-master: "../../memoire-maitrise"
%%% End:

%! TEX root = memoire-maitrise.tex
\chapter*{Conclusion}
\phantomsection\addcontentsline{toc}{chapter}{Conclusion}

Au cours de ce mémoire, nous avons d'abord pu nous pencher sur les principaux éléments théoriques permettant de s'intéresser à l'analyse des géosymboles.
Par ce tour d'horizon, nous nous sommes également intéressés à la place de l'identité dans les communautés \lgbt{} pour comprendre la pertinence d'une analyse géoculturelle de celles-ci.
Nous avons poursuivi avec une proposition méthodologique permettant un travail de terrain et une recherche en archives, le tout appuyé par des outils comme les \sig{}, l'analyse de médias et la photographie, ces deux derniers étant le propre des méthodologies visuelles alors que la première est traditionnellement plutôt rattachée à la géographie plus empirique.
La présentation et l'analyse des résultats nous ont permis ensuite de montrer la diversité des communautés \lgbt{} au Québec en permettant de déborder le Village gai, unité géographique bien connue du public et déjà analysée par la recherche.
Ainsi, plusieurs événements ont pu être analysés en-dehors de celui-ci et même de la ville de Montréal; Québec et plusieurs autres villes du Québec se sont révélées être habitées par des communautés \lgbt{} de plus ou moins grande importance, avec une visibilité variant de quelques publicités dans les médias \lgbt à des événements d'envergures prenant place durant plusieurs journées dans les espaces publics des villes traitées.
Pour la suite de cette conclusion, nous reviendrons sur les points importants abordés durant l'analyse et le cadre théorique pour terminer par les limites de la recherche ainsi que les perspectives futures de recherche.

\section{Une multiplicité de communautés}
\label{sec:une_multiplicite_de_communautes}

Comme nous avons pu le voir dans notre analyse de données, le paysage géosymbolique des communautés \lgbt{} varie énormément, que ce soit selon les axes du genre ou de l'orientation sexuelle, ou encore d'autres qu'on ne lie pas automatiquement à la sexualité, comme la classe sociale ou l'ethnicité.
Ces variations débordant la simple notion de sexualité, elles confirment les propos de X \todo{faire une citation adéquate de Jackson et Son}, comme quoi nous ne pouvons réduire notre analyse des communautés et identités sexuelles --- hétérosexuelles dans le cas de Jackson --- au simple champ social de la sexualité, celle-ci se manifestant dans l'ensemble des sphères de la société, au même titre que le capitalisme ou l'hégémonie blanche, par exemple.

Ces communautés occupent donc divers espaces temporaires ou permanents, certains se juxtaposant et d'autres se superposant, que ce positionnement soit temporel ou spatial.
En effet, comme nous l'avons vus, certains espaces dykes vont se justaposer temporellement et spatiallement aux espaces de la communauté gaie masculine organisant la Fierté de Montréal, en évitant volontaire le Village gai et en choissant un moment de l'année précédent immédiatement les festivités de la Fierté.
D'autres communautés vont profiter de ce moment de l'année pour organiser des activités parralèles durant le même moment, mais dans des espaces différents, comme pour le festival Qouleur.
Les communautés de la ville de Québec vont décaler plutôt leurs activités pour éviter une compétition directe; en fait, comme nous l'avons vus, certaines communautés vont partager des activités communes et occuper un espace beaucoup plus grand.
Certaines, plus petites, vont voir leur espaces prendre une importance moindre, et ceci est d'autant apparent par le recourt à des symboles plus communs et propagés dans les médias les plus connus, comme le Fugues en tentant du même coups d'attirer des membres des communautés plus importantes.

\section*{Sens et utilité}
\label{sec:sens_et_utilit_}

\begin{quotation}
  Ces relations se construisent comme une appropriation symbolique de l'espace, sous l'effet de forces qui tantôt unissent, tantôt opposent les acteurs sociaux.
  D'où l'idée qu'il existe, dans une société ou un milieu donné, plusieurs « types » et plusieurs « niveaux » de territorialités, celles-ci pouvant être symétriques ou non, selon la nature des échanges qui s'établissent dans le système (simples relations bilatérales ou coûts supérieurs à consentir qui mettent en danger la structure de ce système).\citep[41]{Courville1991}
\end{quotation}

\section{Limites}
\label{sec:limites}

Nous l'avons souligné déjà à quelques reprises dans ce mémoire, mais plusieurs limites s'imposent à l'étendue des résultats et à l'analyse que nous établissons du territoire des communautés \lgbt{} du Québec.

Y'a-t-il une place à la critique des différentes manifestations dont nous avons effectué la revue?
Il ne s'agit pas du but premier de notre travail.
Par contre, nous pouvons constater certaines des tendances que nous avons traitées dans le premier chapitre, à savoir que les enjeux d'hétéronormativité semblent se manifester.

Si nous revenons sur la conception geertzienne de la culture dans son analyse sémiotique, nous pouvons remarquer que notre travail s'est très peu attardé au \emph{modèles de} d'un/des modèles culturels \lgbt{}.
En effet, nous pouvons considérer \latin{a posteriori} que l'analyse des structures non-symboliques de la réalité n'a pas un sens particulier pour un sous-groupe comme les communautés \lgbt{}; cette conception du réel pourrait être celle de la société majoritaire, du modèle culturel occidental, américain, ou québecois selon la lentille.
Si ce travail ne s'y est pas attardé, ce pourrait être une avenue intéressante~\todo{à compléter}.
Une limite important de notre recherche s'est dessinée durant l'analyse des données.
Nous avons remarqué rapidement la faible quantité de données portant sur les communautés lesbiennes, trans bisexuelles et racisées et nous croyons que, bien que les journaux Fugues et Sortie aillent pour mission de traiter des sujets touchant l'ensemble des communautés \lgbt{}, ceux-ci offrent une représentation beaucoup plus grande de la communauté gaie masculine blanche à leur lectorat que des autres.
Bien qu'il nous ait semblé que peu de médias s'intéressaient aux autres communautés \lgbt, nous avons noté la présence de certains médias s'adressant particulièrement aux communautés lesbiennes, comme le magazine Sapho Mag et Lez Spread the Word.
Nous nous sommes intéressés dans la collecte à leur présence dans diverses activités organisées mais pas sur les symboles que ceux-ci peuvent porter eux-mêmes comme médias.
Il serait essentiel d'envisager dans des travaux futurs de se pencher sur ces médias pour cette raison, mais également pour agrandir notre regard sur les communautés \lgbt québecoises et possiblement repérer d'autres types d'espaces structurants pour la communauté lesbienne.
Étant donné que cette communauté apparait, au regard de notre recherche et d'autres travaux, articulée autour d'espaces temporaires, de tels médias sont essentiels pour prendre connaissance de ces espaces.
Par l'importance accordée aux médias sociaux dans la diffusion d'événements, nous avons été à même de situer certains espaces, principalement politiques comme la Marche Dyke. 
Par contre, nous croyons que les médias imprimés et numériques occupent encore une place importante dans cette diffusion et pour le partage d'informations aux communautés, comme nous le montre les médias Fugues et Sortie; ainsi, ces médias lesbiens méritent une analyse supplémentaire.

Un sujet que nous avons peu abordé dans ce mémoire est la place de la sexualité dans la construction identitaire et de sa relation avec la territorialité.
Nous nous sommes intéressés aux différents saunas et à quelques événements privés, mais la méthodologie que nous avons utilisée ne nous a pas permis de faire émerger d'éléments importants quant aux activités sexuelles des individus.
Dans des communautés liées en grande partie par l'orientation sexuelle, il est envisageable que les rapports sexuels puissent alimenter cette construction identitaire et territoriale.
On retrouve quelques exemples prometteurs quant à cette possibilité dans la littérature; notamment, \citet{Hennen2013} s'est intéressée aux communautés cuirs et bears, montrant que dans celles-ci certains bars étaient particulièrement important, et que l'accent mis sur certains pratiques sexuelles ou certains critères de désirabilité peuvent causer une scission dans une même communauté, et provoquer la constitution d'une seconde autour de critères de désirabilité différents.

\section*{Pistes de recherche futures}
\label{sec:pistes_de_recherches}

À la suite des différents résultats de ce mémoire ainsi que les conclusions soulevées par les différentes approches méthodologiques sur lesquelles s'assoit ce travail, il apparait maintenant nécessaire de poursuivre notre démarche auprès des différents groupes rencontrés sur le terrain.
En effet, le portrait dressé reste fortement influencé par ma perspective personnelle de chercheur, autant comme géographe que comme membre de la communauté \lgbt{}, avec mes a priori et une volonté de demeurer objectif qui est nécessairement partielle.

Cet exercice auprès de la population pourrait prendre diverses formes.
D'abord, nous pensons qu'il est envisageable de contacter certains organismes représentant ces groupes, qu'il s'agisse du GRIS-Montréal ou Qouleur par exemple.
Un deuxième travail de recherche pourrait se donner comme objectif d'offrir, en plus des conclusions soulevées par ce mémoire, un approfondissement analytique de l'occupation de l'espace urbain par la population qu'ils représentent \todo{reformuler}.

Cet approfondissement pourrait prendre la forme de la cartographie participative des espaces \lgbt{} par la population et pour celles-ci.
En effet, au-delà des cartographies qu'on retrouve à l'intérieur du Fugues, dans le domaine de la recherche \todo{trouver la citation de Podmore pour sa   cartographie} ou exceptionnelle dans le cadre de certains événements~\parencite{Pervers/Cite2015}, aucun outil ne centralise l'ensemble de ces connaissances.
Comme nous le soulevons dans cette recherche, les espaces \lgbt{} dans les villes de Montréal ou de Québec sont multiples et méritent, en concordance avec la volonté de certains groupes comme ceux de la Marche Dyke, une meilleure visibilité.
Cette visibilité pourrait offrir aux individus d'orientation ou de genre variés de retrouver les gens qui leur ressemble et obtenir des ressources adaptées à ceux et celles-ci, que ce soit des lieux de socialisation comme les bars réputés sécuritaires ou des cliniques offrant des soins particuliers.

Également, nous avons peu été traité d'un ensemble d'espaces dans cette recherche, soit les villes régionales et villages où s'organisent ou vivent des individus des minorités sexuelles.
L'accès à des données récentes détaillée était rendu complexe par le choix des médias que nous avons fait au début de cette recherche.
Si le magazine Fugue nous a permis de trouver des données limitées quant à l'ensemble du spectre \lgbt, ce portrait était également centré sur la communauté montréalaise.
Peu d'informations portaient sur les communautés régionales, et il nous apparait qu'un travail de plus grande ampleur serait nécessaire auprès de celles-ci, surtout au niveau géohistorique.
Si ce travail a été fait par de nombreux auteurs en ce qui concerne la ville de Montréal, avec les apports de Podmore, Chamberland et de Higgins, les villes régionales et leur histoire reste peu abordé.
Une telle perspective s'est montrée efficace pour la métropole, et nous croyons qu'elle ouvrirait la possibilité à une étude plus approfondie des géosymboles en région. 
Par contre, en l'absence de ces données, il est difficile de décrire plus particulièrement les autres villes Québécoises possédant une communauté de minorités sexuelles.

On peut toutefois nommer les villes de Rimouski, Gatineau, Saguenay et Trois-Rivières comme candidates à une analyse plus approfondie.
Ces villes, par leur inscription au sein d'une structure régionale urbanisée et par leur proximité à d'autres centres urbains importants de l'est du Canada, nous apparaissent comme candidates intéressantes pour un travail subséquent.
En effet, comme nous l'avons décrit dans les chapitres précédents, nous possédons des données recensées dans ces diverses villes, soit des géosymboles ou des adresses et des contacts prouvant l'existence de telles communautés.

Comme nous l'avons souligné précédemment, la sexualité des individus pourrait être une piste à suivre quant à l'analyse des territoires et espaces des minorités sexuelles.
En suivant la piste offerte par Hennen que nous avons nommé précédemment, la théorie des champs sexuels pourraient être une piste intéressante pour l'analyse des territoires sexuels.
Prenant ses racines dans les travaux de Bourdieu par le concept de champ et celui d'Habitus, cette théorie tente de rendre compte de la sociabilité du désir sexuel tout en portant une attention particulière à l'espace, que celui-ci soit matériel ou numérique.
Par contre, cette théorie s'intéresse moins à la culture, bien qu'il nous apparait d'intéressant de l'intégrer dans une analyse plus large des identités et cultures sexuelles pour rendre compte du désirs et des pratiques sexuelles.
Cette théorie propose également le concept de district pour rendre compte de la superposition et de la juxtaposition de multiples champs sexuels pour permettre, par exemple, l'analyse d'un quartier comme le village gai ou le plateau Mont-Royal ou diverses communautés et sous-communautés peuvent exister autour de multiples identités sexuelles ou de genre.

%%% Local Variables:
%%% mode: latex
%%% TeX-master: "../../memoire-maitrise"
%%% End:



%\bibliography{}                 % production de la bibliographie
%\printbibliography[nottype=manual,nottype=online]
%\printbibliography[type=manual,type=online,sorting=nyt]
\printbibliography%[sorting=nyt]
\appendix                       % annexes le cas échéant
%!TEX root = memoire-maitrise.tex
\chapter{Matériel de collecte de données}     % numérotée

% Please add the following required packages to your document preamble:
% \usepackage{booktabs}
% \begin{table}[]
% \centering
\begin{longtable}{ p{.20\textwidth}  p{.80\textwidth} } 
  
% \begin{tabular}{@{}ll@{}}
\toprule
Question                   & Réponse                                                                                                                       \\ \midrule
Ville                      & \begin{tabular}[c]{@{}l@{}}\text{\Circle} Montréal\\ \text{\Circle} Québec\end{tabular}                                                                  \\ \midrule
Nom                        & \emph{texte}                                                                                                                       \\ \midrule
Médium                     & \begin{tabular}[c]{@{}l@{}}\text{\Circle} Événement\\ \text{\Circle} Mobilier urbain\\ \text{\Circle} Bâtiment\end{tabular}                                          \\ \midrule
Mixité                     & \begin{tabular}[c]{@{}l@{}}\text{\Circle} Mixte\\ \text{\Circle} Non-mixte\end{tabular}                                                                  \\ \midrule
Type de non-mixité         & \begin{tabular}[c]{@{}l@{}}\text{\Square} Trans\\ \text{\Square} Lesbienne\\ \text{\Square} Personnes de couleur\\ \text{\Square} Autochtone\\ \text{\Square} Autre\end{tabular} \\ \midrule
Événement connu            & \begin{tabular}[c]{@{}l@{}}\text{\Circle} Fierté 2015\\ \text{\Circle} Fierté trans\\ \text{\Circle} Pervers/Cité\\ \text{\Circle} Qouleur\\ \text{\Circle} Non/Autre\end{tabular}           \\ \midrule
Persistance                & \begin{tabular}[c]{@{}l@{}}\text{\Circle}  Permanent\\ \text{\Circle}  Temporaire\end{tabular}                                                             \\ \midrule
Notes écrites              & \emph{texte}                                                                                                                       \\ \midrule
Notes audios               & \emph{enregistrement au dictaphone}                                                                                                \\ \midrule
Environnement sonore       & \emph{enregistrement au dictaphone}                                                                                                \\ \midrule
Photographie du géosymbole & \emph{fichier matriciel}                                                                                                           \\ \midrule
Temps                      & \emph{généré automatiquement}                                                                                                      \\ \midrule
Longitude                  & \emph{généré automatiquement} \\ \midrule
Latitude                   & \emph{généré automatiquement}                                                                                                      \\ \bottomrule
\caption[Questionnaire pour \anglais{Cloud GIS}]{Questionnaire construit sur l'application Cloud GIS pour la collecte de données sur terrain}
\label{ann:cloudgis}
\end{longtable}
% \end{tabular}
% \end{table}

%%% Local Variables:
%%% mode: latex
%%% TeX-master: "memoire-maitrise"
%%% End:
                % annexe A
\end{document}
