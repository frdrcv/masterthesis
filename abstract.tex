\chapter*{Abstract}                      % ne pas numéroter
\phantomsection\addcontentsline{toc}{chapter}{Abstract} % inclure dans TdM

\begin{otherlanguage*}{english}
This research aims to identify the geosymbols used by the gay, lesbian, bisexual, transgender and queer communities as well as those of other identities linked to sexual orientation or gender marking their territories.
  This approach is based on the theoretical works of many authors in cultural geography and try to establish the link with the contributions of queer theory and sexual geography.
  Using data collected in fieldwork in Quebec City and Montreal during the summer of 2015, the \emph{Fugues} and \emph{Sortie} print media, archival data and social networks, we have been able to build a diversified portrayal of the different territories of these communities, especially in urban settings.
  We discovered that Montreal possesses a plurality of these spaces, the gay village being one of many, certain being temporary and other permanent, depending on the need to wish they answer.
  Other spaces also exist outside of the metropolis, and if we find proofs of their existence in the studied media, this research opens the door to a bigger exploration of rural territories and small towns.
\end{otherlanguage*}
