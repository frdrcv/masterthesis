%!TEX root = memoire-maitrise.tex
\chapter*{Introduction}         % ne pas numéroter
\phantomsection\addcontentsline{toc}{chapter}{Introduction} % inclure dans TdM

%\chapterprecishere{<< Everyone needs a place. It shouldn't be inside of someone else.” \par\raggedleft--- \textup{Richard Siken}, Crush}

Les intérêts associés à cette recherche sont multiples.
Sans prétendre innover au niveau conceptuel, cette étude pourrait permettre d'abord d'associer un champ de la géographie culturelle, l'étude des géosymboles, à la géographie \emph{queer} ou sexuelle, de développer une méthodologie pour l'étude des géosymboles à la jonction de ces deux champs, et enfin, de mettre à l'étude des espaces, les villes Québecoises, peu traitées encore en études gaies et lesbiennes sauf exceptions \parencite{Chamberland1993a,Podmore2006,Podmore2001,Hebert2012,Hunt2008,Laprade2014}.

Il sera possible de développer une méthodologie pour la reconnaissance, la description et l'usage des géosymboles des espaces \emph{queers} en milieu urbain.
Il apparait important de dépasser le concept de territoire au sein de la
géographie: si les géosymboles demeurent pertinents comme il sera démontré dans la partie~\ref{sec:problematique}, il convient de complexifier l'usage de ceux-ci en géographie en prenant en compte des éléments immatériels ou temporaires dans l'espace, sachant que certains groupes culturels n'ont pas d'assises territoriales stabilisées et tangibles, mais font plutôt usage de l'espace de manière ponctuelle ou subversive \parencite{Talburt2012}.

Le domaine de la géographie sexuelle n'est pas un domaine d'étude courant dans
la recherche francophone en géographie: la majeure partie des travaux sont effectués en Anglais dans des villes Américaines ou en Europe, en Belgique ou en France \parencite{Blidon2010,Blidon2006,Cattan2010,Deligne2006}, quoiqu'on retrouve de plus en plus de travaux de recherches dans d'autre pays développés ou en voie de développement, en Chine ou à Singapour par exemple \parencite{Oswin2014a,Kong2012}.
Cela pose un problème sachant que des différences culturelles majeures existent entre États et régions notamment. 
Il apparaît donc nécessaire d'enrichir le domaine de la géographie sexuelle au Québec, un territoire peu étudié en français notamment en-dehors de la ville de Montréal.


\section*{Objectifs et hypothèse} % (fold)
\label{sec:objectifs_et_hypotheses}
Nous décrirons dans cette section l'hypothèse découlant de la problématique ainsi que les différents objectifs visés par cette recherche.

% section objectifs_et_hypotheses (end)

%MODIFICATION--------------------------------------
\subsection*{Objectif général} % (fold)
\label{sub:objectif_general}
Ce projet de recherche vise à dresser un portrait des géosymboles des espaces \emph{queers} en milieux urbains, particulièrement Montréal et Québec.
% subsection objectif_general (end)

\subsection*{Objectifs spécifiques} % (fold)
\label{sub:objectifs_specifiques}

\begin{itemize}
	\item Développer une méthodologie pour identifier et répertorier les géosymboles marquant les espaces \emph{queers} en tenant compte du caractère éphémère ou permanent, matériel ou immatériel de ceux-ci \;
	\item Rechercher et répertorier les géosymboles \emph{queers} en tenant en compte de l'environnement et de la forme de ceux-ci dans une variété de milieux urbains et les géolocaliser \;
	\item Dresser les différences et ressemblances entre les différentes expressions spatiales et symboliques de la diversité sexuelle, notamment selon le genre et l'orientation sexuelle \;
  \item Faire progresser la connaissance sur les espaces \lgbt{} urbains de la province du Québec en prenant d'abord comme terrain d'étude les villes de Québec et de Montréal puis les villes de taille inférieures selon la présence d'une communauté \lgbt.
\end{itemize}

% subsection objectifs_specifiques (end)

\subsection*{Question de recherche} % (fold)
\label{sub:hypothese}

Les groupes et individus \lgbt{} occupent l'espace d'une façon particulière selon leur propre identité et cette occupation s'exprime par une variété de géosymboles marquant la différence ou l'inclusion, de manière permanente ou temporaire.
Ces géosymboles sont apparus à partir de la révolution sexuelle, plus précisément depuis les émeutes de Stonewall aux États-Unis et se sont répandus ailleurs en occident, dont au Québec.
%AJOUT-----------------------------
Ces géosymboles apparaissent dans les milieux urbains selon des paramètres propres à l'histoire des géosymboles, du milieu urbain, de la taille de ce milieu et de sa place dans la hiérarchie urbaine.

Pour conclure, le prochain chapitre servira à la confirmation de ces choix d'objectifs de recherche par un approfondissement de la problématique de recherche.

% subsection hypothese (end)

%%% Local Variables:
%%% mode: latex
%%% TeX-master: "memoire-maitrise"
%%% End:
