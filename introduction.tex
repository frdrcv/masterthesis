%!TEX root = memoire-maitrise.tex
\chapter*{Introduction}         % ne pas numéroter
\phantomsection\addcontentsline{toc}{chapter}{Introduction} % inclure dans TdM

%\chapterprecishere{<< Everyone needs a place. It shouldn't be inside of someone else.” \par\raggedleft--- \textup{Richard Siken}, Crush}
Avant d'entâmer mes études universitaires à Québec, j'ai vécu loin des grandes villes; d'abord dans la région de Bellechasse, puis durant une bonne partie de mon enfance et de mon adolescence dans la ville de Rimouski, dans l'est du Québec.
Rimouski n'est pas la ville régionale typique organisée autour d'une industrie particulière; au contraire, il s'agit d'une ville réputée pour ses services et sa proximité avec le fleuve, possédant de nombreuses institutions gouvernementales et d'éducation.
Sans vouloir porter de jugement vis-à-vis ce type de ville, les valeurs étaient plus traditionnelles que dans une ville comme Montréal, mais restait tout de même assez progressives comparativement à l'image que l'on peut se faire des villages et villes industrielles.
Par contre, pour ceux qui débordaient un tant soit peu de la norme, les espaces où se retrouver étaient peu nombreux, à moins que cette déviance soit fondée sur l'attachement à un style musical, comme la culture punk ou metal.
Toutes deux étaient d'ailleurs bien représentées durant mon adolescence; j'y ait d'ailleurs participé.
Par contre, si la déviance était basée sur l'orientation sexuelle\ldots

Adolescent, parler de diversité sexuelle, ça se faisait dans les cours d'éducation sexuelle, quelques fois par années, par un ou une enseignante plus ou moins intéressée.
Même s'il s'agissait d'un début, c'était tout de même mieux à certains égars qu'aujourd'hui; ces cours ont été annulés pour être remplacés par des interludes par-ci par-là dans les cours considérés comme plus sérieux, comme ceux de français, de biologie, etc.
Lorsque venait la puberté et qu'on commençait à se poser des questions sur ses intérêts sexuels, sur nos amours, il y avait peu de place vers qui se tourner.
Il restait l'Internet, avec ses premiers sites de rencontres rudimentaires (mais graphiques) pour hommes de 18 ans et plus et les canaux \irc{} (beaucoup plus austères) où il était possible de publier sa petite annonce, en espérant trouver l'amour ou, plus souvent qu'autrement, une baise d'un soir.
Une fois la décision prise de se rencontrer, plusieurs options s'offraient à nous; on allait chez l'autre s'il avait un appartement, on se retrouvait dans le Tim Hortons du centre-ville pour faire connaissance, ou l'on se retrouvait en-dehors de la ville, en forêt sur le bord de la rivière Rimouski où d'autres se rencontraient aussi, chacun dans sa voiture, les fenêtres embuées.
J'ai appris tardivement qu'il s'organisait des soirées dans une taverne près du cégep, puis dans une autre, après que la précédente ferma ses portes pour être réouvert par le même propriétaire sur une autre rue.
À moins de l'apprendre par Internet, rien n'indiquait clairement que durant cette soirée-là, il était possible de trouver des gens pas tout à fait straights.
Il n'y avait pas vraiment de femmes, lesbiennes ou non, ou d'individus s'identifiant socialement comme trans.
On y voyait par contre tel chauffeur de taxi, ou tel ami du secondaire, et on rencontrait un nombre non-néglieable de nouvelles personnes qu'on ne croyait pas avoir jamais croisée, malgré la taille modeste de la ville.
Enfin, j'ai aussi entendu parler de la discothèque retro du centre-ville où plusieurs personnes non-straights allait danser sans que lieu soit considéré comme le bar gay de ville.
Il faut dire aussi que c'est là que les mineurs allaient essayer d'entrer avant d'avoir atteint la majorité et où certaines personnes plus âgées, femmes ou hommes, s'intéressaient aux jeunes qui y dansaient.

Tous ces lieux, je les ai connus ou fréquentés il y a plusieurs années, en les découvrant par le bouche-à-oreille (numérique comme matériel).
Peu de choses les identifiaient, et le reste de la population rimouskoise pouvaient passer à proximité de ces espaces, le jour comme la nuit, sans nécessairement remarquer qu'il s'agissait d'espaces queers.
Mais quand je suis arrivé à Québec, quand j'ai été à Montréal dans le Village gai, ou plus récemment, quand j'ai été à New-York ou à Philadelphie dans les centres-villes, j'ai pu voir des drapeaux multicolores, ceux qu'on voyaient parfois à la télévision et un peu partout sur Internet en allant sur les bons sites.
Le symbole de l'arc-en-ciel, bien que représentant l'ensemble de la communauté \lgbt{}, m'a toujours apparu représentant la communauté gaie masculine, et parfois aussi les femmes lesbiennes.
Bien que sachant l'existence des personnes trans, intersexes et bisexuelles, celles-ci ne semblaient pas présentes où je posais mon regard.
S'agissait-il d'un phénomène de rareté, ou étais-je tout simplement pas apte à regarder aux endroits, à m'intéresser aux bonnes personnes?
Ou peut-être que je considérais mon prochain comme étant hétérosexuel ou gay, sans autres issues possibles?

L'idée d'écrire ce mémoire m'est venue durant mes études au baccalauréat en géographie.
Domaine réputé s'intéresser à l'espace, souvent selon l'un ou l'autre des axes physiques ou humains, peu laissait envisager qu'il était possible de s'intéresser à des groupes, à des personnes plus marginalisées.
J'ai eu la chance de rencontrer certaines personnes, certains groupes, qui m'ont permi de constater que cette possibilité existait bel et bien.
Il m'est donc apparu possible d'utiliser cette discipline pour répondre à ces interrogations que j'ai cultivé durant bien des années.

Je m'inscris, dans cette recherche, dans le domaine de la géographie culturelle, en mettant de l'avant les travaux déjà effectués auprès des communautés \lgbt{}.
Les intérêts associés à cette recherche sont multiples.
Sans prétendre innover au niveau conceptuel, cette étude pourrait permettre d'abord d'associer un champ de la géographie culturelle, l'étude des géosymboles, à la géographie \emph{queer} ou sexuelle, de développer une méthodologie pour l'étude des géosymboles à la jonction de ces deux champs, et enfin, de mettre à l'étude des espaces, les villes Québecoises, peu traitées encore en études gaies et lesbiennes sauf exceptions \parencite{Chamberland1993a,Podmore2006,Podmore2001,Hebert2012,Hunt2008,Laprade2014}.

Il sera possible de développer une méthodologie pour la reconnaissance, la description et l'usage des géosymboles des espaces \emph{queers} en milieu urbain.
Il apparait important de dépasser le concept de territoire au sein de la
géographie: si les géosymboles demeurent pertinents comme il sera démontré dans la partie~\ref{sec:problematique}, il convient de complexifier l'usage de ceux-ci en géographie en prenant en compte des éléments immatériels ou temporaires dans l'espace, sachant que certains groupes culturels n'ont pas d'assises territoriales stabilisées et tangibles, mais font plutôt usage de l'espace de manière ponctuelle ou subversive \parencite{Talburt2012}.

Le domaine de la géographie sexuelle n'est pas un domaine d'étude courant dans
la recherche francophone en géographie: la majeure partie des travaux sont effectués en Anglais dans des villes Américaines ou en Europe, en Belgique ou en France \parencite{Blidon2010,Blidon2006,Cattan2010,Deligne2006}, quoiqu'on retrouve de plus en plus de travaux de recherches dans d'autre pays développés ou en voie de développement, en Chine ou à Singapour par exemple \parencite{Oswin2014a,Kong2012}.
Cela pose un problème sachant que des différences culturelles majeures existent entre États et régions notamment. 
Il apparaît donc nécessaire d'enrichir le domaine de la géographie sexuelle au Québec, un territoire peu étudié en français notamment en-dehors de la ville de Montréal.


\section*{Objectifs et hypothèse}
Ce mémoire vise rendre compte de notre recherche visant à dresser un portrait des géosymboles des espaces \emph{queers} en milieux urbains, plus particulièrement ceux des villes de Montréal et de Québec, mais aussi des villes de plus petite envergure.
Pour arriver à atteindre cet but général, nous nous avons également fixé des objectifs plus spécifiques.
D'abord, cette recherche a permis de développer une méthodologie pour identifier et répertorier les géosymboles marquant les espaces \emph{queers} en tenant compte du caractère éphémère ou permanent, matériel ou immatériel de ceux-ci. 
Ensuite, nous avons pris à tâche de rechercher et répertorier les géosymboles \emph{queers} en tenant en compte de l'environnement et de la forme de ceux-ci dans une variété relative de milieux urbains (métropolitains et non-métropolitaints) pour ensuite les géolocaliser.
Nous avons aussi voulu dresser les différences et ressemblances entre les différentes expressions spatiales et symboliques de la diversité sexuelle, notamment selon le genre et l'orientation sexuelle.
Enfin, nous avons cherché à faire progresser la connaissance portant sur les espaces \lgbt{} urbains de la province du Québec en prenant d'abord comme terrain d'étude les villes de Québec et de Montréal puis les villes de taille inférieures selon la présence d'une communauté \lgbt{}.
% subsection objectifs_specifiques (end)

\subsection*{Question de recherche} % (fold)
\label{sub:hypothese}
\todo{À simplifier}
L'ensemble des objectifs précédents ont servi à répondre à la question de recherche que nous exposerons dans les prochaines lignes.
Nous supposons que les groupes et individus \lgbt{} occupent l'espace d'une façon particulière selon leur propre identité.
Cette occupation s'exprime par une variété de géosymboles marquant la différence ou l'inclusion des individus selon, encore une fois, cette identité.
L'existence de ces espaces, et des géosymboles les marquant, est permanente ou temporaire, selon les modalités particulières de l'espace.
Ces géosymboles marquants les espaces des communautés \lgbt sont récents; leur apparition serait survenue à partir de la \emph{révolution sexuelle}, une période durant laquelle les normes sexuelles se seraient en apparence assouplies.
Plus précisément, pour les communautés \lgbt{}, ce changement débuterait à partir des émeutes de Stonewall aux États-Unis et se serait répandu ailleurs en occident, dont au Québec.
%AJOUT-----------------------------
Ces géosymboles se seraient répandus d'abord dans les milieux urbains selon des paramètres propres à l'histoire des géosymboles, du milieu urbain, de la taille de ce milieu et de sa place dans la hiérarchie urbaine.

Pour conclure, le prochain chapitre servira à la confirmation de ces choix d'objectifs de recherche par un approfondissement de la problématique de recherche.

%%% Local Variables:
%%% mode: latex
%%% TeX-master: "memoire-maitrise"
%%% End:
