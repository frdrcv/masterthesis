%! TEX root = memoire-maitrise.tex
\chapter*{Conclusion}
\phantomsection\addcontentsline{toc}{chapter}{Conclusion}

Au cours de ce mémoire, nous avons d'abord pu nous pencher sur les principaux éléments théoriques permettant de s'intéresser à l'analyse des géosymboles.
Par ce tour d'horizon, nous nous sommes également intéressés à la place de l'identité dans les communautés \lgbt{} pour comprendre la pertinence d'une analyse géoculturelle de celles-ci.
Nous avons poursuivi avec une proposition méthodologique permettant un travail de terrain et une recherche en archives, le tout appuyé par des outils comme les \sig{}, l'analyse de médias et la photographie, ces deux derniers étant le propre des méthodologies visuelles alors que la première est traditionnellement plutôt rattachée à la géographie plus empirique.
La présentation et l'analyse des résultats nous ont permis ensuite de montrer la diversité des communautés \lgbt{} au Québec en permettant de déborder le Village gai, unité géographique bien connue du public et déjà analysée par la recherche.
Ainsi, plusieurs événements ont pu être analysés en-dehors de celui-ci et même de la ville de Montréal; Québec et plusieurs autres villes du Québec se sont révélées être habitées par des communautés \lgbt{} de plus ou moins grande importance, avec une visibilité variant de quelques publicités dans les médias \lgbt à des événements d'envergures prenant place durant plusieurs journées dans les espaces publics des villes traitées.
Pour la suite de cette conclusion, nous reviendrons sur les points importants abordés durant l'analyse et le cadre théorique pour terminer par les limites de la recherche ainsi que les perspectives futures de recherche.

\section{Une multiplicité de communautés}
\label{sec:une_multiplicite_de_communautes}

Comme nous avons pu le voir dans notre analyse de données, le paysage géosymbolique des communautés \lgbt{} varie énormément, que ce soit selon les axes du genre ou de l'orientation sexuelle, ou encore d'autres qu'on ne lie pas automatiquement à la sexualité, comme la classe sociale ou l'ethnicité.
Ces variations débordant la simple notion de sexualité, elles confirment les propos de X \todo{faire une citation adéquate de Jackson et Son}, comme quoi nous ne pouvons réduire notre analyse des communautés et identités sexuelles --- hétérosexuelles dans le cas de Jackson --- au simple champ social de la sexualité, celle-ci se manifestant dans l'ensemble des sphères de la société, au même titre que le capitalisme ou l'hégémonie blanche, par exemple.

Ces communautés occupent donc divers espaces temporaires ou permanents, certains se juxtaposant et d'autres se superposant, que ce positionnement soit temporel ou spatial.
En effet, comme nous l'avons vus, certains espaces dykes vont se juxtaposer temporellement et spatialement aux espaces de la communauté gaie masculine organisant la Fierté de Montréal, en évitant volontaire le Village gai et en choisissant un moment de l'année précédent immédiatement les festivités de la Fierté.
D'autres communautés vont profiter de ce moment de l'année pour organiser des activités parallèles durant le même moment, mais dans des espaces différents, comme pour le festival Qouleur.
Les communautés de la ville de Québec vont décaler plutôt leurs activités pour éviter une compétition directe; en fait, comme nous l'avons vus, certaines communautés vont partager des activités communes et occuper un espace beaucoup plus grand.
Certaines, plus petites, vont voir leur espaces prendre une importance moindre, et ceci est d'autant apparent par le recourt à des symboles plus communs et propagés dans les médias les plus connus, comme le Fugues en tentant du même coups d'attirer des membres des communautés plus importantes.

\section*{Sens et utilité}
\label{sec:sens_et_utilit_}

\begin{quotation}
  Ces relations se construisent comme une appropriation symbolique de l'espace, sous l'effet de forces qui tantôt unissent, tantôt opposent les acteurs sociaux.
  D'où l'idée qu'il existe, dans une société ou un milieu donné, plusieurs « types » et plusieurs « niveaux » de territorialités, celles-ci pouvant être symétriques ou non, selon la nature des échanges qui s'établissent dans le système (simples relations bilatérales ou coûts supérieurs à consentir qui mettent en danger la structure de ce système).\citep[41]{Courville1991}
\end{quotation}

\section{Limites}
\label{sec:limites}

Nous l'avons souligné déjà à quelques reprises dans ce mémoire, mais plusieurs limites s'imposent à l'étendue des résultats et à l'analyse que nous établissons du territoire des communautés \lgbt{} du Québec.

Y'a-t-il une place à la critique des différentes manifestations dont nous avons effectué la revue?
Il ne s'agit pas du but premier de notre travail.
Par contre, nous pouvons constater certaines des tendances que nous avons traitées dans le premier chapitre, à savoir que les enjeux d'hétéronormativité semblent se manifester.

Si nous revenons sur la conception geertzienne de la culture dans son analyse sémiotique, nous pouvons remarquer que notre travail s'est très peu attardé au \emph{modèles de} d'un/des modèles culturels \lgbt{}.
En effet, nous pouvons considérer \latin{a posteriori} que l'analyse des structures non-symboliques de la réalité n'a pas un sens particulier pour un sous-groupe comme les communautés \lgbt{}; cette conception du réel pourrait être celle de la société majoritaire, du modèle culturel occidental, américain, ou québecois selon la lentille.
Si ce travail ne s'y est pas attardé, ce pourrait être une avenue intéressante~\todo{à compléter}.
Une limite important de notre recherche s'est dessinée durant l'analyse des données.
Nous avons remarqué rapidement la faible quantité de données portant sur les communautés lesbiennes, trans bisexuelles et racisées et nous croyons que, bien que les journaux Fugues et Sortie aillent pour mission de traiter des sujets touchant l'ensemble des communautés \lgbt{}, ceux-ci offrent une représentation beaucoup plus grande de la communauté gaie masculine blanche à leur lectorat que des autres.
Bien qu'il nous ait semblé que peu de médias s'intéressaient aux autres communautés \lgbt, nous avons noté la présence de certains médias s'adressant particulièrement aux communautés lesbiennes, comme le magazine Sapho Mag et Lez Spread the Word.
Nous nous sommes intéressés dans la collecte à leur présence dans diverses activités organisées mais pas sur les symboles que ceux-ci peuvent porter eux-mêmes comme médias.
Il serait essentiel d'envisager dans des travaux futurs de se pencher sur ces médias pour cette raison, mais également pour agrandir notre regard sur les communautés \lgbt québecoises et possiblement repérer d'autres types d'espaces structurants pour la communauté lesbienne.
Étant donné que cette communauté apparait, au regard de notre recherche et d'autres travaux, articulée autour d'espaces temporaires, de tels médias sont essentiels pour prendre connaissance de ces espaces.
Par l'importance accordée aux médias sociaux dans la diffusion d'événements, nous avons été à même de situer certains espaces, principalement politiques comme la Marche Dyke. 
Par contre, nous croyons que les médias imprimés et numériques occupent encore une place importante dans cette diffusion et pour le partage d'informations aux communautés, comme nous le montre les médias Fugues et Sortie; ainsi, ces médias lesbiens méritent une analyse supplémentaire.

Un sujet que nous avons peu abordé dans ce mémoire est la place de la sexualité dans la construction identitaire et de sa relation avec la territorialité.
Nous nous sommes intéressés aux différents saunas et à quelques événements privés, mais la méthodologie que nous avons utilisée ne nous a pas permis de faire émerger d'éléments importants quant aux activités sexuelles des individus.
Dans des communautés liées en grande partie par l'orientation sexuelle, il est envisageable que les rapports sexuels puissent alimenter cette construction identitaire et territoriale.
On retrouve quelques exemples prometteurs quant à cette possibilité dans la littérature; notamment, \citet{Hennen2013} s'est intéressée aux communautés cuirs et bears, montrant que dans celles-ci certains bars étaient particulièrement important, et que l'accent mis sur certains pratiques sexuelles ou certains critères de désirabilité peuvent causer une scission dans une même communauté, et provoquer la constitution d'une seconde autour de critères de désirabilité différents.

\section*{Pistes de recherche futures}
\label{sec:pistes_de_recherches}

À la suite des différents résultats de ce mémoire ainsi que les conclusions soulevées par les différentes approches méthodologiques sur lesquelles s'assoit ce travail, il apparait maintenant nécessaire de poursuivre notre démarche auprès des différents groupes rencontrés sur le terrain.
En effet, le portrait dressé reste fortement influencé par ma perspective personnelle de chercheur, autant comme géographe que comme membre de la communauté \lgbt{}, avec mes a priori et une volonté de demeurer objectif qui est nécessairement partielle.

Cet exercice auprès de la population pourrait prendre diverses formes.
D'abord, nous pensons qu'il est envisageable de contacter certains organismes représentant ces groupes, qu'il s'agisse du GRIS-Montréal ou Qouleur par exemple.
Un deuxième travail de recherche pourrait se donner comme objectif d'offrir, en plus des conclusions soulevées par ce mémoire, un approfondissement analytique de l'occupation de l'espace urbain par la population qu'ils représentent \todo{reformuler}.

Cet approfondissement pourrait prendre la forme de la cartographie participative des espaces \lgbt{} par la population et pour celles-ci.
En effet, au-delà des cartographies qu'on retrouve à l'intérieur du Fugues, dans le domaine de la recherche \todo{trouver la citation de Podmore pour sa   cartographie} ou exceptionnelle dans le cadre de certains événements~\parencite{Pervers/Cite2015}, aucun outil ne centralise l'ensemble de ces connaissances.
Comme nous le soulevons dans cette recherche, les espaces \lgbt{} dans les villes de Montréal ou de Québec sont multiples et méritent, en concordance avec la volonté de certains groupes comme ceux de la Marche Dyke, une meilleure visibilité.
Cette visibilité pourrait offrir aux individus d'orientation ou de genre variés de retrouver les gens qui leur ressemble et obtenir des ressources adaptées à ceux et celles-ci, que ce soit des lieux de socialisation comme les bars réputés sécuritaires ou des cliniques offrant des soins particuliers.

Également, nous avons peu été traité d'un ensemble d'espaces dans cette recherche, soit les villes régionales et villages où s'organisent ou vivent des individus des minorités sexuelles.
L'accès à des données récentes détaillée était rendu complexe par le choix des médias que nous avons fait au début de cette recherche.
Si le magazine Fugue nous a permis de trouver des données limitées quant à l'ensemble du spectre \lgbt, ce portrait était également centré sur la communauté montréalaise.
Peu d'informations portaient sur les communautés régionales, et il nous apparait qu'un travail de plus grande ampleur serait nécessaire auprès de celles-ci, surtout au niveau géohistorique.
Si ce travail a été fait par de nombreux auteurs en ce qui concerne la ville de Montréal, avec les apports de Podmore, Chamberland et de Higgins, les villes régionales et leur histoire reste peu abordé.
Une telle perspective s'est montrée efficace pour la métropole, et nous croyons qu'elle ouvrirait la possibilité à une étude plus approfondie des géosymboles en région. 
Par contre, en l'absence de ces données, il est difficile de décrire plus particulièrement les autres villes Québécoises possédant une communauté de minorités sexuelles.

On peut toutefois nommer les villes de Rimouski, Gatineau, Saguenay et Trois-Rivières comme candidates à une analyse plus approfondie.
Ces villes, par leur inscription au sein d'une structure régionale urbanisée et par leur proximité à d'autres centres urbains importants de l'est du Canada, nous apparaissent comme candidates intéressantes pour un travail subséquent.
En effet, comme nous l'avons décrit dans les chapitres précédents, nous possédons des données recensées dans ces diverses villes, soit des géosymboles ou des adresses et des contacts prouvant l'existence de telles communautés.

Comme nous l'avons souligné précédemment, la sexualité des individus pourrait être une piste à suivre quant à l'analyse des territoires et espaces des minorités sexuelles.
En suivant la piste offerte par Hennen que nous avons nommé précédemment, la théorie des champs sexuels pourraient être une piste intéressante pour l'analyse des territoires sexuels.
Prenant ses racines dans les travaux de Bourdieu par le concept de champ et celui d'Habitus, cette théorie tente de rendre compte de la sociabilité du désir sexuel tout en portant une attention particulière à l'espace, que celui-ci soit matériel ou numérique.
Par contre, cette théorie s'intéresse moins à la culture, bien qu'il nous apparait d'intéressant de l'intégrer dans une analyse plus large des identités et cultures sexuelles pour rendre compte du désirs et des pratiques sexuelles.
Cette théorie propose également le concept de district pour rendre compte de la superposition et de la juxtaposition de multiples champs sexuels pour permettre, par exemple, l'analyse d'un quartier comme le village gai ou le plateau Mont-Royal ou diverses communautés et sous-communautés peuvent exister autour de multiples identités sexuelles ou de genre.

%%% Local Variables:
%%% mode: latex
%%% TeX-master: "../../memoire-maitrise"
%%% End:

