\chapter*{Conclusion}         % ne pas numéroter
\phantomsection\addcontentsline{toc}{chapter}{Conclusion} % dans TdM

% Une thèse ou un mémoire devrait normalement se terminer par une
% conclusion, placée avant les annexes, le cas échéant. Celle-ci est
% traitée comme un chapitre normal, sauf qu'elle n'est pas numérotée.

Comme nous avons pu le voir dans notre analyse de données, le paysage géosymbolique des communautés \lgbt{} varie énormément, que ce soit selon les axes du genre ou de l'orientation sexuelle, ou encore d'autres qu'on ne lie pas automatiquement à la sexualité, comme la classe sociale ou l'ethnicité.
Ces variations débordant la simple notion de sexualité, elles confirment les propos de X \todo{faire une citation adéquate de Jackson et Son}, comme quoi nous ne pouvons réduire notre analyse de l'hétérosexualité au simple champ social de la sexualité, celle-ci se manifestant dans l'ensemble des sphères de la société, au même titre que le capitalisme ou l'hégémonie blanche, par exemple.


Nous l'avons souligné déjà à quelques reprises dans ce mémoire, mais plusieurs limites s'imposent à l'étendue des résultats et à l'analyse que nous établissons du territoire des communautés \lgbt{} du Québec.

Y'a-t-il une place à la critique des différentes manifestations dont nous avons effectué la revue?
Il ne s'agit pas du but premier de notre travail.
Par contre, nous pouvons constater certaines des tendances que nous avons traitées dans le premier chapitre, à savoir que les enjeux d'hétéronormativité semblent se manifester.

Si nous revenons sur la conception geertzienne de la culture dans son analyse sémiotique, nous pouvons remarquer que notre travail s'est très peu attardé au \emph{modèles de} d'un/des modèles culturels \lgbt{}.
En effet, nous pouvons considérer \latin{a posteriori} que l'analyse des structures non-symboliques de la réalité n'a pas un sens particulier pour un sous-groupe comme les communautés \lgbt{}; cette conception du réel pourrait être celle de la société majoritaire, du modèle culturel occidental, américain, ou québecois selon la lentille.
Si ce travail ne s'y est pas attardé, ce pourrait être une avenue intéressante~\todo{à compléter}.

\section*{Types de symboles}
\label{sec:types_de_symboles}


\section*{Sens et utilité}
\label{sec:sens_et_utilit_}
\begin{quotation}
  Ces relations se construisent comme une appropriation symbolique de l'espace, sous l'effet de forces qui tantôt unissent, tantôt opposent les acteurs sociaux.
  D'où l'idée qu'il existe, dans une société ou un milieu donné, plusieurs « types » et plusieurs « niveaux » de territorialités, celles-ci pouvant être symétriques ou non, selon la nature des échanges qui s'établissent dans le système (simples relations bilatérales ou coûts supérieurs à consentir qui mettent en danger la structure de ce système).\citep[41]{Courville1991}
\end{quotation}

\section*{Types d'espaces rencontrés}
\label{sec:types_d_espaces_rencontr_s}


\section*{Pistes de recherche futures}
\label{sec:pistes_de_recherches}

À la suite des différents résultats de ce mémoire ainsi que les conclusions soulevées par les différentes approches méthodologiques sur lesquelles s'assoit ce travail, il apparait maintenant nécessaire de poursuivre notre démarche auprès des différents groupes rencontrés sur le terrain.
En effet, le portrait dressé reste fortement influencé par ma perspective personnelle de chercheur, autant comme géographe que comme membre de la communauté \lgbt{}, avec mes a priori et une volonté de demeurer objectif qui est nécessairement partielle.

Cet exercice auprès de la population pourrait prendre diverses formes.
D'abord, nous pensons qu'il est envisageable de contacter certains organismes représentant ces groupes, qu'il s'agisse du GRIS-Montréal ou Qouleur par exemple.
Un deuxième travail de recherche pourrait se donner comme objectif d'offrir, en plus des conclusions soulevées par ce mémoire, un approfondissement analytique de l'occupation de l'espace urbain par la population qu'ils représentent \todo{reformuler}.

Cet approfondissement pourrait prendre la forme de la cartographie participative des espaces \lgbt{} par la population et pour celles-ci.
En effet, au-delà des cartographies qu'on retrouve à l'intérieur du Fugues, dans le domaine de la recherche \todo{trouver la citation de Podmore pour sa   cartographie} ou exceptionnelle dans le cadre de certains événements~\parencite{Pervers/Cite2015}, aucun outil ne centralise l'ensemble de ces connaissances.
Comme nous le soulevons dans cette recherche, les espaces \lgbt{} dans les villes de Montréal ou de Québec sont multiples et méritent, en concordance avec la volonté de certains groupes comme ceux de la Marche Dyke, une meilleure visibilité.
Cette visibilité pourrait offrir aux individus d'orientation ou de genre variés de retrouver les gens qui leur ressemble et obtenir des ressources adaptées à ceux et celles-ci, que ce soit des lieux de socialisation comme les bars réputés sécuritaires ou des cliniques offrant des soins particuliers.

Également, nous avons peu été traité d'un ensemble d'espaces dans cette recherche, soit les villes régionales et villages où s'organisent ou vivent des individus des minorités sexuelles.


Un travail de plus grande ampleur dans le champ géohistorique ouvrirait la possibilité à une étude plus approfondie des géosymboles, mais, en l'absence de ces données, il est difficile de décrire plus particulièrement les autres villes Québécoises possédant une communauté de minorités sexuelles.
On peut toutefois nommer les villes de Rimouski, Gatineau, Saguenay et Trois-Rivières comme candidates à une analyse plus approfondie.
Ces villes, par leur inscription au sein d'une structure régionale urbanisée et par leur proximité à d'autres centres urbains importants de l'est du Canada, nous apparaissent comme candidates intéressantes pour un travail subséquent.
En effet, comme nous l'avons décrit dans les chapitres précédents, nous possédons des données recensées dans ces diverses villes, soit des géosymboles ou des adresses et des contacts prouvant l'existence de telles communautés.

%%% Local Variables:
%%% mode: latex
%%% TeX-master: "../../memoire-maitrise"
%%% End:

